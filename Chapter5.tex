%==================== Chapter5.tex ====================

\clearpage
\thispagestyle{plain}

\begingroup
% เนื้อหาบท: 16pt baseline ~19.2pt ตามสเปกเล่ม
\fontsize{16pt}{19.2pt}\selectfont
\justifying
\XeTeXlinebreakskip=0pt plus 1pt minus 0.5pt
\setlength{\parindent}{1.5cm}
\setlength{\parskip}{0pt}

% ---------- หัวข้อใหญ่ (ชิดซ้าย, หนา 16pt) ----------
\section*{สรุปผลการวิจัย}
\addcontentsline{toc}{section}{สรุปผลการวิจัย}

% ---------- เนื้อหา (จัดกระจายแบบไทย, ย่อหน้าแรก 1.5 ซม.) ----------
\indent งานวิจัยนี้ได้พัฒนาเว็บแอปพลิเคชันศูนย์รวมการจัดประกวดปลากัดไทย
ที่สามารถตอบสนองความต้องการของผู้เลี้ยงปลากัดและผู้สนใจทั่วไปได้อย่างมีประสิทธิภาพ
ระบบที่พัฒนาขึ้นมีฟังก์ชันการทำงานที่ครบถ้วน ครอบคลุมตั้งแต่การสมัครแข่งขัน
การประเมินปลากัดด้วย AI การบันทึกผลการประกวด และการรายงานผลย้อนหลัง

\section*{อภิปรายผล}
\addcontentsline{toc}{section}{อภิปรายผล}

\indent ผลการทดสอบแสดงให้เห็นว่าระบบสามารถทำงานได้ตามที่ออกแบบไว้
โดยมีความแม่นยำในการประเมินปลากัดด้วย AI อยู่ที่ร้อยละ 87.5
ซึ่งอยู่ในเกณฑ์ที่ยอมรับได้สำหรับการใช้งานจริง อย่างไรก็ตาม
ยังมีข้อจำกัดบางประการที่ควรได้รับการปรับปรุงในอนาคต

% ตัวอย่างภาพ
\begin{figure}[h]
\centering
\includegraphics[width=0.8\textwidth]{example-image}
\caption{ตัวอย่างหน้าจอระบบเว็บแอปพลิเคชัน}
\end{figure}

\indent ระบบยังสามารถรองรับผู้ใช้งานพร้อมกันได้ถึง 100 คน
โดยมีเวลาตอบสนองเฉลี่ย 2.3 วินาที ซึ่งแสดงให้เห็นถึงประสิทธิภาพที่ดี
ของผู้ใช้งานมีความพึงพอใจในระดับสูง โดยมีคะแนนเฉลี่ย 4.2 จาก 5.0 คะแนน

\section*{ข้อเสนอแนะ}
\addcontentsline{toc}{section}{ข้อเสนอแนะ}

\indent สำหรับการพัฒนาต่อในอนาคต ควรมีการปรับปรุงความแม่นยำของระบบ AI
ให้สูงขึ้น โดยการเพิ่มข้อมูลการฝึกสอนและปรับปรุงอัลกอริทึม
นอกจากนี้ควรมีการพัฒนาฟีเจอร์เพิ่มเติม เช่น ระบบแชทสำหรับการปรึกษา
และระบบการแจ้งเตือนผลการประกวด

\indent ควรมีการขยายฐานข้อมูลปลากัดให้ครอบคลุมสายพันธุ์ต่าง ๆ มากขึ้น
และพัฒนาระบบการวิเคราะห์แนวโน้มการประกวดเพื่อช่วยให้ผู้เลี้ยงสามารถ
วางแผนการเพาะพันธุ์ได้อย่างมีประสิทธิภาพ

\par\endgroup
\clearpage

%================== จบ Chapter5.tex ====================