%==================== chapter3_28.tex ====================

\clearpage
\thispagestyle{empty}

\begingroup
\fontsize{16pt}{19.2pt}\selectfont
\justifying
\XeTeXlinebreakskip=0pt plus 1pt minus 0.5pt
\setlength{\parindent}{1.5cm}
\setlength{\parskip}{0pt}

% ---------- หัวบท + เขียนสารบัญบทก่อนหัวข้อย่อย ----------
\phantomsection
\addcontentsline{toc}{chapter}{บทที่ 5 บทสรุป}
\begin{center}
	{\bfseries\fontsize{18pt}{21.6pt}\selectfont บทที่ 5}
\end{center}

\vspace{\baselineskip}

% ---------- ชื่อบท ----------
\begin{center}
	{\bfseries\fontsize{18pt}{21.6pt}\selectfont บทสรุป}
\end{center}

\vspace{\baselineskip}

\indent จากการศึกษาค้นคว้า เรื่องเว็บ แอปพลิเคชัน ศูนย์ รวมการจัด ประกวดปลากัด ไทย และมี
การนำ Ai มาใช้ในการช่วยตรวจสอบประเภทปลากัด ว่าเป็นปลากัดประเภทไหน โดยมีวัตถุประสงค์
เพื่อ พัฒนาระบบเว็บ แอปพลิเคชัน ศูนย์ รวมการจัด ประกวดปลากัด ไทย โดยเราได้ ศึกษาและพัฒนา
ระบบเป็น Web Application โดยใช้ React, Visual Studio code, Supabase
การสร้างโมเดลจาก ResNet-50 เป็นการนำมาใช้ ในการสร้างต้นแบบบน Web App
Appication เนื่องจากมีการสนับสนุนร่วมกับ React,Node.js,Express โดย Web App Appication
สามารถทำการแยกประเภทปลากัด ได้ 3 ประเภทในตอนนี้สามารถทำการจำแนกได้โดยการอัปโหลดรูปภาพ 3 รูปภาพแต่เราเอารูปภาพรูปแรกเพื่อให้โมเดลดูว่าคือปลากัดประเภทไหน

\vspace{\baselineskip}

% ---------- หัวข้อใหญ่ (ชิดซ้าย, หนา 16pt) ----------
\section*{ปัญหาและอุปสรรค}
\addcontentsline{toc}{section}{ปัญหาและอุปสรรค}

\begin{sloppypar}
	\begin{enumerate}
		\item แหล่งข้อมูลในการทำ Dataset ที่เป็นรูปภาพปลากัดของแต่ละประเภท หาได้ค่อนข้างยากเพราะ
		ต้องใช้ภาพจำนวนมากต่อปลา 1 ตัว จึงต้องทำการขออนุญาตใช้ภาพจากกลุ่ม Facebook ชุมชน
		คนเลี้ยงปลากัด ที่อนุเคราะห์ให้ภาพปลากัดมาทำ Dataset
		\item เนื่องจาก Dataset ที่มีอยู่ยังไม่ครอบคลุมประเภทปลากัดพื้นบ้านทั้งหมด จึงทำให้โมเดลสามารถทำนายได้เพียง 3 คลาส ได้แก่ ปลากัดพื้นบ้านมหาชัย ปลากัดพื้นบ้านอีสานหางลาย และปลากัดพื้นบ้านภาคใต้
		\item ข้อจำกัดด้านการฝึก (Training) และการใช้งานโมเดลบนแพลตฟอร์มโฮสต์
		\begin{enumerate}
			\item ข้อจำกัดเมื่อฝึกโมเดลจริง: การฝึก ResNet50 แบบ Fine-tune ต้องใช้ทรัพยากรค่อนข้างมาก (ทั้งด้านเวลาและหน่วยประมวลผล) โดยเฉพาะเมื่อมีการทำ Augmentation และต้องจูน Hyperparameters หลายรอบ ส่งผลให้รอบการทดลอง (experiment iteration) ช้าลงและมีค่าใช้จ่ายทรัพยากรสูงเมื่อเทียบกับขนาดชุดข้อมูลที่เพิ่มขึ้น
			\item ข้อจำกัดเมื่อเผยแพร่บน Hugging Face Spaces (CPU Basic): โครงสร้างพื้นฐานที่ใช้งานแบบ \emph{CPU basic} (2 vCPU, RAM 16\,GB) ไม่มี GPU ทำให้เวลาอนุมาน (inference latency) สูงกว่าโหมดที่มี GPU โดยเฉพาะเมื่อรับภาพความละเอียดสูงหรือประมวลผลแบบหลายคำขอพร้อมกัน (concurrent requests) ส่งผลต่อประสบการณ์ผู้ใช้ในช่วงที่มีผู้ใช้งานพร้อมกัน
			\item ข้อจำกัดด้านพื้นที่และทรัพยากรหน่วยความจำ: แม้โมเดล ResNet50 จะไม่ใหญ่มากเมื่อเทียบกับสถาปัตยกรรมสมัยใหม่ แต่การรันพร้อม Pipeline ก่อน–หลังการประมวลผล (pre/post-processing) และการเก็บไฟล์เสริม (เช่น class mapping, ตัวอย่างสาธิต) อาจใช้หน่วยความจำจนกระทบเสถียรภาพของแอปเมื่อเกิดโหลดชั่วขณะ
			\item ผลกระทบจากการพักการทำงานอัตโนมัติ (Sleep after 48 hr of inactivity): เมื่อ Space เข้าสถานะพัก ระบบจะต้อง “อุ่นเครื่อง” (cold start) ใหม่ในคำขอแรกหลังตื่น ส่งผลให้การตอบสนองครั้งแรกช้าผิดปกติ และหากมีงานประมวลผลเป็นชุด (batch) หรืองานที่รออยู่ อาจสะดุดหรือหมดเวลา (timeout) ได้
			\item ความต่อเนื่องของบริการและการอัปเดต: การอัปเดตเวอร์ชันไลบรารีหรือรีสตาร์ทคอนเทนเนอร์อาจทำให้แคชโมเดลถูกล้าง ต้องดาวน์โหลด/โหลดโมเดลใหม่ (model warm-up) ทำให้ช่วงเวลาพร้อมใช้งานจริงลดลงหากไม่มีระบบแคชหรือกลไกเตรียมความพร้อม
		\end{enumerate}
		\item ผลกระทบต่อระบบเว็บแอป “ศูนย์รวมการจัดประกวดปลากัดไทย”
		\begin{enumerate}
			\item ประสบการณ์ผู้ใช้: ช่วงเวลาหน่วงสูงในคำขอแรกหลัง Space ตื่น หรือเมื่อมีผู้ใช้พร้อมกันหลายราย อาจทำให้ผู้ใช้รู้สึกว่าระบบช้า/ไม่เสถียร
			\item ข้อจำกัดด้านคุณภาพการทำนาย: เพราะชุดข้อมูลยังครอบคลุมได้เพียง 3 คลาส โมเดลจะไม่สามารถระบุประเภทอื่น ๆ ได้ ทำให้ต้องออกแบบ UX ให้รองรับกรณี “นอกคลาส” (out-of-distribution) เช่น แจ้งเตือน/เสนอให้เลือกประเภทด้วยตนเอง
		\end{enumerate}
	\end{enumerate}
\end{sloppypar}


% ---------- หัวข้อใหญ่ (ชิดซ้าย, หนา 16pt) ----------
\section*{ข้อเสนอแนะ}
\addcontentsline{toc}{section}{ข้อเสนอแนะ}

\begin{sloppypar}
	\begin{enumerate}
		\item รองรับการรับชำระเงินค่าสมัครเข้าร่วมประกวด (Payments)
		\begin{enumerate}
			\item แนวทางระบบ: แยก \emph{Payment Service} ออกเป็นโมดูล/ไมโครเซอร์วิส รับคำสั่งชำระเงินและรอ \emph{webhook} ยืนยันสถานะจากผู้ให้บริการชำระเงิน (PG) จากนั้นอัปเดตฐานข้อมูล (เช่น payments, orders) และเปลี่ยนสถานะการสมัครใน registrations เป็น paid.
			\item การเชื่อมต่อ: ใช้ Hosted Checkout/PaymentIntent ของผู้ให้บริการ (ลดภาระ PCI-DSS) พร้อมรองรับวิธีชำระเงินที่แพร่หลาย เช่น บัตร, โอน/QR, e-Wallet. ตั้งค่า \emph{webhook} ที่ปลอดภัย (ตรวจลายเซ็น/secret, IP allowlist).
			\item สคีมาฐานข้อมูล (ตัวอย่าง): \\
			payments(id, user\_id, registration\_id, amount, currency, provider, status, charge\_id, created\_at) \\
			payment\_logs(payment\_id, event\_type, payload, created\_at)
			\item ประสบการณ์ผู้ใช้: สร้างหน้าสรุปค่าธรรมเนียม, ปุ่ม ``ชำระเงิน'' และหน้า \emph{Return URL} ที่แสดงผล \emph{pending/success/failure} แบบชัดเจน พร้อมส่งหลักฐาน/ใบเสร็จ (อีเมล/ไฟล์ PDF)
			\item ความปลอดภัยและกฎระเบียบ: ไม่สัมผัสข้อมูลบัตรโดยตรง (ใช้ PG หน้าเว็บ), เปิด HTTPS บังคับ, เก็บเฉพาะ \emph{token/charge\_id}, ทำ \emph{idempotency} ในฝั่งเซิร์ฟเวอร์กันยิงซ้ำ, ล็อกธุรกรรมทุกครั้ง
		\end{enumerate}
		
		\item เปรียบเทียบปลากัดที่ชนะการประกวดกับปลากัดของผู้ใช้ (Similarity/Ranking)
		\begin{enumerate}
			\item แนวทางโมเดล: ใช้ \emph{feature extractor} เดียวกับโมเดลหลัก (ResNet50 ที่ Fine-tune แล้ว) ดึง \emph{embedding} จากชั้นก่อน FC แล้วทำ \emph{similarity search} (Cosine/Euclidean) กับฐานข้อมูลภาพผู้ชนะ
			\item เร่งความเร็วค้นหา: จัดทำดัชนีด้วย FAISS/Annoy/HNSW ให้ค้นหาใกล้เคียงแบบประมาณ (ANN) ได้เร็วเมื่อจำนวนภาพผู้ชนะเพิ่มขึ้น
			\item เมตริกและคำอธิบายผล: แสดง Top-$k$ ความใกล้เคียง พร้อมคะแนน/ระยะห่าง และ \emph{saliency/Grad-CAM} เพื่ออธิบายว่าบริเวณใดมีผลต่อความใกล้เคียง (เพิ่มความโปร่งใส)
			\item ส่วนติดต่อผู้ใช้: หน้าเปรียบเทียบแบบ \emph{side-by-side} (ภาพผู้ใช้ vs. ภาพผู้ชนะ) + ชิปคุณลักษณะเด่น (สีครีบ, ลายหาง, รูปทรงครีบ) และคำแนะนำการปรับปรุง (ถ้ามี)
			\item สคีมาฐานข้อมูล (ตัวอย่าง): \\
			winners(id, contest\_id, image\_url, class, embedding, meta) \\
			user\_images(id, user\_id, image\_url, embedding, created\_at)
			\item ถูกต้องและยุติธรรม: แจ้งผู้ใช้ว่าเป็นการ ``เปรียบเทียบเชิงลักษณะภาพ'' ไม่ใช่การตัดสินผลประกวด ลดความเข้าใจผิดเรื่องเกณฑ์กรรมการ
		\end{enumerate}
		
		\item เพิ่มจำนวนคลาสให้ครบประเภทปลากัดพื้นบ้านของไทย (Data Expansion \& Model Update)
		\begin{enumerate}
			\item กลยุทธ์ข้อมูล: ทำแคมเปญรับรูปอย่างเป็นระบบ (แบบฟอร์ม + เงื่อนไขสิทธิ์ใช้งาน), ใช้ \emph{active learning} ให้โมเดลช่วยคัดตัวอย่างที่ไม่มั่นใจเพื่อส่งให้ผู้เชี่ยวชาญช่วยติดป้ายกำกับ
			\item คุณภาพป้ายกำกับ: ใช้ผู้เชี่ยวชาญ 2 คนขึ้นไปต่อภาพ (double-blind) แล้วทำ consensus; เก็บ \emph{label provenance} และความเชื่อมั่นต่อคลาส
			\item ลดปัญหาคลาสไม่สมดุล: ใช้ \emph{class-balanced loss}, \emph{focal loss}, \emph{re-weighting}, และ \emph{augmentation} ตามลักษณะคลาส (ไม่บิดเพศ/ลักษณะจำแนก)
			\item ป้องกันข้อมูลรั่วไหล: แบ่ง \emph{train/val/test} แบบ \emph{grouped by owner/farm} ไม่ให้ภาพจากผู้ส่งเดียวกันไปอยู่หลายเซ็ต
			\item รองรับกรณีคลาสอื่น/นอกคลาส: เพิ่มคลาส ``unknown'' และตัวตรวจจับ \emph{out-of-distribution} (เช่น \emph{energy-based} หรือ \emph{Mahalanobis}) เพื่อบอกผู้ใช้อย่างปลอดภัยเมื่อโมเดลไม่มั่นใจ
			\item วงจรอัปเดตโมเดล: วาง \emph{model registry} (เวอร์ชัน, เมตริก, วันที่) และทดสอบ \emph{offline \& A/B} ก่อนสลับใช้งานจริง เพื่อลด \emph{regression}
		\end{enumerate}
		
		\item ระบบการแจ้งเตือนภายนอกสำหรับผู้เชี่ยวชาญ/ผู้จัดการประกวด (Notifications)
		\begin{enumerate}
			\item โครงสร้างเหตุการณ์: นิยามอีเวนต์หลัก (เช่น ``มีภาพรอประเมิน'', ``สมัครประกวดใหม่'', ``ต้องอนุมัติ/ปฏิเสธ'', ``ชำระเงินสำเร็จ''). ใช้ \emph{event bus} กลาง (เช่น database trigger $\rightarrow$ queue) แล้วค่อยกระจายไปช่องทางปลายทาง
			\item ช่องทางแจ้งเตือน: 
			\begin{itemize}
				\item \emph{Email}: ส่งสรุปรายวัน+เรียลไทม์เมื่อสำคัญ
				\item \emph{Web Push}: ผ่าน FCM สำหรับผู้ใช้ที่ยินยอม
				\item \emph{Chat Ops}: LINE Notify/Telegram Bot สำหรับทีมงาน (ห้องเฉพาะกิจ)
			\end{itemize}
			\item การบังคับใช้และกำกับสิทธิ์: ตั้งค่าการสมัครรับแจ้งเตือนรายบทบาท (ผู้เชี่ยวชาญ/ผู้จัด), ฟิลเตอร์ชนิดอีเวนต์, และช่วงเวลาเงียบ (Do-Not-Disturb)
			\item สคีมาฐานข้อมูล (ตัวอย่าง): \\
			notifications(id, user\_id, type, payload, status, channel, created\_at) \\
			subscriptions(user\_id, channel, event\_type, enabled, quiet\_hours)
			\item ความเชื่อถือได้: ใช้คิวงานพร้อม \emph{retry/backoff}, ทำ \emph{dead-letter queue}, และบันทึก \emph{delivery logs}. ทุกข้อความมี \emph{idempotency key} ป้องกันส่งซ้ำ
			\item อินทิเกรตกับระบบเดิม: หากใช้ Supabase/PG ให้ใช้ trigger $\rightarrow$ \emph{Edge Function} ยิงแจ้งเตือนเมื่อแถวในตาราง submissions, assignments, registrations, payments เปลี่ยนสถานะ
		\end{enumerate}
	\end{enumerate}
	
	\vspace{0.5em}
	\noindent Roadmap แนะนำ (ลำดับทำงาน): 
	\begin{enumerate}
		\item เปิดรับชำระเงิน (เพื่อลดภาระงานมือและสร้างรายได้ทันที)
		\item ระบบแจ้งเตือนภายนอก (ลดงานค้าง ช่วยการประสานงาน)
		\item เปรียบเทียบผู้ชนะ vs. ภาพผู้ใช้ (เพิ่มคุณค่า/แรงจูงใจ)
		\item ขยายคลาสให้ครบ (ลงทุนระยะกลาง-ยาว ยกระดับคุณภาพโมเดล)
	\end{enumerate}
\end{sloppypar}
