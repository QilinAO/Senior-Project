%==================== chapter3_28.tex ====================

\clearpage
\thispagestyle{empty}

\begingroup
\fontsize{16pt}{19.2pt}\selectfont
\justifying
\XeTeXlinebreakskip=0pt plus 1pt minus 0.5pt
\setlength{\parindent}{1.5cm}
\setlength{\parskip}{0pt}

% ---------- หัวบท + เขียนสารบัญบทก่อนหัวข้อย่อย ----------
\phantomsection
\addcontentsline{toc}{chapter}{บทที่ 5 บทสรุป}
\begin{center}
	{\bfseries\fontsize{18pt}{21.6pt}\selectfont บทที่ 5}
\end{center}

\vspace{\baselineskip}

% ---------- ชื่อบท ----------
\begin{center}
	{\bfseries\fontsize{18pt}{21.6pt}\selectfont บทสรุป}
\end{center}

\vspace{\baselineskip}

\indent จากการศึกษาค้นคว้า เรื่องเว็บ แอปพลิเคชัน ศูนย์ รวมการจัด ประกวดปลากัด ไทย และมี
การนำ Ai มาใช้ในการช่วยตรวจสอบประเภทปลากัด ว่าเป็นปลากัดประเภทไหน โดยมีวัตถุประสงค์
เพื่อ พัฒนาระบบเว็บ แอปพลิเคชัน ศูนย์ รวมการจัด ประกวดปลากัด ไทย โดยเราได้ ศึกษาและพัฒนา
ระบบเป็น Web Application โดยใช้ React, Visual Studio code, Supabase
การสร้างโมเดลจาก ResNet-50 เป็นการนำมาใช้ ในการสร้างต้นแบบบน Web App
Appication เนื่องจากมีการสนับสนุนร่วมกับ React,Node.js,Express โดย Web App Appication
สามารถทำการแยกประเภทปลากัด ได้ 3 ประเภทในตอนนี้สามารถทำการจำแนกได้โดยการอัปโหลดรูปภาพ 3 รูปภาพแต่เราเอารูปภาพรูปแรกเพื่อให้โมเดลดูว่าคือปลากัดประเภทไหน

\vspace{\baselineskip}

% ---------- หัวข้อใหญ่ (ชิดซ้าย, หนา 16pt) ----------
\section*{ปัญหาและอุปสรรค}
\addcontentsline{toc}{section}{ปัญหาและอุปสรรค}

\begin{sloppypar}
	\begin{enumerate}
		\item แหล่งข้อมูลในการทำ Dataset ที่เป็นรูปภาพปลากัดของแต่ละประเภท หาได้ค่อนข้างยากเพราะ
		ต้องใช้ภาพจำนวนมากต่อปลา 1 ตัว จึงต้องทำการขออณุญาตใช้ภาพจากกลุ่ม Facebook ชุมชน
		คนเลี้ยงปลากัด ที่อนุเคราะห์ให้ภาพปลากัดมาทำ Dataset
	\end{enumerate}
\end{sloppypar}

% ---------- หัวข้อใหญ่ (ชิดซ้าย, หนา 16pt) ----------
\section*{ข้อเสนอแนะ}
\addcontentsline{toc}{section}{ข้อเสนอแนะ}

\begin{sloppypar}
	\begin{enumerate}
		\item Web Application ทำให้สามารถรับชำระเงินค่าสมัครการเข้าร่วมการประกวด
		\item Web Application เปรียบเทียบปลากัดที่ชนะการประกวด กับปลากัดของตนเอง
		\item เพิ่มจำนวนคลาสให้ครบถ้วนตาม ตามประเภทปลากัดพื้นบ้านของไทย
		\item เพิ่มระบบการแจ้งเตือนภายนอก เพื่อให้ผู้เชี่ยวชาญหรือผู้จัดการประกวด ทราบได้ทันท่วงทีว่ามี
		กิจกรรมอะไรเข้ามา อย่างเช่น ผู้เชี่ยวมีปลากัดรอประเมิน ผู้จัดการประกวด มีกดอนุมัติเข้าร่วม
		การประกวด
	\end{enumerate}
\end{sloppypar}