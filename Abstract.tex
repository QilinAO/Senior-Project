%==================== abstract.tex ====================

\clearpage
\phantomsection
\addcontentsline{toc}{chapter}{บทคัดย่อ}
% \thispagestyle{empty} % ถ้าไม่เอาเลขหน้า ให้เอาคอมเมนต์ออก

\begingroup
% ทั้งหน้า 13pt (baseline ~15.6pt)
\fontsize{13pt}{15.6pt}\selectfont

% จัดเต็มบรรทัดแบบไทย + ย่อหน้า 1.5 ซม.
\justifying
\XeTeXlinebreakskip=0pt plus 1pt minus 0.5pt
\setlength{\parindent}{1.5cm}
\setlength{\parskip}{0pt}

% ---------- จัดบรรทัด "ป้ายกำกับ : ค่า" ให้ตรงคอลัมน์ ----------
% วัดความกว้างป้ายกำกับที่ยาวสุด (ใช้ตัวหนา + มีโคลอน)
\newlength{\FieldLabelWidth}
\newlength{\FieldSep} \setlength{\FieldSep}{3em} % ระยะเว้นระหว่างฉลากกับค่า

\settowidth{\FieldLabelWidth}{\bfseries ประเภทสารนิพนธ์ :}

% ป้ายกำกับบรรทัดแรก "ตัวหนา 13pt" + ค่า "13pt ธรรมดา" อยู่บรรทัดเดียวกัน
% ถ้าค่ายาวจะตัดบรรทัดภายในคอลัมน์ขวาและชิดซ้ายใต้คอลัมน์ค่าเดิมเสมอ
\newcommand{\Field}[2]{%
	\noindent
	\makebox[\FieldLabelWidth][l]{\bfseries #1 :}%
	\hspace{\FieldSep}%
	\parbox[t]{\dimexpr\linewidth-\FieldLabelWidth-\FieldSep\relax}{#2}%
	\par
}

% ---------- บล็อกข้อมูลหัวเรื่อง ----------
\Field{เรื่อง}{เว็บแอปพลิเคชันศูนย์รวมการจัดประกวดปลากัดไทย}
\Field{ผู้ศึกษาค้นคว้า}{เอกสิทธิ์ อัศวดารา}
\Field{อาจารย์ที่ปรึกษา}{ผศ.\,ดร.\,สุรางคนา ระวังยศ}
\Field{ประเภทสารนิพนธ์}{ภาคนิพนธ์ ปริญญาตรี สาขาวิชาวิทยาการคอมพิวเตอร์}
\Field{คำสำคัญ}{ปลากัดไทย, การประกวด, เว็บแอปพลิเคชัน}

% เว้น 1 บรรทัด
\vspace{\baselineskip}

% ---------- หัวข้อ "บทคัดย่อ" ----------
\phantomsection
\addcontentsline{toc}{chapter}{บทคัดย่อ}

\begin{center}
	\bfseries บทคัดย่อ
\end{center}

% เว้น 1 บรรทัด
\vspace{\baselineskip}

% ---------- เนื้อหาบทคัดย่อ ----------
การแข่งขันปลากัดไทยในปัจจุบันมีการแพร่หลายผ่านสื่อออนไลน์ เช่น กลุ่ม Facebook ต่าง ๆ
ซึ่งทำให้ข้อมูลการประกวดกระจัดกระจาย ไม่เป็นระบบ และยากต่อการอ้างอิง
งานวิจัยนี้จึงพัฒนาเว็บแอปพลิเคชันศูนย์รวมการจัดประกวดปลากัดไทย
เพื่อรวบรวมข้อมูลการประกวด การจัดเก็บผลการตัดสิน และการเผยแพร่ข่าวสารอย่างเป็นระบบ
โดยมีเป้าหมายเพื่ออำนวยความสะดวกแก่นักเพาะเลี้ยงและผู้สนใจทั่วไป
รวมถึงยกระดับคุณภาพการประกวดให้มีมาตรฐานเดียวกันผ่านการประยุกต์ใช้เทคโนโลยีสารสนเทศ
ระบบถูกออกแบบและพัฒนาด้วยเฟรมเวิร์กสมัยใหม่ รองรับการใช้งานทั้งบนคอมพิวเตอร์และอุปกรณ์พกพา
มีฟังก์ชันการสมัครแข่งขัน การบันทึกผลการประกวด การรายงานผลย้อนหลัง และการจัดเก็บข้อมูลผู้ใช้งาน
ผลการทดสอบการใช้งานจริงพบว่าระบบสามารถทำงานได้ตามที่ออกแบบ และช่วยลดความซับซ้อนในการจัดการข้อมูลการประกวดปลากัดได้อย่างมีประสิทธิภาพ

\par\endgroup
\clearpage

%================== จบ abstract.tex ==================
