%==================== chapter3_29/1.tex ====================

\clearpage
\thispagestyle{plain}

\begingroup
\fontsize{16pt}{19.2pt}\selectfont
\justifying
\XeTeXlinebreakskip=0pt plus 1pt minus 0.5pt
\setlength{\parindent}{1.5cm}
\setlength{\parskip}{0pt}



\section*{ขั้นตอนการทำงานการทดลอง}
\addcontentsline{toc}{section}{ขั้นตอนการทำงานการทดลอง}
\indent ระบบประกอบด้วย 4 ขั้นหลัก: (1) Data Preparation สแกนและทำความสะอาดรูป แปลงเป็น RGB/Resize แบ่ง Train/Validation และคัดรูปซ้ำด้วย Average Hash ก่อนบันทึกเป็นโครงสร้าง ImageFolder; (2) Data Loading ใช้ SafeImageFolder+DataLoader พร้อม Augmentation (RandomResizedCrop, Flip, ColorJitter, Affine) และ Normalization; (3) Training ใช้ ResNet50 (pretrained ImageNet) เป็นตัวเข้ารหัสคุณลักษณะ ร่วมกับหัวจำแนก (GlobalAvgPool+Linear) เทรนด้วย AdamW, Label Smoothing, Warmup+Cosine LR, Gradient Accumulation/Clipping, AMP (bf16/fp16), และรองรับ torch.compile; (4) Evaluation \& Artifacts ประเมินด้วย Accuracy และ F1-macro พร้อมรายงาน per-class/Confusion Matrix และบันทึก best.pt, last.pt, metrics.json, class\_to\_idx.json.


\begin{figure}[h]
	\centering
	\includegraphics[width=0.3\linewidth]{Image Classification Pipeline.drawio}
	\caption{Image Classification Pipeline}
\end{figure}

\newpage

\begin{sloppypar}
	\begin{enumerate} %[start=4]  % เริ่มที่ 2 (ต่อจาก 1)
		\item \textbf{รับข้อมูลเข้า:} รูปภาพถูกจัดมาจากหลายโฟลเดอร์ แทนแต่ละ “คลาส” ของป้ายกำกับ
		\item \textbf{Data Preparation:}
			\begin{enumerate}
				\item เปิดรูปอย่างปลอดภัย แปลงเป็น RGB/ย่อขนาด
				\item คัดรูปซ้ำแบบหยาบ (average hash)
				\item แบ่งเป็นชุดฝึก (train) และตรวจสอบ (val)
				\item บันทึกเป็นโครงสร้าง ImageFolder พร้อมสถิติการเตรียมข้อมูล
			\end{enumerate}
		\item \textbf{Data Loading:}
			\begin{enumerate}
			\item ใช้ SafeImageFolder + DataLoader โหลดรูปเป็นแบตช์
			\item ใส่ augmentation สำหรับ train และ normaliz ให้เข้ากับ ImageNet
			\end{enumerate}
		\item \textbf{โมเดล (Encoder + Head):}
			\begin{enumerate}
			\item ResNet-50 (pretrained) ทำหน้าที่ “เข้ารหัส” สกัดฟีเจอร์จากภาพ
			\item อัปเดตพารามิเตอร์ด้วย AdamW, ใช้ Warmup→Cosine LR
			\item รองรับ AMP (bf16/fp16), gradient accumulation/clip, และ (ถ้ามี) torch.compile
			\end{enumerate}
		\item \textbf{การฝึก (Training Loop):}
			\begin{enumerate}
			\item คำนวณ Cross-Entropy (มี label smoothing)
			\item ส่วนหัว (Global Avg Pool + Linear) แปลงฟีเจอร์เป็น logits ตามจำนวนคลาส
			\end{enumerate}
		\item \textbf{การประเมิน (Validation):}
			\begin{enumerate}
			\item วัด Accuracy, F1-macro, รายงานรายคลาส และ Confusion Matrix
			\item ใช้ F1-macro เลือกโมเดลที่ดีที่สุด และตรวจ Early Stopping
			\end{enumerate}
		\item \textbf{อาร์ติแฟกต์/ผลลัพธ์:}
					\begin{enumerate}
			\item เก็บ best.pt, last.pt, metrics.json, class\_to\_idx.json
			\item พร้อมนำไปทดสอบ/เสิร์ฟใช้งานต่อในขั้นตอนถัดไป
		\end{enumerate}
	\end{enumerate}
\end{sloppypar}

\clearpage