%==================== chapter3_29/1.tex ====================

\clearpage
\thispagestyle{plain}

\begingroup
\fontsize{16pt}{19.2pt}\selectfont
\justifying
\XeTeXlinebreakskip=0pt plus 1pt minus 0.5pt
\setlength{\parindent}{1.5cm}
\setlength{\parskip}{0pt}


\begin{figure}[h]
	\centering
	\includegraphics[width=0.2\linewidth]{ResNet-50.drawio}
	\caption{ขั้นตอนการทำงาน ResNet-50}
\end{figure}

\begin{sloppypar}
	\begin{enumerate} %[start=4]  % เริ่มที่ 2 (ต่อจาก 1)
		\item อินพุตภาพ 3×224×224 -> ผ่าน Conv 7×7, stride 2 แล้ว MaxPool 3×3, stride 2 เพื่อลดขนาดแผนที่คุณลักษณะ (feature map) อย่างรวดเร็ว
		\item เข้า สเตจของ Residual Bottleneck 4 ช่วงต่อเนื่อง (มีทางลัด residual ทุกบล็อก):
		\begin{enumerate}
			\item Stage1: ขนาดคงที่ 56×56, ช่องออก 256 (บล็อก ×3)
			\item Stage2: ลดสเกลเป็น 28×28 ที่บล็อกแรก (stride 2), ช่องออก 512 (บล็อก ×4)
			\item Stage3: ลดสเกลเป็น 14×14 ที่บล็อกแรก, ช่องออก 1024 (บล็อก ×6)
			\item Stage4: ลดสเกลเป็น 7×7 ที่บล็อกแรก, ช่องออก 2048 (บล็อก ×3)
			แต่ละบล็อกเป็นลำดับ 1×1 (ลด/ขยายช่อง) -> 3×3 (สกัดลวดลายเชิงพื้นที่) -> 1×1 (ขยายช่อง) แล้วบวกกับเส้นลัด (shortcut) เพื่อช่วยให้เทรนลึกได้
		\end{enumerate}
		\item ได้แผนที่คุณลักษณะสุดท้ายขนาด 2048×7×7 -> ทำ Global Average Pooling ให้เป็นเวกเตอร์ 2048-D
		\item ส่งเข้า ชั้น Linear (FC) แปลงเป็น logits = จำนวนคลาส
	\end{enumerate}
\end{sloppypar}