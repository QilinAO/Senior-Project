%==================== chapter4.tex ====================

\clearpage
\thispagestyle{empty}

\begingroup
\fontsize{16pt}{19.2pt}\selectfont
\justifying
\XeTeXlinebreakskip=0pt plus 1pt minus 0.5pt
\setlength{\parindent}{1.5cm}
\setlength{\parskip}{0pt}

% ---------- หัวบท + เขียนสารบัญบทก่อนหัวข้อย่อย ----------
\phantomsection
\addcontentsline{toc}{chapter}{บทที่ 4 ผลการดำเนินงาน}
\begin{center}
	{\bfseries\fontsize{18pt}{21.6pt}\selectfont บทที่ 4}
\end{center}

\vspace{\baselineskip}

% ---------- ชื่อบท ----------
\begin{center}
	{\bfseries\fontsize{18pt}{21.6pt}\selectfont ผลการดำเนินงาน}
\end{center}

\vspace{\baselineskip}

% ---------- หัวข้อใหญ่ (ชิดซ้าย, หนา 16pt) ----------
\section*{ผลการทดสอบแบบจำลองปัญญาประดิษฐ์}
\addcontentsline{toc}{section}{ผลการทดสอบแบบจำลองปัญญาประดิษฐ์}

% ---------- เนื้อหา (จัดกระจายแบบไทย, ย่อหน้าแรก 1.5 ซม.) ----------
\indent ในการพัฒนาเว็บแอปพลิเคชัน “ศูนย์รวมการจัดประกวดปลากัดไทย” ผู้วิจัยได้สร้างระบบ
จัดเก็บข้อมูลการแข่งขัน ข่าวสาร และเชื่อมต่อกับแบบจำลองสำหรับการจำแนกสายพันธุ์ปลากัด ซึ่ง
โมเดลหลักที่เลือกใช้คือ ResNet50 (pre-trained บน ImageNet) พร้อมการปรับ Fine-tune
ให้เหมาะกับข้อมูลที่รวบรวมมา โดยก่อนการฝึกโมเดลมีการเตรียมข้อมูลภาพ ดังนี้

\begin{sloppypar}
	\begin{enumerate} %[start=4]  % เริ่มที่ 2 (ต่อจาก 1)
		\item \textbf{การตรวจสอบและกรองไฟล์ ภาพ:} ลือกเฉพาะไฟล์ ที่ รองรับ ได้แก่ .jpg, .jpeg, .png, .bmp,
		.webp
		\item \textbf{การทำความสะอาด (Cleaning):} เปิดภาพอย่างปลอดภัยด้วยไลบรารี PIL, แปลงเป็น RGB,ปรับขนาดด้านยาวไม่เกินที่กำหนด และบันทึกใหม่เป็น JPEG
		\item \textbf{การกำจัดภาพซ้ำ (Deduplication):} ใช้วิธี Average Hash เพื่อตรวจจับและตัดภาพที่ซ้ำหรือเกือบซ้ำ
		\item \textbf{การแบ่ง ชุด ข้อมูล (Splitting):} แบ่ง ข้อมูล เป็น Training set และ Validation set ในอัตรา 85:15 โดยสุ่มแยกต่อคลาส
	\end{enumerate}
\end{sloppypar}

\indent หลังการเตรียมข้อมูลได้ชุดข้อมูลรวมจำนวน 356 ภาพแบ่งเป็น Train 294 ภาพและ Validation 50 ภาพ

\begin{table}[h]
	\caption{จำนวนภาพปลากัดหลังการเตรียมข้อมูล}
	{\tablefont
		\setlength{\tabcolsep}{6pt}%
		\begin{tabularx}{\linewidth}{@{}
				>{\raggedright\arraybackslash}X
				>{\centering\arraybackslash}p{2.2cm}
				>{\centering\arraybackslash}p{2.6cm}
				>{\centering\arraybackslash}p{2.2cm}
				@{}}
			\Xhline{1.5pt}
			\bfseries คลาส & \bfseries Train & \bfseries Validation & \bfseries รวม \\
			\Xhline{0.5pt}
			ปลากัดพื้นบ้านภาคอีสานหางลาย & 101 & 17 & 118 \\
			\Xhline{0.5pt}
			ปลากัดพื้นบ้านภาคใต้ & 114 & 20 & 134 \\
			\Xhline{0.5pt}
			ปลากัดพื้นบ้านมหาชัย & 79 & 13 & 92 \\
			\Xhline{0.5pt}
			รวม & 294 & 50 & 344 \\
			\Xhline{1.5pt}
	\end{tabularx}}
\end{table}

\endgroup