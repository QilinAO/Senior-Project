%==================== chapter3_24.tex ====================

\clearpage
\thispagestyle{plain}

\begingroup
\fontsize{16pt}{19.2pt}\selectfont
\justifying
\XeTeXlinebreakskip=0pt plus 1pt minus 0.5pt
\setlength{\parindent}{1.5cm}
\setlength{\parskip}{0pt}

% ---------- หัวข้อใหญ่ (ชิดซ้าย, หนา 16pt) ----------
\section*{Sequence Diagra}
\addcontentsline{toc}{section}{Sequence Diagra}

% ---------- เนื้อหา (จัดกระจายแบบไทย, ย่อหน้าแรก 1.5 ซม.) ----------
\indent Sequence Diagram เป็นแผนผังแสดงการทำงานแบบลำดับปฏิสัมพันธ์ โดยเว็บแอปพลิเคชันศูนย์รวมการจัดประกวดปลากัดไทย มีองค์ประกอบ Sequence Diagram ดังนี้

% ===== enumerate แบบ robust (ไม่ใช้ \dimexpr เพื่อกัน error) =====
\setlist[enumerate,1]{%
	label=\arabic*., align=left,
	leftmargin=1.5cm, labelindent=0pt,
	labelwidth=\LoneLabelWidth, labelsep=\LoneLabelSep,
	itemsep=0pt, topsep=0.5\baselineskip
}

% ระดับ 2: ให้คอลัมน์ข้อความตรงตามที่ตั้ง (ชิฟต์ด้วย \ExtraAlign)
\setlist[enumerate,2]{%
	label*=\arabic*., align=left,
	leftmargin=*,
	labelwidth=\LtwoLabelWidth, labelsep=\LtwoLabelSep,
	labelindent=\dimexpr \LoneContentCol + \ExtraAlign - \LtwoLabelWidth - \LtwoLabelSep\relax,
	itemsep=0pt, topsep=0pt
}
\setlist[enumerate,3]{%
	label*=\arabic*., align=left,
	leftmargin=*, labelsep=-0.6em,
	labelindent=-1em, % ให้คอลัมน์ข้อความตรงเส้นเดียวกับระดับ 2
	widest=9.9.9,
	itemsep=0pt, topsep=0pt
}

\begin{enumerate}
	\item การเข้าสู่ระบบ
	\item การสมัครสมาชิก
	\item การออกจากระบบ
	\item การกู้คืนเซสชันเมื่อเข้าแอป
	\item การอัปเดตโปรไฟล์ผู้ใช้
	\item การอัปโหลดรูปโปรไฟล์
	\item ผู้ใช้ส่งปลากัดเพื่อประเมินคุณภาพ
	\item ผู้ใช้ส่งปลากัดเข้าร่วมการแข่งขัน
	\item ผู้เชี่ยวชาญดึงคิวงานประเมินคุณภาพ
	\item ผู้เชี่ยวชาญตอบรับ/ปฏิเสธงานประเมิน
	\item ผู้เชี่ยวชาญส่งคะแนนประเมินคุณภาพ
	\item ดึงเกณฑ์การให้คะแนนตามประเภทปลา
	\item ผู้เชี่ยวชาญดูรายการประกวดที่ต้องตัดสิน (มีฟอลแบ็ก)
	\item ผู้เชี่ยวชาญตอบรับ/ปฏิเสธคำเชิญเป็นกรรมการ
	\item ดึงรายชื่อปลาที่เข้าประกวดในกิจกรรม
	\item ผู้เชี่ยวชาญส่งคะแนนการแข่งขัน
	\item ผู้จัดการสร้างกิจกรรม/ข่าว
	\item ผู้จัดการแก้ไขกิจกรรม
	\item ผู้จัดการมอบหมาย/ถอดถอนกรรมการและส่งแจ้งเตือน
	\item ผู้จัดการปิดการแข่งขันและประกาศผล
	\item ผู้ใช้เปิดศูนย์แจ้งเตือนและทำเครื่องหมายอ่านทั้งหมด
	\item วิเคราะห์รูปภาพด้วย AI สำหรับการแข่งขัน (เดี่ยว/หลายรูป)
	\item ผู้ใช้ดูผลการแข่งขันแบบสาธารณะ
	\item ประวัติการทำงานของผู้เชี่ยวชาญ (สลับแท็บ คุณภาพ/การแข่งขัน)
	\item ประวัติการส่งของผู้ใช้ (คุณภาพ/การแข่งขัน)
\end{enumerate}

\vspace{\baselineskip}

\begin{figure}[h]
	\centering
	\includegraphics[width=0.8\linewidth]{SQ1}
	\caption{การเข้าสู่ระบบ}
\end{figure}

\indent ไดอะแกรมนี้แสดงลำดับเหตุการณ์เมื่อผู้ใช้พยายามเข้าสู่ระบบ หน้า Login รับอินพุตจากผู้ใช้และตรวจสอบความครบถ้วน จากนั้นเรียก AuthContext.signin ซึ่งภายในจะเด้งไปเรียก authService.loginUser เพื่อยิง POST /auth/signin ผ่าน apiService ไปยัง Backend เมื่อสำเร็จเซิร์ฟเวอร์ส่ง token และ profile กลับมา ฝั่งฟรอนต์จะบันทึกลง localStorage แล้วอัปเดตสเตต user ใน AuthContext จากนั้นหน้า Login แสดง Toast สำเร็จและเปลี่ยนเส้นทางตาม role ของผู้ใช้ หากล้มเหลว หน้า Login แสดงข้อความผิดพลาดโดยไม่เปลี่ยนเส้นทาง

\newpage

\begin{figure}[h]
	\centering
	\includegraphics[width=0.8\linewidth]{SQ2}
	\caption{การสมัครสมาชิก}
\end{figure}

\indent แผนภาพนี้อธิบายการสร้างบัญชีผู้ใช้ใหม่ หน้า SignUp ตรวจความครบถ้วนของข้อมูลก่อนเรียก authService.signupUser ซึ่งยิง POST /auth/signup ผ่าน apiService ไปยัง Backend เมื่อสำเร็จจะตอบ 201 Created กลับมา หน้า SignUp แสดง Toast สำเร็จและนำผู้ใช้ไปหน้า Login ในกรณีผิดพลาด หน้า SignUp แสดงสาเหตุข้อผิดพลาดเพื่อให้แก้ไขข้อมูลและลองใหม่


\newpage

\vspace{\baselineskip}

\begin{figure}[h]
	\centering
	\includegraphics[width=0.8\linewidth]{SQ3}
	\caption{การออกจากระบบ}
\end{figure}

\indent ไดอะแกรมนี้อธิบายขั้นตอนการออกจากระบบของผู้ใช้ ผู้ใช้กดปุ่ม “ออกจากระบบ” ใน Navbar จากนั้นเรียก AuthContext.signout ซึ่งภายในใช้ authService.signoutUser ยิง POST /auth/signout ไป Backend แม้ตอบล้มเหลวก็เคลียร์โทเคนและโปรไฟล์ใน localStorage เสมอ แล้วตั้ง user=null ใน AuthContext จากนั้น Navbar นำผู้ใช้กลับหน้า Login หรือหน้าแรก

\newpage

\vspace{\baselineskip}

\begin{figure}[h]
	\centering
	\includegraphics[width=0.8\linewidth]{SQ4}
	\caption{การกู้คืนเซสชันเมื่อเข้าแอป}
\end{figure}

\indent การกู้คืนเซสชันอัตโนมัติเมื่อแอปเริ่มทำงาน เมื่อ AuthProvider เมาท์ครั้งแรก จะตรวจว่ามีโทเคนใน localStorage ไหม ถ้ามีจะพยายามอ่าน profile จากแคชก่อน ถ้าไม่มีแคชจะยิง GET /auth/profile ไป Backend หากสำเร็จจะอัปเดต user และแคชโปรไฟล์ไว้ หากล้มเหลว (เช่น 401) จะเคลียร์แคชและถือว่าไม่ได้ล็อกอิน

\vspace{\baselineskip}

\begin{figure}[h]
	\centering
	\includegraphics[width=0.8\linewidth]{SQ5}
	\caption{การอัปเดตโปรไฟล์ผู้ใช้}
\end{figure}

\indent อธิบายการอัปโหลดรูปโปรไฟล์ใหม่ หน้าโปรไฟล์ตรวจความถูกต้องของฟิลด์ จากนั้นเรียก userService.updateProfile เพื่อยิง PUT /users/profile ไป Backend เมื่อสำเร็จจะได้ profile ที่อัปเดตกลับมาและแสดง Toast สำเร็จ ถ้าเกิดข้อผิดพลาดจะแสดงข้อความเตือนและไม่เปลี่ยนสเตตภายในหน้า



\newpage

\begin{figure}[h]
	\centering
	\includegraphics[width=0.8\linewidth]{SQ6}
	\caption{การอัปโหลดรูปโปรไฟล์}
\end{figure}

\indent อธิบายการอัปโหลดรูปโปรไฟล์ใหม่ ผู้ใช้เลือกไฟล์รูป จากนั้นหน้าโปรไฟล์เรียก userService.uploadProfilePicture(file) ซึ่งส่ง FormData ไป POST /users/profile/picture เมื่อสำเร็จ Backend ส่ง URL รูปใหม่กลับมา หน้าโปรไฟล์อัปเดตรูปทันทีและแจ้ง Toast สำเร็จ ถ้าอัปโหลดล้มเหลวจะแจ้งข้อความผิดพลาด

\vspace{\baselineskip}

\begin{figure}[h]
	\centering
	\includegraphics[width=0.8\linewidth]{SQ7}
	\caption{ผู้ใช้ส่งปลากัดเพื่อประเมินคุณภาพ}
\end{figure}

\indent อธิบายการส่งปลากัดเพื่อ “ประเมินคุณภาพ” โดยมีการช่วยตรวจจาก AI ก่อนส่ง หน้า BettaEvaluationForm ให้ผู้ใช้อัปโหลดรูป 3 รูปและเลือกประเภทปลา ก่อนส่งจะเรียก modelService.analyzeBatchImages เพื่อเช็คความสอดคล้องของรูปและคาดเดาประเภท ถ้ารูปไม่ตรงกันระบบจะแจ้งและคัดทิ้งเฉพาะรูปที่ไม่ผ่าน เมื่อข้อมูลครบและผ่าน AI แล้วจึงส่ง POST /submissions ไป Backend หากสำเร็จแจ้ง Toast และพาไปหน้า /history ถ้าล้มเหลวแจ้งข้อผิดพลาดและคงแบบฟอร์มให้แก้ไข

\newpage

\begin{figure}[h]
	\centering
	\includegraphics[width=0.8\linewidth]{SQ8}
	\caption{ผู้ใช้ส่งปลากัดเข้าร่วมการแข่งขัน}
\end{figure}

\indent อธิบายการส่งปลากัดเข้าร่วม “การแข่งขัน” เมื่อเปิด SubmissionFormModal ผู้ใช้กรอกข้อมูลและแนบรูป พร้อม contest\_id คอมโพเนนต์ตรวจว่ามีไอดีกิจกรรมและรูปตามจำนวนที่กำหนด จากนั้นเรียก userService.submitBettaForCompetition ส่ง FormData ไป POST /submissions/compete หากสำเร็จ Modal ปิดและแสดง Toast สำเร็จ กรณีไม่ผ่านจะแสดงข้อความและให้แก้ไขแล้วลองใหม่

\vspace{\baselineskip}

\begin{figure}[h]
	\centering
	\includegraphics[width=0.8\linewidth]{SQ9}
	\caption{ผู้เชี่ยวชาญดึงคิวงานประเมินคุณภาพ}
\end{figure}

\indent แสดงการโหลด “คิวงานประเมินคุณภาพ” ของผู้เชี่ยวชาญ หน้า EvaluationQueue เรียก expertService.getEvaluationQueue ยิง GET /experts/queue เพื่อดึงสองลิสต์คือ “รอการตอบรับ” และ “ที่ต้องให้คะแนน” แล้วแสดงผลแบบแท็บ ผู้ใช้กดสลับแท็บและกดรับงานหรือไปหน้ากรอกคะแนนต่อไป

\newpage

\begin{figure}[h]
	\centering
	\includegraphics[width=0.8\linewidth]{SQ10}
	\caption{ผู้เชี่ยวชาญตอบรับ/ปฏิเสธงานประเมิน}
\end{figure}

\indent อธิบายการตอบรับหรือปฏิเสธงานในคิว จากหน้า EvaluationQueue เมื่อกด “ตอบรับ” หรือ “ปฏิเสธ” จะเรียก expertService.respondToEvaluation ยิง POST /experts/assignments/:id/respond ส่งสถานะและเหตุผล (ถ้ามี) กลับไปยังเซิร์ฟเวอร์ เมื่อสำเร็จหน้ารีโหลดคิวและแจ้งผลด้วย Toast ขณะที่ข้อผิดพลาดจะแสดงข้อความและไม่เปลี่ยนสถานะใน UI

\vspace{\baselineskip}

\begin{figure}[h]
	\centering
	\includegraphics[width=0.8\linewidth]{SQ11}
	\caption{ผู้เชี่ยวชาญส่งคะแนนประเมินคุณภาพ}
\end{figure}

\indent ผังนี้เป็นการส่งคะแนน “ประเมินคุณภาพ” เมื่อเปิด ScoringFormModal และกรอกคะแนนครบแล้ว ผู้เชี่ยวชาญกดยืนยัน ระบบเรียก expertService.submitQualityScores ยิง POST /experts/assignments/:id/score แนบคะแนนย่อยและคะแนนรวม ถ้าบันทึกสำเร็จ โมดอลปิด แสดง Toast สำเร็จ และรายการงานในคิวจะถูกย้ายไปสถานะ “ประเมินเสร็จแล้ว”

\newpage

\begin{figure}[h]
	\centering
	\includegraphics[width=0.8\linewidth]{SQ12}
	\caption{ดึงเกณฑ์การให้คะแนนตามประเภทปลา}
\end{figure}

\indent ผังนี้เป็นการดึง “เกณฑ์การให้คะแนน” แบบไดนามิกตามประเภทปลา ขณะเปิดฟอร์มให้คะแนน ระบบเรียก expertService.getScoringSchema(bettaType,{contestId}) เพื่อยิง GET /experts/scoring-schema พร้อมพารามิเตอร์ betta\_type และ contest\_id (ถ้ามี) เซิร์ฟเวอร์ส่ง schema กลับมาให้เรนเดอร์ช่องคะแนนตามประเภทปลาที่เลือก ลดโอกาสกรอกผิดและรองรับกิจกรรมที่เกณฑ์ต่างกัน

\vspace{\baselineskip}

\begin{figure}[h]
	\centering
	\includegraphics[width=0.8\linewidth]{SQ13}
	\caption{ผู้เชี่ยวชาญดูรายการประกวดที่ต้องตัดสิน (มีฟอลแบ็ก)}
\end{figure}

\indent แผนภาพนี้แสดงการดึง “รายการประกวดที่ต้องตัดสิน” พร้อมเส้นทางสำรอง expertService.getJudgingContests จะพยายามเรียก GET /experts/judging ก่อน หากไม่มีข้อมูลหรือเกิดข้อผิดพลาดจึง fallback ไป GET /experts/contests/judging ทั้งสองกรณีเซิร์ฟเวอร์จะตอบลิสต์ invitations และ myContests กลับมาเพื่อแสดงในแดชบอร์ด ช่วยให้ผู้เชี่ยวชาญเห็นภาพรวมงานที่ต้องตัดสิน

\newpage

\begin{figure}[h]
	\centering
	\includegraphics[width=0.8\linewidth]{SQ14}
	\caption{ผู้เชี่ยวชาญตอบรับ/ปฏิเสธคำเชิญเป็นกรรมการ}
\end{figure}

\indent แผนภาพนี้อธิบายการตอบรับ/ปฏิเสธ “คำเชิญเป็นกรรมการ” หน้ารับเชิญเรียก expertService.respondToJudgeInvitation ซึ่งจะเลือกยิง POST /experts/contests/:id/accept เมื่อรับเชิญ หรือ POST /experts/contests/:id/decline เมื่อปฏิเสธ (พร้อมเหตุผลถ้ามี) เซิร์ฟเวอร์ตอบกลับผลลัพธ์ให้ UI ปรับสถานะรายการเชิญและแสดง Toast แจ้งผล

\vspace{\baselineskip}

\begin{figure}[h]
	\centering
	\includegraphics[width=0.8\linewidth]{SQ15}
	\caption{ดึงรายชื่อปลาที่เข้าประกวดในกิจกรรม}
\end{figure}

\indent เมื่อผู้เชี่ยวชาญเปิดหน้ากิจกรรม ระบบเรียก expertService.getFishInContest(contestId) ยิง GET /experts/judging/:contestId/submissions เพื่อดึงลิสต์ผลงานทั้งหมด นำมาสร้างรายการ/การ์ดพร้อมตัวกรอง เพื่อให้เลือกไปหน้ากรอกคะแนนทีละชิ้นได้สะดวก

\newpage

\begin{figure}[h]
	\centering
	\includegraphics[width=0.8\linewidth]{SQ16}
	\caption{ผู้เชี่ยวชาญส่งคะแนนการแข่งขัน}
\end{figure}

\indent แผนภาพนี้อธิบายการส่งคะแนน “การแข่งขัน” เมื่อกรอกคะแนนครบใน CompetitionScoringForm จะเรียก expertService.submitCompetitionScore(submissionId,data) ยิง POST /experts/judging/submissions/:id/score ถ้าบันทึกสำเร็จ UI แสดง Toast และอัปเดตสถานะของผลงานนั้น ๆ ในลิสต์ของกิจกรรม

\vspace{\baselineskip}

\begin{figure}[h]
	\centering
	\includegraphics[width=0.8\linewidth]{SQ17}
	\caption{ผู้จัดการสร้างกิจกรรม/ข่าว}
\end{figure}

\indent แผนภาพนี้อธิบายการสร้างกิจกรรม/ข่าวใหม่โดยผู้จัดการ หน้า ContestManagement เมื่อกรอกฟอร์มครบและกด “สร้าง” จะเรียก managerService.createContestOrNews(FormData) ยิง POST /manager/contests ไป Backend หากสร้างสำเร็จจะส่ง contest ที่เพิ่งสร้างกลับมา UI แสดง Toast และนำไปหน้าแก้ไขหรือหน้าแสดงรายละเอียดกิจกรรม

\newpage

\begin{figure}[h]
	\centering
	\includegraphics[width=0.8\linewidth]{SQ18}
	\caption{ผู้จัดการแก้ไขกิจกรรม}
\end{figure}

\indent แผนภาพนี้อธิบายการแก้ไขกิจกรรม ใน EditContestModal เมื่อผู้จัดการกดบันทึก จะเรียก managerService.updateMyContest(contestId,data) ยิง PUT /manager/contests/:id เซิร์ฟเวอร์ตอบข้อมูลล่าสุดกลับมาเพื่อนำไปแทนที่บน UI และแสดง Toast ยืนยันการอัปเดตสำเร็จ

\vspace{\baselineskip}

\begin{figure}[h]
	\centering
	\includegraphics[width=0.8\linewidth]{SQ19}
	\caption{ผู้จัดการมอบหมาย/ถอดถอนกรรมการและส่งแจ้งเตือน}
\end{figure}

\indent แผนภาพนี้อธิบายการมอบหมาย/ถอดถอนกรรมการ และการแจ้งเตือนกรณีถอด บนหน้า AssignJudges ผู้จัดการดึงรายชื่อผู้เชี่ยวชาญด้วย GET /manager/experts?contest\_id เพื่อเลือกมอบหมาย (POST /manager/contests/:id/judges) หรือถอด (DELETE /manager/contests/:id/judges/:judgeId) เมื่อถอดสำเร็จ ระบบเรียก notifyJudgeRemoval ส่ง POST /manager/notifications หากเอ็นด์พอยต์นี้ไม่พร้อมใช้จะ fallback ไป POST /notifications ให้ผู้เชี่ยวชาญได้รับการแจ้งเตือนการถอดเสมอ

\newpage 

\begin{figure}[h]
	\centering
	\includegraphics[width=0.8\linewidth]{SQ20}
	\caption{ผู้จัดการปิดการแข่งขันและประกาศผล}
\end{figure}

\indent แผนภาพนี้อธิบายการ “ปิดประกวด/ประกาศผล” โดยผู้จัดการ ในหน้า LiveContestRoom ผู้จัดการคลิก “ประกาศผล” ระบบเรียก managerService.finalizeContest(contestId) ยิง POST /manager/contests/:id/finalize เซิร์ฟเวอร์ประมวลผลรวมคะแนนและสถานะ แล้วส่งผลลัพธ์สรุปกลับมา UI แสดงอันดับ สรุปคะแนน และสถิติที่เกี่ยวข้อง

\vspace{\baselineskip} 

\begin{figure}[h]
	\centering
	\includegraphics[width=0.8\linewidth]{SQ21}
	\caption{ผู้ใช้เปิดศูนย์แจ้งเตือนและทำเครื่องหมายอ่านทั้งหมด}
\end{figure}

\indent แผนภาพนี้ครอบคลุมการเปิดศูนย์แจ้งเตือนและทำ “อ่านแล้วทั้งหมด” ผู้ใช้กดไอคอนกระดิ่งใน Navbar ระบบเรียก notificationService.getMyNotifications ยิง GET /notifications?params เพื่อดึงรายการล่าสุดมาแสดง หากผู้ใช้คลิก “ทำทั้งหมดเป็นอ่านแล้ว” จะเรียก notificationService.markAllNotificationsRead ยิง PATCH /notifications/read-all เมื่อสำเร็จ UI อัปเดตสถานะทั้งหมดเป็นอ่านแล้วทันที

\vspace{\baselineskip} 

\begin{figure}[h]
	\centering
	\includegraphics[width=0.8\linewidth]{SQ22}
	\caption{วิเคราะห์รูปภาพด้วย AI สำหรับการแข่งขัน (เดี่ยว/หลายรูป)}
\end{figure}

\indent เมื่อผู้ใช้แนบรูปใน SubmissionFormModal ระบบเรียก modelService.analyzeForCompetition(formData, images) ถ้ามี 1 รูปจะยิง POST /model/analyze-single ถ้ามีหลายรูปจะยิง POST /model/analyze-batch เมื่อผลมาถึงจะสรุปประเภท/ความมั่นใจและแสดงคำแนะนำใน UI เพื่อช่วยตัดสินใจก่อนส่งเข้าระบบจริง

\vspace{\baselineskip} 

\begin{figure}[h]
	\centering
	\includegraphics[width=0.8\linewidth]{SQ23}
	\caption{ผู้ใช้ดูผลการแข่งขันแบบสาธารณะ}
\end{figure}

\indent แผนภาพนี้อธิบายการดูผลการแข่งขันสาธารณะ หน้าผลการแข่งขันเรียก userService.getPublicContestResults(contestId) ยิง GET /public/contests/:id/results เพื่อดึงผลลัพธ์หลังประกาศอย่างเป็นทางการ เมื่อได้ข้อมูลแล้วจะแสดงอันดับ คะแนน และข้อมูลประกอบอื่น ๆ ให้ผู้เยี่ยมชมทั่วไปดูได้โดยไม่ต้องล็อกอิน

\newpage 

\begin{figure}[h]
	\centering
	\includegraphics[width=0.8\linewidth]{SQ24}
	\caption{ประวัติการทำงานของผู้เชี่ยวชาญ (สลับแท็บ คุณภาพ/การแข่งขัน)}
\end{figure}

\indent แผนภาพนี้อธิบายการดูประวัติการทำงานของผู้เชี่ยวชาญ แยกเป็นแท็บ “คุณภาพ/การแข่งขัน” หน้า EvaluationHistory มีแท็บสองประเภท เมื่อเลือก “คุณภาพ” ระบบจะเรียก GET /experts/history/evaluations ถ้าเลือก “การแข่งขัน” จะเรียก GET /experts/history/contests จากนั้นเรนเดอร์ตารางผล โดยจัดรูปแบบวันที่ คะแนน และชื่อผลงาน/การแข่งขันอย่างเหมาะสม

\newpage 
\vspace{\baselineskip} 

\begin{figure}[h]
	\centering
	\includegraphics[width=0.8\linewidth]{SQ25}
	\caption{ประวัติการส่งของผู้ใช้ (คุณภาพ/การแข่งขัน)}
\end{figure}

\indent ผนภาพนี้อธิบายการดูประวัติการส่งของผู้ใช้ แยก “ประเมินคุณภาพ/การแข่งขัน” หน้า History ของผู้ใช้เรียก userService.getMyEvaluationHistory เพื่อดูลิสต์การส่งประเมินคุณภาพ และเรียก userService.getMyCompetitionHistory เพื่อดูลิสต์การแข่งขันที่เคยเข้าร่วม จากนั้น UI แสดงรายการพร้อมรายละเอียดคะแนนและสถานะให้ผู้ใช้ติดตามย้อนหลังได้
\endgroup