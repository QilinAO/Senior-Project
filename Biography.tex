\clearpage
\thispagestyle{empty}


% ทั้งหน้า 13pt (baseline ~15.6pt)
\fontsize{16pt}{24pt}\selectfont

% จัดเต็มบรรทัดแบบไทย + ย่อหน้า 1.5 ซม.
\justifying
\XeTeXlinebreakskip=0pt plus 1pt minus 0.5pt
\setlength{\parindent}{1.5cm}
\setlength{\parskip}{0pt}

\phantomsection
\addcontentsline{toc}{chapter}{ประวัติผู้วิจัย}
\begin{center}
	{\bfseries\fontsize{16pt}{19.2pt}\selectfont ประวัติผู้วิจัย}
\end{center}
\vspace{\baselineskip}



% ---------- จัดบรรทัด "ป้ายกำกับ : ค่า" ให้ตรงคอลัมน์ ----------

\settowidth{\FieldLabelWidth}{\bfseries ประเภทสารนิพนธ์ :}

% ป้ายกำกับบรรทัดแรก "ตัวหนา 13pt" + ค่า "13pt ธรรมดา" อยู่บรรทัดเดียวกัน
% ถ้าค่ายาวจะตัดบรรทัดภายในคอลัมน์ขวาและชิดซ้ายใต้คอลัมน์ค่าเดิมเสมอ
\newcommand{\Field}[2]{%
	\noindent
	\makebox[\FieldLabelWidth][l]{\bfseries #1 :}%
	\hspace{\FieldSep}%
	\parbox[t]{\dimexpr\linewidth-\FieldLabelWidth-\FieldSep\relax}{#2}%
	\par
}

% ---------- บล็อกข้อมูลหัวเรื่อง ----------
\Field{ชื่อ-สกุล}{เอกสิทธิ์ อัศวดารา}
\Field{วัน เดือน ปี เกิด}{17 กรกฎาคม 2546}
\Field{สถานที่เกิด}{จังหวัดพิษณุโลก}
\Field{วุฒิการศึกษา}{-}
\Field{ที่อยู่ปัจจุบัน}{66/4 หมู่ที่ 4 ตำบลชาติตระการ อพเภาชาติตระการ จังหวัดพิษณุโลก 65170}

% เว้น 1 บรรทัด
\vspace{\baselineskip}