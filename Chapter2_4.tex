%==================== chapter2_4.tex ====================

\clearpage
\thispagestyle{plain}

\begingroup
\fontsize{16pt}{19.2pt}\selectfont
\justifying
\XeTeXlinebreakskip=0pt plus 1pt minus 0.5pt
\setlength{\parindent}{1.5cm}
\setlength{\parskip}{0pt}

% ---------- การพัฒนาเว็บแอปพลิเคชัน ----------
\noindent{\bfseries\fontsize{16pt}{19.2pt}\selectfont การพัฒนาเว็บแอปพลิเคชัน}\par

\ThaiPara{กระบวนการหลัก ได้แก่ การกำหนดแนวคิดและเป้าหมาย การวิเคราะห์ความต้องการ
	การออกแบบ UX/UI การทำต้นแบบ การพัฒนา (Front-end/Back-end) การทดสอบ
	(การใช้งาน ความปลอดภัย ประสิทธิภาพ) และการเผยแพร่ พร้อมดูแลปรับปรุงตามผลตอบรับของผู้ใช้}

\vspace{\baselineskip}

% ---------- งานวิจัยที่เกี่ยวข้อง ----------
\noindent{\bfseries\fontsize{16pt}{19.2pt}\selectfont งานวิจัยที่เกี่ยวข้อง}\par

\indent {ตัวอย่างประเด็นที่เกี่ยวข้อง เช่น 
	 งานวิจัยเกี่ยวกับการจำแนกภาพขวดแบบเซ็ตเปิดด้วยโครงข่ายประสาทเทียมแบบคอนโวลูชัน (Convolutional Neural Network หรือ CNN) โดย ศุภณัฐ จินตวัฒน์สกุล \cite{jintawatsakoon2019openset} มีวัตถุประสงค์หลักในการพัฒนาโมเดลที่สามารถจำแนกประเภทของขวดที่เปิดแล้วได้อย่างมีประสิทธิภาพ โดยการใช้เทคนิค Deep Learning เพื่อวิเคราะห์และประมวลผลภาพที่ถ่ายในสภาพแวดล้อมต่าง ๆ ในงานวิจัยนี้ ผู้วิจัยได้เสนอการใช้โครงข่ายประสาทเทียมแบบคอนโวลูชันซึ่งสามารถดึงคุณลักษณะสำคัญจากภาพ เช่น รูปร่าง สี และลวดลายของขวด โดยใช้ภาษาPython ร่วมกับไลบรารี TensorFlow เพื่อพัฒนาและฝึกสอนโมเดล การใช้ชุดข้อมูลที่มีภาพของขวดแบบเซ็ตเปิดที่ติดป้ายกำกับช่วยให้โมเดลสามารถเรียนรู้และปรับปรุงความแม่นยำในการจำแนกประเภทได้ ผลลัพธ์จากการทดลองแสดงให้เห็นว่าโมเดลที่พัฒนาขึ้นมีค่าความแม่นยำสูงในการจำแนกประเภทขวดเปิด ซึ่งสามารถนำไปประยุกต์ใช้ในหลายสาขา เช่น การควบคุมคุณภาพในอุตสาหกรรม หรือการจัดการขวดในระบบอัตโนมัติ ดังนั้นงานวิจัยนี้จึงมีความสำคัญในการพัฒนาเทคโนโลยี   การจำแนกภาพที่มีประสิทธิภาพและช่วยส่งเสริมการใช้งานในวงกว้าง.งานวิจัยเกี่ยวกับ "Responsive Portfolio Website Using React" \cite{muksikarat2016rwd} มีวัตถุประสงค์หลักในการพัฒนาเว็บไซต์พอร์ตโฟลิโอที่สามารถปรับตัวให้เข้ากับอุปกรณ์ต่าง ๆ ได้อย่างมีประสิทธิภาพ เช่นโทรศัพท์มือถือ แท็บเล็ต และคอมพิวเตอร์ โดยใช้เทคโนโลยี React ซึ่งเป็นไลบรารี Java Scriptที่นิยมในการสร้างส่วนประกอบที่มีความโต้ตอบสูง การเลือกใช้ React ช่วยให้การพัฒนาเว็บไซต์สามารถแบ่งแยกส่วนประกอบต่าง ๆ ได้ง่ายและทำให้โค้ดมีความเรียบร้อยและสามารถดูแลรักษาได้ง่ายขึ้น ในกระบวนการพัฒนา Siva Rama Lingham N และคณะ \cite{10425667} ได้ใช้ CSS และCSS Frameworks เช่น Bootstrap หรือ Tailwind CSS เพื่อสร้างเลย์เอาต์ที่ตอบสนองต่อการเปลี่ยนแปลงขนาดหน้าจอ โดยสามารถแสดงข้อมูลได้อย่างเหมาะสมและมีความน่าสนใจในทุกอุปกรณ์ นอกจากนี้ยังมีการใช้ JavaScript เพื่อเพิ่มการโต้ตอบระหว่างผู้ใช้กับเว็บไซต์ เช่น การจัดการสถานะของข้อมูล การตอบสนองต่อการกระทำของผู้ใช้ และการนำเสนอข้อมูลในรูปแบบที่น่าสนใจ ผลลัพธ์ที่ได้จากงานวิจัยนี้คือเว็บไซต์พอร์ตโฟลิโอที่ไม่เพียงแต่มีความสวยงามและใช้งานง่าย แต่ยังสามารถแสดงผลงานและข้อมูลของผู้ใช้ได้อย่างมีประสิทธิภาพในทุกแพลตฟอร์ม เว็บไซต์นี้ช่วยให้ผู้ใช้สามารถนำเสนอผลงานของตนได้อย่างมีสไตล์ และเป็นเครื่องมือที่มีประโยชน์ในการสร้างภาพลักษณ์ที่ดีในสายงานที่ตนสนใจ โดยรวมแล้วงานวิจัยนี้มีความสำคัญในการพัฒนาเทคโนโลยีเว็บที่ตอบสนองความต้องการของผู้ใช้ในยุคดิจิทัลที่มีการเปลี่ยนแปลงอย่างรวดเร็ว 
	 
	 \newpage 
	 
	 R.Archana and P.S. Eliahim Leevaraj \cite{8016501} งานวิจัยนี้สำรวจและประเมินโมเดลการเรียนรู้เชิง ลึก (Deep Learning) ในการประมวลผลภาพดิจิทัล เช่น การลดสัญญาณรบกวน การเพิ่มคุณภาพภาพ การ แบ่งส่วน การสกัดคุณลักษณะ และการจำแนกประเภท โดยเปรียบเทียบกับวิธีการประมวลผลภาพแบบดั้งเดิม โดยใช้ เทคนิคการประมวลผลภาพแบบดั้งเดิม เช่น การปรับความคมชัด การกรองสัญญาณรบกวน และการ แบ่งส่วนภาพ รวมถึงการใช้โมเดลการเรียนรู้เชิงลึก เช่น Convolutional Neural Networks (CNNs) และ Recurrent Neural Networks (RNNs) เพื่อเพิ่มประสิทธิภาพในการประมวลผลภาพ โดยมี 4 ขั้นตอน ได้แก่ 1.การเตรียมรูปภาพ เป็นการลดสัญญาณรบกวนและเพิ่มคุณภาพของรูป 2. การแบ่งส่วนภาพ เป็นการแยก ภาพออกเป็นส่วนๆ ตามลักษณะเฉพาะ 3.การสกัดคุณลักษณะ เป็นการดึงข้อมูลสำคัญจากภาพ 4.การจำแยก ประเภท เป็นการจัดหมวดหมู่ภาพตามเนื้อหา ผลลัพธ์ของงานวิจัยนี้แสดงให้เห็นว่าโมเดล DeepLabV3 และ U-Net มีความแม่นยำสูงกว่า FCN ในการแบ่งส่วนภาพจากมุมมองด้านบน ซึ่งมีความสำคัญในหลายๆ การใช้ งาน เช่น การเฝ้าระวัง การตรวจสอบฝูงชน และการโต้ตอบระหว่างมนุษย์กับคอมพิวเตอร์.}
	 
	

