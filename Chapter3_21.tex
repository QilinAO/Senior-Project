%==================== chapter3_21.tex ====================

\clearpage
\thispagestyle{plain}

\begingroup
\fontsize{16pt}{19.2pt}\selectfont
\justifying
\XeTeXlinebreakskip=0pt plus 1pt minus 0.5pt
\setlength{\parindent}{1.5cm}
\setlength{\parskip}{0pt}

\vspace{\baselineskip}

% ================================= Class: SupabaseAdmin ========================
\begin{table}[h]
	\caption{Class Description : SupabaseAdmin}
	{\tablefont\setlength{\tabcolsep}{6pt}%
		\begin{tabularx}{\linewidth}{@{} >{\raggedright\arraybackslash}p{3.6cm} D @{}}
			\Xhline{1.5pt}
			\textbf{Class Name :} & SupabaseAdmin \\
			\Xhline{0.5pt}
			\textbf{Description :} & Client กลางสำหรับติดต่อกับ Supabase โดยใช้ SERVICE\_ROLE\_KEY ข้ามกฎ RLS เพื่อจัดการข้อมูลระดับ Admin \\
			\Xhline{0.5pt}
			\textbf{Attribute :} & - \\
			\Xhline{0.5pt}
			\textbf{Method :} &
			\begin{tabular}{@{}l@{}}
				rpc(function, params): เรียกใช้ Database Function \\
				from(table): เลือกตารางเพื่อดำเนินการ (CRUD) \\
				storage(): เข้าถึงส่วนจัดการไฟล์ (Storage)
			\end{tabular} \\
			\Xhline{1.5pt}
	\end{tabularx}}
\end{table}

% ============================= Class: NotificationService ======================
\begin{table}[h]
	\caption{Class Description : NotificationService}
	{\tablefont\setlength{\tabcolsep}{6pt}%
		\begin{tabularx}{\linewidth}{@{} >{\raggedright\arraybackslash}p{3.6cm} D @{}}
			\Xhline{1.5pt}
			\textbf{Class Name :} & NotificationService \\
			\Xhline{0.5pt}
			\textbf{Description :} & Service กลางสำหรับสร้างและจัดการการแจ้งเตือน (Notifications) ไปยังผู้ใช้ในเหตุการณ์ต่าง ๆ \\
			\Xhline{0.5pt}
			\textbf{Attribute :} & - \\
			\Xhline{0.5pt}
			\textbf{Method :} &
			\begin{tabular}{@{}l@{}}
				createNotification(userId, message): สร้างการแจ้งเตือนใหม่สำหรับผู้ใช้ที่ระบุ
			\end{tabular} \\
			\Xhline{1.5pt}
	\end{tabularx}}
\end{table}

% =================================== Class: ApiService =========================
\begin{table}[h]
	\caption{Class Description : ApiService}
	{\tablefont\setlength{\tabcolsep}{6pt}%
		\begin{tabularx}{\linewidth}{@{} >{\raggedright\arraybackslash}p{3.6cm} D @{}}
			\Xhline{1.5pt}
			\textbf{Class Name :} & ApiService \\
			\Xhline{0.5pt}
			\textbf{Description :} & Service กลางของ Frontend ทำหน้าที่เป็น Wrapper ของ \texttt{fetch} API เพื่อสื่อสารกับ Backend ทั้งหมด จัดการ Token, Headers และ Error เบื้องต้น \\
			\Xhline{0.5pt}
			\textbf{Attribute :} & - \\
			\Xhline{0.5pt}
			\textbf{Method :} &
			\begin{tabular}{@{}l@{}}
			get(endpoint): ส่ง GET request \\
			post(endpoint, body): ส่ง POST request \\
			put(endpoint, body): ส่ง PUT request \\
			delete(endpoint): ส่ง DELETE request
			\end{tabular} \\
			\Xhline{1.5pt}
	\end{tabularx}}
\end{table}