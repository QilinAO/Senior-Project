%==================== Appendix.tex ====================

\clearpage
\thispagestyle{plain}

\begingroup
% เนื้อหาภาคผนวก: 16pt baseline ~19.2pt ตามสเปกเล่ม
\fontsize{16pt}{19.2pt}\selectfont
\justifying
\XeTeXlinebreakskip=0pt plus 1pt minus 0.5pt
\setlength{\parindent}{1.5cm}
\setlength{\parskip}{0pt}

% ---------- หัวข้อใหญ่ (ชิดซ้าย, หนา 16pt) ----------
\noindent{\bfseries\fontsize{16pt}{19.2pt}\selectfont ภาคผนวก ก}\par
\noindent{\bfseries\fontsize{16pt}{19.2pt}\selectfont รายละเอียดการทดสอบระบบ}\par

% ---------- เนื้อหา (จัดกระจายแบบไทย, ย่อหน้าแรก 1.5 ซม.) ----------
\indent ภาคผนวกนี้แสดงรายละเอียดการทดสอบระบบเว็บแอปพลิเคชันศูนย์รวมการจัดประกวดปลากัดไทย
รวมถึงผลการทดสอบในแต่ละด้านและข้อมูลเพิ่มเติมที่เกี่ยวข้อง

\noindent{\bfseries\fontsize{16pt}{19.2pt}\selectfont ก.1 การทดสอบการทำงานของระบบ}\par

\indent การทดสอบการทำงานของระบบครอบคลุมการทดสอบฟังก์ชันต่าง ๆ ของระบบ
รวมถึงการสมัครแข่งขัน การอัปโหลดรูปภาพปลากัด การประเมินด้วย AI
และการบันทึกผลการประกวด

\noindent{\bfseries\fontsize{16pt}{19.2pt}\selectfont ก.2 การทดสอบประสิทธิภาพ}\par

\indent การทดสอบประสิทธิภาพของระบบใช้เครื่องมือ Load Testing
เพื่อจำลองการใช้งานจริงและวัดประสิทธิภาพของระบบในสภาวะต่าง ๆ

\noindent{\bfseries\fontsize{16pt}{19.2pt}\selectfont ภาคผนวก ข}\par
\noindent{\bfseries\fontsize{16pt}{19.2pt}\selectfont รหัสต้นฉบับโปรแกรม}\par

\indent ภาคผนวกนี้แสดงรหัสต้นฉบับของโปรแกรมที่สำคัญบางส่วน
เพื่อให้ผู้อ่านสามารถเข้าใจการทำงานของระบบได้ดียิ่งขึ้น

\par\endgroup
\clearpage

%================== จบ Appendix.tex ====================