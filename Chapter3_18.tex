%==================== chapter3_18.tex ====================

\clearpage
\thispagestyle{plain}

\begingroup
\fontsize{16pt}{19.2pt}\selectfont
\justifying
\XeTeXlinebreakskip=0pt plus 1pt minus 0.5pt
\setlength{\parindent}{1.5cm}
\setlength{\parskip}{0pt}

\vspace{\baselineskip}

% ============================ Class: Profile ============================
\begin{table}[h]
	\caption{Class Description : Profile}
	{\tablefont\setlength{\tabcolsep}{6pt}%
		\begin{tabularx}{\linewidth}{@{} >{\raggedright\arraybackslash}p{3.6cm} X @{}}
			\Xhline{1.5pt}
			\textbf{Class Name :} & Profile \\  % ปิดแถวด้วย \\
			\Xhline{0.5pt}
			\textbf{Description :} & โมเดลผู้ใช้หลักของระบบ ใช้แทนผู้ใช้งานทุกบทบาท และเก็บข้อมูลโปรไฟล์รวมถึงความเชี่ยวชาญ \\
			\Xhline{0.5pt}
			\textbf{Attribute :} &
			\begin{tabular}{@{}l@{}}
				id: UUID — รหัสผู้ใช้ (PK) \\
				username: string — ชื่อผู้ใช้ \\
				first\_name: string — ชื่อ \\
				last\_name: string — นามสกุล \\
				email: string — อีเมล \\
				avatar\_url: string — รูปโปรไฟล์ \\
				created\_at: datetime — เวลาเริ่มสร้าง \\
				updated\_at: datetime — เวลาอัปเดต \\
				role: string — บทบาท \\
				specialities: JSON — ความเชี่ยวชาญของผู้เชี่ยวชาญ
			\end{tabular} \\
			\Xhline{0.5pt}
			\textbf{Method :} &
			\begin{tabular}{@{}l@{}}
				fetchProfile(): ดึงโปรไฟล์ผู้ใช้ปัจจุบัน \\
				updateProfile(profileData): อัปเดตข้อมูลโปรไฟล์ \\
				uploadProfilePicture(file): อัปโหลดรูปโปรไฟล์
			\end{tabular} \\
			\Xhline{1.5pt}
	\end{tabularx}}
\end{table}

\newpage

% ============================ Class: Contest ============================
\begin{table}[h]
	\caption{Class Description : Contest}
	{\tablefont\setlength{\tabcolsep}{6pt}%
		\begin{tabularx}{\linewidth}{@{} >{\raggedright\arraybackslash}p{3.6cm} X @{}}
			\Xhline{1.5pt}
			\textbf{Class Name :} & Contest \\ 
			\Xhline{0.5pt}
			\textbf{Description :} & กิจกรรมการประกวด/ข่าวสารที่จัดโดยผู้จัดการการแข่งขัน ใช้ควบคุมสถานะและช่วงเวลา \\
			\Xhline{0.5pt}
			\textbf{Attribute :} &
			\begin{tabular}{@{}l@{}}
				id: UUID — รหัสการประกวด (PK) \\
				name: string — ชื่อการประกวด \\
				short\_description: string — คำอธิบายสั้น \\
				full\_description: string — คำอธิบายเต็ม \\
				poster\_url: string — โปสเตอร์ \\
				category: string — หมวดหมู่ \\
				start\_date: datetime — วันเริ่ม \\
				end\_date: datetime — วันจบ \\
				status: string — สถานะ \\
				is\_vote\_open: boolean — เปิดโหวตหรือไม่ \\
				created\_by: UUID — ผู้สร้าง (Profile.id)
			\end{tabular} \\
			\Xhline{0.5pt}
			\textbf{Method :} &
			\begin{tabular}{@{}l@{}}
				createContestOrNews(formData): สร้างกิจกรรม/ข่าว \\
				getMyContests(): ดึงรายการกิจกรรมของผู้จัดการ \\
				getContestDetail(contestId, isContest): ดึงรายละเอียดกิจกรรม \\
				updateMyContest(contestId, data): อัปเดตกิจกรรม \\
				deleteMyContest(contestId): ลบกิจกรรม \\
				updateContestStatus(contestId, status): เปลี่ยนสถานะ \\
				finalizeContest(contestId): ปิดและประกาศผล \\
				getContestSubmissions(contestId): ดึงรายการผู้สมัคร \\
				getScoringProgress(contestId): ความคืบหน้าการให้คะแนน \\
				getAllResults(): ดึงสรุปผลคะแนนทั้งหมด
			\end{tabular} \\
			\Xhline{1.5pt}
	\end{tabularx}}
\end{table}

\newpage

% ============================ Class: Submission ============================
\begin{table}[h]
	\caption{Class Description : Submission}
	{\tablefont\setlength{\tabcolsep}{6pt}%
		\begin{tabularx}{\linewidth}{@{} >{\raggedright\arraybackslash}p{3.6cm} X @{}}
			\Xhline{1.5pt}
			\textbf{Class Name :} & Submission \\
			\Xhline{0.5pt}
			\textbf{Description :} & การส่งผลงานปลากัดของผู้ใช้ ทั้งเพื่อการประเมินคุณภาพและเพื่อเข้าร่วมประกวด \\
			\Xhline{0.5pt}
			\textbf{Attribute :} &
			\begin{tabular}{@{}l@{}}
				id: UUID — รหัสการส่ง (PK) \\
				owner\_id: UUID — เจ้าของ (Profile.id) \\
				contest\_id: UUID — การประกวดที่เข้าร่วม (nullable เมื่อประเมินคุณภาพ) \\
				fish\_name: string — ชื่อปลา \\
				fish\_type: string — ประเภท/สายพันธุ์ \\
				fish\_age\_months: int — อายุ (เดือน) \\
				fish\_image\_urls: string[] — ลิงก์รูป \\
				fish\_video\_url: string — ลิงก์วิดีโอ \\
				status: string — สถานะ \\
				submitted\_at: datetime — เวลาส่ง \\
				final\_score: decimal — คะแนนสุดท้าย (ถ้ามี)
			\end{tabular} \\
			\Xhline{0.5pt}
			\textbf{Method :} &
			\begin{tabular}{@{}l@{}}
				submitBettaForEvaluation(formData): ส่งประเมินคุณภาพ \\
				submitBettaForCompetition(formData): ส่งเข้าประกวด \\
				getScoresForSubmission(submissionId): ดึงคะแนนของผลงาน
			\end{tabular} \\
			\Xhline{1.5pt}
	\end{tabularx}}
\end{table}

\clearpage