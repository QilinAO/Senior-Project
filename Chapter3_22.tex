%==================== chapter3_22.tex ====================

\clearpage
\thispagestyle{plain}

\begingroup
\fontsize{16pt}{19.2pt}\selectfont
\justifying
\XeTeXlinebreakskip=0pt plus 1pt minus 0.5pt
\setlength{\parindent}{1.5cm}
\setlength{\parskip}{0pt}

\vspace{\baselineskip}

% =============================== Class: [Role]ServiceFE ========================
\begin{table}[h]
	\caption{Class Description : [Role]ServiceFE (เช่น ManagerServiceFE, ExpertServiceFE)}
	{\tablefont\setlength{\tabcolsep}{6pt}%
		\begin{tabularx}{\linewidth}{@{} >{\raggedright\arraybackslash}p{3.6cm} D @{}}
			\Xhline{1.5pt}
			\textbf{Class Name :} & [Role]ServiceFE \\
			\Xhline{0.5pt}
			\textbf{Description :} & กลุ่ม Service ฝั่ง Frontend แยก Logic การเรียกใช้ API ตามบทบาทผู้ใช้ เพื่อให้ Component ไม่เรียก ApiService โดยตรง \\
			\Xhline{0.5pt}
			\textbf{Attribute :} & - \\
			\Xhline{0.5pt}
			\textbf{Method :} &
			\begin{tabular}{@{}l@{}}
				getMyContests(): เรียก ApiService.get ไปยัง endpoint ของ Manager \\
				getEvaluationQueue(): เรียก ApiService.get ไปยัง endpoint ของ Expert
			\end{tabular} \\
			\Xhline{1.5pt}
	\end{tabularx}}
\end{table}

% ============================ Class: [Page/Component] ==========================
\begin{table}[h]
	\caption{Class Description : [Page/Component] (เช่น LiveContestRoom, EvaluationQueue)}
	{\tablefont\setlength{\tabcolsep}{6pt}%
		\begin{tabularx}{\linewidth}{@{} >{\raggedright\arraybackslash}p{3.6cm} D @{}}
			\Xhline{1.5pt}
			\textbf{Class Name :} & [Page/Component] \\
			\Xhline{0.5pt}
			\textbf{Description :} & Component แสดงผล UI สำหรับบทบาทต่าง ๆ โดยเรียกใช้ Service (FE) ที่เกี่ยวข้องเพื่อนำข้อมูลมาแสดงและส่งกลับ Backend \\
			\Xhline{0.5pt}
			\textbf{Attribute :} &
			\begin{tabular}{@{}l@{}}
				selectedContest / queue: state สำหรับเก็บข้อมูลที่ดึงมา
			\end{tabular} \\
			\Xhline{0.5pt}
			\textbf{Method :} &
			\begin{tabular}{@{}l@{}}
				handle...(): ฟังก์ชันจัดการเหตุการณ์จากผู้ใช้ (เช่น คลิก) และเรียกใช้ Service
			\end{tabular} \\
			\Xhline{1.5pt}
	\end{tabularx}}
\end{table}

% ============================= Class: SubmissionFormModal ======================
\begin{table}[h]
	\caption{Class Description : SubmissionFormModal}
	{\tablefont\setlength{\tabcolsep}{6pt}%
		\begin{tabularx}{\linewidth}{@{} >{\raggedright\arraybackslash}p{3.6cm} D @{}}
			\Xhline{1.5pt}
			\textbf{Class Name :} & SubmissionFormModal \\
			\Xhline{0.5pt}
			\textbf{Description :} & Modal สำหรับรับข้อมูลการส่งปลากัดจากผู้ใช้ จัดการ state ของรูปภาพ วิดีโอ และข้อมูลในฟอร์ม \\
			\Xhline{0.5pt}
			\textbf{Attribute :} &
			\begin{tabular}{@{}l@{}}
				images: state สำหรับเก็บไฟล์รูปภาพ \\
				video: state สำหรับเก็บไฟล์วิดีโอ
			\end{tabular} \\
			\Xhline{0.5pt}
			\textbf{Method :} &
			\begin{tabular}{@{}l@{}}
				onSubmit(): เรียกใช้เมื่อผู้ใช้ยืนยันการส่งข้อมูล โดยจะเรียก UserServiceFE \\และ ModelServiceFE
			\end{tabular} \\
			\Xhline{1.5pt}
	\end{tabularx}}
\end{table}
