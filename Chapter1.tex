%==================== chapter2.tex ====================

\clearpage
\thispagestyle{empty}

\begingroup
% เนื้อหาบท: 16pt baseline ~19.2pt ตามสเปกเล่ม
\fontsize{16pt}{19.2pt}\selectfont
\justifying
\XeTeXlinebreakskip=0pt plus 1pt minus 0.5pt
\setlength{\parindent}{1.5cm}
\setlength{\parskip}{0pt}

% ---------- หัวบท + เขียนสารบัญบทก่อนหัวข้อย่อย ----------
\phantomsection
\addcontentsline{toc}{chapter}{บทที่ 1 บทนำ}
\begin{center}
{\bfseries\fontsize{18pt}{21.6pt}\selectfont บทที่ 1}
\end{center}

\vspace{\baselineskip}

% ---------- ชื่อบท (บทนำ) ----------
\begin{center}
{\bfseries\fontsize{18pt}{21.6pt}\selectfont บทนำ}
\end{center}

\vspace{\baselineskip}

% ---------- หัวข้อใหญ่ (ชิดซ้าย, หนา 16pt) ----------
\section*{ความเป็นมาและความสำคัญ}
\addcontentsline{toc}{section}{ความเป็นมาและความสำคัญ}

% ---------- เนื้อหา (จัดกระจายแบบไทย, ย่อหน้าแรก 1.5 ซม.) ----------
\indent ปลากัดไทยเป็นสัตว์น้ำที่มีความงดงามและเอกลักษณ์เฉพาะตัว ซึ่งเป็นที่นิยมในการเลี้ยงดู
ทั้งในด้านความสวยงามและการแข่งขัน \cite{ngthai2021betta} ปลากัดมีลักษณะทางกายภาพที่โดดเด่น เช่น สีสันสดใส
ครีบ ที่ ยาวสง่า งาม และลวดลายที่ เป็น เอกลักษณ์ ทำให้ ปลากัด เป็น สัตว์ เลี้ยงที่ ถูกใจผู้ ชื่น ชอบสัตว์
น้ำสวยงาม นอกจากความสวยงามแล้ว การแข่งขันปลากัดยังเป็นอีกหนึ่งกิจกรรมที่ได้รับความนิยม
อย่างมาก โดยเฉพาะการแข่งขัน ด้านความสวยงามที่ เน้น การประเมิน ลักษณะภายนอกของปลากัด
เช่น สีสันที่คมชัด ความสมบูณ์ของครีบ และการเคลื่อนไหวที่สง่างาม การประกวดปลากัดไม่ได้จำกัด
เฉพาะในระดับท้องถิ่นเท่านั้น แต่ยังมีการจัดงานระดับสากล ซึ่งดึงดูดนักเพาะพันธุ์และผู้เลี้ยงปลากัด
จากทั่วโลก ปลากัดไทยได้กลายเป็นปลาสวยงามที่มีปริมาณการส่งออกสูงเป็นอันดับ 1 ของประเทศ
โดยมีการผลิตและส่งออกปลากัดไปกว่า 80 ประเทศทั่วโลก อาทิ สหรัฐอเมริกา ฝรั่งเศส สิงคโปร์ จีน
และอิหร่าน ซึ่งมีปริมาณการส่งออกเฉลี่ยมากกว่า 20 ล้านตัวต่อปี สร้างรายได้ให้ประเทศปีละกว่า \cite{nstda2020betta}
200 ล้านบาท จุด เด่น ของปลากัด ไทยที่ ได้ รับ ความนิยมจากทั่ว โลก คือ ความหลากหลายของครีบ
หาง เช่น ครีบ สั้น ครีบ ยาว หางแบบพระจันทร์ ครึ่ง ดวง (Halfmoon) หางมงกุฎ (Crowtail) 2 หาง
(Doubletail) หรือครีบหูใหญ่ เช่น หูช้าง (Bigears/Dumbo) รวมไปถึงสีสันที่สวยงามฉูดฉาดสะดุด
ตา ปลากัดไทยยังเป็นที่นิยมเพราะสามารถเลี้ยงในพื้นที่เล็กและดูแลได้ง่าย ทำให้เหมาะสำหรับผู้
ที่มีเวลาน้อยปัจจุบัน เกษตรกรไทยได้มีการพัฒนาสายพันธุ์ให้แปลกใหม่และสวยงาม โดยเฉพาะสีสัน
ที่สามารถเลือกเพาะปลาให้มีสีตามที่ต้องการได้ เช่น ปลากัดสีธงชาติ ที่มีสีขาว น้ำเงิน และแดงอยู่
ในตัวเดียวกัน และยังสามารถจัดเรียงสีให้คล้ายธงชาติได้ อย่างไรก็ตาม ผู้เลี้ยงปลากัดยังขาดระบบที่
ช่วยให้สามารถเข้าถึงบริการประเมินคุณภาพปลากัดได้สะดวกและรวดเร็ว รวมถึงการให้คำปรึกษาด้านการเพาะพันธุ์และการดูแลปลากัด ระบบที่มีอยู่ยังขาดความยืดหยุ่นในการรองรับผู้เลี้ยงที่ต้องการ
ทราบผลการประเมินเบื้องต้น ก่อนส่ง ปลากัดเข้าประกวดจริง ซึ่งทำให้ เกิดความล่าช้าในการเตรียม
ปลากัดเข้าสู่การประกวด จากปัญหานี้ การพัฒนาเว็บแอปพลิเคชันศูนย์รวมการจัดประกวดปลากัด
ไทย จึงเป็นทางเลือกที่สำคัญในการช่วยให้ผู้เลี้ยงปลากัดสามารถเข้าถึงบริการต่าง ๆ เช่น การประเมิน
คุณภาพปลากัดโดย AI หรือผู้เชี่ยวชาญ การรับคำปรึกษาด้านการเพาะพันธุ์ และการติดตามผลการ
ประเมินได้อย่างสะดวกและรวดเร็ว เว็บแอปพลิเคชันนี้จะเป็นเครื่องมือที่ช่วยให้ผู้เลี้ยงมีความพร้อม
มากขึ้นในการพัฒนาปลากัดและเพิ่มโอกาสในการแข่งขัน

\par\endgroup
\clearpage

%================== จบ chapter1.tex ====================