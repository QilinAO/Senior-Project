%==================== chapter2_1.tex ====================

\clearpage
\thispagestyle{plain}

\begingroup
\fontsize{16pt}{19.2pt}\selectfont
\justifying
\XeTeXlinebreakskip=0pt plus 1pt minus 0.5pt
\setlength{\parindent}{1.5cm}
\setlength{\parskip}{0pt}

% ---------- วัตถุประสงค์ของงานวิจัย ----------
\noindent{\bfseries\fontsize{16pt}{19.2pt}\selectfont การประกวดปลากัดป่าในประเทศไทย}\par
\indent เกณฑ์การตัดสินการประกวดปลากัดป่าในประเทศไทย คณะกรรมการจะตัดสิน ตามความ
สวยงามและความสมบูรณ์ของปลากัดป่าตามเกณฑ์มาตรฐานตรงตามสายพันธุ์และประเภทของการ
ประกวด ดังรายละเอียดต่อไปนี้

\vspace{\baselineskip} % เว้น 1 บรรทัด

\section*{ความรู้เกี่ยวกับปลากัด}
\addcontentsline{toc}{section}{ความรู้เกี่ยวกับปลากัด}

\indent การประกวดปลากัด สวยงามในประเทศไทยได้ รับ ความนิยมอย่างแพร่ หลายในหลาย
ประเทศทั่วโลก เกษตรกรไทยมีความรู้และความชำนาญในการพัฒนาสายพันธุ์ปลากัดจนเป็นที่ยอมรับ
ของนักเลี้ยงปลาทั่วโลก จากมูลค่าการส่งออกปลาสวยงามในแต่ละปีพบว่าปลากัดเป็นปลาสวยงามที่
มีมูลค่าการส่งออกสูงเป็นอันดับหนึ่งของประเทศไทย การผสมคัดเลือกพันธุ์ช่วยพัฒนาปลากัดทั้ง “รูป
แบบครีบหาง” (เช่น เดลต้า/Delta, มงกุฎ/Crowntail, ฮาล์ฟมูน/Halfmoon, สองหาง/Doubletail)
และ “สีสัน–ลวดลาย” (สีเดียว ผสม หลายสี เช่น Dragon, Butterfly, Multicolor ฯลฯ)


% --- ตาราง: ลักษณะเฉพาะของปลากัดป่าภาคใต้และเกณฑ์การให้คะแนนในการประกวด (บังคับ 13pt) ---
\begingroup
\renewcommand{\arraystretch}{1.15}
\setlength{\arrayrulewidth}{0.5pt}

% ถ้ามี hook \AtBeginEnvironment{tabularx}{\tablefont} อยู่ ให้กำหนด \tablefont เฉพาะที่นี่เป็น 13pt
\makeatletter
\@ifundefined{tablefont}{\newcommand{\tablefont}{}}{}%
\makeatother
\renewcommand{\tablefont}{\fontsize{13pt}{15.6pt}\selectfont} % << ฟอนต์คงที่ 13pt

\begin{table}[h]
	\captionsetup{justification=raggedright, singlelinecheck=false,
		labelfont=bf, textfont=bf}
	\caption{ลักษณะเฉพาะของปลากัดป่าภาคใต้และเกณฑ์การให้คะแนนในการประกวด}
	\centering
	
	{\tablefont % ย้ำอีกชั้นให้แน่ใจว่าเนื้อหาในตารางใช้ 13pt
		\begin{tabularx}{\textwidth}{@{}>{\raggedright\arraybackslash}p{2.8cm}
				>{\raggedright\arraybackslash}X
				>{\centering\arraybackslash}p{1.6cm}@{}}
			\Xhline{1.5pt}
			\bfseries ลักษณะ & \bfseries ลักษณะเด่นตามมาตรฐาน & \bfseries คะแนน \\
			\hline
			ส่วนหัวและลำตัว &
			สันหัวเป็นเขม่าดำ ที่แก้มมีขีดสีเขียวถึงฟ้าสองขีด บางตัวอาจมีสีเขียวเคลือบเต็มแก้ม;
			ลำตัวทรงกระบอก สัดส่วนสมดุล & 10 \\
			\hline
			ลำตัวและเกล็ด &
			เกล็ดขนาดเล็กถึงปานกลาง เรียงแนบลำตัวสม่ำเสมอ ปลายเกล็ดมีประกายเขียว–ฟ้าตามธรรมชาติ & 10 \\
			\hline
			ครีบหลัง (Dorsal Fin) &
			โคนกว้าง แผ่ได้เต็ม แนวครีบเรียงเป็นระเบียบ ปลายครีบแหลมเล็กน้อย ขอบครีบเรียบไม่บิดงอ & 15 \\
			\hline
			ครีบก้น (Anal Fin) &
			ครีบยาวต่อเนื่อง ขอบครีบสม่ำเสมอ ไม่ฉีกขาดหรือบิดงอ รูปทรงกลมกลืนกับลำตัว & 15 \\
			\hline
			ครีบเอว/ครีบท้อง (Pelvic Fin) &
			เป็นคู่สมมาตร เรียวยาว ปลายครีบมีแต้มสีขาว/ฟ้าได้ กางได้ดี & 5 \\
			\hline
			ครีบหาง (Caudal Fin) &
			ทรงกลมหรือมน แผ่ได้เต็ม ขอบหางเรียบสมมาตร ไม่บิดงอ & 10 \\
			\hline
			การทรงตัวและการว่ายน้ำ &
			ทรงตัวดี ว่ายเป็นจังหวะต่อเนื่อง ไม่เอียงหรืองอผิดปกติ & 10 \\
			\hline
			การพองสู้ &
			กางครีบเต็ม แผ่นปิดเหงือก (เหงือก) แผ่ได้ดี ตอบสนองต่อสิ่งเร้าชัดเจน & 10 \\
			\hline
			ภาพรวม &
			ความสมบูรณ์ของร่างกาย ความกลมกลืนของสัดส่วนและสีสัน ความแข็งแรงโดยรวม & 15 \\
			\Xhline{0.5pt}
			\bfseries คะแนนรวมทั้งสิ้น & & \bfseries 100 \\
			\Xhline{1.5pt}
		\end{tabularx}
	}% end \tablefont
	\caption*{ที่มา: อรุณี รอดลอย, 2018, 128 69}
\end{table}
\endgroup
