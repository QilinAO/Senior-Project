%==================== chapter2.tex ====================

\clearpage
\thispagestyle{empty}

\begingroup
% เนื้อหาบท: 16pt baseline ~19.2pt ตามสเปกเล่ม
\fontsize{16pt}{19.2pt}\selectfont
\justifying
\XeTeXlinebreakskip=0pt plus 1pt minus 0.5pt
\setlength{\parindent}{1.5cm}
\setlength{\parskip}{0pt}

% ---------- หัวบท + เขียนสารบัญบทก่อนหัวข้อย่อย ----------
\phantomsection
\addcontentsline{toc}{chapter}{บทที่ 2 เอกสารที่เกี่ยวข้อง}
\begin{center}
	{\bfseries\fontsize{18pt}{21.6pt}\selectfont บทที่ 2}
\end{center}

\vspace{\baselineskip}

% ---------- ชื่อบท ----------
\begin{center}
	{\bfseries\fontsize{18pt}{21.6pt}\selectfont เอกสารและงานวิจัยที่เกี่ยวข้อง}
\end{center}

\vspace{\baselineskip}

% ---------- หัวข้อใหญ่ (ชิดซ้าย, หนา 16pt) ----------
\section*{เกณฑ์การตัดสินการประกวดปลากัดครีบสั้นในประเทศไทย}
\addcontentsline{toc}{section}{เกณฑ์การตัดสินการประกวดปลากัดครีบสั้นในประเทศไทย}

% ---------- เนื้อหา (จัดกระจายแบบไทย, ย่อหน้าแรก 1.5 ซม.) ----------
\indent เกณฑ์การตัดสินการประกวดปลากัดครีบสั้นในประเทศไทย คณะกรรมการจะตัดสินความ
สวยงามและความสมบูรณ์ของปลากัดครีบสั้นตามเกณฑ์มาตรฐานตรงตามสายพันธุ์และประเภทของ
การประกวด ดังรายละเอียดต่อไปนี้

\vspace{\baselineskip}

% --- ตาราง: เกณฑ์การให้คะแนนปลากัดครีบสั้น ---
\begingroup
\renewcommand{\arraystretch}{1.2}
\setlength{\arrayrulewidth}{0.5pt} % เส้นคั่นแถว 0.5pt

\begin{table}[h]
	
	\caption{เกณฑ์การให้คะแนนปลากัดครีบสั้น}
	\captionsetup[]{}
	\centering
	\begin{tabularx}{\textwidth}{@{}>{\raggedright\arraybackslash}X
			>{\centering\arraybackslash}p{2.5cm}
			>{\centering\arraybackslash}p{3cm}@{}}
		\Xhline{1.5pt} % เส้นบน 1.5pt
		\bfseries ลักษณะ & \bfseries คะแนน & \bfseries คะแนนรวม \\
		\hline
		หัวและตา & 5 & 5 \\
		\hline
		ลำตัวและเกล็ด & 5 & 5 \\
		\hline
		ครีบหลัง (กระโดง) & 10 & 10 \\
		\hline
		ครีบหาง (หาง) & 15 & 15 \\
		\hline
		ครีบก้น (ชายน้ำ) & 10 & 10 \\
		\hline
		ครีบอื่น ๆ เช่น ครีบหู, ครีบอก (ตะเกียบ), แผ่นปิดเหงือก (เหงือก) & 5 & 5 \\
		\hline
		สี และลวดลาย & 20 & 20 \\
		\hline
		การทรงตัว และการว่ายน้ำ & 5 & 5 \\
		\hline
		การพองสู้ & 5 & 5 \\
		\hline
		ภาพรวม & 20 & 20 \\
		\Xhline{0.5pt}
		\bfseries คะแนนรวมทั้งสิ้น & \bfseries 100 & \bfseries 100 \\
		\Xhline{1.5pt} % เส้นล่าง 1.5pt
	\end{tabularx}
	\caption*{ที่มา: อรุณี รอดลอย, 2018, 128 55}
\end{table}
\endgroup

\newpage

\vspace{\baselineskip}

% --- ตาราง: เกณฑ์การให้คะแนนปลากัดครีบยาว (2 คอลัมน์) ---
\begingroup
\renewcommand{\arraystretch}{1.2}
\setlength{\arrayrulewidth}{0.5pt} % เส้นคั่นแถว 0.5pt

\begin{table}[h]
	\captionsetup{justification=raggedright, singlelinecheck=false,
		labelfont=bf, textfont=bf} % ชื่อตารางชิดซ้าย + ตัวหนา
	\caption{เกณฑ์การให้คะแนนปลากัดครีบยาว}
	\centering
	\begin{tabularx}{\textwidth}{@{}>{\raggedright\arraybackslash}X
			>{\centering\arraybackslash}p{3cm}@{}}
		\Xhline{1.5pt} % เส้นบน 1.5pt
		\bfseries ลักษณะ & \bfseries คะแนน \\
		\hline
		หัวและตา & 5 \\
		\hline
		ลำตัวและเกล็ด & 5 \\
		\hline
		ครีบหลัง (กระโดง) & 10 \\
		\hline
		ครีบหาง (หาง) & 15 \\
		\hline
		ครีบก้น (ชายน้ำ) & 10 \\
		\hline
		ครีบอื่น ๆ เช่น ครีบหู, ครีบอก (ตะเกียบ), แผ่นปิดเหงือก (เหงือก) & 5 \\
		\hline
		สี และลวดลาย & 20 \\
		\hline
		การทรงตัว และการว่ายน้ำ & 5 \\
		\hline
		การพองสู้ & 5 \\
		\hline
		ภาพรวม & 20 \\
		\Xhline{0.5pt}
		\bfseries คะแนนรวมทั้งสิ้น & \bfseries 100 \\
		\Xhline{1.5pt} % เส้นล่าง 1.5pt
	\end{tabularx}
	\caption*{ที่มา: อรุณี รอดลอย, 2018, 128 56}
\end{table}
\endgroup


\clearpage