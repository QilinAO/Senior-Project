%==================== chapter3.tex ====================

\clearpage
\thispagestyle{empty}

\begingroup
% เนื้อหาบท: 16pt baseline ~19.2pt ตามสเปกเล่ม
\fontsize{16pt}{19.2pt}\selectfont
\justifying
\XeTeXlinebreakskip=0pt plus 1pt minus 0.5pt
\setlength{\parindent}{1.5cm}
\setlength{\parskip}{0pt}

% ---------- หัวบท + เขียนสารบัญบทก่อนหัวข้อย่อย ----------
\phantomsection
\addcontentsline{toc}{chapter}{บทที่ 3 วิธีดำเนินงานวิจัย}
\begin{center}
	{\bfseries\fontsize{18pt}{21.6pt}\selectfont บทที่ 3}
\end{center}

\vspace{\baselineskip}

% ---------- ชื่อบท (วิธีดำเนินงานวิจัย) ----------
\begin{center}
	{\bfseries\fontsize{18pt}{21.6pt}\selectfont วิธีดำเนินงานวิจัย}
\end{center}

\vspace{\baselineskip}

% ---------- หัวข้อใหญ่ (ชิดซ้าย, หนา 16pt) ----------
\section*{การวิเคราะห์และการออกแบบระบบ}
\addcontentsline{toc}{section}{การวิเคราะห์และการออกแบบระบบ}

% ---------- เนื้อหา (จัดกระจายแบบไทย, ย่อหน้าแรก 1.5 ซม.) ----------
\indent ในการพัฒนาระบบเว็บแอปพลิเคชันศูนย์รวมการจัดประกวดปลากัดไทยนั้นจำเป็นต้องมีการ
ออกแบบระบบเพื่อชี้ให้เห็นโครงสร้างและหลักการทำงานของโครงงานนี้โดยผู้จัดทำได้ศึกษาค้นคว้า
และวิเคราะห์ความต้องการขิงผู้ใช้งานระบบจากนั้นได้นำรายละเอียดที่ได้จากการศึกษาและวิเคราะห์
นำมาออกแบบระบบซึ่งสามารถแบ่งออกเป็น

% ตั้งค่าให้เหมือนสเปกเดิมทุกบท
\setlength{\LoneLabelSep}{0.5em}
\settowidth{\LoneLabelWidth}{9.}
\setlength{\LoneContentCol}{\dimexpr 1.5cm + \LoneLabelWidth + \LoneLabelSep\relax}

\setlength{\LtwoLabelSep}{0.5em}
\settowidth{\LtwoLabelWidth}{9.9.}
\setlength{\ExtraAlign}{-2.8em}

% ระดับ 1: อินเดนต์ 1.5 ซม.
\setlist[enumerate,1]{%
	label=\arabic*., align=left,
	leftmargin=1.5cm, labelindent=0pt,
	labelwidth=\LoneLabelWidth, labelsep=\LoneLabelSep,
	itemsep=0pt, topsep=0.5\baselineskip
}

\begin{enumerate}
	\item Use Case Diagram
	\item Entity-Relation Diagram
	\item Class Diagram
	\item Sequence Diagram
	\item Activity Diagram
	\item Interface
\end{enumerate}

\clearpage