

% --- ตาราง: ลักษณะเฉพาะของปลากัดป่าภาคใต้และเกณฑ์การให้คะแนนในการประกวด (บังคับ 13pt) ---
\begingroup
\renewcommand{\arraystretch}{1.15}
\setlength{\arrayrulewidth}{0.5pt}

% ถ้ามี hook \AtBeginEnvironment{tabularx}{\tablefont} อยู่ ให้กำหนด \tablefont เฉพาะที่นี่เป็น 13pt
\makeatletter
\@ifundefined{tablefont}{\newcommand{\tablefont}{}}{}%
\makeatother
\renewcommand{\tablefont}{\fontsize{13pt}{15.6pt}\selectfont} % << ฟอนต์คงที่ 13pt

\begin{table}[h]
	\captionsetup{justification=raggedright, singlelinecheck=false,
		labelfont=bf, textfont=bf}
	\caption{ลักษณะเฉพาะของปลากัดป่าภาคอีสานและเกณฑ์การให้คะแนนในการประกวด}
	\centering
	
	{\tablefont % ย้ำอีกชั้นให้แน่ใจว่าเนื้อหาในตารางใช้ 13pt
		\begin{tabularx}{\textwidth}{@{}>{\raggedright\arraybackslash}p{2.8cm}
				>{\raggedright\arraybackslash}X
				>{\centering\arraybackslash}p{1.6cm}@{}}
			\Xhline{1.5pt}
			\bfseries ลักษณะ & \bfseries ลักษณะเด่นตามมาตรฐาน & \bfseries คะแนน \\
			\hline
			ส่วนหัวและลำตัว &
			สันหัวเป็นเขม่าดำ สีของเกล็ดมีสีเขียววาวขึ้นบริเวณแผ่นปิดเหงือก
			(แก้ม) ลำตัวทรงกระบอกเรียวยาวกว่าชนิดอื่น (ในกลุ่มก่อหวอด) & 10 \\
			\hline
			แก้มและเกล็ด &
			เกล็ดเรียงแน่นเป็นระเบียบตลอดทั้งลำตัว สีของเกล็ดมีโทนสีเขียว
			ถือว่าเป็นลักษณะเด่น & 15 \\
			\hline
			ครีบหลังหรือกระโดง
			(Dorsal Fin) &
			มีก้านครีบเดี่ยว ก้านครีบมีโทนสีน้ำตาลถึงดำ เยื่อครีบมีพื้นสีเขียว
			หรือฟ้า มีลายสีดำ & 10 \\
			\hline
			ครีบก้นหรือชายน้ำ
			(Anal Fin) &
			ก้านครีบเดี่ยว พื้นเนื้อและก้านครีบมีโทนสีแดง มีแถบสีเขียวหรือ
			ฟ้าแซมระหว่างก้านครีบจากโคนถึงปลายของแต่ละเส้น & 10 \\
			\hline
			ครีบท้องหรือตะเกียบ
			(Pelvic fin) &
			เป็นครีบคู่ต้องยาวเรียว มีโทนสีแดง ก้านครีบเส้นแรกมีสีดำ หรือสี
			น้ำตาลแดง หรือสีขาวสะท้อนสีฟ้า ปลายตะเกียบต้องมีสีขาว และ
			ไม่แตก & 10 \\
			\hline
			ครีบหาง
			(Caudal Fin) &
			ก้านครีบแตกสองเท่านั้น พื้นเนื้อและก้านครีบเป็นสีแดง มีแถบสี
			เขียวหรือฟ้าแซมระหว่างก้านครีบลากจากโคนก้านไปจนสุดปลาย
			ก้านหางของแต่ละเส้น และในระหว่างปลายก้านครีบหางที่แตก
			สองจะมีสีเขียวถึงฟ้าแซมทุกช่อง & 15 \\
			\hline
			การพองสู้และการว่ายน้ำ &
			การว่ายน้ำสง่างาม ปราดเปรียว ปลาควรพองสู้ & 10 \\
			\hline
			ภาพรวม &
			ความสมบูรณ์ของร่างกาย ความกลมกลืนของสัดส่วนและสีสัน ความแข็งแรงโดยรวม & 20 \\
			\Xhline{0.5pt}
			\bfseries คะแนนรวมทั้งสิ้น & & \bfseries 100 \\
			\Xhline{1.5pt}
		\end{tabularx}
	}% end \tablefont
	%\caption*{ที่มา: อรุณี รอดลอย, 2018, 128 69}
\end{table}
\endgroup