%==================== acknowledgement.tex ====================

\clearpage
% \phantomsection
%\thispagestyle{empty}

\begingroup
\fontsize{16pt}{19.2pt}\selectfont

% จัดกระจายเต็มบรรทัดแบบไทย
\justifying
\XeTeXlinebreakskip=0pt plus 1pt minus 0.5pt

% ย่อหน้า 1.5 ซม. ชัดเจน (เผื่อ main ตั้งค่าอย่างอื่นไว้)
\setlength{\parindent}{1.5cm}
\setlength{\parskip}{0pt}

% เว้น 1 บรรทัดจากขอบบน
\vspace*{-\topskip}\vspace*{\baselineskip}

% หัวเรื่อง
\phantomsection
\addcontentsline{toc}{chapter}{กิตติกรรมประกาศ}

\begin{center}
	\textbf{กิตติกรรมประกาศ}
\end{center}

\vspace{\baselineskip}

% --- ย่อหน้า 1 ---
การวิจัยและการทำโครงงานเรื่องเว็บแอปพลิเคชันศูนย์รวมการจัดประกวดปลากัดไทย
สำเร็จลงได้ด้วยความกรุณาอย่างสูงจากอาจารย์ซึ่งเป็นอาจารย์ที่ปรึกษา
ผศ.\,ดร.\,สุรางคนา ระวังยศ และอาจารย์ผู้ให้ความรู้เกี่ยวกับปลากัด การคัดแยกปลากัด
การแบ่งประเภทปลากัด ซึ่งเป็นอาจารย์ที่ให้ความรู้ในด้านปลากัดโดยตรง
โดย ผศ.\,ดร.\,เกรียงไกร สีตะพันธุ์ ได้ให้แนวคิดที่ดีและแก้ไขข้อบกพร่องต่าง ๆ
รวมถึงการเก็บ \textit{Dataset} ขอขอบคุณกลุ่มชุมชนบน Facebook
ที่ให้ความอนุเคราะห์รูปภาพปลากัดพร้อมระบุประเภทปลากัดสำหรับงานวิจัยครั้งนี้
ตลอดระยะเวลาในการวิจัย และคณะกรรมการทุกท่านที่ให้คำแนะนำและให้คำปรึกษาด้วยความเอาใจใส่เป็นอย่างยิ่ง
จนการศึกษาค้นคว้าด้วยตนเองเสร็จสมบูรณ์ คณะผู้ศึกษาค้นคว้าขอกราบขอบพระคุณอย่างสูงไว้ ณ ที่นี้

% --- ย่อหน้า 2 ---
\par ขอกราบขอบพระคุณคณะกรรมการสอบโครงงาน ได้แก่ อาจารย์วรกฤต แสนโภชน์
และอาจารย์ธนวัฒน์ แซ่เอียบ รวมถึงอาจารย์ประจำสาขาวิทยาการคอมพิวเตอร์
คณะเทคโนโลยีสารสนเทศและการสื่อสาร มหาวิทยาลัยพะเยา
ที่ให้คำแนะนำอันมีค่าและความช่วยเหลือตลอดกระบวนการทำวิจัย
จนทำโครงงานนี้สำเร็จลุล่วงไปได้ด้วยดี ขอบคุณเพื่อน ๆ ร่วมสาขาที่คอยช่วยเหลือ
คอยให้กำลังใจ และให้คำแนะนำที่มีประโยชน์ต่อผู้วิจัย

% --- ย่อหน้า 3 ---
\par สุดท้ายนี้ คณะผู้จัดทำงานวิจัยและโครงงานนี้ขอขอบพระคุณทุกท่านที่มีส่วนร่วมในการให้คำแนะนำ
และสนับสนุนโครงงานนี้ด้วยความซาบซึ้งอย่างยิ่ง

% เว้น 1 บรรทัด แล้วชิดขอบขวาใส่ชื่อ
\vspace{\baselineskip}
\begin{flushright}
	เอกสิทธิ์ อัศวดารา
\end{flushright}

\par\endgroup
\clearpage

%================== จบ acknowledgement.tex ==================
