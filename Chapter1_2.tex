%==================== chapter1_2.tex ====================

\clearpage
\thispagestyle{plain}

\begingroup
% ฟอนต์เนื้อหา 16pt baseline ~19.2pt
\fontsize{16pt}{19.2pt}\selectfont
\justifying
\XeTeXlinebreakskip=0pt plus 1pt minus 0.5pt
\setlength{\parindent}{1.5cm}
\setlength{\parskip}{0pt}

% ---------- โปรแกรมที่ใช้ในการพัฒนา ----------
\section*{โปรแกรมที่ใช้ในการพัฒนา}
\addcontentsline{toc}{section}{โปรแกรมที่ใช้ในการพัฒนา}

% ตั้งค่าให้เหมือนสเปกเดิมทุกบท
\setlength{\LoneLabelSep}{0.5em}
\settowidth{\LoneLabelWidth}{9.}
\setlength{\LoneContentCol}{\dimexpr 1.5cm + \LoneLabelWidth + \LoneLabelSep\relax}

\setlength{\LtwoLabelSep}{0.5em}
\settowidth{\LtwoLabelWidth}{9.9.}
\setlength{\ExtraAlign}{-2.8em}

% ระดับ 1: อินเดนต์ 1.5 ซม.
\setlist[enumerate,1]{%
	label=\arabic*., align=left,
	leftmargin=1.5cm, labelindent=0pt,
	labelwidth=\LoneLabelWidth, labelsep=\LoneLabelSep,
	itemsep=0pt, topsep=0.5\baselineskip
}

% (เผื่อมีระดับ 2 ในหน้านี้ภายหลัง) จัดคอลัมน์เหมือนบทก่อน
\setlist[enumerate,2]{%
	label*=\arabic*., align=left,
	leftmargin=*,
	labelwidth=\LtwoLabelWidth, labelsep=\LtwoLabelSep,
	labelindent=\dimexpr \LoneContentCol + \ExtraAlign - \LtwoLabelWidth - \LtwoLabelSep\relax,
	itemsep=0pt, topsep=0pt
}

\begin{enumerate}
	\item Visual Studio Code
	\item React
	\item Supabase
	\item Axios
	\item Express
	\item React Testing Library
	\item Figma
\end{enumerate}

\par\endgroup
\clearpage

%================== จบ chapter1_2.tex ====================
