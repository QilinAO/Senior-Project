%==================== Approval.tex ====================

\clearpage
% \thispagestyle{empty}

\begingroup
\fontsize{16pt}{19.2pt}\selectfont
\sloppy % ผ่อนกฎจัดบรรทัดเฉพาะหน้าแบบฟอร์ม

% ----- ปุ่มปรับได้ -----
\newlength{\SignRule} \setlength{\SignRule}{5cm}   % ความยาวเส้นจุด (5cm/11cm แล้วแต่ต้องการ)
\newlength{\SignSep}  \setlength{\SignSep}{0.75em}  % ช่องว่างหลังเส้นจุด
\newlength{\DotUnit}  \setlength{\DotUnit}{0.15em}  % ยิ่งเล็กยิ่งจุดถี่

% วาดเส้นจุด: หน่วยจุดยิ่งเล็กยิ่งถี่
\newcommand{\dotleaders}{\leaders\hbox to \DotUnit{\hss.\hss}\hfill}

% #1 = ข้อความตำแหน่ง (ต่อจากเส้นจุด; ยาวแค่ไหนก็ตัดบรรทัดเอง)
% #2 = ชื่อในวงเล็บ (บรรทัดถัดไป ชิดซ้าย)
\newcommand{\signrow}[2]{%
	% ย่อเฉพาะบรรทัดแรก = ความยาวเส้นจุด + ช่องว่าง
	{\parshape 2
		\dimexpr\SignRule+\SignSep\relax \dimexpr\linewidth-\SignRule-\SignSep\relax
		0pt \linewidth
		% ซ้อนเส้นจุดให้เริ่มที่ "ขอบซ้ายจริง" โดยยื่นกลับไปทางซ้ายด้วย \llap
		\noindent\llap{%
			\makebox[\dimexpr\SignRule+\SignSep\relax][l]{%
				\makebox[\SignRule][l]{\dotleaders}%
			}%
		}%
		\raggedright #1\par}%
	% ชื่อในวงเล็บ บรรทัดใหม่ ชิดซ้าย
	\noindent(#2)\par\vspace{\baselineskip}%
}


% ---------- ส่วนหัว (กึ่งกลาง) ----------
\begin{center}
	ภาคนิพนธ์\\[-0.15\baselineskip]
	เรื่อง\\[1\baselineskip]
	
	เว็บแอปพลิเคชันศูนย์รวมการจัดประกวดปลากัดไทย\\[1\baselineskip]
	
	ของ เอกสิทธิ์ อัศวดารา\\[1\baselineskip]
	
	ได้รับการพิจารณาอนุมัติให้เป็นส่วนหนึ่งของการศึกษา
	หลักสูตรปริญญาวิทยาศาสตรบัณฑิต\\
	สาขาวิชาวิทยาการคอมพิวเตอร์
	ของมหาวิทยาลัยพะเยา
\end{center}

\vspace{\baselineskip}

% ---------- ลำดับผู้ลงนาม (ห้ามครอบด้วย center) ----------
\signrow{อาจารย์ที่ปรึกษา}{ผู้ช่วยศาสตราจารย์ ดร.สุรางคนา ระวังยศ}

\signrow{กรรมการ}{อาจารย์วรกฤต แสนโภชน์}

\signrow{กรรมการ}{อาจารย์ธนวัฒน์ แซ่เอียบ}

% แถวที่คุณต้องการให้ “ต่อท้ายยาว ๆ แล้วตัดบรรทัดเอง”
\signrow{ประธานหลักสูตรวิทยาศาสตรบัณฑิต \\ สาขาวิชาวิทยาการคอมพิวเตอร์
	คณะเทคโนโลยีสารสนเทศและการสื่อสาร มหาวิทยาลัยพะเยา}{อาจารย์วรกฤต แสนโภชน์}

\par\endgroup
\clearpage

%================== จบ Approval.tex ==================
