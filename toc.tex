%==================== toc.tex (ฉบับแก้ไขใหม่) ====================
\clearpage
%\thispagestyle{empty} % หากไม่ต้องการเลขหน้าในหน้าสารบัญ ให้เอา % ออก

% --- 1. ปิดการสร้างหัวข้ออัตโนมัติของ tocloft ---
% เราจะสร้างหัวข้อเอง จึงสั่งให้ tocloft ไม่ต้องแสดงผลอะไร
\renewcommand{\cfttoctitlefont}{\null}
\renewcommand{\cftaftertoctitle}{\null}

% --- 2. สร้างหัวข้อสารบัญด้วยตนเอง ---
{%
	\fontsize{16pt}{19.2pt}\selectfont % กำหนดขนาดฟอนต์
	\noindent\bfseries\centerline{สารบัญ} % "สารบัญ" ตัวหนา กึ่งกลาง
	\noindent\hfill หน้า\par % "หน้า" ชิดขวา บรรทัดถัดมา
	\vspace{0.5\baselineskip} % เว้นระยะเล็กน้อยก่อนเริ่มรายการสารบัญ
}

% --- 3. คงการตั้งค่ารูปแบบรายการในสารบัญ (เหมือนเดิม) ---
\renewcommand{\cftchapfont}{\fontsize{16pt}{19.2pt}\selectfont}
\renewcommand{\cftchappagefont}{\fontsize{16pt}{19.2pt}\selectfont}
\renewcommand{\cftchapleader}{\cftdotfill{\cftdotsep}}
\renewcommand{\cftchapindent}{0pt}
\renewcommand{\cftchapnumwidth}{0pt}
\setlength{\cftbeforechapskip}{0pt}
\setlength{\cftchapindent}{0pt}

% --- 4. สร้างสารบัญ ---
\tableofcontents

%================== จบ toc.tex ==================