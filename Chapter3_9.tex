%==================== chapter3_9.tex ====================

\clearpage
\thispagestyle{plain}

\begingroup
\fontsize{16pt}{19.2pt}\selectfont
\justifying
\XeTeXlinebreakskip=0pt plus 1pt minus 0.5pt
\setlength{\parindent}{1.5cm}
\setlength{\parskip}{0pt}


% ================== Use Case ให้คะแนนในการแข่งขัน ==================
\begin{table}[h]
	\caption{Use Case ให้คะแนนในการแข่งขัน}
	{\tablefont
		\setlength{\tabcolsep}{6pt}%
		\begin{tabularx}{\linewidth}{@{} >{\justifying\arraybackslash}X >{\raggedleft\arraybackslash}p{4.2cm} @{}}
			\Xhline{1.5pt}
			\textbf{Use Case Title:}\enspace ให้คะแนนในการแข่งขัน & \UseCaseID[uc:register] \\
			\Xhline{0.5pt}
			\textbf{Primary Actor:}\enspace ผู้เชี่ยวชาญด้านปลากัด & \\
			\Xhline{0.5pt}
			\textbf{Stakeholder Actor:}\enspace - & \\
			\Xhline{0.5pt}
			\textbf{Main Flow:}\enspace ทำการให้คะแนนในการแข่งเพื่อที่ผู้จัดจะสามารถทำการประกาศผู้ชนะในการแข่งขันได้ & \\
			\Xhline{0.5pt}
			\textbf{Exception Flow ที่ 1:}\enspace การที่จะให้คะแนนในการแช่งขันปลากัดได้นั้นจำเป็นต้องได้รับเชิญให้เป็นกรรมก่อนถึงจะสามารถทำ
			การให้คะแนนในการแข่งขันได้ & \\
			\Xhline{1.5pt}
		\end{tabularx}
	}
\end{table}
% =====================================================

% ================== Use Case จัดการบัญชีผู้ใช้ เพิ่ม/ลบ/แก้ไข ข้อมูล ==================
\begin{table}[h]
	\caption{Use Case จัดการบัญชีผู้ใช้ เพิ่ม/ลบ/แก้ไข ข้อมูล}
	{\tablefont
		\setlength{\tabcolsep}{6pt}%
		\begin{tabularx}{\linewidth}{@{} >{\justifying\arraybackslash}X >{\raggedleft\arraybackslash}p{4.2cm} @{}}
			\Xhline{1.5pt}
			\textbf{Use Case Title:}\enspace จัดการบัญชีผู้ใช้ เพิ่ม/ลบ/แก้ไข ข้อมูล & \UseCaseID[uc:register] \\
			\Xhline{0.5pt}
			\textbf{Primary Actor:}\enspace ผู้บริหารจัดการข้อมูลที่เกิดขึ้นในระบบ & \\
			\Xhline{0.5pt}
			\textbf{Stakeholder Actor:}\enspace - & \\
			\Xhline{0.5pt}
			\textbf{Main Flow:}\enspace สามารถทำการ แก้ไข/เพิ่ม/ลบ ข้อมูลของผู้ใช้ในระบบทั้งหมดได้ & \\
			\Xhline{0.5pt}
			\textbf{Exception Flow ที่ 1:}\enspace กรณีที่ผู้บริหารจัดการข้อมูลที่เกิดขึ้นในระบบทำการ เพิ่ม/ลบ/แก้ไข ไม่ถูกต้องระบบจะทำการแสดงข้อความเตือนข้อผิดผลาดแล้วให้การก้ไขก่อนอัพเดทข้อมูล & \\
			\Xhline{1.5pt}
		\end{tabularx}
	}
\end{table}
% =====================================================

% ================== Use Case การจัดการการแจ้งเตือน ==================
\begin{table}[h]
	\caption{Use Case การจัดการการแจ้งเตือน}
	{\tablefont
		\setlength{\tabcolsep}{6pt}%
		\begin{tabularx}{\linewidth}{@{} >{\justifying\arraybackslash}X >{\raggedleft\arraybackslash}p{4.2cm} @{}}
			\Xhline{1.5pt}
			\textbf{Use Case Title:}\enspace การจัดการการแจ้งเตือน & \UseCaseID[uc:register] \\
			\Xhline{0.5pt}
			\textbf{Primary Actor:}\enspace ผู้บริหารจัดการข้อมูลที่เกิดขึ้นในระบบ & \\
			\Xhline{0.5pt}
			\textbf{Stakeholder Actor:}\enspace - & \\
			\Xhline{0.5pt}
			\textbf{Main Flow:}\enspace สามารถทำการเพิ่มลบข้อมูลต่าง ๆ ที่จะทำการแจ้งให้แก่ผู้ใช้ทราบ & \\
			\Xhline{1.5pt}
		\end{tabularx}
	}
\end{table}
% =====================================================