%!TEX TS-program = xelatex
\documentclass[a4paper, oneside]{report}

% ===== ระยะขอบ A4 ตามข้อกำหนด =====
% top=3.81cm, left=3.81cm, right=2.54cm, bottom=2.54cm
\usepackage[top=3.81cm, left=3.81cm, right=2.54cm, bottom=2.54cm, headheight=1cm]{geometry}

% ----- แพ็กเกจที่ใช้งาน -----
\usepackage{ragged2e}
\usepackage{enumitem}
\usepackage{tabularx,array}
\usepackage{graphicx}
\usepackage{makecell}
\usepackage{booktabs}
\usepackage{etoolbox}
\usepackage{titlesec}
\usepackage{indentfirst}
\usepackage{needspace}
\usepackage{caption}
\usepackage{adjustbox}
\usepackage{tocloft}

% ===== ชื่อหัวเรื่องเป็นภาษาไทย =====
\renewcommand{\contentsname}{สารบัญ}
\renewcommand{\listtablename}{สารบัญตาราง}
\renewcommand{\listfigurename}{สารบัญภาพ}

% ===== รูปแบบหัวเรื่อง + คำว่า "หน้า" ใต้หัวเรื่อง (ช่องว่างหลัง "หน้า" = 0) =====
\renewcommand{\cfttoctitlefont}{\centering\bfseries\fontsize{16pt}{19.2pt}\selectfont}
\renewcommand{\cftaftertoctitle}{%
	\par\vspace{\baselineskip}% เว้น 1 บรรทัดระหว่าง "สารบัญ" กับ "หน้า"
	\hfill\bfseries\fontsize{16pt}{19.2pt}\selectfont หน้า\par
}

\renewcommand{\cftlottitlefont}{\centering\bfseries\fontsize{16pt}{19.2pt}\selectfont}
\renewcommand{\cftafterlottitle}{%
	\par\vspace{\baselineskip}%
	\hfill\bfseries\fontsize{16pt}{19.2pt}\selectfont หน้า\par
}

\renewcommand{\cftloftitlefont}{\centering\bfseries\fontsize{16pt}{19.2pt}\selectfont}
\renewcommand{\cftafterloftitle}{%
	\par\vspace{\baselineskip}%
	\hfill\bfseries\fontsize{16pt}{19.2pt}\selectfont หน้า\par
}





% ===== การตั้งค่าสารบัญ =====

% ---- ดันหัว "สารบัญ/ตาราง/ภาพ" ขึ้นชิดขอบบน ----
\setlength{\cftbeforetoctitleskip}{-2\baselineskip}
\setlength{\cftbeforelottitleskip}{-2\baselineskip}
\setlength{\cftbeforeloftitleskip}{-2\baselineskip}

\setlength{\cftaftertoctitleskip}{0pt}
\setlength{\cftafterlottitleskip}{0pt}
\setlength{\cftafterloftitleskip}{0pt}



% (คงระยะหลังหัวไว้ตามเดิมที่ตั้งไว้แล้ว ไม่ต้องแก้)
% \cftaftertoctitle / \cftafterlottitle / \cftafterloftitle


% --- บท (Chapter) ---
\renewcommand{\cftchapfont}{\fontsize{16pt}{19.2pt}\bfseries} % ตัวหนา
\renewcommand{\cftchappagefont}{\fontsize{16pt}{19.2pt}\bfseries} % ตัวหนา
\renewcommand{\cftchapleader}{\cftdotfill{\cftdotsep}} % เส้นจุด
\setlength{\cftbeforechapskip}{1em} % ระยะห่างก่อนหน้า
\setlength{\cftchapnumwidth}{5em} % ความกว้างเลขบท

% --- หัวข้อ (Section) ---
\renewcommand{\cftsecfont}{\fontsize{16pt}{19.2pt}\selectfont} % ปกติ
\renewcommand{\cftsecpagefont}{\fontsize{16pt}{19.2pt}\selectfont} % ปกติ
\renewcommand{\cftsecleader}{\cftdotfill{\cftdotsep}} % เส้นจุด
\setlength{\cftbeforesecskip}{0pt} % ระยะห่างก่อนหน้า
\setlength{\cftsecindent}{2em} % เยื้องจากขอบ
\setlength{\cftsecnumwidth}{3em} % ความกว้างเลขหัวข้อ

% --- หัวข้อย่อย (Subsection) ---
\renewcommand{\cftsubsecfont}{\fontsize{16pt}{19.2pt}\selectfont} % ปกติ
\renewcommand{\cftsubsecpagefont}{\fontsize{16pt}{19.2pt}\selectfont} % ปกติ
\renewcommand{\cftsubsecleader}{\cftdotfill{\cftdotsep}} % เส้นจุด
\setlength{\cftbeforesubsecskip}{0pt} % ระยะห่างก่อนหน้า
\setlength{\cftsubsecindent}{4em} % เยื้องจากขอบ
\setlength{\cftsubsecnumwidth}{4em} % ความกว้างเลขหัวข้อย่อย

% ===== จัดเลขหน้า =====
\usepackage{fancyhdr}

\graphicspath{{/home/anming/Dropbox/ProJect/}{/home/anming/Dropbox/Apps/drawio-diagrams/}}


% ย่อหน้าแบบไทย (เรียกใช้เป็นครั้งคราว)
\providecommand{\ThaiPara}[1]{%
	\par\begingroup
	\justifying
	\XeTeXlinebreakskip=0pt plus 1pt minus 0.5pt
	\indent #1\par
	\endgroup
}

% ----- ฟอนต์ไทย -----
\usepackage{fontspec}
\setmainfont{TH Sarabun New}[
Path=/home/anming/.local/share/fonts/THSarabun/,
UprightFont    ={THSarabunNew.ttf},
BoldFont       ={THSarabunNew Bold.ttf},
ItalicFont     ={THSarabunNew.ttf},
BoldItalicFont ={THSarabunNew Bold.ttf}
]
\XeTeXlinebreaklocale "th"
\XeTeXlinebreakskip = 0pt plus 1pt

% ===== ขนาดตัวอักษรทั้งเล่ม =====
\makeatletter
\renewcommand\normalsize{\@setfontsize\normalsize{16pt}{19.2pt}}
\normalsize
\renewcommand\small{\@setfontsize\small{13pt}{15.6pt}}
\renewcommand\footnotesize{\@setfontsize\footnotesize{13pt}{15.6pt}}
\renewcommand\scriptsize{\@setfontsize\scriptsize{13pt}{15.6pt}}
\renewcommand\tiny{\@setfontsize\tiny{13pt}{15.6pt}}
\renewcommand\large{\@setfontsize\large{18pt}{21.6pt}}
\renewcommand\Large{\@setfontsize\Large{18pt}{21.6pt}}
\renewcommand\LARGE{\@setfontsize\LARGE{18pt}{21.6pt}}
\renewcommand\huge{\@setfontsize\huge{18pt}{21.6pt}}
\renewcommand\Huge{\@setfontsize\Huge{18pt}{21.6pt}}
\makeatother

% ===== จัดหน้า =====
\setlength{\parindent}{1.5cm}
\setlength{\parskip}{0pt}
\emergencystretch=2em

% ===== ชื่อรูป/ตาราง =====
\captionsetup[table]{name=ตารางที่}
\captionsetup[figure]{name=ภาพที่}
\captionsetup[table]{justification=raggedright,singlelinecheck=false, labelfont=bf,textfont=bf}
\captionsetup[figure]{justification=centering, singlelinecheck=false, labelfont=bf, textfont=bf}
\AtBeginEnvironment{figure}{\centering}

% ===== ฟอนต์ตารางเริ่มต้น =====
\newcommand{\tablefont}{\fontsize{13pt}{15.6pt}\selectfont}
\AtBeginEnvironment{tabularx}{\tablefont\setlength{\parindent}{0pt}}

% ===== ไฮเปอร์ลิงก์ =====
\usepackage[hidelinks,unicode]{hyperref}

% ===== บรรณานุกรม: Numeric ต่อเนื่องทั้งเล่ม ตามลำดับอ้างในเนื้อหา =====
\usepackage{csquotes}
\usepackage[
backend=biber,
style=numeric-comp,
sorting=none,
sortcites=true,
doi=false, isbn=false, url=true
]{biblatex}
\addbibresource{./references.bib}

% หัวข้อ "บรรณานุกรม" ไทย: หนา 16pt กึ่งกลาง + (ไม่เพิ่มใน TOC)
\defbibheading{thai-bib}{
	\clearpage
	\centering\bfseries\fontsize{16pt}{19.2pt}\selectfont บรรณานุกรม\par
	\vspace{\baselineskip}
}
% แขวนบรรทัด (รายการยาวบรรทัดต่อไปเยื้อง 1.50 ซม.)
\setlength{\bibhang}{1.5cm}

% --- Use Case ID auto-increment ---
\newcounter{usecaseid}
\newcommand{\UseCaseID}[1][]{%
	\refstepcounter{usecaseid}%
	\textbf{Use Case ID: \theusecaseid}%
	\if\relax\detokenize{#1}\relax\else\label{#1}\fi
}

% คอลัมน์ Description แบบไม่เยื้อง + จัดกระจายไทย
\newcolumntype{D}{>{\noindent\parindent0pt\justifying\arraybackslash}X}

% ===== enumerate layout (global, same spec) =====
\makeatletter
\@ifundefined{LoneLabelSep}{\newlength{\LoneLabelSep}}{}
\@ifundefined{LoneLabelWidth}{\newlength{\LoneLabelWidth}}{}
\@ifundefined{LoneContentCol}{\newlength{\LoneContentCol}}{}
\@ifundefined{LtwoLabelSep}{\newlength{\LtwoLabelSep}}{}
\@ifundefined{LtwoLabelWidth}{\newlength{\LtwoLabelWidth}}{}
\@ifundefined{ExtraAlign}{\newlength{\ExtraAlign}}{}
\@ifundefined{LthreeLabelSep}{\newlength{\LthreeLabelSep}}{}
\@ifundefined{LthreeLabelWidth}{\newlength{\LthreeLabelWidth}}{}
\@ifundefined{NestedStep}{\newlength{\NestedStep}}{}
\makeatother

\setlength{\LoneLabelSep}{0.5em}
\settowidth{\LoneLabelWidth}{9.}
\setlength{\LoneContentCol}{\dimexpr 1.5cm + \LoneLabelWidth + \LoneLabelSep\relax}

\setlength{\LtwoLabelSep}{0.5em}
\settowidth{\LtwoLabelWidth}{9.9.}
\setlength{\ExtraAlign}{-2.8em}

\setlength{\LthreeLabelSep}{0.5em}
\settowidth{\LthreeLabelWidth}{9.9.9.}
\setlength{\NestedStep}{0pt}

% Level 1
\setlist[enumerate,1]{%
	label=\arabic*., align=left,
	leftmargin=1.5cm, labelindent=0pt,
	labelwidth=\LoneLabelWidth, labelsep=\LoneLabelSep,
	itemsep=0pt, topsep=0.5\baselineskip
}
% Level 2
\setlist[enumerate,2]{%
	label*=\arabic*., align=left,
	leftmargin=*,
	labelwidth=\LtwoLabelWidth, labelsep=\LtwoLabelSep,
	labelindent=\dimexpr \LoneContentCol + \ExtraAlign - \LtwoLabelWidth - \LtwoLabelSep\relax,
	itemsep=0pt, topsep=0pt
}
% Level 3
\setlist[enumerate,3]{%
	label*=\arabic*., align=left,
	leftmargin=*,
	labelwidth=\LthreeLabelWidth, labelsep=\LthreeLabelSep,
	labelindent=\dimexpr \LoneContentCol + \ExtraAlign + \NestedStep
	- \LthreeLabelWidth - \LthreeLabelSep\relax,
	itemsep=0pt, topsep=0pt
}

% ===== การตั้งค่าเลขหน้า =====
\pagestyle{fancy}
\fancyhf{}
\renewcommand{\headrulewidth}{0pt}
% เลขหน้าอยู่มุมบนขวา ระยะจากขอบบน 2.5 cm และขอบขวา 2.5 cm
\fancyhead[R]{\fontsize{16pt}{19.2pt}\selectfont\thepage}

\fancypagestyle{plain}{%
	\fancyhf{}%
	\renewcommand{\headrulewidth}{0pt}%
	\fancyhead[R]{\fontsize{16pt}{19.2pt}\selectfont\thepage}%
}

% กำหนดรูปแบบเลขหน้าสำหรับส่วนนำ (อักษรไทย)
\newcommand{\thaiPageNumber}[1]{%
	\ifcase#1\or ก\or ข\or ค\or ง\or จ\or ฉ\or ช\or ซ\or ฌ\or ญ\or
	\or ด\or ต\or ถ\or ท\or ธ\or น\or บ\or ป\or ผ\or ฝ\or พ\or ฟ\or ภ\or ม\or ย\or ร\or ล\or ว\or ศ\or ษ\or ส\or ห\or ฬ\or อ\or ฮ\fi
}

\begin{document}
	
	% ===== ส่วนนำ =====
	\pagestyle{empty}
	\IfFileExists{Cover.tex}{%==================== Cover.tex ====================

% ---------- ปกนอก (หน้า 1) ----------
\clearpage
\thispagestyle{empty}
\begingroup
\fontsize{16pt}{19.2pt}\selectfont % ทั้งหน้าตัวหนา 16pt ตามสเปก
\bfseries

% ส่วนต้น (ชื่อเรื่อง ชิดขอบบนของหน้ากระดาษ)
% จัดแบบ "ปิรามิดคว่ำ" อย่างหยาบโดยแบ่งบรรทัดแล้วจัดกึ่งกลาง
\noindent
\begin{minipage}[t]{\textwidth}
	\centering
	เว็บแอปพลิเคชันศูนย์รวมการจัดประกวดปลากัดไทย
\end{minipage}

% ส่วนกลาง (ชื่อผู้ทำ อยู่กึ่งกลางระหว่างท้ายชื่อเรื่องกับต้นส่วนท้าย)
\vspace*{\fill}
\begin{center}
	เอกสิทธิ์ อัศวดารา
\end{center}

% ส่วนท้าย (ข้อความบล็อกด้านล่าง)
\vspace*{\fill}
\noindent
\begin{minipage}[b]{\textwidth}
	\centering
	ภาคนิพนธ์เสนอมหาวิทยาลัยพะเยา เพื่อเป็นส่วนหนึ่งของการศึกษา\\
	หลักสูตรปริญญาตรี วิทยาศาสตรบัณฑิต\\
	สาขาวิทยาการคอมพิวเตอร์\\
	กันยายน พ.ศ. 2568\\
	ลิขสิทธิ์ของมหาวิทยาลัยพะเยา
\end{minipage}
\par
\endgroup
\clearpage

% ---------- หน้าเปล่า (หน้า 2) ----------
\thispagestyle{empty}\null\clearpage

% ---------- ปกใน (ไทย) (หน้า 3) ----------
\thispagestyle{empty}
\begingroup

\noindent
\begin{minipage}[t]{\textwidth}
	\centering
	เว็บแอปพลิเคชันศูนย์รวมการจัดประกวดปลากัดไทย
\end{minipage}

\vspace*{\fill}
\begin{center}
	เอกสิทธิ์ อัศวดารา
\end{center}

\vspace*{\fill}
\noindent
\begin{minipage}[b]{\textwidth}
	\centering
	ภาคนิพนธ์เสนอมหาวิทยาลัยพะเยา เพื่อเป็นส่วนหนึ่งของการศึกษา\\
	หลักสูตรปริญญาตรี วิทยาศาสตรบัณฑิต\\
	สาขาวิทยาการคอมพิวเตอร์\\
	กันยายน พ.ศ. 2568\\
	ลิขสิทธิ์ของมหาวิทยาลัยพะเยา
\end{minipage}

\par
\endgroup
\clearpage

% ---------- ปกใน (อังกฤษ) (หน้า 4) ----------
\thispagestyle{empty}
\begingroup


% Title (approx. translation; แก้คำแปลได้ตามที่ต้องการ)
\noindent
\begin{minipage}[t]{\textwidth}
	\centering
	Web Application for the Centralized\\
	Thai Betta Fish Competition
\end{minipage}

\vspace*{\fill}
\begin{center}
	% เปลี่ยนเป็นสะกดชื่ออังกฤษที่ต้องการได้
	Aekkasit Oatsawadara
\end{center}

\vspace*{\fill}
\noindent
\begin{minipage}[b]{\textwidth}
	\centering
	A Term Paper submitted to the University of Phayao in Partial Fulfillment of the\\
	Requirements for the bachelor’s degree. Bachelor of Science.\\
	Department of Computer Science\\
	Computer Science Program\\
	September 2025\\
	Copyright belongs to the University of Phayao
\end{minipage}

\par
\endgroup
\clearpage

%================== จบ Cover.tex ==================
}{}
	\IfFileExists{Approval.tex}{%==================== Approval.tex ====================

\clearpage
% \thispagestyle{empty}

\begingroup
\fontsize{16pt}{19.2pt}\selectfont
\sloppy % ผ่อนกฎจัดบรรทัดเฉพาะหน้าแบบฟอร์ม

% ----- ปุ่มปรับได้ -----
\newlength{\SignRule} \setlength{\SignRule}{5cm}   % ความยาวเส้นจุด (5cm/11cm แล้วแต่ต้องการ)
\newlength{\SignSep}  \setlength{\SignSep}{0.75em}  % ช่องว่างหลังเส้นจุด
\newlength{\DotUnit}  \setlength{\DotUnit}{0.15em}  % ยิ่งเล็กยิ่งจุดถี่

% วาดเส้นจุด: หน่วยจุดยิ่งเล็กยิ่งถี่
\newcommand{\dotleaders}{\leaders\hbox to \DotUnit{\hss.\hss}\hfill}

% #1 = ข้อความตำแหน่ง (ต่อจากเส้นจุด; ยาวแค่ไหนก็ตัดบรรทัดเอง)
% #2 = ชื่อในวงเล็บ (บรรทัดถัดไป ชิดซ้าย)
\newcommand{\signrow}[2]{%
	% ย่อเฉพาะบรรทัดแรก = ความยาวเส้นจุด + ช่องว่าง
	{\parshape 2
		\dimexpr\SignRule+\SignSep\relax \dimexpr\linewidth-\SignRule-\SignSep\relax
		0pt \linewidth
		% ซ้อนเส้นจุดให้เริ่มที่ "ขอบซ้ายจริง" โดยยื่นกลับไปทางซ้ายด้วย \llap
		\noindent\llap{%
			\makebox[\dimexpr\SignRule+\SignSep\relax][l]{%
				\makebox[\SignRule][l]{\dotleaders}%
			}%
		}%
		\raggedright #1\par}%
	% ชื่อในวงเล็บ บรรทัดใหม่ ชิดซ้าย
	\noindent(#2)\par\vspace{\baselineskip}%
}


% ---------- ส่วนหัว (กึ่งกลาง) ----------
\begin{center}
	ภาคนิพนธ์\\[-0.15\baselineskip]
	เรื่อง\\[1\baselineskip]
	
	เว็บแอปพลิเคชันศูนย์รวมการจัดประกวดปลากัดไทย\\[1\baselineskip]
	
	ของ เอกสิทธิ์ อัศวดารา\\[1\baselineskip]
	
	ได้รับการพิจารณาอนุมัติให้เป็นส่วนหนึ่งของการศึกษา
	หลักสูตรปริญญาวิทยาศาสตรบัณฑิต\\
	สาขาวิชาวิทยาการคอมพิวเตอร์
	ของมหาวิทยาลัยพะเยา
\end{center}

\vspace{\baselineskip}

% ---------- ลำดับผู้ลงนาม (ห้ามครอบด้วย center) ----------
\signrow{อาจารย์ที่ปรึกษา}{ผู้ช่วยศาสตราจารย์ ดร.สุรางคนา ระวังยศ}

\signrow{กรรมการ}{อาจารย์วรกฤต แสนโภชน์}

\signrow{กรรมการ}{อาจารย์ธนวัฒน์ แซ่เอียบ}

% แถวที่คุณต้องการให้ “ต่อท้ายยาว ๆ แล้วตัดบรรทัดเอง”
\signrow{ประธานหลักสูตรวิทยาศาสตรบัณฑิต \\ สาขาวิชาวิทยาการคอมพิวเตอร์
	คณะเทคโนโลยีสารสนเทศและการสื่อสาร มหาวิทยาลัยพะเยา}{อาจารย์วรกฤต แสนโภชน์}

\par\endgroup
\clearpage

%================== จบ Approval.tex ==================
}{}
	
	% เริ่มนับหน้าส่วนนำด้วยอักษรไทย
	\clearpage
	\pagenumbering{roman}
	\setcounter{page}{1}
	\pagestyle{fancy}
	\renewcommand{\thepage}{\thaiPageNumber{\value{page}}}
	
	% แสดงเลขหน้าเฉพาะ "บทคัดย่อไทย" และ "กิตติกรรมประกาศ"
	\IfFileExists{Abstract.tex}{%==================== abstract.tex ====================

\clearpage
\phantomsection
% \thispagestyle{empty} % ถ้าไม่เอาเลขหน้า ให้เอาคอมเมนต์ออก

\begingroup
% ทั้งหน้า 13pt (baseline ~15.6pt)
\fontsize{13pt}{15.6pt}\selectfont

% จัดเต็มบรรทัดแบบไทย + ย่อหน้า 1.5 ซม.
\justifying
\XeTeXlinebreakskip=0pt plus 1pt minus 0.5pt
\setlength{\parindent}{1.5cm}
\setlength{\parskip}{0pt}

% ---------- จัดบรรทัด "ป้ายกำกับ : ค่า" ให้ตรงคอลัมน์ ----------
% วัดความกว้างป้ายกำกับที่ยาวสุด (ใช้ตัวหนา + มีโคลอน)
\newlength{\FieldLabelWidth}
\newlength{\FieldSep} \setlength{\FieldSep}{3em} % ระยะเว้นระหว่างฉลากกับค่า

\settowidth{\FieldLabelWidth}{\bfseries ประเภทสารนิพนธ์ :}

% ป้ายกำกับบรรทัดแรก "ตัวหนา 13pt" + ค่า "13pt ธรรมดา" อยู่บรรทัดเดียวกัน
% ถ้าค่ายาวจะตัดบรรทัดภายในคอลัมน์ขวาและชิดซ้ายใต้คอลัมน์ค่าเดิมเสมอ
\newcommand{\Field}[2]{%
	\noindent
	\makebox[\FieldLabelWidth][l]{\bfseries #1 :}%
	\hspace{\FieldSep}%
	\parbox[t]{\dimexpr\linewidth-\FieldLabelWidth-\FieldSep\relax}{#2}%
	\par
}

% ---------- บล็อกข้อมูลหัวเรื่อง ----------
\Field{เรื่อง}{เว็บแอปพลิเคชันศูนย์รวมการจัดประกวดปลากัดไทย}
\Field{ผู้ศึกษาค้นคว้า}{เอกสิทธิ์ อัศวดารา}
\Field{อาจารย์ที่ปรึกษา}{ผศ.\,ดร.\,สุรางคนา ระวังยศ}
\Field{ประเภทสารนิพนธ์}{ภาคนิพนธ์ ปริญญาตรี สาขาวิชาวิทยาการคอมพิวเตอร์}
\Field{คำสำคัญ}{ปลากัดไทย, การประกวด, เว็บแอปพลิเคชัน}

% เว้น 1 บรรทัด
\vspace{\baselineskip}

% ---------- หัวข้อ "บทคัดย่อ" ----------
\phantomsection
\addcontentsline{toc}{chapter}{บทคัดย่อ}

\begin{center}
	\bfseries บทคัดย่อ
\end{center}

% เว้น 1 บรรทัด
\vspace{\baselineskip}

% ---------- เนื้อหาบทคัดย่อ ----------
การแข่งขันปลากัดไทยในปัจจุบันมีการแพร่หลายผ่านสื่อออนไลน์ เช่น กลุ่ม Facebook ต่าง ๆ
ซึ่งทำให้ข้อมูลการประกวดกระจัดกระจาย ไม่เป็นระบบ และยากต่อการอ้างอิง
งานวิจัยนี้จึงพัฒนาเว็บแอปพลิเคชันศูนย์รวมการจัดประกวดปลากัดไทย
เพื่อรวบรวมข้อมูลการประกวด การจัดเก็บผลการตัดสิน และการเผยแพร่ข่าวสารอย่างเป็นระบบ
โดยมีเป้าหมายเพื่ออำนวยความสะดวกแก่นักเพาะเลี้ยงและผู้สนใจทั่วไป
รวมถึงยกระดับคุณภาพการประกวดให้มีมาตรฐานเดียวกันผ่านการประยุกต์ใช้เทคโนโลยีสารสนเทศ
ระบบถูกออกแบบและพัฒนาด้วยเฟรมเวิร์กสมัยใหม่ รองรับการใช้งานทั้งบนคอมพิวเตอร์และอุปกรณ์พกพา
มีฟังก์ชันการสมัครแข่งขัน การบันทึกผลการประกวด การรายงานผลย้อนหลัง และการจัดเก็บข้อมูลผู้ใช้งาน
ผลการทดสอบการใช้งานจริงพบว่าระบบสามารถทำงานได้ตามที่ออกแบบ และช่วยลดความซับซ้อนในการจัดการข้อมูลการประกวดปลากัดได้อย่างมีประสิทธิภาพ

\par\endgroup
\clearpage

%================== จบ abstract.tex ==================
}{}
	\IfFileExists{Acknowledgement.tex}{%==================== acknowledgement.tex ====================

\clearpage
% \phantomsection
%\thispagestyle{empty}

\begingroup
\fontsize{16pt}{19.2pt}\selectfont

% จัดกระจายเต็มบรรทัดแบบไทย
\justifying
\XeTeXlinebreakskip=0pt plus 1pt minus 0.5pt

% ย่อหน้า 1.5 ซม. ชัดเจน (เผื่อ main ตั้งค่าอย่างอื่นไว้)
\setlength{\parindent}{1.5cm}
\setlength{\parskip}{0pt}

% เว้น 1 บรรทัดจากขอบบน
\vspace*{-\topskip}\vspace*{\baselineskip}

% หัวเรื่อง
\phantomsection
\addcontentsline{toc}{chapter}{กิตติกรรมประกาศ}

\begin{center}
	\textbf{กิตติกรรมประกาศ}
\end{center}

\vspace{\baselineskip}

% --- ย่อหน้า 1 ---
การวิจัยและการทำโครงงานเรื่องเว็บแอปพลิเคชันศูนย์รวมการจัดประกวดปลากัดไทย
สำเร็จลงได้ด้วยความกรุณาอย่างสูงจากอาจารย์ซึ่งเป็นอาจารย์ที่ปรึกษา
ผศ.\,ดร.\,สุรางคนา ระวังยศ และอาจารย์ผู้ให้ความรู้เกี่ยวกับปลากัด การคัดแยกปลากัด
การแบ่งประเภทปลากัด ซึ่งเป็นอาจารย์ที่ให้ความรู้ในด้านปลากัดโดยตรง
โดย ผศ.\,ดร.\,เกรียงไกร สีตะพันธุ์ ได้ให้แนวคิดที่ดีและแก้ไขข้อบกพร่องต่าง ๆ
รวมถึงการเก็บ \textit{Dataset} ขอขอบคุณกลุ่มชุมชนบน Facebook
ที่ให้ความอนุเคราะห์รูปภาพปลากัดพร้อมระบุประเภทปลากัดสำหรับงานวิจัยครั้งนี้
ตลอดระยะเวลาในการวิจัย และคณะกรรมการทุกท่านที่ให้คำแนะนำและให้คำปรึกษาด้วยความเอาใจใส่เป็นอย่างยิ่ง
จนการศึกษาค้นคว้าด้วยตนเองเสร็จสมบูรณ์ คณะผู้ศึกษาค้นคว้าขอกราบขอบพระคุณอย่างสูงไว้ ณ ที่นี้

% --- ย่อหน้า 2 ---
\par ขอกราบขอบพระคุณคณะกรรมการสอบโครงงาน ได้แก่ อาจารย์วรกฤต แสนโภชน์
และอาจารย์ธนวัฒน์ แซ่เอียบ รวมถึงอาจารย์ประจำสาขาวิทยาการคอมพิวเตอร์
คณะเทคโนโลยีสารสนเทศและการสื่อสาร มหาวิทยาลัยพะเยา
ที่ให้คำแนะนำอันมีค่าและความช่วยเหลือตลอดกระบวนการทำวิจัย
จนทำโครงงานนี้สำเร็จลุล่วงไปได้ด้วยดี ขอบคุณเพื่อน ๆ ร่วมสาขาที่คอยช่วยเหลือ
คอยให้กำลังใจ และให้คำแนะนำที่มีประโยชน์ต่อผู้วิจัย

% --- ย่อหน้า 3 ---
\par สุดท้ายนี้ คณะผู้จัดทำงานวิจัยและโครงงานนี้ขอขอบพระคุณทุกท่านที่มีส่วนร่วมในการให้คำแนะนำ
และสนับสนุนโครงงานนี้ด้วยความซาบซึ้งอย่างยิ่ง

% เว้น 1 บรรทัด แล้วชิดขอบขวาใส่ชื่อ
\vspace{\baselineskip}
\begin{flushright}
	เอกสิทธิ์ อัศวดารา
\end{flushright}

\par\endgroup
\clearpage

%================== จบ acknowledgement.tex ==================
}{}
	
	% หน้า "สารบัญ/สารบัญตาราง/สารบัญภาพ" ไม่แสดงเลข แต่ยังนับ
	\clearpage
	\thispagestyle{empty}
\begin{center}
		\tableofcontents
\end{center}
	
	% สารบัญตาราง
	\clearpage
	\thispagestyle{empty}
\begin{center}
		\listoftables
\end{center}

	
	% สารบัญภาพ
	\clearpage
	\thispagestyle{empty}
\begin{center}
		\listoffigures
\end{center}
	
	% ===== ส่วนเนื้อหา =====
	\clearpage
	\pagenumbering{arabic}
	\setcounter{page}{1}
	\pagestyle{fancy}
	\renewcommand{\thepage}{\arabic{page}}
	
	% บทที่ 1 (ใช้หน้าใน Chapter1.tex เป็นหน้าบท)
	\thispagestyle{plain}
	\IfFileExists{Chapter1.tex}{%==================== chapter2.tex ====================

\clearpage
\thispagestyle{empty}

\begingroup
% เนื้อหาบท: 16pt baseline ~19.2pt ตามสเปกเล่ม
\fontsize{16pt}{19.2pt}\selectfont
\justifying
\XeTeXlinebreakskip=0pt plus 1pt minus 0.5pt
\setlength{\parindent}{1.5cm}
\setlength{\parskip}{0pt}

% ---------- หัวบท + เขียนสารบัญบทก่อนหัวข้อย่อย ----------
\phantomsection
\addcontentsline{toc}{chapter}{บทที่ 1 บทนำ}
\begin{center}
{\bfseries\fontsize{18pt}{21.6pt}\selectfont บทที่ 1}
\end{center}

\vspace{\baselineskip}

% ---------- ชื่อบท (บทนำ) ----------
\begin{center}
{\bfseries\fontsize{18pt}{21.6pt}\selectfont บทนำ}
\end{center}

\vspace{\baselineskip}

% ---------- หัวข้อใหญ่ (ชิดซ้าย, หนา 16pt) ----------
\section*{ความเป็นมาและความสำคัญ}
\addcontentsline{toc}{section}{ความเป็นมาและความสำคัญ}

% ---------- เนื้อหา (จัดกระจายแบบไทย, ย่อหน้าแรก 1.5 ซม.) ----------
\indent ปลากัดไทยเป็นสัตว์น้ำที่มีความงดงามและเอกลักษณ์เฉพาะตัว ซึ่งเป็นที่นิยมในการเลี้ยงดู
ทั้งในด้านความสวยงามและการแข่งขัน \cite{ngthai2021betta} ปลากัดมีลักษณะทางกายภาพที่โดดเด่น เช่น สีสันสดใส
ครีบ ที่ ยาวสง่า งาม และลวดลายที่ เป็น เอกลักษณ์ ทำให้ ปลากัด เป็น สัตว์ เลี้ยงที่ ถูกใจผู้ ชื่น ชอบสัตว์
น้ำสวยงาม นอกจากความสวยงามแล้ว การแข่งขันปลากัดยังเป็นอีกหนึ่งกิจกรรมที่ได้รับความนิยม
อย่างมาก โดยเฉพาะการแข่งขัน ด้านความสวยงามที่ เน้น การประเมิน ลักษณะภายนอกของปลากัด
เช่น สีสันที่คมชัด ความสมบูณ์ของครีบ และการเคลื่อนไหวที่สง่างาม การประกวดปลากัดไม่ได้จำกัด
เฉพาะในระดับท้องถิ่นเท่านั้น แต่ยังมีการจัดงานระดับสากล ซึ่งดึงดูดนักเพาะพันธุ์และผู้เลี้ยงปลากัด
จากทั่วโลก ปลากัดไทยได้กลายเป็นปลาสวยงามที่มีปริมาณการส่งออกสูงเป็นอันดับ 1 ของประเทศ
โดยมีการผลิตและส่งออกปลากัดไปกว่า 80 ประเทศทั่วโลก อาทิ สหรัฐอเมริกา ฝรั่งเศส สิงคโปร์ จีน
และอิหร่าน ซึ่งมีปริมาณการส่งออกเฉลี่ยมากกว่า 20 ล้านตัวต่อปี สร้างรายได้ให้ประเทศปีละกว่า \cite{nstda2020betta}
200 ล้านบาท จุด เด่น ของปลากัด ไทยที่ ได้ รับ ความนิยมจากทั่ว โลก คือ ความหลากหลายของครีบ
หาง เช่น ครีบ สั้น ครีบ ยาว หางแบบพระจันทร์ ครึ่ง ดวง (Halfmoon) หางมงกุฎ (Crowtail) 2 หาง
(Doubletail) หรือครีบหูใหญ่ เช่น หูช้าง (Bigears/Dumbo) รวมไปถึงสีสันที่สวยงามฉูดฉาดสะดุด
ตา ปลากัดไทยยังเป็นที่นิยมเพราะสามารถเลี้ยงในพื้นที่เล็กและดูแลได้ง่าย ทำให้เหมาะสำหรับผู้
ที่มีเวลาน้อยปัจจุบัน เกษตรกรไทยได้มีการพัฒนาสายพันธุ์ให้แปลกใหม่และสวยงาม โดยเฉพาะสีสัน
ที่สามารถเลือกเพาะปลาให้มีสีตามที่ต้องการได้ เช่น ปลากัดสีธงชาติ ที่มีสีขาว น้ำเงิน และแดงอยู่
ในตัวเดียวกัน และยังสามารถจัดเรียงสีให้คล้ายธงชาติได้ อย่างไรก็ตาม ผู้เลี้ยงปลากัดยังขาดระบบที่
ช่วยให้สามารถเข้าถึงบริการประเมินคุณภาพปลากัดได้สะดวกและรวดเร็ว รวมถึงการให้คำปรึกษาด้านการเพาะพันธุ์และการดูแลปลากัด ระบบที่มีอยู่ยังขาดความยืดหยุ่นในการรองรับผู้เลี้ยงที่ต้องการ
ทราบผลการประเมินเบื้องต้น ก่อนส่ง ปลากัดเข้าประกวดจริง ซึ่งทำให้ เกิดความล่าช้าในการเตรียม
ปลากัดเข้าสู่การประกวด จากปัญหานี้ การพัฒนาเว็บแอปพลิเคชันศูนย์รวมการจัดประกวดปลากัด
ไทย จึงเป็นทางเลือกที่สำคัญในการช่วยให้ผู้เลี้ยงปลากัดสามารถเข้าถึงบริการต่าง ๆ เช่น การประเมิน
คุณภาพปลากัดโดย AI หรือผู้เชี่ยวชาญ การรับคำปรึกษาด้านการเพาะพันธุ์ และการติดตามผลการ
ประเมินได้อย่างสะดวกและรวดเร็ว เว็บแอปพลิเคชันนี้จะเป็นเครื่องมือที่ช่วยให้ผู้เลี้ยงมีความพร้อม
มากขึ้นในการพัฒนาปลากัดและเพิ่มโอกาสในการแข่งขัน

\par\endgroup
\clearpage

%================== จบ chapter1.tex ====================}{}
	\IfFileExists{Chapter1_1.tex}{%==================== chapter1_1.tex ====================

\clearpage
\thispagestyle{plain}

\begingroup
\fontsize{16pt}{19.2pt}\selectfont
\justifying
\XeTeXlinebreakskip=0pt plus 1pt minus 0.5pt
\setlength{\parindent}{1.5cm}
\setlength{\parskip}{0pt}

% ---------- วัตถุประสงค์ของงานวิจัย ----------
\section*{วัตถุประสงค์ของงานวิจัย}
\addcontentsline{toc}{section}{วัตถุประสงค์ของงานวิจัย}
\indent 1.\,พัฒนาแพลตฟอร์มที่เป็นศูนย์รวมการจัดประกวดปลากัดไทยบนเว็บแอปพลิเคชัน

\vspace{\baselineskip} % เว้น 1 บรรทัด

% ---------- ขอบเขตของงานวิจัย ----------
\section*{ขอบเขตของงานวิจัย}
\addcontentsline{toc}{section}{ขอบเขตของงานวิจัย}

\indent งานวิจัยนี้มุ่งเน้นการออกแบบและพัฒนาเว็บแอปพลิเคชันศูนย์รวมการจัดประกวดปลากัดไทย
โดยใช้หลักการ Responsive Design รองรับการใช้งานทุกแพลตฟอร์ม โดยมีขอบเขตดังนี้

% ===== รูปแบบ enumerate ให้ตรงสเปกเดิม (พร้อมตัวแปรครบ) =====
% ต้องมี \usepackage{enumitem}

% กำหนดค่าแต่ละตัว
\setlength{\LoneLabelSep}{0.5em}                                 % ระยะเลขกับข้อความ (ระดับ 1)
\settowidth{\LoneLabelWidth}{9.}                                 % กว้างพอ "9."
\setlength{\LoneContentCol}{\dimexpr 1.5cm + \LoneLabelWidth + \LoneLabelSep\relax}

\setlength{\LtwoLabelSep}{0.5em}                                 % ระยะเลขกับข้อความ (ระดับ 2)
\settowidth{\LtwoLabelWidth}{9.9.}                               % กว้างพอ "9.9."
\setlength{\ExtraAlign}{-2.8em}                                  % ปรับชิดเพิ่ม/ลดตามต้องการ

\setlength{\LthreeLabelSep}{0.5em}                               % ระยะเลขกับข้อความ (ระดับ 3)
\settowidth{\LthreeLabelWidth}{9.9.9.}                           % กว้างพอ "9.9.9."
\setlength{\NestedStep}{2em}                                     % ให้ระดับ 3 ลึกกว่า level 2 อีก 2em (ปรับได้)

% ระดับ 1
\setlist[enumerate,1]{%
	label=\arabic*., align=left,
	leftmargin=1.5cm, labelindent=0pt,
	labelwidth=\LoneLabelWidth, labelsep=\LoneLabelSep,
	itemsep=0pt, topsep=0.5\baselineskip
}

% ระดับ 2 (ชิฟต์ด้วย \ExtraAlign ให้ตรงสเปกเดิม)
\setlist[enumerate,2]{%
	label*=\arabic*., align=left,
	leftmargin=*,
	labelwidth=\LtwoLabelWidth, labelsep=\LtwoLabelSep,
	labelindent=\dimexpr \LoneContentCol + \ExtraAlign - \LtwoLabelWidth - \LtwoLabelSep\relax,
	itemsep=0pt, topsep=0pt
}

% ระดับ 3 (จุดเริ่มข้อความ = \LoneContentCol + \ExtraAlign + \NestedStep)
\setlist[enumerate,3]{%
	label*=\arabic*., align=left,
	leftmargin=*,
	labelwidth=\LthreeLabelWidth, labelsep=\LthreeLabelSep,
	labelindent=\dimexpr \LoneContentCol + \ExtraAlign + \NestedStep
	- \LthreeLabelWidth - \LthreeLabelSep\relax,
	itemsep=0pt, topsep=0pt
}

\begin{enumerate}
	\item ผู้ใช้งานที่ไม่มีบัญชีผู้ใช้งาน
	\begin{enumerate}
		\item สามารถลงทะเบียนสมัครสมาชิกได้
		\item สามารถเข้าชมข่าวสารเกี่ยวกับการแข่งขันหรือข้อมูลอื่น ๆ ได้
	\end{enumerate}
	
	\item ผู้เลี้ยงปลากัด
	\begin{enumerate}
		\item ลงทะเบียนและจัดการบัญชีผู้ใช้งานของตนเอง
		\item อัปโหลดข้อมูลปลากัด เช่น รูปภาพและรายละเอียดต่าง ๆ
		\item ขอรับบริการประเมินคุณภาพปลากัดโดยผู้เชี่ยวชาญ
		\item ติดตามผลการประเมินและประวัติการประเมิน
		\item รับการแจ้งเตือนกิจกรรมและข้อมูลใหม่จากระบบ
		\item เข้าร่วมการประกวดปลากัด
	\end{enumerate}
	
	\item ผู้จัดการประกวดปลากัด
	\begin{enumerate}
		\item เพิ่มรายละเอียดการจัดการประกวด
		\item ลงทะเบียนและจัดการบัญชีผู้ใช้งาน
		\item กำหนดการจัดประกวดปลากัดไทย
		\item กำหนดกรรมการตัดสินในการประกวดแต่ละครั้ง
		\item ประกาศผลการแข่งขัน
		\item ดูประวัติการจัดการประกวด
		\item ดูผลคะแนนการแข่งขันทั้งหมดของผู้เลี้ยงปลากัด
	\end{enumerate}
	
	\item ผู้เชี่ยวชาญด้านปลากัด
	\begin{enumerate}
		\item ลงทะเบียนและจัดการโปรไฟล์ผู้เชี่ยวชาญ
		\item ตรวจสอบข้อมูลปลากัดที่ถูกอัปโหลดโดยผู้เลี้ยง
		\item ให้บริการประเมินคุณภาพปลากัดและบันทึกผลการประเมิน
		\item เข้าร่วมการประเมินแบบกลุ่มและกิจกรรมออนไลน์
	\end{enumerate}
\newpage
	\item ผู้บริหารจัดการข้อมูล
	\begin{enumerate}
		\item ดูแลและจัดการระบบทั้งหมด
		\item จัดการข้อมูลผู้ใช้งาน เช่น การปรับปรุงและลบข้อมูล
		\item ติดตามผลการประเมินและกิจกรรมต่าง ๆ ภายในระบบ
		\item จัดการการแจ้งเตือนและข้อมูลสำคัญของระบบ
	\end{enumerate}
\end{enumerate}

% --- เว้น 1 บรรทัด ---
\vspace{\baselineskip}

% ---------- แนวคิดและหลักการ ----------
\noindent{\bfseries\fontsize{16pt}{19.2pt}\selectfont แนวคิดและหลักการ}\par
\indent แนวคิดและหลักการของการพัฒนาเว็บแอปพลิเคชันศูนย์รวมการจัดประกวดปลากัดไทย
คือการสร้างแพลตฟอร์มที่สามารถเชื่อมโยงผู้เลี้ยงปลากัด ผู้เชี่ยวชาญ และผู้จัดการประกวดเข้าด้วยกันอย่างมีประสิทธิภาพ
โดยมีเป้าหมายเพื่อเพิ่มความสะดวกในการเข้าถึงบริการต่าง ๆ เช่น การประเมินคุณภาพปลากัด การให้คำปรึกษาด้านการเพาะพันธุ์
และการติดตามผลการประกวด ระบบจะรองรับการใช้งานทั้งบนอุปกรณ์พกพาและเว็บเบราว์เซอร์ โดยใช้หลักการออกแบบแบบ
Responsive Design เพื่อให้ผู้ใช้งานสามารถเข้าถึงข้อมูลและบริการได้ทุกที่ทุกเวลา

% --- เว้น 1 บรรทัด ---
\vspace{\baselineskip}

% ---------- เครื่องมือที่ใช้ในการพัฒนาโปรแกรม ----------
\noindent{\bfseries\fontsize{16pt}{19.2pt}\selectfont เครื่องมือที่ใช้ในการพัฒนาโปรแกรม}\par

\begin{enumerate}
	\item คอมพิวเตอร์ส่วนบุคคล (Personal Computer)
	\begin{enumerate}
		\item Device name: MacBook Air (M1, 2020)
		\item Processor: Apple M1 chip, 8-core CPU (4 performance cores,\\4 efficiency cores)
		\item Installed RAM: 8 GB unified memory
		\item System type: 64-bit operating system, ARM-based processor
	\end{enumerate}
\end{enumerate}

\par\endgroup
\clearpage
\clearpage
%================== จบ chapter1_1.tex ====================
}{}
	\IfFileExists{Chapter1_2.tex}{%==================== chapter1_2.tex ====================

\clearpage
\thispagestyle{plain}

\begingroup
% ฟอนต์เนื้อหา 16pt baseline ~19.2pt
\fontsize{16pt}{19.2pt}\selectfont
\justifying
\XeTeXlinebreakskip=0pt plus 1pt minus 0.5pt
\setlength{\parindent}{1.5cm}
\setlength{\parskip}{0pt}

% ---------- โปรแกรมที่ใช้ในการพัฒนา ----------
\section*{โปรแกรมที่ใช้ในการพัฒนา}
\addcontentsline{toc}{section}{โปรแกรมที่ใช้ในการพัฒนา}

% ตั้งค่าให้เหมือนสเปกเดิมทุกบท
\setlength{\LoneLabelSep}{0.5em}
\settowidth{\LoneLabelWidth}{9.}
\setlength{\LoneContentCol}{\dimexpr 1.5cm + \LoneLabelWidth + \LoneLabelSep\relax}

\setlength{\LtwoLabelSep}{0.5em}
\settowidth{\LtwoLabelWidth}{9.9.}
\setlength{\ExtraAlign}{-2.8em}

% ระดับ 1: อินเดนต์ 1.5 ซม.
\setlist[enumerate,1]{%
	label=\arabic*., align=left,
	leftmargin=1.5cm, labelindent=0pt,
	labelwidth=\LoneLabelWidth, labelsep=\LoneLabelSep,
	itemsep=0pt, topsep=0.5\baselineskip
}

% (เผื่อมีระดับ 2 ในหน้านี้ภายหลัง) จัดคอลัมน์เหมือนบทก่อน
\setlist[enumerate,2]{%
	label*=\arabic*., align=left,
	leftmargin=*,
	labelwidth=\LtwoLabelWidth, labelsep=\LtwoLabelSep,
	labelindent=\dimexpr \LoneContentCol + \ExtraAlign - \LtwoLabelWidth - \LtwoLabelSep\relax,
	itemsep=0pt, topsep=0pt
}

\begin{enumerate}
	\item Visual Studio Code
	\item React
	\item Supabase
	\item Axios
	\item Express
	\item React Testing Library
	\item Figma
\end{enumerate}

\par\endgroup
\clearpage

%================== จบ chapter1_2.tex ====================
}{}
	
	% บทที่ 2 (ใช้หน้าใน Chapter2.tex เป็นหน้าบท)
	\thispagestyle{plain}
	\IfFileExists{Chapter2.tex}{%==================== chapter2.tex ====================

\clearpage
\thispagestyle{empty}

\begingroup
% เนื้อหาบท: 16pt baseline ~19.2pt ตามสเปกเล่ม
\fontsize{16pt}{19.2pt}\selectfont
\justifying
\XeTeXlinebreakskip=0pt plus 1pt minus 0.5pt
\setlength{\parindent}{1.5cm}
\setlength{\parskip}{0pt}

% ---------- หัวบท + เขียนสารบัญบทก่อนหัวข้อย่อย ----------
\phantomsection
\addcontentsline{toc}{chapter}{บทที่ 2 เอกสารที่เกี่ยวข้อง}
\begin{center}
	{\bfseries\fontsize{18pt}{21.6pt}\selectfont บทที่ 2}
\end{center}

\vspace{\baselineskip}

% ---------- ชื่อบท ----------
\begin{center}
	{\bfseries\fontsize{18pt}{21.6pt}\selectfont เอกสารและงานวิจัยที่เกี่ยวข้อง}
\end{center}

\vspace{\baselineskip}

% ---------- หัวข้อใหญ่ (ชิดซ้าย, หนา 16pt) ----------
\section*{เกณฑ์การตัดสินการประกวดปลากัดครีบสั้นในประเทศไทย}
\addcontentsline{toc}{section}{เกณฑ์การตัดสินการประกวดปลากัดครีบสั้นในประเทศไทย}

% ---------- เนื้อหา (จัดกระจายแบบไทย, ย่อหน้าแรก 1.5 ซม.) ----------
\indent เกณฑ์การตัดสินการประกวดปลากัดครีบสั้นในประเทศไทย คณะกรรมการจะตัดสินความ
สวยงามและความสมบูรณ์ของปลากัดครีบสั้นตามเกณฑ์มาตรฐานตรงตามสายพันธุ์และประเภทของ
การประกวด ดังรายละเอียดต่อไปนี้

\vspace{\baselineskip}

% --- ตาราง: เกณฑ์การให้คะแนนปลากัดครีบสั้น ---
\begingroup
\renewcommand{\arraystretch}{1.2}
\setlength{\arrayrulewidth}{0.5pt} % เส้นคั่นแถว 0.5pt

\begin{table}[h]
	
	\caption{เกณฑ์การให้คะแนนปลากัดครีบสั้น}
	\captionsetup[]{}
	\centering
	\begin{tabularx}{\textwidth}{@{}>{\raggedright\arraybackslash}X
			>{\centering\arraybackslash}p{2.5cm}
			>{\centering\arraybackslash}p{3cm}@{}}
		\Xhline{1.5pt} % เส้นบน 1.5pt
		\bfseries ลักษณะ & \bfseries คะแนน & \bfseries คะแนนรวม \\
		\hline
		หัวและตา & 5 & 5 \\
		\hline
		ลำตัวและเกล็ด & 5 & 5 \\
		\hline
		ครีบหลัง (กระโดง) & 10 & 10 \\
		\hline
		ครีบหาง (หาง) & 15 & 15 \\
		\hline
		ครีบก้น (ชายน้ำ) & 10 & 10 \\
		\hline
		ครีบอื่น ๆ เช่น ครีบหู, ครีบอก (ตะเกียบ), แผ่นปิดเหงือก (เหงือก) & 5 & 5 \\
		\hline
		สี และลวดลาย & 20 & 20 \\
		\hline
		การทรงตัว และการว่ายน้ำ & 5 & 5 \\
		\hline
		การพองสู้ & 5 & 5 \\
		\hline
		ภาพรวม & 20 & 20 \\
		\Xhline{0.5pt}
		\bfseries คะแนนรวมทั้งสิ้น & \bfseries 100 & \bfseries 100 \\
		\Xhline{1.5pt} % เส้นล่าง 1.5pt
	\end{tabularx}
	\caption*{ที่มา: อรุณี รอดลอย, 2018, 128 55}
\end{table}
\endgroup

\newpage

\vspace{\baselineskip}

% --- ตาราง: เกณฑ์การให้คะแนนปลากัดครีบยาว (2 คอลัมน์) ---
\begingroup
\renewcommand{\arraystretch}{1.2}
\setlength{\arrayrulewidth}{0.5pt} % เส้นคั่นแถว 0.5pt

\begin{table}[h]
	\captionsetup{justification=raggedright, singlelinecheck=false,
		labelfont=bf, textfont=bf} % ชื่อตารางชิดซ้าย + ตัวหนา
	\caption{เกณฑ์การให้คะแนนปลากัดครีบยาว}
	\centering
	\begin{tabularx}{\textwidth}{@{}>{\raggedright\arraybackslash}X
			>{\centering\arraybackslash}p{3cm}@{}}
		\Xhline{1.5pt} % เส้นบน 1.5pt
		\bfseries ลักษณะ & \bfseries คะแนน \\
		\hline
		หัวและตา & 5 \\
		\hline
		ลำตัวและเกล็ด & 5 \\
		\hline
		ครีบหลัง (กระโดง) & 10 \\
		\hline
		ครีบหาง (หาง) & 15 \\
		\hline
		ครีบก้น (ชายน้ำ) & 10 \\
		\hline
		ครีบอื่น ๆ เช่น ครีบหู, ครีบอก (ตะเกียบ), แผ่นปิดเหงือก (เหงือก) & 5 \\
		\hline
		สี และลวดลาย & 20 \\
		\hline
		การทรงตัว และการว่ายน้ำ & 5 \\
		\hline
		การพองสู้ & 5 \\
		\hline
		ภาพรวม & 20 \\
		\Xhline{0.5pt}
		\bfseries คะแนนรวมทั้งสิ้น & \bfseries 100 \\
		\Xhline{1.5pt} % เส้นล่าง 1.5pt
	\end{tabularx}
	\caption*{ที่มา: อรุณี รอดลอย, 2018, 128 56}
\end{table}
\endgroup


\clearpage}{}
	\IfFileExists{Chapter2_1.tex}{\include{Chapter2_1}}{}
	\IfFileExists{Chapter2_2.tex}{%==================== chapter2_2.tex ====================

\clearpage
\thispagestyle{plain}

\begingroup
\fontsize{16pt}{19.2pt}\selectfont
\justifying
\XeTeXlinebreakskip=0pt plus 1pt minus 0.5pt
\setlength{\parindent}{1.5cm}
\setlength{\parskip}{0pt}

\section*{Convolutional Neural Network (CNN)}
\addcontentsline{toc}{section}{Convolutional Neural Network (CNN)}

\ThaiPara{\cite{lecun1998gradient}~โครงข่ายประสาทเทียมแบบคอนโวลูชัน (Convolutional Neural Network: CNN)
	ถูกออกแบบมาเพื่อจัดการข้อมูลที่มีโครงสร้างพื้นที่ เช่น รูปภาพ หัวใจคือชั้นคอนโวลูชันซึ่งใช้ฟิลเตอร์
	(kernel) เลื่อนไปบนภาพเพื่อสกัดคุณลักษณะ (เช่น ขอบและลวดลาย) ทำให้จำนวนพารามิเตอร์ลดลง
	และคำนวณได้มีประสิทธิภาพ นอกจากนี้ยังมีแนวคิด \emph{padding} เพื่อคงขนาดข้อมูล
	\emph{stride} เพื่อกำหนดก้าวการเลื่อนฟิลเตอร์ และชั้น \emph{pooling} (เช่น \emph{max pooling})
	เพื่อลดขนาดข้อมูล ก่อนส่งต่อไปยังชั้นเชื่อมต่อสมบูรณ์และ \emph{softmax} เพื่อจำแนกประเภท}

\vspace{\baselineskip}

% คำบรรยายรูปให้ justified เฉพาะรูปนี้
\begin{figure}[h]
	\centering
	\includegraphics[width=0.8\linewidth]{ml-blog19-lenet5}
	\caption{สถาปัตยกรรมพื้นฐานของ Convolutional Neural Network (CNN)}
\end{figure}



\section*{Image classification (การจำแนกภาพ)}
\addcontentsline{toc}{section}{Image classification (การจำแนกภาพ)}

% ---------- Image classification ----------
\noindent{\bfseries\fontsize{16pt}{19.2pt}\selectfont }\par

\ThaiPara{\cite{superannotate2023imageclassification}~การจำแนกภาพคือการกำหนดป้ายกำกับให้รูปภาพจากชุดป้ายที่กำหนดล่วงหน้า
	ปัจจุบันโมเดลเชิงลึก โดยเฉพาะ CNN สามารถเรียนรู้คุณลักษณะสำคัญจากพิกเซลดิบแบบปลายทางถึงปลายทาง
	(\emph{end-to-end}) และส่งออกค่าความน่าจะเป็นของแต่ละคลาสได้อย่างมีประสิทธิภาพ}

\vspace{\baselineskip}

\begin{figure}[h]
	\centering
	\includegraphics[width=0.8\linewidth]{png1}
	\caption{สถาปัตยกรรมพื้นฐานของการจำแนกภาพ}
\end{figure}

\par\endgroup
\clearpage
%================== จบ chapter2_1.tex ====================
}{}
	\IfFileExists{Chapter2_3.tex}{%==================== chapter2_3.tex ====================

\clearpage
\thispagestyle{plain}

\begingroup
\fontsize{16pt}{19.2pt}\selectfont
\justifying
\XeTeXlinebreakskip=0pt plus 1pt minus 0.5pt
\setlength{\parindent}{1.5cm}
\setlength{\parskip}{0pt}

% ---------- การออกแบบเว็บไซต์ที่รองรับการใช้งานบนทุกขนาดของหน้าจอ ----------
\section*{การออกแบบเว็บไซต์ที่รองรับการใช้งานบนทุกขนาดของหน้าจอ (Responsive web design)}
\addcontentsline{toc}{section}{การออกแบบเว็บไซต์ที่รองรับการใช้งานบนทุกขนาดของหน้าจอ (Responsive web design)}

\indent \cite{kinsta2024responsive} การออกแบบเว็บให้รองรับอุปกรณ์หลายชนิด (มือ ถือ แท็บเล็ต เดสก์ท็อป) ด้วยที่อยู่เว็บ
และโค้ดชุดเดียวโดยเลย์เอาต์ปรับตัวตามขนาดหน้าจออัตโนมัติช่วยคงประสบการณ์ใช้งานที่สม่ำเสมอ

\begin{figure}[h]
	\centering
	\includegraphics[width=0.8\linewidth]{responsive-adaptive-design-768x425}
	\caption{ภาพตัวอย่าง Responsive web design}
\end{figure}


% ---------- การออกแบบเว็บไซต์ที่รองรับการใช้งานบนทุกขนาดของหน้าจอ ----------

\section*{วงจรการพัฒนาซอฟต์แวร์}
\addcontentsline{toc}{section}{วงจรการพัฒนาซอฟต์แวร์}

\indent วงจรการพัฒนาซอฟต์แวร์ (Software Development Life Cycle: SDLC) มักแบ่งเป็น 6
ขั้นตอน ดังนี้

\begin{enumerate}
	\item \textbf{การวางแผน (Planning)} กำหนดเวลา ขอบเขต ทรัพยากร ผู้มีส่วนเกี่ยวข้อง และความเสี่ยง
	\item \textbf{เก็บรวบรวมและวิเคราะห์ความต้องการ (Requirements)} รวบรวมและทบทวนความต้องการของผู้ใช้/ระบบ
	\item \textbf{ออกแบบซอฟต์แวร์ (Design)} สถาปัตยกรรม ฐานข้อมูล UI/UX ความปลอดภัย เครือข่าย ฯลฯ
	\item \textbf{พัฒนา (Development)} พัฒนาแต่ละฟีเจอร์และรวมเป็นระบบตามดีไซน์
	\item \textbf{ทดสอบ (Testing)} สร้าง Test case ตรวจสอบตาม Requirement และใช้ Automated test ตามความเหมาะสม
	\item \textbf{บำรุงรักษา (Operations \& Maintenance)} เผยแพร่ แก้ไขบั๊ก เพิ่มฟีเจอร์และปรับปรุง คุณภาพอย่างต่อเนื่อง
\end{enumerate}}{}
	\IfFileExists{Chapter2_4.tex}{%==================== chapter2_4.tex ====================

\clearpage
\thispagestyle{plain}

\begingroup
\fontsize{16pt}{19.2pt}\selectfont
\justifying
\XeTeXlinebreakskip=0pt plus 1pt minus 0.5pt
\setlength{\parindent}{1.5cm}
\setlength{\parskip}{0pt}

% ---------- การพัฒนาเว็บแอปพลิเคชัน ----------
\noindent{\bfseries\fontsize{16pt}{19.2pt}\selectfont การพัฒนาเว็บแอปพลิเคชัน}\par

\ThaiPara{กระบวนการหลัก ได้แก่ การกำหนดแนวคิดและเป้าหมาย การวิเคราะห์ความต้องการ
	การออกแบบ UX/UI การทำต้นแบบ การพัฒนา (Front-end/Back-end) การทดสอบ
	(การใช้งาน ความปลอดภัย ประสิทธิภาพ) และการเผยแพร่ พร้อมดูแลปรับปรุงตามผลตอบรับของผู้ใช้}

\vspace{\baselineskip}

% ---------- งานวิจัยที่เกี่ยวข้อง ----------
\noindent{\bfseries\fontsize{16pt}{19.2pt}\selectfont งานวิจัยที่เกี่ยวข้อง}\par

\indent {ตัวอย่างประเด็นที่เกี่ยวข้อง เช่น 
	 งานวิจัยเกี่ยวกับการจำแนกภาพขวดแบบเซ็ตเปิดด้วยโครงข่ายประสาทเทียมแบบคอนโวลูชัน (Convolutional Neural Network หรือ CNN) โดย ศุภณัฐ จินตวัฒน์สกุล \cite{jintawatsakoon2019openset} มีวัตถุประสงค์หลักในการพัฒนาโมเดลที่สามารถจำแนกประเภทของขวดที่เปิดแล้วได้อย่างมีประสิทธิภาพ โดยการใช้เทคนิค Deep Learning เพื่อวิเคราะห์และประมวลผลภาพที่ถ่ายในสภาพแวดล้อมต่าง ๆ ในงานวิจัยนี้ ผู้วิจัยได้เสนอการใช้โครงข่ายประสาทเทียมแบบคอนโวลูชันซึ่งสามารถดึงคุณลักษณะสำคัญจากภาพ เช่น รูปร่าง สี และลวดลายของขวด โดยใช้ภาษาPython ร่วมกับไลบรารี TensorFlow เพื่อพัฒนาและฝึกสอนโมเดล การใช้ชุดข้อมูลที่มีภาพของขวดแบบเซ็ตเปิดที่ติดป้ายกำกับช่วยให้โมเดลสามารถเรียนรู้และปรับปรุงความแม่นยำในการจำแนกประเภทได้ ผลลัพธ์จากการทดลองแสดงให้เห็นว่าโมเดลที่พัฒนาขึ้นมีค่าความแม่นยำสูงในการจำแนกประเภทขวดเปิด ซึ่งสามารถนำไปประยุกต์ใช้ในหลายสาขา เช่น การควบคุมคุณภาพในอุตสาหกรรม หรือการจัดการขวดในระบบอัตโนมัติ ดังนั้นงานวิจัยนี้จึงมีความสำคัญในการพัฒนาเทคโนโลยี   การจำแนกภาพที่มีประสิทธิภาพและช่วยส่งเสริมการใช้งานในวงกว้าง.งานวิจัยเกี่ยวกับ "Responsive Portfolio Website Using React" \cite{muksikarat2016rwd} มีวัตถุประสงค์หลักในการพัฒนาเว็บไซต์พอร์ตโฟลิโอที่สามารถปรับตัวให้เข้ากับอุปกรณ์ต่าง ๆ ได้อย่างมีประสิทธิภาพ เช่นโทรศัพท์มือถือ แท็บเล็ต และคอมพิวเตอร์ โดยใช้เทคโนโลยี React ซึ่งเป็นไลบรารี Java Scriptที่นิยมในการสร้างส่วนประกอบที่มีความโต้ตอบสูง การเลือกใช้ React ช่วยให้การพัฒนาเว็บไซต์สามารถแบ่งแยกส่วนประกอบต่าง ๆ ได้ง่ายและทำให้โค้ดมีความเรียบร้อยและสามารถดูแลรักษาได้ง่ายขึ้น ในกระบวนการพัฒนา Siva Rama Lingham N และคณะ \cite{10425667} ได้ใช้ CSS และCSS Frameworks เช่น Bootstrap หรือ Tailwind CSS เพื่อสร้างเลย์เอาต์ที่ตอบสนองต่อการเปลี่ยนแปลงขนาดหน้าจอ โดยสามารถแสดงข้อมูลได้อย่างเหมาะสมและมีความน่าสนใจในทุกอุปกรณ์ นอกจากนี้ยังมีการใช้ JavaScript เพื่อเพิ่มการโต้ตอบระหว่างผู้ใช้กับเว็บไซต์ เช่น การจัดการสถานะของข้อมูล การตอบสนองต่อการกระทำของผู้ใช้ และการนำเสนอข้อมูลในรูปแบบที่น่าสนใจ ผลลัพธ์ที่ได้จากงานวิจัยนี้คือเว็บไซต์พอร์ตโฟลิโอที่ไม่เพียงแต่มีความสวยงามและใช้งานง่าย แต่ยังสามารถแสดงผลงานและข้อมูลของผู้ใช้ได้อย่างมีประสิทธิภาพในทุกแพลตฟอร์ม เว็บไซต์นี้ช่วยให้ผู้ใช้สามารถนำเสนอผลงานของตนได้อย่างมีสไตล์ และเป็นเครื่องมือที่มีประโยชน์ในการสร้างภาพลักษณ์ที่ดีในสายงานที่ตนสนใจ โดยรวมแล้วงานวิจัยนี้มีความสำคัญในการพัฒนาเทคโนโลยีเว็บที่ตอบสนองความต้องการของผู้ใช้ในยุคดิจิทัลที่มีการเปลี่ยนแปลงอย่างรวดเร็ว 
	 
	 \newpage 
	 
	 R.Archana and P.S. Eliahim Leevaraj \cite{8016501} งานวิจัยนี้สำรวจและประเมินโมเดลการเรียนรู้เชิง ลึก (Deep Learning) ในการประมวลผลภาพดิจิทัล เช่น การลดสัญญาณรบกวน การเพิ่มคุณภาพภาพ การ แบ่งส่วน การสกัดคุณลักษณะ และการจำแนกประเภท โดยเปรียบเทียบกับวิธีการประมวลผลภาพแบบดั้งเดิม โดยใช้ เทคนิคการประมวลผลภาพแบบดั้งเดิม เช่น การปรับความคมชัด การกรองสัญญาณรบกวน และการ แบ่งส่วนภาพ รวมถึงการใช้โมเดลการเรียนรู้เชิงลึก เช่น Convolutional Neural Networks (CNNs) และ Recurrent Neural Networks (RNNs) เพื่อเพิ่มประสิทธิภาพในการประมวลผลภาพ โดยมี 4 ขั้นตอน ได้แก่ 1.การเตรียมรูปภาพ เป็นการลดสัญญาณรบกวนและเพิ่มคุณภาพของรูป 2. การแบ่งส่วนภาพ เป็นการแยก ภาพออกเป็นส่วนๆ ตามลักษณะเฉพาะ 3.การสกัดคุณลักษณะ เป็นการดึงข้อมูลสำคัญจากภาพ 4.การจำแยก ประเภท เป็นการจัดหมวดหมู่ภาพตามเนื้อหา ผลลัพธ์ของงานวิจัยนี้แสดงให้เห็นว่าโมเดล DeepLabV3 และ U-Net มีความแม่นยำสูงกว่า FCN ในการแบ่งส่วนภาพจากมุมมองด้านบน ซึ่งมีความสำคัญในหลายๆ การใช้ งาน เช่น การเฝ้าระวัง การตรวจสอบฝูงชน และการโต้ตอบระหว่างมนุษย์กับคอมพิวเตอร์.}
	 
	

}{}
	
	% บทที่ 3 (ใช้หน้าใน Chapter3.tex เป็นหน้าบท)
	\thispagestyle{plain}
	\IfFileExists{Chapter3.tex}{%==================== chapter3.tex ====================

\clearpage
\thispagestyle{empty}

\begingroup
% เนื้อหาบท: 16pt baseline ~19.2pt ตามสเปกเล่ม
\fontsize{16pt}{19.2pt}\selectfont
\justifying
\XeTeXlinebreakskip=0pt plus 1pt minus 0.5pt
\setlength{\parindent}{1.5cm}
\setlength{\parskip}{0pt}

% ---------- หัวบท + เขียนสารบัญบทก่อนหัวข้อย่อย ----------
\phantomsection
\addcontentsline{toc}{chapter}{บทที่ 3 วิธีดำเนินงานวิจัย}
\begin{center}
	{\bfseries\fontsize{18pt}{21.6pt}\selectfont บทที่ 3}
\end{center}

\vspace{\baselineskip}

% ---------- ชื่อบท (วิธีดำเนินงานวิจัย) ----------
\begin{center}
	{\bfseries\fontsize{18pt}{21.6pt}\selectfont วิธีดำเนินงานวิจัย}
\end{center}

\vspace{\baselineskip}

% ---------- หัวข้อใหญ่ (ชิดซ้าย, หนา 16pt) ----------
\section*{การวิเคราะห์และการออกแบบระบบ}
\addcontentsline{toc}{section}{การวิเคราะห์และการออกแบบระบบ}

% ---------- เนื้อหา (จัดกระจายแบบไทย, ย่อหน้าแรก 1.5 ซม.) ----------
\indent ในการพัฒนาระบบเว็บแอปพลิเคชันศูนย์รวมการจัดประกวดปลากัดไทยนั้นจำเป็นต้องมีการ
ออกแบบระบบเพื่อชี้ให้เห็นโครงสร้างและหลักการทำงานของโครงงานนี้โดยผู้จัดทำได้ศึกษาค้นคว้า
และวิเคราะห์ความต้องการขิงผู้ใช้งานระบบจากนั้นได้นำรายละเอียดที่ได้จากการศึกษาและวิเคราะห์
นำมาออกแบบระบบซึ่งสามารถแบ่งออกเป็น

% ตั้งค่าให้เหมือนสเปกเดิมทุกบท
\setlength{\LoneLabelSep}{0.5em}
\settowidth{\LoneLabelWidth}{9.}
\setlength{\LoneContentCol}{\dimexpr 1.5cm + \LoneLabelWidth + \LoneLabelSep\relax}

\setlength{\LtwoLabelSep}{0.5em}
\settowidth{\LtwoLabelWidth}{9.9.}
\setlength{\ExtraAlign}{-2.8em}

% ระดับ 1: อินเดนต์ 1.5 ซม.
\setlist[enumerate,1]{%
	label=\arabic*., align=left,
	leftmargin=1.5cm, labelindent=0pt,
	labelwidth=\LoneLabelWidth, labelsep=\LoneLabelSep,
	itemsep=0pt, topsep=0.5\baselineskip
}

\begin{enumerate}
	\item Use Case Diagram
	\item Entity-Relation Diagram
	\item Class Diagram
	\item Sequence Diagram
	\item Activity Diagram
	\item Interface
\end{enumerate}

\clearpage}{}
	\IfFileExists{Chapter3_1.tex}{%==================== chapter3_1.tex ====================

\clearpage
\thispagestyle{plain}

\begingroup
\fontsize{16pt}{19.2pt}\selectfont
\justifying
\XeTeXlinebreakskip=0pt plus 1pt minus 0.5pt
\setlength{\parindent}{1.5cm}
\setlength{\parskip}{0pt}

% ---------- หัวข้อใหญ่ (ชิดซ้าย, หนา 16pt) ----------

\section*{Use Case Diagram}
\addcontentsline{toc}{section}{Use Case Diagram}

% ---------- เนื้อหา (จัดกระจายแบบไทย, ย่อหน้าแรก 1.5 ซม.) ----------
\indent Use Case Diagram ที่เป็นการจำลองการทำงานของผู้เข้าประกวดผู้เชี่ยวชาญและฝ่าย
ประชาสัมพันธ์ซึ่งจะเห็นได้ว่าระบบนี้ประกอบไปด้วย 27 Use Case Diagram คือ

\vspace{\baselineskip}

\begin{figure}[h]
	\centering
	\includegraphics[width=0.8\linewidth]{Pasted image}
	\caption{Use Case ระบบเว็บแอปพลิเคชันศูนย์รวมการจัดประกวดปลากัดไทย}
\end{figure}

% ตั้งค่าให้เหมือนสเปกเดิมทุกบท
\setlength{\LoneLabelSep}{0.5em}
\settowidth{\LoneLabelWidth}{9.}
\setlength{\LoneContentCol}{\dimexpr 1.5cm + \LoneLabelWidth + \LoneLabelSep\relax}

\setlength{\LtwoLabelSep}{0.5em}
\settowidth{\LtwoLabelWidth}{9.9.}
\setlength{\ExtraAlign}{-2.8em}

% ระดับ 1: อินเดนต์ 1.5 ซม.
\setlist[enumerate,1]{%
	label=\arabic*., align=left,
	leftmargin=1.5cm, labelindent=0pt,
	labelwidth=\LoneLabelWidth, labelsep=\LoneLabelSep,
	itemsep=0pt, topsep=0.5\baselineskip
}

\begin{enumerate}
	\item Use Case: เข้าสู่ระบบ
	\item Use Case: สมัครสมาชิก
	\item Use Case: ขอรับบริการประเมินคุณภาพปลากัดจากผู้เชี่ยวชาญ
	\item Use Case: อัปโหลดรูปภาพ/วิดีโอ
	\item Use Case: กรอกข้อมูลปลากัด
	\item Use Case: ดูประวัติผลการประเมินคะแนน
	\item Use Case: เปรียบเทียบผลคะแนนจากประวัติผลการประเมิน
	\item Use Case: ดูประวัติการแข่งขัน
	\item Use Case: ข่าวสารการประกวด
	\item Use Case: สมัครเข้าร่วมการประกวด
	\item Use Case: กรอกข้อมูลปลากัด
	\item Use Case: อัปโหลดรูปภาพ/วิดีโอ
	\item Use Case: รับการแจ้งเตือนจากเว็บแอปพลิเคชัน
	\item Use Case: จัดการบัญชีผู้ใช้งาน
	\item Use Case: สร้างอีเว้นท์การประกวด
	\item Use Case: เพิ่มรายรายละเอียดการจัดการประกวด
	\item Use Case: กำหนดกรรมการในการตัดสิน
	\item Use Case: ประกาศผลการแข่งขัน
	\item Use Case: ดูผลคะแนนการแข่งขันทั้งหมด
	\item Use Case: ดูประวัติการจัดประกวด
	\item Use Case: ให้บริการประเมินคุณภาพปลากัดทำการประเมินคะแนนให้กับผู้เลี้ยงปลากัด
	\item Use Case: เป็นกรรมการในการตัดสิน
	\item Use Case: ให้คะแนนในการแข่งขัน
	\item Use Case: จัดการข้อมูลผู้ใช้ เพิ่ม/ลบ/แก้ไข ข้อมูล
	\item Use Case: จัดการการแจ้งเตือน
	\item Use Case: ดูแลและจัดการข้อมูลอื่นๆ เช่นรูปภาพ วิดีโอ ผลคะแนนต่างๆ
	\item Use Case: ผู้บริหารจัดการข้อมูลที่เกิดขึ้นในระบบ เข้าสู่ระบบ
\end{enumerate}

\vspace{\baselineskip}

% ================== Use Case: Login ==================
\begin{table}[h]
	\caption{Use Case สำหรับการเข้าสู่ระบบ}
	{\tablefont
		\setlength{\tabcolsep}{6pt}%
		\begin{tabularx}{\linewidth}{@{} >{\justifying\arraybackslash}X >{\raggedleft\arraybackslash}p{4.2cm} @{}}
			\Xhline{1.5pt}
			\textbf{Use Case Title:}\enspace เข้าสู่ระบบ & \UseCaseID[uc:login] \\
			\Xhline{0.5pt}
			\textbf{Primary Actor:}\enspace ผู้เลี้ยงปลากัด, ผู้จัดการประกวด, ผู้เชี่ยวชาญด้านปลากัด, ผู้บริหารจัดการข้อมูลที่เกิดขึ้นในระบบ & \\
			\Xhline{0.5pt}
			\textbf{Stakeholder Actor:}\enspace - & \\
			\Xhline{0.5pt}
			\textbf{Main Flow:}\enspace สามารถกรอก username และ password เข้าสู่ระบบใช้งานได้เลยกรณีมีบัญชีผู้ใช้งานอยู่แล้ว & \\
			\Xhline{0.5pt}
			\textbf{Exception Flow ที่ 1:}\enspace กรณีที่ผู้ใช้งานยังไม่มีบัญชีเข้าใช้งาน ระบบจะแจ้งเตือนให้ผู้ใช้สมัครสมาชิกก่อนเข้าใช้งานเว็บแอพพลิเคชัน & \\
			\Xhline{1.5pt}
		\end{tabularx}
	}
\end{table}
% =====================================================



\clearpage}{}
	\IfFileExists{Chapter3_2.tex}{%==================== chapter3_2.tex ====================

\clearpage
\thispagestyle{plain}

\begingroup
\fontsize{16pt}{19.2pt}\selectfont
\justifying
\XeTeXlinebreakskip=0pt plus 1pt minus 0.5pt
\setlength{\parindent}{1.5cm}
\setlength{\parskip}{0pt}

% ================== Use Case: Register ==================
\begin{table}[h]
	\caption{Use Case สมัครสมาชิก}
	{\tablefont
		\setlength{\tabcolsep}{6pt}%
		\begin{tabularx}{\linewidth}{@{} >{\justifying\arraybackslash}X >{\raggedleft\arraybackslash}p{4.2cm} @{}}
			\Xhline{1.5pt}
			\textbf{Use Case Title:}\enspace สมัครสมาชิก & \UseCaseID[uc:register] \\
			\Xhline{0.5pt}
			\textbf{Primary Actor:}\enspace ผู้เลี้ยงปลากัด, ผู้จัดการประกวด, ผู้เชี่ยวชาญด้านปลากัด & \\
			\Xhline{0.5pt}
			\textbf{Stakeholder Actor:}\enspace - & \\
			\Xhline{0.5pt}
			\textbf{Main Flow:}\enspace ผู้ใช้ที่ยังไม่มีบัญชีเข้าใช้งาน สามารถสมัครสมาชิกเพื่อเข้าใช้งานระบบได้ & \\
			\Xhline{0.5pt}
			\textbf{Exception Flow ที่ 1:}\enspace กรณีผู้ใช้กรอกข้อมูลไม่ครบถ้วนหรือไม่ตรงตามเงื่อนไข ระบบจะทำการแจ้งข้อความเตือน & \\
			\Xhline{0.5pt}
			\textbf{Exception Flow ที่ 2:}\enspace กรณีที่ผู้ใช้มีบัญชีอยู่แล้ว ระบบจะทำการแจ้งเตือนให้ทราบ & \\
			\Xhline{1.5pt}
		\end{tabularx}
	}
\end{table}
% =====================================================

% ================== Use Case ขอรับบริการประเมิณคุณภาพปลากัดจากผู้เชี่ยวชาญ ==================
\begin{table}[h]
	\caption{Use Case ขอรับบริการประเมิณคุณภาพปลากัดจากผู้เชี่ยวชาญ}
	{\tablefont
		\setlength{\tabcolsep}{6pt}%
		\begin{tabularx}{\linewidth}{@{} >{\justifying\arraybackslash}X >{\raggedleft\arraybackslash}p{4.2cm} @{}}
			\Xhline{1.5pt}
			\textbf{Use Case Title:}\enspace ขอรับบริการประเมิณคุณภาพปลากัดจากผู้เชี่ยวชาญ & \UseCaseID[uc:register] \\
			\Xhline{0.5pt}
			\textbf{Primary Actor:}\enspace ผู้เลี้ยงปลากัด & \\
			\Xhline{0.5pt}
			\textbf{Stakeholder Actor:}\enspace - & \\
			\Xhline{0.5pt}
			\textbf{Main Flow:}\enspace ผู้ใช้สามารถขอรับบริการประเมินคุณภาพปลากัดปลากัดเพื่อประประเมินคุณภาพปลากัดของผู้เลี้ยงได้
			จากผู้เชี่ยวชาญ & \\
			\Xhline{0.5pt}
			\textbf{Exception Flow ที่ 1:}\enspace ผู้ใช้ต้องทำการอัปโหลดข้อมูลเกี่ยวกับปลากัดเข้ามาก่อนถึงจะสามารถส่งไปให้ผู้เชี่ยวประเมินคุณภาพให้ได้ & \\
			\Xhline{1.5pt}
		\end{tabularx}
	}
\end{table}
% =====================================================

% ================== Use Case อัปโหลดรูปภาพ/วิดีโอ ==================
\begin{table}[h]
	\caption{Use Case อัปโหลดรูปภาพ/วิดีโอ}
	{\tablefont
		\setlength{\tabcolsep}{6pt}%
		\begin{tabularx}{\linewidth}{@{} >{\justifying\arraybackslash}X >{\raggedleft\arraybackslash}p{4.2cm} @{}}
			\Xhline{1.5pt}
			\textbf{Use Case Title:}\enspace อัปโหลดรูปภาพ/วิดีโอ & \UseCaseID[uc:register] \\
			\Xhline{0.5pt}
			\textbf{Primary Actor:}\enspace ผู้เลี้ยงปลากัด & \\
			\Xhline{0.5pt}
			\textbf{Stakeholder Actor:}\enspace - & \\
			\Xhline{0.5pt}
			\textbf{Main Flow:}\enspace ทำการเพิ่มรูปภาพ/วิดีโอ ลงบนเว็บแอพพลิเคชันเพื่อทำการประเมินผลคะแนนโดยวิธีการต่างๆ & \\
			\Xhline{1.5pt}
		\end{tabularx}
	}
\end{table}
% =====================================================

\clearpage}{}
	\IfFileExists{Chapter3_3.tex}{%==================== chapter3_3.tex ====================

\clearpage
\thispagestyle{plain}

\begingroup
\fontsize{16pt}{19.2pt}\selectfont
\justifying
\XeTeXlinebreakskip=0pt plus 1pt minus 0.5pt
\setlength{\parindent}{1.5cm}
\setlength{\parskip}{0pt}

% ================== Use Case กรอกข้อมูลปลากัด ==================
\begin{table}[h]
	\caption{Use Case กรอกข้อมูลปลากัด}
	{\tablefont
		\setlength{\tabcolsep}{6pt}%
		\begin{tabularx}{\linewidth}{@{} >{\justifying\arraybackslash}X >{\raggedleft\arraybackslash}p{4.2cm} @{}}
			\Xhline{1.5pt}
			\textbf{Use Case Title:}\enspace กรอกข้อมูลปลากัด & \UseCaseID[uc:register] \\
			\Xhline{0.5pt}
			\textbf{Primary Actor:}\enspace ผู้เลี้ยงปลากัด & \\
			\Xhline{0.5pt}
			\textbf{Stakeholder Actor:}\enspace - & \\
			\Xhline{0.5pt}
			\textbf{Main Flow:}\enspace ผู้ใช้ทำการกรอกรายละเอียดข้อมูลปลากัด เช่น ชื่อปลากัดอายุปลากัดประเภทปลากัด & \\
			\Xhline{1.5pt}
		\end{tabularx}
	}
\end{table}
% =====================================================

% ================== Use Case ดูประวัติผลการประเมินคะแนน ==================
\begin{table}[h]
	\caption{Use Case ดูประวัติผลการประเมินคะแนน}
	{\tablefont
		\setlength{\tabcolsep}{6pt}%
		\begin{tabularx}{\linewidth}{@{} >{\justifying\arraybackslash}X >{\raggedleft\arraybackslash}p{4.2cm} @{}}
			\Xhline{1.5pt}
			\textbf{Use Case Title:}\enspace ดูประวัติผลการประเมินคะแนน & \UseCaseID[uc:register] \\
			\Xhline{0.5pt}
			\textbf{Primary Actor:}\enspace ผู้เลี้ยงปลากัด & \\
			\Xhline{0.5pt}
			\textbf{Stakeholder Actor:}\enspace - & \\
			\Xhline{0.5pt}
			\textbf{Main Flow:}\enspace ผู้เลี้ยงปลากัดสามารถเข้าดูประวัคิการการประเมินได้ทั้งหมดที่ตนเองเคยได้ทำการประเมินผลไว้ & \\
			\Xhline{0.5pt}
			\textbf{Exception Flow ที่ 1:}\enspace จัดการการแจ้งเตือนและข้อมูลสำคัญของระบบ , ผู้ใช้สามารถทำการเปรียบทียบผลคะแนนจากปลาตัวอื่นหรือว่าปลาตัวเดียวกันที่เคยประเมินไว้ก่อนหน้านี้ & \\
			\Xhline{1.5pt}
		\end{tabularx}
	}
\end{table}
% =====================================================

% ================== Use Case เปรียบเทียบผลคะแนนจากประวัติการประเมิน ==================
\begin{table}[h]
	\caption{Use Case เปรียบเทียบผลคะแนนจากประวัติการประเมิน}
	{\tablefont
		\setlength{\tabcolsep}{6pt}%
		\begin{tabularx}{\linewidth}{@{} >{\justifying\arraybackslash}X >{\raggedleft\arraybackslash}p{4.2cm} @{}}
			\Xhline{1.5pt}
			\textbf{Use Case Title:}\enspace เปรียบเทียบผลคะแนนจากประวัติการประเมิน & \UseCaseID[uc:register] \\
			\Xhline{0.5pt}
			\textbf{Primary Actor:}\enspace ผู้เลี้ยงปลากัด & \\
			\Xhline{0.5pt}
			\textbf{Stakeholder Actor:}\enspace - & \\
			\Xhline{0.5pt}
			\textbf{Main Flow:}\enspace สามารถเปรียบเทียบผลคะแนนกับปลาตัวอื่น ๆ ได้หรือว่าจะเป็นปลาตัวเดียวกันก็ได้เช่นกัน
			เพื่อให้เห็นการวิวัฒนาการของปลา หรือ ความแตกตางระหว่างปลา 2 ตัว & \\
			\Xhline{1.5pt}
		\end{tabularx}
	}
\end{table}
% =====================================================
\clearpage
}{}
	\IfFileExists{Chapter3_4.tex}{%==================== chapter3_4.tex ====================

\clearpage
\thispagestyle{plain}

\begingroup
\fontsize{16pt}{19.2pt}\selectfont
\justifying
\XeTeXlinebreakskip=0pt plus 1pt minus 0.5pt
\setlength{\parindent}{1.5cm}
\setlength{\parskip}{0pt}

% ================== Use Case ดูประวัติการแข่งขัน ==================
\begin{table}[h]
	\caption{Use Case ดูประวัติการแข่งขัน}
	{\tablefont
		\setlength{\tabcolsep}{6pt}%
		\begin{tabularx}{\linewidth}{@{} >{\justifying\arraybackslash}X >{\raggedleft\arraybackslash}p{4.2cm} @{}}
			\Xhline{1.5pt}
			\textbf{Use Case Title:}\enspace ดูประวัติการแข่งขัน & \UseCaseID[uc:register] \\
			\Xhline{0.5pt}
			\textbf{Primary Actor:}\enspace ผู้เลี้ยงปลากัด & \\
			\Xhline{0.5pt}
			\textbf{Stakeholder Actor:}\enspace - & \\
			\Xhline{0.5pt}
			\textbf{Main Flow:}\enspace สามารถดูประวัติการเข้าร่วมการแข่งขันได้ทั้งหมด เพื่อดูว่าผู้เลี้ยงทำการแข่งขันไปทั้งหมดกี่ครั้งแล้ว & \\
			\Xhline{1.5pt}
		\end{tabularx}
	}
\end{table}
% =====================================================

% ================== Use Case ข่าวสารการประกวด ==================
\begin{table}[h]
	\caption{Use Case ข่าวสารการประกวด}
	{\tablefont
		\setlength{\tabcolsep}{6pt}%
		\begin{tabularx}{\linewidth}{@{} >{\justifying\arraybackslash}X >{\raggedleft\arraybackslash}p{4.2cm} @{}}
			\Xhline{1.5pt}
			\textbf{Use Case Title:}\enspace ข่าวสารการประกวด & \UseCaseID[uc:register] \\
			\Xhline{0.5pt}
			\textbf{Primary Actor:}\enspace ผู้เลี้ยงปลากัด, ผู้จัดการประกวด, ผู้เชี่ยวชาญด้านปลากัด & \\
			\Xhline{0.5pt}
			\textbf{Stakeholder Actor:}\enspace - & \\
			\Xhline{0.5pt}
			\textbf{Main Flow:}\enspace ข้าชมข่าวสารการประกวดต่างๆในเว็บ แต่ละข่าวสารอาจมีการรับสมัครเข้าร่วมการแข่งขัน & \\
			\Xhline{0.5pt}
			\textbf{Exception Flow ที่ 1:}\enspace สามารถกดสมัครเข้าร่วมการประกวดปลากัดได้ & \\
			\Xhline{0.5pt}
			\textbf{Exception Flow ที่ 2:}\enspace การที่จะสมัครเข้าร่วมกันแข่งขันได้นั้น จะต้องมีอีเว้นท์การประกวดก่อนถึงจะเข้าร่วมการแข่งได้ & \\
			\Xhline{1.5pt}
		\end{tabularx}
	}
\end{table}
% =====================================================

% ================== Use Case สมัครเข้าร่วมการประกวด ==================
\begin{table}[h]
	\caption{Use Case สมัครเข้าร่วมการประกวด}
	{\tablefont
		\setlength{\tabcolsep}{6pt}%
		\begin{tabularx}{\linewidth}{@{} >{\justifying\arraybackslash}X >{\raggedleft\arraybackslash}p{4.2cm} @{}}
			\Xhline{1.5pt}
			\textbf{Use Case Title:}\enspace สมัครเข้าร่วมการประกวด & \UseCaseID[uc:register] \\
			\Xhline{0.5pt}
			\textbf{Primary Actor:}\enspace ผู้เลี้ยงปลากัด & \\
			\Xhline{0.5pt}
			\textbf{Stakeholder Actor:}\enspace - & \\
			\Xhline{0.5pt}
			\textbf{Main Flow:}\enspace สามารถกดสมัครเข้าร่วมการประกวดปลากัด ในกิจกรรมนั้นๆได้ & \\
			\Xhline{0.5pt}
			\textbf{Exception Flow ที่ 1:}\enspace สามารถกดสมัครเข้าร่วมการประกวดปลากัดได้ & \\
			\Xhline{0.5pt}
			\textbf{Exception Flow ที่ 2:}\enspace การที่จะสมัครเข้าร่วมกันแข่งขันได้นั้น จะต้องมีอีเว้นท์การประกวดก่อนถึงจะเข้าร่วมการแข่ง
			ได้ & \\
			\Xhline{1.5pt}
		\end{tabularx}
	}
\end{table}
% =====================================================
\clearpage}{}
	\IfFileExists{Chapter3_5.tex}{%==================== chapter3_6.tex ====================

\clearpage
\thispagestyle{plain}

\begingroup
\fontsize{16pt}{19.2pt}\selectfont
\justifying
\XeTeXlinebreakskip=0pt plus 1pt minus 0.5pt
\setlength{\parindent}{1.5cm}
\setlength{\parskip}{0pt}

% ================== Use Case กรอกข้อมูลปลากัด ==================
\begin{table}[h]
	\caption{Use Case กรอกข้อมูลปลากัด}
	{\tablefont
		\setlength{\tabcolsep}{6pt}%
		\begin{tabularx}{\linewidth}{@{} >{\justifying\arraybackslash}X >{\raggedleft\arraybackslash}p{4.2cm} @{}}
			\Xhline{1.5pt}
			\textbf{Use Case Title:}\enspace กรอกข้อมูลปลากัด & \UseCaseID[uc:register] \\
			\Xhline{0.5pt}
			\textbf{Primary Actor:}\enspace ผู้เลี้ยงปลากัด & \\
			\Xhline{0.5pt}
			\textbf{Stakeholder Actor:}\enspace - & \\
			\Xhline{0.5pt}
			\textbf{Main Flow:}\enspace กรอกรายละเอียดข้อมูลปลากัดเพื่อสมัครเข้าร่วมการแข่งขันปลากัด & \\
			\Xhline{1.5pt}
		\end{tabularx}
	}
\end{table}
% =====================================================

% ================== Use Case อัปโหลดรูปภาพ/วิดีโอ ==================
\begin{table}[h]
	\caption{Use Case อัปโหลดรูปภาพ/วิดีโอ}
	{\tablefont
		\setlength{\tabcolsep}{6pt}%
		\begin{tabularx}{\linewidth}{@{} >{\justifying\arraybackslash}X >{\raggedleft\arraybackslash}p{4.2cm} @{}}
			\Xhline{1.5pt}
			\textbf{Use Case Title:}\enspace อัปโหลดรูปภาพ/วิดีโอ & \UseCaseID[uc:register] \\
			\Xhline{0.5pt}
			\textbf{Primary Actor:}\enspace ผู้เลี้ยงปลากัด & \\
			\Xhline{0.5pt}
			\textbf{Stakeholder Actor:}\enspace - & \\
			\Xhline{0.5pt}
			\textbf{Main Flow:}\enspace ทำการอัปโหลดรูปภาพ/วิดีโอ ของปลากัดเพื่อสมัคนเข้าประกวด & \\
			\Xhline{1.5pt}
		\end{tabularx}
	}
\end{table}
% =====================================================

% ================== Use Case รับการแจ้งเตือนจากระบบ ==================
\begin{table}[h]
	\caption{Use Case รับการแจ้งเตือนจากระบบ}
	{\tablefont
		\setlength{\tabcolsep}{6pt}%
		\begin{tabularx}{\linewidth}{@{} >{\justifying\arraybackslash}X >{\raggedleft\arraybackslash}p{4.2cm} @{}}
			\Xhline{1.5pt}
			\textbf{Use Case Title:}\enspace รับการแจ้งเตือนจากระบบ & \UseCaseID[uc:register] \\
			\Xhline{0.5pt}
			\textbf{Primary Actor:}\enspace ผู้เลี้ยงปลากัด, ผู้จัดการประกวด, ผู้เชี่ยวชาญด้านปลากัด & \\
			\Xhline{0.5pt}
			\textbf{Stakeholder Actor:}\enspace - & \\
			\Xhline{0.5pt}
			\textbf{Main Flow:}\enspace การแจ้งเตือนการแข่งขัน การแจ้งเตือนเกี่ยวกับข่าวสารการแข่งขันต่าง ๆ & \\
			\Xhline{1.5pt}
		\end{tabularx}
	}
\end{table}
% =====================================================

\clearpage}{}
	\IfFileExists{Chapter3_6.tex}{%==================== chapter3_2.tex ====================

\clearpage
\thispagestyle{plain}

\begingroup
\fontsize{16pt}{19.2pt}\selectfont
\justifying
\XeTeXlinebreakskip=0pt plus 1pt minus 0.5pt
\setlength{\parindent}{1.5cm}
\setlength{\parskip}{0pt}

% ================== Use Case จัดการบัญชีผู้ใช้งาน ==================
\begin{table}[h]
	\caption{Use Case จัดการบัญชีผู้ใช้งาน}
	{\tablefont
		\setlength{\tabcolsep}{6pt}%
		\begin{tabularx}{\linewidth}{@{} >{\justifying\arraybackslash}X >{\raggedleft\arraybackslash}p{4.2cm} @{}}
			\Xhline{1.5pt}
			\textbf{Use Case Title:}\enspace จัดการบัญชีผู้ใช้งาน & \UseCaseID[uc:register] \\
			\Xhline{0.5pt}
			\textbf{Primary Actor:}\enspace ผู้เลี้ยงปลากัด, ผู้จัดการประกวด, ผู้เชี่ยวชาญด้านปลากัด & \\
			\Xhline{0.5pt}
			\textbf{Stakeholder Actor:}\enspace - & \\
			\Xhline{0.5pt}
			\textbf{Main Flow:}\enspace สามารถทำการจัดการบัญชี ผู้ ใช้ งานของตนเองได้ เช่น แก้ไขรูป โปรไฟล์ แก้ไขชื่อ-สกุล แก้ ไข
			ยูสเซอร์เนม แก้ไขพาสเวิร์ด & \\
			\Xhline{1.5pt}
		\end{tabularx}
	}
\end{table}
% =====================================================

% ================== Use Case สร้างอีเว้นท์การประกวด ==================
\begin{table}[h]
	\caption{Use Case สร้างอีเว้นท์การประกวด}
	{\tablefont
		\setlength{\tabcolsep}{6pt}%
		\begin{tabularx}{\linewidth}{@{} >{\justifying\arraybackslash}X >{\raggedleft\arraybackslash}p{4.2cm} @{}}
			\Xhline{1.5pt}
			\textbf{Use Case Title:}\enspace สร้างอีเว้นท์การประกวด & \UseCaseID[uc:register] \\
			\Xhline{0.5pt}
			\textbf{Primary Actor:}\enspace ผู้จัดการประกวด & \\
			\Xhline{0.5pt}
			\textbf{Stakeholder Actor:}\enspace - & \\
			\Xhline{0.5pt}
			\textbf{Main Flow:}\enspace ทำการสร้างอีเว้นท์การประกวดเพื่อจัดการแข่งขันการประกวดปลากัด
			ยูสเซอร์เนม แก้ไขพาสเวิร์ด & \\
			\Xhline{0.5pt}
			\textbf{Exception Flow ที่ 1:}\enspace ผู้จัดการประกวดปลากัดจะต้องทำการเพิ่มรายละเอียดข้อมูลต่างๆที่เกี่ยวข้องกับการแข่งขัน & \\
			\Xhline{1.5pt}
		\end{tabularx}
	}
\end{table}
% =====================================================

% ================== Use Case เพิ่มรายละเอียดการจัดการประกวด ==================
\begin{table}[h]
	\caption{Use Case เพิ่มรายละเอียดการจัดการประกวด}
	{\tablefont
		\setlength{\tabcolsep}{6pt}%
		\begin{tabularx}{\linewidth}{@{} >{\justifying\arraybackslash}X >{\raggedleft\arraybackslash}p{4.2cm} @{}}
			\Xhline{1.5pt}
			\textbf{Use Case Title:}\enspace เพิ่มรายละเอียดการจัดการประกวด & \UseCaseID[uc:register] \\
			\Xhline{0.5pt}
			\textbf{Primary Actor:}\enspace ผู้จัดการประกวด & \\
			\Xhline{0.5pt}
			\textbf{Stakeholder Actor:}\enspace - & \\
			\Xhline{0.5pt}
			\textbf{Main Flow:}\enspace พิ่ มรายละเอียดข้อมูล เกี่ยวกับ การประกวดต่างๆ เพื่อ ให้ ผู้ เลี้ยงทราบถึง รายละเอียดขอมูล
			ต่างๆในการแข่งขันครั้งนั้น & \\
			\Xhline{0.5pt}
			\textbf{Exception Flow ที่ 1:}\enspace ผู้จัดการประกวดปลากัดจะต้องทำการเพิ่มรายละเอียดข้อมูลต่างๆที่เกี่ยวข้องกับการแข่งขัน & \\
			\Xhline{1.5pt}
		\end{tabularx}
	}
\end{table}
% =====================================================

\clearpage}{}
	\IfFileExists{Chapter3_7.tex}{%==================== chapter3_7.tex ====================

\clearpage
\thispagestyle{plain}

\begingroup
\fontsize{16pt}{19.2pt}\selectfont
\justifying
\XeTeXlinebreakskip=0pt plus 1pt minus 0.5pt
\setlength{\parindent}{1.5cm}
\setlength{\parskip}{0pt}

% ================== Use Case กำหนดกรรมการในการตัดสิน ==================
\begin{table}[h]
	\caption{Use Case กำหนดกรรมการในการตัดสิน}
	{\tablefont
		\setlength{\tabcolsep}{6pt}%
		\begin{tabularx}{\linewidth}{@{} >{\justifying\arraybackslash}X >{\raggedleft\arraybackslash}p{4.2cm} @{}}
			\Xhline{1.5pt}
			\textbf{Use Case Title:}\enspace กำหนดกรรมการในการตัดสิน & \UseCaseID[uc:register] \\
			\Xhline{0.5pt}
			\textbf{Primary Actor:}\enspace ผู้จัดการประกวด & \\
			\Xhline{0.5pt}
			\textbf{Stakeholder Actor:}\enspace - & \\
			\Xhline{0.5pt}
			\textbf{Main Flow:}\enspace กำหนดผู้เชี่ยวชาญเพื่อให้มาเป็นกรรมการในการตัดสินในแต่ ครั้งและกำหนด จำนวนกรรมการว่ากรรมการในการตัดสินการแข่งขันนี้ทั้งหมดกี่คน & \\
			\Xhline{1.5pt}
		\end{tabularx}
	}
\end{table}
% =====================================================

% ================== Use Case ประกาศผลการแข่งขัน ==================
\begin{table}[h]
	\caption{Use Case ประกาศผลการแข่งขัน}
	{\tablefont
		\setlength{\tabcolsep}{6pt}%
		\begin{tabularx}{\linewidth}{@{} >{\justifying\arraybackslash}X >{\raggedleft\arraybackslash}p{4.2cm} @{}}
			\Xhline{1.5pt}
			\textbf{Use Case Title:}\enspace ประกาศผลการแข่งขัน & \UseCaseID[uc:register] \\
			\Xhline{0.5pt}
			\textbf{Primary Actor:}\enspace ผู้จัดการประกวด & \\
			\Xhline{0.5pt}
			\textbf{Stakeholder Actor:}\enspace - & \\
			\Xhline{0.5pt}
			\textbf{Main Flow:}\enspace ทำการประกาศผลคะแนนการแข่งขันตามที่กรรมการได้ให้คะแนนไว้ & \\
			\Xhline{0.5pt}
			\textbf{Exception Flow ที่ 1:}\enspace กรรมการต้องทำการให้คะแนนการประเมินก่อนถึงจะสามารถประกาศผลการแข่งได้ & \\
			\Xhline{1.5pt}
		\end{tabularx}
	}
\end{table}
% =====================================================

% ================== Use Case ดูผลคะแนนการแข่งขัน ==================
\begin{table}[h]
	\caption{Use Case ดูผลคะแนนการแข่งขัน}
	{\tablefont
		\setlength{\tabcolsep}{6pt}%
		\begin{tabularx}{\linewidth}{@{} >{\justifying\arraybackslash}X >{\raggedleft\arraybackslash}p{4.2cm} @{}}
			\Xhline{1.5pt}
			\textbf{Use Case Title:}\enspace ดูผลคะแนนการแข่งขัน & \UseCaseID[uc:register] \\
			\Xhline{0.5pt}
			\textbf{Primary Actor:}\enspace ผู้จัดการประกวด & \\
			\Xhline{0.5pt}
			\textbf{Stakeholder Actor:}\enspace - & \\
			\Xhline{0.5pt}
			\textbf{Main Flow:}\enspace สามารถดูผลคะแนนที่กรรมการทำการให้คะแนน & \\
			\Xhline{0.5pt}
			\textbf{Exception Flow ที่ 1:}\enspace ผู้จัดการประกวดต้องทราบผลคะแนนที่กรรมให้คะแนนได้ก่อนถึงจะทำการประกาศผู้ทื่ชนะ
			ในการแข่งขันนั้นได้ & \\
			\Xhline{1.5pt}
		\end{tabularx}
	}
\end{table}
% =====================================================

\clearpage}{}
	\IfFileExists{Chapter3_8.tex}{%==================== chapter3_8.tex ====================

\clearpage
\thispagestyle{plain}

\begingroup
\fontsize{16pt}{19.2pt}\selectfont
\justifying
\XeTeXlinebreakskip=0pt plus 1pt minus 0.5pt
\setlength{\parindent}{1.5cm}
\setlength{\parskip}{0pt}

% ================== Use Case ดูประวัติการจัดการประกวด ==================
\begin{table}[h]
	\caption{Use Case ดูประวัติการจัดการประกวด}
	{\tablefont
		\setlength{\tabcolsep}{6pt}%
		\begin{tabularx}{\linewidth}{@{} >{\justifying\arraybackslash}X >{\raggedleft\arraybackslash}p{4.2cm} @{}}
			\Xhline{1.5pt}
			\textbf{Use Case Title:}\enspace ดูประวัติการจัดประกวด & \UseCaseID[uc:register] \\
			\Xhline{0.5pt}
			\textbf{Primary Actor:}\enspace ผู้จัดการประกวด & \\
			\Xhline{0.5pt}
			\textbf{Stakeholder Actor:}\enspace - & \\
			\Xhline{0.5pt}
			\textbf{Main Flow:}\enspace สามารถดูประวัติการจัดประกวดทั้งหมดได้ & \\
			\Xhline{1.5pt}
		\end{tabularx}
	}
\end{table}
% =====================================================

% ================== Use Case ให้บริการประเมินคุณภาพปลากัดทำการประเมินคะแนนให้กับผู้เลี้ยงปลากัด ==================
\begin{table}[h]
	\caption{Use Case ให้บริการประเมินคุณภาพปลากัดทำการประเมินคะแนนให้กับ
		ผู้เลี้ยงปลากัด}
	{\tablefont
		\setlength{\tabcolsep}{6pt}%
		\begin{tabularx}{\linewidth}{@{} >{\justifying\arraybackslash}X >{\raggedleft\arraybackslash}p{4.2cm} @{}}
			\Xhline{1.5pt}
			\textbf{Use Case Title:}\enspace ให้บริการประเมินคุณภาพปลากัดทำการประเมินคะแนนให้กับ & \UseCaseID[uc:register] \\
			\Xhline{0.5pt}
			\textbf{Primary Actor:}\enspace ผู้เชี่ยวชาญด้านปลากัด & \\
			\Xhline{0.5pt}
			\textbf{Stakeholder Actor:}\enspace - & \\
			\Xhline{0.5pt}
			\textbf{Main Flow:}\enspace สามารถทำการให้ คะแนนการประเมินแก่ผู้เลี้ยงปลากัด และส่งผลคะแนนกลับไปหาผู้เลี้ยงปลากัดเพื่อให้ผู้เลี้ยงได้ทราบผลคะแนน & \\
			\Xhline{1.5pt}
		\end{tabularx}
	}
\end{table}
% =====================================================

% ================== Use Case เป็นกรรมการในการตัดสิน ==================
\begin{table}[h]
	\caption{Use Case เป็นกรรมการในการตัดสิน}
	{\tablefont
		\setlength{\tabcolsep}{6pt}%
		\begin{tabularx}{\linewidth}{@{} >{\justifying\arraybackslash}X >{\raggedleft\arraybackslash}p{4.2cm} @{}}
			\Xhline{1.5pt}
			\textbf{Use Case Title:}\enspace เป็นกรรมการในการตัดสิน & \UseCaseID[uc:register] \\
			\Xhline{0.5pt}
			\textbf{Primary Actor:}\enspace ผู้เชี่ยวชาญด้านปลากัด & \\
			\Xhline{0.5pt}
			\textbf{Stakeholder Actor:}\enspace - & \\
			\Xhline{0.5pt}
			\textbf{Main Flow:}\enspace ทำหน้าที่เป็นกรรมการในการตัดสิน & \\
			\Xhline{0.5pt}
			\textbf{Exception Flow ที่ 1:}\enspace การที่จะเป็นกรรมการในการตัดสินได้จะต้องได้รับเชิญจากผู้จัดการประกวดเท่านั้น & \\
			\Xhline{1.5pt}
		\end{tabularx}
	}
\end{table}
% =====================================================}{}
	\IfFileExists{Chapter3_9.tex}{%==================== chapter3_9.tex ====================

\clearpage
\thispagestyle{plain}

\begingroup
\fontsize{16pt}{19.2pt}\selectfont
\justifying
\XeTeXlinebreakskip=0pt plus 1pt minus 0.5pt
\setlength{\parindent}{1.5cm}
\setlength{\parskip}{0pt}


% ================== Use Case ให้คะแนนในการแข่งขัน ==================
\begin{table}[h]
	\caption{Use Case ให้คะแนนในการแข่งขัน}
	{\tablefont
		\setlength{\tabcolsep}{6pt}%
		\begin{tabularx}{\linewidth}{@{} >{\justifying\arraybackslash}X >{\raggedleft\arraybackslash}p{4.2cm} @{}}
			\Xhline{1.5pt}
			\textbf{Use Case Title:}\enspace ให้คะแนนในการแข่งขัน & \UseCaseID[uc:register] \\
			\Xhline{0.5pt}
			\textbf{Primary Actor:}\enspace ผู้เชี่ยวชาญด้านปลากัด & \\
			\Xhline{0.5pt}
			\textbf{Stakeholder Actor:}\enspace - & \\
			\Xhline{0.5pt}
			\textbf{Main Flow:}\enspace ทำการให้คะแนนในการแข่งเพื่อที่ผู้จัดจะสามารถทำการประกาศผู้ชนะในการแข่งขันได้ & \\
			\Xhline{0.5pt}
			\textbf{Exception Flow ที่ 1:}\enspace การที่จะให้คะแนนในการแช่งขันปลากัดได้นั้นจำเป็นต้องได้รับเชิญให้เป็นกรรมก่อนถึงจะสามารถทำ
			การให้คะแนนในการแข่งขันได้ & \\
			\Xhline{1.5pt}
		\end{tabularx}
	}
\end{table}
% =====================================================

% ================== Use Case จัดการบัญชีผู้ใช้ เพิ่ม/ลบ/แก้ไข ข้อมูล ==================
\begin{table}[h]
	\caption{Use Case จัดการบัญชีผู้ใช้ เพิ่ม/ลบ/แก้ไข ข้อมูล}
	{\tablefont
		\setlength{\tabcolsep}{6pt}%
		\begin{tabularx}{\linewidth}{@{} >{\justifying\arraybackslash}X >{\raggedleft\arraybackslash}p{4.2cm} @{}}
			\Xhline{1.5pt}
			\textbf{Use Case Title:}\enspace จัดการบัญชีผู้ใช้ เพิ่ม/ลบ/แก้ไข ข้อมูล & \UseCaseID[uc:register] \\
			\Xhline{0.5pt}
			\textbf{Primary Actor:}\enspace ผู้บริหารจัดการข้อมูลที่เกิดขึ้นในระบบ & \\
			\Xhline{0.5pt}
			\textbf{Stakeholder Actor:}\enspace - & \\
			\Xhline{0.5pt}
			\textbf{Main Flow:}\enspace สามารถทำการ แก้ไข/เพิ่ม/ลบ ข้อมูลของผู้ใช้ในระบบทั้งหมดได้ & \\
			\Xhline{0.5pt}
			\textbf{Exception Flow ที่ 1:}\enspace กรณีที่ผู้บริหารจัดการข้อมูลที่เกิดขึ้นในระบบทำการ เพิ่ม/ลบ/แก้ไข ไม่ถูกต้องระบบจะทำการแสดงข้อความเตือนข้อผิดผลาดแล้วให้การก้ไขก่อนอัพเดทข้อมูล & \\
			\Xhline{1.5pt}
		\end{tabularx}
	}
\end{table}
% =====================================================

% ================== Use Case การจัดการการแจ้งเตือน ==================
\begin{table}[h]
	\caption{Use Case การจัดการการแจ้งเตือน}
	{\tablefont
		\setlength{\tabcolsep}{6pt}%
		\begin{tabularx}{\linewidth}{@{} >{\justifying\arraybackslash}X >{\raggedleft\arraybackslash}p{4.2cm} @{}}
			\Xhline{1.5pt}
			\textbf{Use Case Title:}\enspace การจัดการการแจ้งเตือน & \UseCaseID[uc:register] \\
			\Xhline{0.5pt}
			\textbf{Primary Actor:}\enspace ผู้บริหารจัดการข้อมูลที่เกิดขึ้นในระบบ & \\
			\Xhline{0.5pt}
			\textbf{Stakeholder Actor:}\enspace - & \\
			\Xhline{0.5pt}
			\textbf{Main Flow:}\enspace สามารถทำการเพิ่มลบข้อมูลต่าง ๆ ที่จะทำการแจ้งให้แก่ผู้ใช้ทราบ & \\
			\Xhline{1.5pt}
		\end{tabularx}
	}
\end{table}
% =====================================================}{}
	\IfFileExists{Chapter3_10.tex}{%==================== chapter3_10.tex ====================

\clearpage
\thispagestyle{plain}

\begingroup
\fontsize{16pt}{19.2pt}\selectfont
\justifying
\XeTeXlinebreakskip=0pt plus 1pt minus 0.5pt
\setlength{\parindent}{1.5cm}
\setlength{\parskip}{0pt}


% ================== Use Case ดูและและจัดการข้อมูลภาพ/วิดีโอ ผลคะแนนต่างๆ ==================
\begin{table}[h]
	\caption{Use Case ดูและและจัดการข้อมูลภาพ/วิดีโอ ผลคะแนนต่างๆ}
	{\tablefont
		\setlength{\tabcolsep}{6pt}%
		\begin{tabularx}{\linewidth}{@{} >{\justifying\arraybackslash}X >{\raggedleft\arraybackslash}p{4.2cm} @{}}
			\Xhline{1.5pt}
			\textbf{Use Case Title:}\enspace ดูและและจัดการข้อมูลภาพ/วิดีโอ ผลคะแนนต่างๆ & \UseCaseID[uc:register] \\
			\Xhline{0.5pt}
			\textbf{Primary Actor:}\enspace ผู้บริหารจัดการข้อมูลที่เกิดขึ้นในระบบ & \\
			\Xhline{0.5pt}
			\textbf{Stakeholder Actor:}\enspace - & \\
			\Xhline{0.5pt}
			\textbf{Main Flow:}\enspace ผู้บริหารจัดการข้อมูลที่เกิดขึ้นในระบบสามารถทำการ ลบ รูปภาพ/วิดีโอได้แต่ไม่สามารถทำการแก้ไขผลคะแนนต่างๆได้ & \\
			\Xhline{1.5pt}
		\end{tabularx}
	}
\end{table}
% =====================================================

% ================== Use Case ผู้บริหารจัดการข้อมูลที่เกิดขึ้นในระบบ เข้าสู่ระบบ ==================
\begin{table}[h]
	\caption{Use Case ผู้บริหารจัดการข้อมูลที่เกิดขึ้นในระบบ เข้าสู่ระบบ}
	{\tablefont
		\setlength{\tabcolsep}{6pt}%
		\begin{tabularx}{\linewidth}{@{} >{\justifying\arraybackslash}X >{\raggedleft\arraybackslash}p{4.2cm} @{}}
			\Xhline{1.5pt}
			\textbf{Use Case Title:}\enspace ผู้บริหารจัดการข้อมูลที่เกิดขึ้นในระบบ เข้าสู่ระบบ & \UseCaseID[uc:register] \\
			\Xhline{0.5pt}
			\textbf{Primary Actor:}\enspace ผู้บริหารจัดการข้อมูลที่เกิดขึ้นในระบบ & \\
			\Xhline{0.5pt}
			\textbf{Stakeholder Actor:}\enspace - & \\
			\Xhline{0.5pt}
			\textbf{Main Flow:}\enspace ทำการเข้าสู่ระบบเพื่อที่จะสามารถทำงานในด้านของผู้บริหารจัดการข้อมูลที่เกิดขึ้นในระบบ
			ได้ & \\
			\Xhline{1.5pt}
		\end{tabularx}
	}
\end{table}
% =====================================================

\clearpage}{}
	\IfFileExists{Chapter3_11.tex}{%==================== chapter3_11.tex ====================

\clearpage
\thispagestyle{plain}

\begingroup
\fontsize{16pt}{19.2pt}\selectfont
\justifying
\XeTeXlinebreakskip=0pt plus 1pt minus 0.5pt
\setlength{\parindent}{1.5cm}
\setlength{\parskip}{0pt}

% ---------- หัวข้อใหญ่ (ชิดซ้าย, หนา 16pt) ----------

\section*{Entity-Relation Diagram}
\addcontentsline{toc}{section}{Entity-Relation Diagram}

% ---------- เนื้อหา (จัดกระจายแบบไทย, ย่อหน้าแรก 1.5 ซม.) ----------
\indent Entity-Relation Diagram: เว็บแอปพลิเคชันศูนย์รวมการจัดการประกวดปลากัดไทย

\vspace{\baselineskip}

\begin{figure}[h]
	\centering
	\includegraphics[width=0.8\linewidth]{png2}
	\caption{ENTITY RELATIONSHIP ระบบเว็บแอปพลิเคชันศูนย์รวมการจัดประกวดปลากัดไทย}
\end{figure}

\clearpage}{}
	\IfFileExists{Chapter3_12.tex}{%==================== chapter3_12.tex ====================

\clearpage
\thispagestyle{plain}

\begingroup
\fontsize{16pt}{19.2pt}\selectfont
\justifying
\XeTeXlinebreakskip=0pt plus 1pt minus 0.5pt
\setlength{\parindent}{1.5cm}
\setlength{\parskip}{0pt}

% ตั้งค่าให้เหมือนสเปกเดิมทุกบท
\setlength{\LoneLabelSep}{0.5em}
\settowidth{\LoneLabelWidth}{9.}
\setlength{\LoneContentCol}{\dimexpr 1.5cm + \LoneLabelWidth + \LoneLabelSep\relax}

\setlength{\LtwoLabelSep}{0.5em}
\settowidth{\LtwoLabelWidth}{9.9.}
\setlength{\ExtraAlign}{-2.8em}

% ระดับ 1: อินเดนต์ 1.5 ซม.
\setlist[enumerate,1]{%
	label=\arabic*., align=left,
	leftmargin=1.5cm, labelindent=0pt,
	labelwidth=\LoneLabelWidth, labelsep=\LoneLabelSep,
	itemsep=0pt, topsep=0.5\baselineskip
}

\begin{enumerate}
	\item Entity – Relation Diagram: Profiles
	\item Entity – Relation Diagram: Contests
	\item Entity – Relation Diagram: Submissions
	\item Entity – Relation Diagram: Assignments
	\item Entity – Relation Diagram: ContestJudges
	\item Entity – Relation Diagram: Notifications
\end{enumerate}

\vspace{\baselineskip}

% ---------- หัวข้อใหญ่ (ชิดซ้าย, หนา 16pt) ----------
\noindent{\bfseries\fontsize{16pt}{19.2pt}\selectfont Data Dictionary: เว็บแอปพลิเคชันศูนย์รวมการจัดการประกวดปลากัดไทย}\par


\vspace{\baselineskip}

% ===== Helper for nicer row height (เฉพาะบล็อกนี้) =====
{\renewcommand{\arraystretch}{1.15}
	
	% ========== ER: PROFILES ==========
	\begin{table}[h]
		\caption{Entity -- Relation Diagram: Profiles}
		{\tablefont
			\setlength{\tabcolsep}{6pt}
			\begin{tabularx}{\linewidth}{@{} >{\raggedright\arraybackslash}p{4.2cm} D >{\centering\arraybackslash}p{2.6cm} >{\centering\arraybackslash}p{1.6cm} @{}}
				\Xhline{1.5pt}
				\textbf{Attribute Name} & \textbf{Description} & \textbf{Data Type} & \textbf{Key Type} \\
				\Xhline{0.5pt}
				\textbf{id}        & รหัสเฉพาะสำหรับระบุผู้ใช้งานแต่ละคน & UUID & PK \\
				\Xhline{0.5pt}
				username           & ชื่อแฝงของผู้ใช้สำหรับแสดงผล (ไม่ซ้ำกัน) & String & \\
				\Xhline{0.5pt}
				first\_name        & ชื่อจริงของผู้ใช้ & String & \\
				\Xhline{0.5pt}
				last\_name         & นามสกุลของผู้ใช้ & String & \\
				\Xhline{0.5pt}
				email              & อีเมลสำหรับเข้าระบบและติดต่อ (ไม่ซ้ำกัน) & String & \\
				\Xhline{0.5pt}
				avatar\_url        & ลิงก์รูปภาพโปรไฟล์ของผู้ใช้ & String & \\
				\Xhline{0.5pt}
				created\_at        & วันและเวลาที่สร้างบัญชีผู้ใช้ & Timestamp & \\
				\Xhline{0.5pt}
				updated\_at        & วันและเวลาที่แก้ไขข้อมูลผู้ใช้ล่าสุด & Timestamp & \\
				\Xhline{0.5pt}
				role               & บทบาทของผู้ใช้ในระบบ & String & \\
				\Xhline{0.5pt}
				specialities       & ความเชี่ยวชาญพิเศษ & JSON & \\
				\Xhline{1.5pt}
		\end{tabularx}}
	\end{table}
	

}{}
	\IfFileExists{Chapter3_13.tex}{%==================== chapter3_13.tex ====================

\clearpage
\thispagestyle{plain}

\begingroup
\fontsize{16pt}{19.2pt}\selectfont
\justifying
\XeTeXlinebreakskip=0pt plus 1pt minus 0.5pt
\setlength{\parindent}{1.5cm}
\setlength{\parskip}{0pt}

% ===== Helper for nicer row height (เฉพาะบล็อกนี้) =====
{\renewcommand{\arraystretch}{1.15}
	
	% ========== ER: Contests ==========
	\begin{table}[h]
		\caption{Entity -- Relation Diagram: Contests}
		{\tablefont
			\setlength{\tabcolsep}{6pt}
			\begin{tabularx}{\linewidth}{@{} >{\raggedright\arraybackslash}p{4.2cm} D >{\centering\arraybackslash}p{2.6cm} >{\centering\arraybackslash}p{1.6cm} @{}}
				\Xhline{1.5pt}
				\textbf{Attribute Name} & \textbf{Description} & \textbf{Data Type} & \textbf{Key Type} \\
				\Xhline{0.5pt}
				\textbf{id}            & รหัสการประกวด & UUID & PK \\
				\Xhline{0.5pt}
				name                   & ชื่อการประกวด & String & \\
				\Xhline{0.5pt}
				short\_description      & คำอธิบายย่อ & String & \\
				\Xhline{0.5pt}
				full\_description       & คำอธิบายเต็ม & String & \\
				\Xhline{0.5pt}
				poster\_url             & ลิงก์โปสเตอร์/ภาพประชาสัมพันธ์ & String & \\
				\Xhline{0.5pt}
				category               & หมวดหมู่ & String & \\
				\Xhline{0.5pt}
				start\_date             & วันเริ่ม & Timestamp & \\
				\Xhline{0.5pt}
				end\_date               & วันสิ้นสุด & Timestamp & \\
				\Xhline{0.5pt}
				status                 & สถานะการประกวด (เช่น draft, open, closed) & String & \\
				\Xhline{0.5pt}
				is\_vote\_open           & เปิดโหวตหรือไม่ & Boolean & \\
				\Xhline{0.5pt}
				\textbf{created\_by}    & ผู้สร้างการประกวด & UUID & FK \\
				\Xhline{1.5pt}
			\end{tabularx}}
	\end{table}}{}
	\IfFileExists{Chapter3_14.tex}{%==================== chapter3_14.tex ====================

\clearpage
\thispagestyle{plain}

\begingroup
\fontsize{16pt}{19.2pt}\selectfont
\justifying
\XeTeXlinebreakskip=0pt plus 1pt minus 0.5pt
\setlength{\parindent}{1.5cm}
\setlength{\parskip}{0pt}

% ===== Helper for nicer row height (เฉพาะบล็อกนี้) =====
{\renewcommand{\arraystretch}{1.15}
	
	% ========== ER: SUBMISSIONSsts ==========
	\begin{table}[h]
		\caption{Entity -- Relation Diagram: Submissions}
		{\tablefont
			\setlength{\tabcolsep}{6pt}
			\begin{tabularx}{\linewidth}{@{} >{\raggedright\arraybackslash}p{4.2cm} D >{\centering\arraybackslash}p{2.6cm} >{\centering\arraybackslash}p{1.6cm} @{}}
				\Xhline{1.5pt}
				\textbf{Attribute Name} & \textbf{Description} & \textbf{Data Type} & \textbf{Key Type} \\
				\Xhline{0.5pt}
				\textbf{id}             & รหัสผลงาน & UUID & PK \\
				\Xhline{0.5pt}
				\textbf{owner\_id}       & เจ้าของผลงาน & UUID & FK \\
				\Xhline{0.5pt}
				\textbf{contest\_id}     & อ้างถึงการประกวด & UUID & FK \\
				\Xhline{0.5pt}
				fish\_name               & ชื่อปลา & String & \\
				\Xhline{0.5pt}
				fish\_type               & ประเภทปลา & String & \\
				\Xhline{0.5pt}
				fish\_age\_months         & อายุ (เดือน) & Integer & \\
				\Xhline{0.5pt}
				fish\_image\_urls         & ลิงก์รูปภาพ (หลายรูป) & Array & \\
				\Xhline{0.5pt}
				fish\_video\_url          & ลิงก์วิดีโอ & String & \\
				\Xhline{0.5pt}
				status                  & สถานะผลงาน (เช่น pending, assigned, scored) & String & \\
				\Xhline{0.5pt}
				submitted\_at            & เวลาที่ส่ง & Timestamp & \\
				\Xhline{0.5pt}
				final\_score             & คะแนนสุดท้าย & Decimal & \\
				\Xhline{1.5pt}
			\end{tabularx}}
	\end{table}}{}
	\IfFileExists{Chapter3_15.tex}{%==================== chapter3_15.tex ====================

\clearpage
\thispagestyle{plain}

\begingroup
\fontsize{16pt}{19.2pt}\selectfont
\justifying
\XeTeXlinebreakskip=0pt plus 1pt minus 0.5pt
\setlength{\parindent}{1.5cm}
\setlength{\parskip}{0pt}

% ===== Helper for nicer row height (เฉพาะบล็อกนี้) =====
{\renewcommand{\arraystretch}{1.15}
	
	% ========== ER: ASSIGNMENTS ==========
	\begin{table}[h]
		\caption{Entity -- Relation Diagram: Assignments}
		{\tablefont
			\setlength{\tabcolsep}{6pt}
			\begin{tabularx}{\linewidth}{@{} >{\raggedright\arraybackslash}p{4.2cm} D >{\centering\arraybackslash}p{2.6cm} >{\centering\arraybackslash}p{1.6cm} @{}}
				\Xhline{1.5pt}
				\textbf{Attribute Name} & \textbf{Description} & \textbf{Data Type} & \textbf{Key Type} \\
				\Xhline{0.5pt}
				\textbf{id}             & รหัสการประเมิน & Long & PK \\
				\Xhline{0.5pt}
				\textbf{submission\_id}  & อ้างถึงผลงาน  & UUID & FK \\
				\Xhline{0.5pt}
				\textbf{evaluator\_id}   & ผู้ประเมิน/ผู้เชี่ยวชาญ  & UUID & FK \\
				\Xhline{0.5pt}
				status                  & สถานะ (เช่น assigned, scored) & String & \\
				\Xhline{0.5pt}
				scores                  & รายละเอียดคะแนน (เช่นรายหัวข้อ) & JSON & \\
				\Xhline{0.5pt}
				total\_score             & คะแนนรวม & Decimal & \\
				\Xhline{0.5pt}
				assigned\_at             & เวลามอบหมาย & Timestamp & \\
				\Xhline{0.5pt}
				evaluated\_at            & เวลาประเมินเสร็จ & Timestamp & \\
				\Xhline{1.5pt}
			\end{tabularx}}
	\end{table}
	
\vspace{\baselineskip}
	
	% ========== ER: CONTEST_JUDGES ==========
	\begin{table}[h]
		\caption{Entity -- Relation Diagram: CONTEST\_JUDGES}
		{\tablefont
			\setlength{\tabcolsep}{6pt}
			\begin{tabularx}{\linewidth}{@{} >{\raggedright\arraybackslash}p{4.2cm} D >{\centering\arraybackslash}p{2.6cm} >{\centering\arraybackslash}p{1.6cm} @{}}
				\Xhline{1.5pt}
				\textbf{Attribute Name} & \textbf{Description} & \textbf{Data Type} & \textbf{Key Type} \\
				\Xhline{0.5pt}
				\textbf{id}          & รหัสกรรมการในการประกวด & Long & PK \\
				\Xhline{0.5pt}
				\textbf{judge\_id}    & กรรมการ  & UUID & FK \\
				\Xhline{0.5pt}
				\textbf{contest\_id}  & การประกวด  & UUID & FK \\
				\Xhline{0.5pt}
				status               & สถานะการตอบรับ & String & \\
				\Xhline{0.5pt}
				assigned\_at          & เวลาที่เชิญ & Timestamp & \\
				\Xhline{1.5pt}
			\end{tabularx}}
	\end{table}
\clearpage}{}
	\IfFileExists{Chapter3_16.tex}{%==================== chapter3_16.tex ====================

\clearpage
\thispagestyle{plain}

\begingroup
\fontsize{16pt}{19.2pt}\selectfont
\justifying
\XeTeXlinebreakskip=0pt plus 1pt minus 0.5pt
\setlength{\parindent}{1.5cm}
\setlength{\parskip}{0pt}

% ===== Helper for nicer row height (เฉพาะบล็อกนี้) =====
{\renewcommand{\arraystretch}{1.15}
	
	% ========== ER: NOTIFICATIONS ==========
	\begin{table}[h]
		\caption{Entity -- Relation Diagram: Notifications}
		{\tablefont
			\setlength{\tabcolsep}{6pt}
			\begin{tabularx}{\linewidth}{@{} >{\raggedright\arraybackslash}p{4.2cm} D >{\centering\arraybackslash}p{2.6cm} >{\centering\arraybackslash}p{1.6cm} @{}}
				 \Xhline{1.5pt}
				\textbf{Attribute Name} & \textbf{Description} & \textbf{Data Type} & \textbf{Key Type} \\
				\Xhline{0.5pt}
				\textbf{id}        & รหัสการแจ้งเตือน & Long & PK \\
				\Xhline{0.5pt}
				\textbf{user\_id}   & ผู้ใช้ที่ได้รับ  & UUID & FK \\
				\Xhline{0.5pt}
				message            & ข้อความ & String & \\
				\Xhline{0.5pt}
				link\_to            & ลิงก์ & String & \\
				\Xhline{0.5pt}
				is\_read            & อ่านแล้วหรือไม่ & Boolean & \\
				\Xhline{0.5pt}
				type               & ประเภท & String & \\
				\Xhline{1.5pt}
			\end{tabularx}}
	\end{table}}{}
	\IfFileExists{Chapter3_17.tex}{%==================== chapter3_17.tex ====================

\clearpage
\thispagestyle{plain}

\begingroup
\fontsize{16pt}{19.2pt}\selectfont
\justifying
\XeTeXlinebreakskip=0pt plus 1pt minus 0.5pt
\setlength{\parindent}{1.5cm}
\setlength{\parskip}{0pt}

% ---------- หัวข้อใหญ่ (ชิดซ้าย, หนา 16pt) ----------

\section*{Class Diagram}
\addcontentsline{toc}{section}{Class Diagram}

% ---------- เนื้อหา (จัดกระจายแบบไทย, ย่อหน้าแรก 1.5 ซม.) ----------
\indent Class Diagram: เว็บแอปพลิเคชันศูนย์รวมการจัดการประกวดปลากัดไทย

\vspace{\baselineskip}

\begin{figure}[h]
	\centering
	\includegraphics[width=0.95\linewidth]{ClassDiagram}
	\caption{Class Diagram ระบบเว็บแอปพลิเคชันศูนย์รวมการจัดประกวดปลากัดไทย}
\end{figure}

\clearpage}{}
	\IfFileExists{Chapter3_18.tex}{%==================== chapter3_18.tex ====================

\clearpage
\thispagestyle{plain}

\begingroup
\fontsize{16pt}{19.2pt}\selectfont
\justifying
\XeTeXlinebreakskip=0pt plus 1pt minus 0.5pt
\setlength{\parindent}{1.5cm}
\setlength{\parskip}{0pt}

\vspace{\baselineskip}

% ============================ Class: Profile ============================
\begin{table}[h]
	\caption{Class Description : Profile}
	{\tablefont\setlength{\tabcolsep}{6pt}%
		\begin{tabularx}{\linewidth}{@{} >{\raggedright\arraybackslash}p{3.6cm} X @{}}
			\Xhline{1.5pt}
			\textbf{Class Name :} & Profile \\  % ปิดแถวด้วย \\
			\Xhline{0.5pt}
			\textbf{Description :} & โมเดลผู้ใช้หลักของระบบ ใช้แทนผู้ใช้งานทุกบทบาท และเก็บข้อมูลโปรไฟล์รวมถึงความเชี่ยวชาญ \\
			\Xhline{0.5pt}
			\textbf{Attribute :} &
			\begin{tabular}{@{}l@{}}
				id: UUID — รหัสผู้ใช้ (PK) \\
				username: string — ชื่อผู้ใช้ \\
				first\_name: string — ชื่อ \\
				last\_name: string — นามสกุล \\
				email: string — อีเมล \\
				avatar\_url: string — รูปโปรไฟล์ \\
				created\_at: datetime — เวลาเริ่มสร้าง \\
				updated\_at: datetime — เวลาอัปเดต \\
				role: string — บทบาท \\
				specialities: JSON — ความเชี่ยวชาญของผู้เชี่ยวชาญ
			\end{tabular} \\
			\Xhline{0.5pt}
			\textbf{Method :} &
			\begin{tabular}{@{}l@{}}
				fetchProfile(): ดึงโปรไฟล์ผู้ใช้ปัจจุบัน \\
				updateProfile(profileData): อัปเดตข้อมูลโปรไฟล์ \\
				uploadProfilePicture(file): อัปโหลดรูปโปรไฟล์
			\end{tabular} \\
			\Xhline{1.5pt}
	\end{tabularx}}
\end{table}

\newpage

% ============================ Class: Contest ============================
\begin{table}[h]
	\caption{Class Description : Contest}
	{\tablefont\setlength{\tabcolsep}{6pt}%
		\begin{tabularx}{\linewidth}{@{} >{\raggedright\arraybackslash}p{3.6cm} X @{}}
			\Xhline{1.5pt}
			\textbf{Class Name :} & Contest \\ 
			\Xhline{0.5pt}
			\textbf{Description :} & กิจกรรมการประกวด/ข่าวสารที่จัดโดยผู้จัดการการแข่งขัน ใช้ควบคุมสถานะและช่วงเวลา \\
			\Xhline{0.5pt}
			\textbf{Attribute :} &
			\begin{tabular}{@{}l@{}}
				id: UUID — รหัสการประกวด (PK) \\
				name: string — ชื่อการประกวด \\
				short\_description: string — คำอธิบายสั้น \\
				full\_description: string — คำอธิบายเต็ม \\
				poster\_url: string — โปสเตอร์ \\
				category: string — หมวดหมู่ \\
				start\_date: datetime — วันเริ่ม \\
				end\_date: datetime — วันจบ \\
				status: string — สถานะ \\
				is\_vote\_open: boolean — เปิดโหวตหรือไม่ \\
				created\_by: UUID — ผู้สร้าง (Profile.id)
			\end{tabular} \\
			\Xhline{0.5pt}
			\textbf{Method :} &
			\begin{tabular}{@{}l@{}}
				createContestOrNews(formData): สร้างกิจกรรม/ข่าว \\
				getMyContests(): ดึงรายการกิจกรรมของผู้จัดการ \\
				getContestDetail(contestId, isContest): ดึงรายละเอียดกิจกรรม \\
				updateMyContest(contestId, data): อัปเดตกิจกรรม \\
				deleteMyContest(contestId): ลบกิจกรรม \\
				updateContestStatus(contestId, status): เปลี่ยนสถานะ \\
				finalizeContest(contestId): ปิดและประกาศผล \\
				getContestSubmissions(contestId): ดึงรายการผู้สมัคร \\
				getScoringProgress(contestId): ความคืบหน้าการให้คะแนน \\
				getAllResults(): ดึงสรุปผลคะแนนทั้งหมด
			\end{tabular} \\
			\Xhline{1.5pt}
	\end{tabularx}}
\end{table}

\newpage

% ============================ Class: Submission ============================
\begin{table}[h]
	\caption{Class Description : Submission}
	{\tablefont\setlength{\tabcolsep}{6pt}%
		\begin{tabularx}{\linewidth}{@{} >{\raggedright\arraybackslash}p{3.6cm} X @{}}
			\Xhline{1.5pt}
			\textbf{Class Name :} & Submission \\
			\Xhline{0.5pt}
			\textbf{Description :} & การส่งผลงานปลากัดของผู้ใช้ ทั้งเพื่อการประเมินคุณภาพและเพื่อเข้าร่วมประกวด \\
			\Xhline{0.5pt}
			\textbf{Attribute :} &
			\begin{tabular}{@{}l@{}}
				id: UUID — รหัสการส่ง (PK) \\
				owner\_id: UUID — เจ้าของ (Profile.id) \\
				contest\_id: UUID — การประกวดที่เข้าร่วม (nullable เมื่อประเมินคุณภาพ) \\
				fish\_name: string — ชื่อปลา \\
				fish\_type: string — ประเภท/สายพันธุ์ \\
				fish\_age\_months: int — อายุ (เดือน) \\
				fish\_image\_urls: string[] — ลิงก์รูป \\
				fish\_video\_url: string — ลิงก์วิดีโอ \\
				status: string — สถานะ \\
				submitted\_at: datetime — เวลาส่ง \\
				final\_score: decimal — คะแนนสุดท้าย (ถ้ามี)
			\end{tabular} \\
			\Xhline{0.5pt}
			\textbf{Method :} &
			\begin{tabular}{@{}l@{}}
				submitBettaForEvaluation(formData): ส่งประเมินคุณภาพ \\
				submitBettaForCompetition(formData): ส่งเข้าประกวด \\
				getScoresForSubmission(submissionId): ดึงคะแนนของผลงาน
			\end{tabular} \\
			\Xhline{1.5pt}
	\end{tabularx}}
\end{table}

\clearpage}{}
	\IfFileExists{Chapter3_19.tex}{%==================== chapter3_19.tex ====================

\clearpage
\thispagestyle{plain}

\begingroup
\fontsize{16pt}{19.2pt}\selectfont
\justifying
\XeTeXlinebreakskip=0pt plus 1pt minus 0.5pt
\setlength{\parindent}{1.5cm}
\setlength{\parskip}{0pt}

\vspace{\baselineskip}


% ============================ Class: ContestJudge ============================
\begin{table}[h]
	\caption{Class Description : ContestJudge}
	{\tablefont\setlength{\tabcolsep}{6pt}%
		\begin{tabularx}{\linewidth}{@{} >{\raggedright\arraybackslash}p{3.6cm} X @{}}
			\Xhline{1.5pt}
			\textbf{Class Name :} & ContestJudge \\
			\Xhline{0.5pt}
			\textbf{Description :} & การเป็นกรรมการของผู้เชี่ยวชาญในกิจกรรมการประกวด รวมถึงสถานะคำเชิญ \\
			\Xhline{0.5pt}
			\textbf{Attribute :} &
			\begin{tabular}{@{}l@{}}
				id: long — รหัสแถว (PK) \\
				judge\_id: UUID — ผู้เชี่ยวชาญ (Profile.id) \\
				contest\_id: UUID — การประกวด \\
				status: string — สถานะคำเชิญ/บทบาท (invited, accepted, declined ฯลฯ) \\
				assigned\_at: datetime — เวลามอบหมาย
			\end{tabular} \\
			\Xhline{0.5pt}
			\textbf{Method :} &
			\begin{tabular}{@{}l@{}}
				getJudgingContests(): ดึงรายการประกวดที่เกี่ยวข้อง \\
				respondToJudgeInvitation(contestId, response, reason): ตอบรับ/ปฏิเสธ \\
				assignJudgeToContest(contestId, judgeId): มอบหมายกรรมการ \\
				removeJudgeFromContest(contestId, judgeId): ถอดกรรมการ \\
				notifyJudgeRemoval(contestId, judgeId, options): แจ้งเตือนการถอด
			\end{tabular} \\
			\Xhline{1.5pt}
	\end{tabularx}}
\end{table}

%\newpage

% ============================ Class: Assignment ============================
\begin{table}[h]
	\caption{Class Description : Assignment}
	{\tablefont\setlength{\tabcolsep}{6pt}%
		\begin{tabularx}{\linewidth}{@{} >{\raggedright\arraybackslash}p{3.6cm} X @{}}
			\Xhline{1.5pt}
			\textbf{Class Name :} & Assignment \\
			\Xhline{0.5pt}
			\textbf{Description :} & งานที่มอบหมายให้ผู้เชี่ยวชาญประเมินผลงาน เก็บคะแนนย่อยและผลรวม \\
			\Xhline{0.5pt}
			\textbf{Attribute :} &
			\begin{tabular}{@{}l@{}}
				id: long — รหัสงาน (PK) \\
				submission\_id: UUID — อ้างอิงผลงาน \\
				evaluator\_id: UUID — ผู้ประเมิน (Profile.id) \\
				status: string — สถานะงาน (queued, accepted, rejected, evaluated ฯลฯ) \\
				scores: JSON — รายละเอียดคะแนน \\
				total\_score: decimal — คะแนนรวม \\
				assigned\_at: datetime — เวลาได้รับมอบหมาย \\
				evaluated\_at: datetime — เวลาประเมินเสร็จ
			\end{tabular} \\
			\Xhline{0.5pt}
			\textbf{Method :} &
			\begin{tabular}{@{}l@{}}
				getEvaluationQueue(): ดึงคิวงานประเมินของผู้เชี่ยวชาญ \\
				respondToEvaluation(assignmentId, status, reason): ตอบรับ/ปฏิเสธงาน \\
				submitQualityScores(assignmentId, scoresData): ส่งคะแนนโหมดคุณภาพ \\
				getScoringSchema(bettaType, options): ดึงเกณฑ์การให้คะแนนแบบไดนามิก
			\end{tabular} \\
			\Xhline{1.5pt}
	\end{tabularx}}
\end{table}

\newpage

% ============================ Class: Notification ============================
\begin{table}[h]
	\caption{Class Description : Notification}
	{\tablefont\setlength{\tabcolsep}{6pt}%
		\begin{tabularx}{\linewidth}{@{} >{\raggedright\arraybackslash}p{3.6cm} X @{}}
			\Xhline{1.5pt}
			\textbf{Class Name :} & Notification \\
			\Xhline{0.5pt}
			\textbf{Description :} & การแจ้งเตือนของระบบที่ส่งถึงผู้ใช้แต่ละคน ใช้แจ้งผลการประกวด การมอบหมาย หรือการเปลี่ยนสถานะ \\
			\Xhline{0.5pt}
			\textbf{Attribute :} &
			\begin{tabular}{@{}l@{}}
				id: long — รหัสแจ้งเตือน (PK) \\
				user\_id: UUID — ผู้รับ \\
				message: string — ข้อความ \\
				link\_to: string — ลิงก์ไปยังหน้าที่เกี่ยวข้อง \\
				is\_read: boolean — อ่านแล้วหรือยัง \\
				type: string — ประเภทแจ้งเตือน \\
				created\_at: datetime — เวลาแจ้งเตือน
			\end{tabular} \\
			\Xhline{0.5pt}
			\textbf{Method :} &
			\begin{tabular}{@{}l@{}}
				getMyNotifications(opts): ดึงรายการแจ้งเตือนของผู้ใช้ \\
				markNotificationRead(id): ทำรายการเดียวเป็นอ่านแล้ว \\
				markAllNotificationsRead(): ทำทั้งหมดเป็นอ่านแล้ว \\
				countUnread(): นับจำนวนที่ยังไม่อ่าน
			\end{tabular} \\
			\Xhline{1.5pt}
	\end{tabularx}}
\end{table}

\clearpage}{}

	\IfFileExists{Chapter3_24.tex}{%==================== chapter3_24.tex ====================

\clearpage
\thispagestyle{plain}

\begingroup
\fontsize{16pt}{19.2pt}\selectfont
\justifying
\XeTeXlinebreakskip=0pt plus 1pt minus 0.5pt
\setlength{\parindent}{1.5cm}
\setlength{\parskip}{0pt}

% ---------- หัวข้อใหญ่ (ชิดซ้าย, หนา 16pt) ----------
\section*{Sequence Diagra}
\addcontentsline{toc}{section}{Sequence Diagra}

% ---------- เนื้อหา (จัดกระจายแบบไทย, ย่อหน้าแรก 1.5 ซม.) ----------
\indent Sequence Diagram เป็นแผนผังแสดงการทำงานแบบลำดับปฏิสัมพันธ์ โดยระบบประเมินปลากัดก่อนการแข่งขันโดยผู้เชี่ยวชาญ มีองค์ประกอบ Sequence Diagram ดังนี้

% ===== enumerate แบบ robust (ไม่ใช้ \dimexpr เพื่อกัน error) =====
\setlist[enumerate,1]{%
	label=\arabic*., align=left,
	leftmargin=1.5cm, labelindent=0pt,
	labelwidth=\LoneLabelWidth, labelsep=\LoneLabelSep,
	itemsep=0pt, topsep=0.5\baselineskip
}

% ระดับ 2: ให้คอลัมน์ข้อความตรงตามที่ตั้ง (ชิฟต์ด้วย \ExtraAlign)
\setlist[enumerate,2]{%
	label*=\arabic*., align=left,
	leftmargin=*,
	labelwidth=\LtwoLabelWidth, labelsep=\LtwoLabelSep,
	labelindent=\dimexpr \LoneContentCol + \ExtraAlign - \LtwoLabelWidth - \LtwoLabelSep\relax,
	itemsep=0pt, topsep=0pt
}
\setlist[enumerate,3]{%
	label*=\arabic*., align=left,
	leftmargin=*, labelsep=-0.6em,
	labelindent=-1em, % ให้คอลัมน์ข้อความตรงเส้นเดียวกับระดับ 2
	widest=9.9.9,
	itemsep=0pt, topsep=0pt
}

\begin{enumerate}
	\item การเข้าสู่ระบบ (User Sign In)
	\item ผู้ใช้ส่งปลากัด (พร้อม AI Analysis)
	\item Manager อนุมัติผู้สมัคร
	\item Expert ส่งผลประเมิน
\end{enumerate}

\vspace{\baselineskip}

\begin{figure}[h]
	\centering
	\includegraphics[width=0.8\linewidth]{S1.drawio}
	\caption{การเข้าสู่ระบบ (User Sign In)}
\end{figure}

\indent ไดอะแกรมนี้แสดงลำดับเหตุการณ์เมื่อผู้ใช้พยายามเข้าสู่ระบบ ตั้งแต่การกรอกข้อมูลไปจนถึงการตรวจสอบสิทธิ์และอัปเดตหน้าเว็บ

% --- ขั้นตอนการเข้าสู่ระบบ ---

\begin{sloppypar}
	\begin{enumerate}
		\item \textbf{ขั้นตอนการเข้าสู่ระบบ (User Sign In)}
		\begin{enumerate}
			\item User -> Login.jsx: ผู้ใช้กรอกอีเมลและรหัสผ่านในหน้า Login แล้วกดปุ่ม “เข้าสู่ระบบ”
			\item Login.jsx -> AuthContext: Component Login เรียกใช้ฟังก์ชัน signin() จาก AuthContext ซึ่งเป็นศูนย์กลางจัดการสถานะการล็อกอิน
			\item AuthContext -> authService.js (FE): AuthContext ส่งต่อคำสั่งไปยัง authService (ฝั่ง Frontend) เพื่อเริ่มกระบวนการล็อกอิน
			\item authService.js (FE) -> ApiService.js: authService เรียกใช้ ApiService ซึ่งเป็น Service กลางสำหรับส่ง HTTP Request ทั้งหมด โดยส่ง POST ไปยัง endpoint /auth/signin
			\item ApiService.js -> authRoutes.js (BE): ApiService ส่ง HTTP POST ไปยัง Backend API ที่ /api/auth/signin
			\item authRoutes.js (BE) -> AuthController: Router ฝั่ง Backend ส่งต่อคำขอไปยังฟังก์ชัน handleSignIn() ใน AuthController
			\item AuthController -> AuthService.js (BE): Controller เรียกใช้ AuthService ซึ่งดูแล Business Logic ของการเข้าสู่ระบบ
			\item AuthService.js (BE) -> Supabase: ส่งอีเมลและรหัสผ่านไปตรวจสอบกับ Supabase Auth
			\item Supabase -> AuthService.js (BE): ถ้าถูกต้อง ส่งข้อมูล user และ session (มี JWT Token) กลับมา
			\item AuthService.js (BE) -> Supabase: นำ user.id ไปค้นหาข้อมูลโปรไฟล์จากตาราง profiles
			\item Supabase -> AuthService.js (BE): ส่งข้อมูลโปรไฟล์ (เช่น role, ชื่อ, นามสกุล) กลับมา
			\item AuthService.js (BE) -> AuthController: ส่งผลลัพธ์สุดท้าย (token + profile) กลับให้ Controller
			\item AuthController -> ApiService.js: ส่ง HTTP 200 OK พร้อม token และ profile กลับไปยัง Frontend
			\item ApiService.js -> authService.js (FE): ApiService ส่งต่อผลลัพธ์ให้ authService
			\item authService.js (FE) -> authService.js (FE): บันทึก Token และโปรไฟล์ลงใน localStorage ของเบราว์เซอร์
			\item authService.js (FE) -> AuthContext: ส่งข้อมูล profile กลับให้ AuthContext
			\item AuthContext -> AuthContext: อัปเดต state ภายใน โดยตั้งค่า user และ isAuthenticated = true
			\item AuthContext -> Login.jsx: แจ้งผลสำเร็จกลับไปยัง Component Login
			\item Login.jsx -> User: นำทางผู้ใช้ไปยัง Dashboard ที่เหมาะสมกับ Role
		\end{enumerate}
	\end{enumerate}
\end{sloppypar}

\newpage

\begin{figure}[h]
	\centering
	\includegraphics[width=0.8\linewidth]{S2.drawio}
	\caption{ขั้นตอนการส่งปลากัดเพื่อประเมิน (พร้อม AI)}
\end{figure}

\indent ไดอะแกรมนี้แสดงขั้นตอนที่ซับซ้อนขึ้น โดยแบ่งเป็น 2 ส่วน คือ การวิเคราะห์ด้วย AI (เกิดขึ้นเมื่ออัปโหลดรูป) และการส่งฟอร์มหลัก

\begin{sloppypar}
	\begin{enumerate}
		\item \textbf{ขั้นตอนการส่งปลากัดเพื่อประเมิน (พร้อม AI)}
		\begin{enumerate}
			\item การวิเคราะห์ด้วย AI (เกิดขึ้นอัตโนมัติเมื่อผู้ใช้อัปโหลดรูป)
			\begin{enumerate}
				\item User -> BettaEvaluationForm.jsx: ผู้ใช้เลือกไฟล์รูปภาพเพื่ออัปโหลด
				\item BettaEvaluationForm.jsx -> ModelService.js (FE): Component เรียกใช้ฟังก์ชัน analyzeBettaTypeAuto() เพื่อวิเคราะห์รูปภาพ
				\item ModelService.js (FE) -> ApiService.js: Service ของ Model เรียก ApiService เพื่อส่ง POST request ไปยัง Backend ที่ endpoint /model/analyze-single
				\item ApiService.js -> modelRoutes.js (BE): Request เดินทางถึง Backend และถูกส่งต่อไปยัง ModelController
				\item ModelController -> ModelApiService.js (BE): Controller เรียกใช้ ModelApiService ซึ่งเป็น Service ที่เชื่อมต่อกับ AI ภายนอก
				\item ModelApiService.js (BE) -> HuggingFace API: ส่งรูปภาพไปวิเคราะห์ที่ HuggingFace API 
				\item HuggingFace API -> ModelApiService.js (BE): ส่งผลลัพธ์ (ประเภทปลา/ความน่าจะเป็น) กลับมา
				\item ModelApiService.js (BE) -> ModelController: จัดรูปแบบผลลัพธ์แล้วส่งกลับ
				\item ModelController -> ApiService.js: ส่งผลลัพธ์กลับไปยัง Frontend (HTTP 200 OK)
				\item ApiService.js -> ModelService.js (FE): ส่งต่อผลลัพธ์
				\item ModelService.js (FE) -> BettaEvaluationForm.jsx: ส่งคำแนะนำกลับไปให้ Component
				\item BettaEvaluationForm.jsx -> User: แสดงผลการวิเคราะห์ของ AI บนหน้าจอ
			\end{enumerate} 
\newpage
			\item \textbf{การส่งฟอร์มหลัก} 
				\begin{enumerate}
					\item User -> BettaEvaluationForm.jsx: ผู้ใช้กรอกข้อมูลที่เหลือจนครบถ้วน และกดปุ่ม "ส่งประเมิน"
					\item BettaEvaluationForm.jsx -> UserService.js (FE): Component เรียกใช้ submitBettaForEvaluation()
					\item UserService.js (FE) -> ApiService.js: Service เรียก ApiService เพื่อส่ง POST request ไปยัง /submissions/evaluate พร้อมข้อมูลทั้งหมดจากฟอร์ม
					\item ApiService.js -> submissionRoutes.js (BE): Request เดินทางถึง Backend
					\item submissionRoutes.js (BE) -> SubmissionController: Route ส่งต่อให้ SubmissionController
					\item SubmissionController -> Supabase Storage: Controller อัปโหลดไฟล์รูปภาพและวิดีโอ ไปยัง Supabase Storage
					\item Supabase Storage --> SubmissionController: Supabase Storage คืนค่า URL ของไฟล์ที่อัปโหลดสำเร็จกลับมา
					\item SubmissionController -> SubmissionService.js (BE): Controller เรียกใช้ SubmissionService พร้อมข้อมูลทั้งหมด (รวมถึง URL ของไฟล์)
					\item SubmissionService.js (BE) -> Supabase DB: SubmissionService เรียกใช้ฟังก์ชัน create\_full\_submission (RPC) ในฐานข้อมูลเพื่อบันทึกข้อมูล Submission
					\item Supabase DB --> SubmissionService.js (BE): ฐานข้อมูลคืนค่า submission\_id ใหม่ที่ถูกสร้างขึ้น
					\item SubmissionController -> AutoAssignmentService: Controller เรียกใช้ AutoAssignmentService แบบ Fire-and-forget (ทำงานเบื้องหลัง) เพื่อมอบหมายงานให้ Expert โดยอัตโนมัติ 
					\item AutoAssignmentService -> Supabase DB: Service นี้จะค้นหา Expert ที่เหมาะสมและสร้าง assignment ใหม่ในฐานข้อมูล
					\item SubmissionController --> ApiService.js: Controller ส่ง HTTP Response 201 Created กลับไปยัง Frontend เพื่อยืนยันว่าการส่งผลงานสำเร็จ
					\item ApiService.js --> UserService.js (FE): ApiService ส่งต่อผลลัพธ์
					\item UserService.js (FE) --> BettaEvaluationForm.jsx: Service แจ้งผลสำเร็จกลับไปที่ Component
					\item BettaEvaluationForm.jsx -> User: Component แสดงข้อความ "สำเร็จ" และเปลี่ยนหน้าไปยังหน้าประวัติ (/history)
				\end{enumerate}
		\end{enumerate}
	\end{enumerate}
\end{sloppypar}

\newpage

\vspace{\baselineskip}

\begin{figure}[h]
	\centering
	\includegraphics[width=0.8\linewidth]{S3.drawio}
	\caption{ขั้นตอนการอนุมัติผู้สมัครโดย Manager}
\end{figure}

\indent ไดอะแกรมนี้แสดงการทำงานในหน้า Live Contest Room เมื่อผู้จัดการทำการอนุมัติผู้ที่ส่งปลากัดเข้าร่วมการแข่งขัน

\begin{sloppypar}
	\begin{enumerate}
		\item Manager -> LiveContestRoom.jsx: ผู้จัดการเข้าสู่หน้า Live Room และเลือกการแข่งขันที่ต้องการจะดู
		\item LiveContestRoom.jsx -> ManagerService.js (FE): Component เรียกฟังก์ชัน getContestSubmissions() เพื่อขอดูรายชื่อผู้สมัครทั้งหมด
		\item ManagerService.js (FE) -> ApiService.js: Service ส่ง GET request ไปยัง Backend 
		\item ApiService.js -> managerRoutes.js (BE): Request เดินทางถึง Backend
		\item managerRoutes.js (BE) -> ManagerController: Route ส่งต่อให้ ManagerController
		\item ManagerController -> ManagerService.js (BE): Controller เรียก Service เพื่อดึงข้อมูล
		\item ManagerService.js (BE) -> Supabase DB: Service ค้นหาข้อมูล submissions ทั้งหมดที่เกี่ยวข้องกับการแข่งขันนี้จากฐานข้อมูล
		\item Supabase DB --> ManagerService.js (BE): ฐานข้อมูลส่งรายชื่อผู้สมัครกลับมา 
		\item ManagerService.js (BE) --> ManagerController: Service ส่งข้อมูลกลับให้ Controller
		\item ManagerController --> ApiService.js: Controller ส่ง HTTP Response 200 OK พร้อมรายชื่อผู้สมัครกลับมา
		\item ApiService.js --> ManagerService.js (FE): ApiService ส่งต่อข้อมูล
		\item ManagerService.js (FE) --> LiveContestRoom.jsx: Service ส่งข้อมูลกลับให้ Component
		\item LiveContestRoom.jsx -> Manager: Component แสดงรายชื่อผู้สมัครทั้งหมดบนหน้าจอ
		\item Manager -> LiveContestRoom.jsx: ผู้จัดการกดปุ่ม "อนุมัติ" บนรายการของผู้สมัครที่ต้องการ
		\item LiveContestRoom.jsx -> ManagerService.js (FE): Component เรียกฟังก์ชัน updateSubmissionStatus()
		\item ManagerService.js (FE) -> ApiService.js: Service ส่ง PUT request ไปยัง Backend เพื่ออัปเดตสถานะ
		\item ApiService.js -> managerRoutes.js (BE): Request เดินทางถึง Backend
		\item managerRoutes.js (BE) -> ManagerController: Route ส่งต่อให้ ManagerController
		\item ManagerController -> ManagerService.js (BE): Controller เรียก Service เพื่ออัปเดตสถานะ
		\item ManagerService.js (BE) -> Supabase DB: Service อัปเดตสถานะของ submission ในฐานข้อมูลเป็น 'approved'
		\item Supabase DB --> ManagerService.js (BE): ฐานข้อมูลยืนยันการอัปเดต
		\item ManagerService.js (BE) -> NotificationService: ManagerService เรียก NotificationService เพื่อสร้างการแจ้งเตือน
		\item NotificationService -> Supabase DB: Service บันทึกการแจ้งเตือนใหม่ ลงในฐานข้อมูลเพื่อส่งให้ผู้สมัคร
		\item ManagerService.js (BE) --> ManagerController: ManagerService แจ้งผลสำเร็จกลับ
		\item ManagerController --> ApiService.js: Controller ส่ง HTTP Response 200 OK
		\item ApiService.js --> ManagerService.js (FE): ApiService ส่งต่อผลลัพธ์
		\item ManagerService.js (FE) --> LiveContestRoom.jsx: Service แจ้งผลสำเร็จกลับไปที่ Component
		\item LiveContestRoom.jsx -> LiveContestRoom.jsx: Component เรียกข้อมูลผู้สมัครใหม่อีกครั้ง เพื่อให้หน้าจอเป็นข้อมูลล่าสุด
		\item LiveContestRoom.jsx -> Manager: Component แสดงข้อความ "สำเร็จ" และอัปเดต UI (เช่น ย้ายผู้สมัครจากแท็บ "รออนุมัติ" ไปยัง "อนุมัติแล้ว")
	\end{enumerate}
\end{sloppypar}
\vspace{\baselineskip}

\endgroup}{}
	\IfFileExists{Chapter3_25.tex}{%==================== chapter3_25.tex ====================

\clearpage
\thispagestyle{plain}

\begingroup
\fontsize{16pt}{19.2pt}\selectfont
\justifying
\XeTeXlinebreakskip=0pt plus 1pt minus 0.5pt
\setlength{\parindent}{1.5cm}
\setlength{\parskip}{0pt}

% ---------- หัวข้อใหญ่ (ชิดซ้าย, หนา 16pt) ----------

\section*{User Interface}
\addcontentsline{toc}{section}{User Interface}

\indent การออกแบบส่วนเชื่อมประสานกับผู้ใช้ระบบเว็บแอปพลิเคชันศูนย์รวมการจัดประกวด
ปลากัดไทยโดยการออกแบบส่วนต่อประสานกับผู้ใช้มี 4 ฝั่ง ดังนี้

\begin{sloppypar}
	\begin{enumerate}
		\item ผู้เลี้ยงปลากัด
		\item ผู้เชี่ยวชาญ
		\item ผู้จัดการประกวด
		\item ผู้ดูแลระบบ
	\end{enumerate}
\end{sloppypar}

\begin{sloppypar}
	\begin{enumerate}
		\item \textbf{Web Application ผู้เลี้ยงปลากัด}
	\end{enumerate}
\end{sloppypar}

\begin{figure}[h]
	\centering
	\includegraphics[width=0.8\linewidth]{HP1}
	\caption{หน้าหลักของเว็บไซต์}
\end{figure}

\indent พอเข้ามาที่หน้าแรกผู้ใช้จะเจอเป็นหน้าต้อนรับหลักของเว็บ ตรงส่วนบนสุดจะมีปุ่มให้เลือกใช้งานได้ 3 ปุ่ม
คือ "ส่งปลากัดเข้าประกวด" ถ้ากดปุ่มนี้ก็จะเข้าไปหน้ากรอกข้อมูลเพื่อส่งปลากัดของเราไปให้ผู้เชี่ยวชาญดู หรือส่งเข้าแข่งขันได้เลย กับอีกปุ่มคือ "ดูการประกวด" ที่จะพาไปดูว่าตอนนี้มีประกวดอะไรเปิดรับสมัครอยู่บ้าง จะมีส่วนที่เป็นรูปภาพสไลด์โชว์ (Carousel) โชว์กิจกรรมเด่นๆ ที่กำลังมีอยู่ เราสามารถกดลูกศรซ้าย-ขวาเพื่อเลื่อนดูโปสเตอร์กิจกรรมอื่นๆ ได้ หรือถ้าสนใจกิจกรรมไหนเป็นพิเศษ ก็คลิกที่รูปโปสเตอร์นั้นเพื่อเข้าไปอ่านรายละเอียดของการประกวดได้เลย ในส่วนล่างของหน้า ก็จะมีส่วนของ "ข่าวสารแนะนำ" ครับ จะเป็นการ์ดข่าวต่างๆ เราสามารถคลิกเข้าไปอ่านเนื้อหาข่าวเต็มๆ ของเรื่องที่เราสนใจได้เลย

\newpage

\begin{figure}[h]
	\centering
	\includegraphics[width=0.8\linewidth]{HP2}
	\caption{หน้าข่าวสารทั้งหมดในระบบ}
\end{figure}

\indent ในหน้านี้ ผู้ใช้จะเจอกับรายการ "ข่าวสารและกิจกรรม" ทั้งหมดที่มีในระบบ โดยหลักๆ แล้ว ผู้ใช้จะสามารถ ค้นหาข่าวที่สนใจ ด้านบนสุดของหน้าจะมีช่องค้นหา ผู้ใช้สามารถพิมพ์คำค้นหาที่ต้องการลงไปได้เลยครับ เช่น ชื่อกิจกรรม หรือคำที่อยู่ในเนื้อหาข่าว ระบบก็จะกรองรายการข่าวที่เกี่ยวข้องมาให้ดูทันที
เลือกอ่านข่าวที่ต้องการ ข่าวแต่ละเรื่องจะแสดงเป็นการ์ดสวยงาม ผู้ใช้สามารถคลิกที่การ์ดข่าวเรื่องไหนก็ได้ที่สนใจ เพื่อเข้าไปอ่านรายละเอียดเนื้อหาทั้งหมดของข่าวนั้นๆ ในหน้าถัดไป
เลื่อนดูหน้าต่างๆ ถ้ามีข่าวเยอะมากๆ จนแสดงไม่หมดในหน้าเดียว ด้านล่างสุดจะมีปุ่มตัวเลขและปุ่ม "ก่อนหน้า" กับ "ถัดไป" ให้ผู้ใช้สามารถกดเพื่อเลื่อนไปดูข่าวในหน้าอื่นๆ ได้ครับ

\vspace{\baselineskip}

\begin{figure}[h]
	\centering
	\includegraphics[width=0.8\linewidth]{HP3}
	\caption{หน้ารายการประกวด}
\end{figure}

\indent ในหน้าการประกวด จะแบ่งการใช้งานตามสถานะการล็อกอินของผู้ใช้ \textbf{ถ้ายังไม่ได้ล็อกอิน (ผู้ใช้ทั่วไป)} ดูได้ทุกอย่าง สามารถเข้ามาดูรายการประกวดทั้งหมดที่กำลังเปิดรับสมัครได้ เห็นโปสเตอร์กิจกรรม ชื่อกิจกรรม และรายละเอียดเบื้องต้นได้เหมือนกับคนที่ล็อกอินแล้วทุกอย่างเลย อ่านรายละเอียดได้ แต่จะยังเข้าร่วมไม่ได้ \textbf{ถ้าล็อกอินแล้ว (สมาชิก)} ดูได้ทุกอย่าง และสามารถเข้าร่วมประกวด

\newpage

\begin{figure}[h]
	\centering
	\includegraphics[width=0.8\linewidth]{LG1}
	\caption{หน้าเข้าสู่ระบบ}
\end{figure}

\indent ในหน้านี้ผู้ใช้สามารถเข้าสู่ระบบได้ และสามารถกดสมัครสมาชิกได้ กรณีไม่มีบัญชีเข้าใช้งาน

\vspace{\baselineskip}

\begin{figure}[h]
	\centering
	\includegraphics[width=0.8\linewidth]{RG1}
	\caption{หน้าสมัครสมาชิก}
\end{figure}

\indent ในหน้านี้ผู้ใช้สามารถสมัครสมาชิกได้โดยกรอกรายละเอียดตามแบบฟอร์มได้เลย

\newpage

\begin{figure}[h]
	\centering
	\includegraphics[width=0.8\linewidth]{EV1}
	\caption{หน้าประเมินคุณภาพปลากัด}
\end{figure}

\indent ในหน้านี้ผู้ใช้สามารถใส่ข้อมูลตามแบบฟอร์มได้เลย เช่นใส่รูปภาพ ใส่วิดีโอ ใส่ชื่อปลากัด ใส่ประเภทปลากัด เรามี Ai ช่วยตรวจสอบประเภทปลากัดตอนนี้เราสามารถทำได้ 3 ประเภท ได้แก่ ปลากัดพื้นบ้านมหาชัย ปลากัดพื้นบ้านภาคอีสานหางลาย และปลากัดพื้นบ้านภาคใต้

\vspace{\baselineskip}

\begin{figure}[h]
	\centering
	\includegraphics[width=0.8\linewidth]{HT1}
	\caption{หน้าประวัติการส่งประเมินคุณภาพปลากัด}
\end{figure}

\indent ในหน้านี้ผู้ใช้สามารถดูได้ว่าตนเองส่งปลากัดเข้าร่วมการประเมินคุณภาพไปกี่ครั้งแล้ว และสามารถดูคะแนนที่รับได้ 

\newpage

\begin{figure}[h]
	\centering
	\includegraphics[width=0.8\linewidth]{HT2}
	\caption{หน้าประวัติการส่งประเมินคุณภาพปลากัด}
\end{figure}

\indent ในหน้านี้ผู้ใช้สามารถดูได้ว่าตนเองส่งปลากัดเข้าร่วมการประเกวดไปกี่ครั้งแล้ว และสามารถดูผลได้ว่าได้ลำดับที่เท่าไหร่

\vspace{\baselineskip}

\begin{figure}[h]
	\centering
	\includegraphics[width=0.8\linewidth]{PF1}
	\caption{หน้าโปรไฟล์ผู้ใช้}
\end{figure}

\indent ในหน้านี้ผู้ใช้สามารถแก้ไขข้อมูลส่วนตัวของตนเองได้และสามารถเปลี่ยบยนรูปภาพโปรไฟล์ได้

\newpage
	
	
\begin{sloppypar}
	\begin{enumerate}[start=2]  % เริ่มที่ 2 (ต่อจาก 1)
		\item \textbf{Web Application ผู้เชี่ยวชาญ}
	\end{enumerate}
\end{sloppypar}

\begin{figure}[h]
	\centering
	\includegraphics[width=0.8\linewidth]{EP1}
	\caption{หน้าหลักผู้เชี่ยวชาญ}
\end{figure}

\indent ในหน้านี้ ผู้เชี่ยวชาญสามารถ ดูภาพรวมของงานทั้งหมด ได้อย่างรวดเร็วผ่านการ์ดสรุป 3 อันหลักๆ คือ 
"รอประเมินคุณภาพ" บอกจำนวนปลากัดที่ถูกส่งมาให้ประเมินและกำลังรอให้เราจัดการ
"รอตอบรับการแข่งขัน" บอกจำนวนการแข่งขันที่ผู้จัดการ (Manager) ได้ส่งคำเชิญมาให้เราเป็นกรรมการ
"งานที่เสร็จสิ้น" สรุปจำนวนงานทั้งหมดที่เราได้ประเมินหรือตัดสินไปแล้ว 

\vspace{\baselineskip}

\begin{figure}[h]
	\centering
	\includegraphics[width=0.8\linewidth]{EP2}
	\caption{หน้าคิวงานผู้เชี่ยวชาญ}
\end{figure}

\indent ในหน้านี้ ผู้เชี่ยวชาญสามารถดูงานที่ได้รับมอบหมาย ผู้เชี่ยวชาญจะเห็นรายการปลากัดที่ระบบมอบหมายมาให้ พร้อมข้อมูลเบื้องต้นอย่างรูปภาพ, ชื่อ, และประเภทของปลา ตัดสินใจรับงาน ผู้เชี่ยวชาญมี 2 ตัวเลือก กดปุ่ม "ตอบรับ" (สีเขียว) เพื่อยืนยันว่าจะทำงานประเมินชิ้นนี้ เมื่องานถูกตอบรับแล้ว ก็จะย้ายไปอยู่ในแท็บ "ที่ต้องให้คะแนน"
กดปุ่ม "ปฏิเสธ" (สีแดง) ในกรณีที่ไม่สะดวกประเมิน หรือเห็นว่าข้อมูลที่ส่งมาไม่เหมาะสม ก็สามารถปฏิเสธงานชิ้นนี้ได้

\endgroup}{}
	\IfFileExists{Chapter3_26.tex}{%==================== chapter3_26.tex ====================

\clearpage
\thispagestyle{plain}

\begingroup
\fontsize{16pt}{19.2pt}\selectfont
\justifying
\XeTeXlinebreakskip=0pt plus 1pt minus 0.5pt
\setlength{\parindent}{1.5cm}
\setlength{\parskip}{0pt}

\begin{figure}[h]
	\centering
	\includegraphics[width=0.8\linewidth]{EP3}
	\caption{หน้ารับงานการเป็นกรรมการ}
\end{figure}

\indent ผู้เชี่ยวชาญจะเห็น รายการคำเชิญใหม่ๆ ที่ผู้จัดการประกวด (Manager) ส่งมาให้เพื่อเป็นกรรมการตัดสิน
ในแต่ละรายการคำเชิญ จะมีปุ่มให้ผู้เชี่ยวชาญตัดสินใจ คือ "ตอบรับ" หรือ "ปฏิเสธ" คำเชิญนั้นๆ 

\vspace{\baselineskip}

\begin{figure}[h]
	\centering
	\includegraphics[width=0.8\linewidth]{EP4}
	\caption{หน้าประวัติการทำงาน}
\end{figure}

\indent หน้านี้คือให้ผู้เชี่ยวชาญสามารถ ย้อนกลับมาดูงานทั้งหมดที่เคยทำเสร็จไปแล้วโดยจะมีการแบ่งประวัติออกเป็น 2 ประเภท ซึ่งผู้ใช้สามารถกดเลือกดูได้จากปุ่มด้านบน การประเมินคุณภาพ เมื่อกดปุ่มนี้ (ซึ่งเป็นหน้าที่แสดงในรูป) ผู้เชี่ยวชาญจะเห็น รายการปลากัดที่เคยประเมินคุณภาพไปแล้วทั้งหมด ในรูปแบบตาราง การตัดสินการแข่งขัน 
ถ้ากดปุ่มนี้ ผู้เชี่ยวชาญก็จะเห็น รายชื่อการแข่งขันทั้งหมดที่เคยเข้าไปเป็นกรรมการตัดสิน

\newpage

\begin{figure}[h]
	\centering
	\includegraphics[width=0.8\linewidth]{EP5}
	\caption{หน้าโปรไฟลืผู้เชี่ยวชาญ}
\end{figure}

\indent  เชี่ยวชาญสามารถ ดูและจัดการข้อมูลส่วนตัวของตัวเอง ได้ โดยจะแบ่งการทำงานเป็น 2 ส่วนหลักๆ การดูข้อมูล ในตอนแรกที่เข้ามา ผู้เชี่ยวชาญจะเห็นข้อมูลของตัวเองที่แสดงอยู่ ได้แก่ รูปโปรไฟล์, ชื่อ-นามสกุล, Username และอีเมล ความถนัด (Specialities) จะมีป้าย Tag แสดงประเภทปลากัดที่ตัวเองมีความเชี่ยวชาญเป็นพิเศษ 

\endgroup}{}
	\IfFileExists{Chapter3_27.tex}{%==================== chapter3_27.tex ====================

\clearpage
\thispagestyle{plain}

\begingroup
\fontsize{16pt}{19.2pt}\selectfont
\justifying
\XeTeXlinebreakskip=0pt plus 1pt minus 0.5pt
\setlength{\parindent}{1.5cm}
\setlength{\parskip}{0pt}

\begin{sloppypar}
	\begin{enumerate}[start=4]  % เริ่มที่ 2 (ต่อจาก 1)
		\item \textbf{Web Application ผู้จัดการประกวด}
	\end{enumerate}
\end{sloppypar}

\begin{figure}[h]
	\centering
	\includegraphics[width=0.8\linewidth]{MG1}
	\caption{หน้าหลักของผู้จัดการประกวด}
\end{figure}

\indent หน้านี้คือหน้า แดชบอร์ด (Dashboard) ผู้จัดการประกวด จัดการสามารถ ดูภาพรวม และ เข้าถึงเครื่องมือจัดการ ทั้งหมดได้ สรุปสถานะการประกวด: ตรงกลางหน้าจะมี การ์ดสรุปตัวเลข บอกชัดเจนว่าตอนนี้มีกิจกรรมทั้งหมดกี่รายการ, อยู่ในสถานะ "ร่าง", "กำลังดำเนินการ", หรือ "ประกาศผล" ไปแล้วกี่รายการ ซึ่งข้อมูลนี้จะถูกสรุปเป็น กราฟแท่ง ด้านล่างเพื่อให้เห็นภาพรวมได้ง่ายขึ้น ดูการแจ้งเตือน: ทางด้านขวาจะมีกล่อง "การแจ้งเตือน" คอยบอกข่าวสารสำคัญๆ เช่น มีผู้เชี่ยวชาญตอบรับคำเชิญ หรือมีกิจกรรมที่ใกล้จะเริ่มแล้ว

\vspace{\baselineskip}

\begin{figure}[h]
	\centering
	\includegraphics[width=0.8\linewidth]{MG2}
	\caption{หน้าสร้างกิจกรรมการประกวดหรือข่าวสาร}
\end{figure}

\indent หน้านี้เป็นหน้าที่ผู้จัดการใช้สำหรับ สร้าง "การประกวด" หรือ "ข่าวสาร" ใหม่ๆ เพื่อประกาศบนเว็บไซต์ โดยผู้จัดการจะต้องกรอกข้อมูลต่างๆ ให้ครบถ้วนตามแบบฟอร์ม

\newpage

\begin{figure}[h]
	\centering
	\includegraphics[width=0.8\linewidth]{MG3}
	\caption{หน้ากิจกรรมที่ได้สร้างขึ้นทั้งหมด}
\end{figure}

\indent หน้านี้เป็นเหมือนคลังเก็บกิจกรรมทั้งหมดที่ผู้จัดการเคยสร้างไว้ ไม่ว่าจะเป็น "การประกวด" หรือ "ข่าวสาร" ก็จะมารวมกันอยู่ที่นี่ทั้งหมด โดยผู้จัดการสามารถเข้ามาจัดการกิจกรรมต่างๆ ได้อย่างเต็มที่

\vspace{\baselineskip}

\begin{figure}[h]
	\centering
	\includegraphics[width=0.8\linewidth]{MG4}
	\caption{หน้ามอบหมายกรรมการให้กิจกรรมการแข่งขัน}
\end{figure}

\indent นหน้านี้ ผู้จัดการสามารถเลือกการประกวด ผู้จัดการต้องใช้เมนู dropdown ด้านบนสุดเพื่อ เลือกการประกวด ที่ต้องการจะมอบหมายกรรมการก่อน ตรวจสอบกรรมการปัจจุบัน หลังจากเลือกการประกวดแล้ว ระบบจะแสดง "กรรมการปัจจุบัน" ที่ถูกมอบหมายให้งานนี้แล้วทำให้ผู้จัดการทราบว่าตอนนี้มีใครอยู่ในทีมตัดสินบ้าง

\newpage

\begin{figure}[h]
	\centering
	\includegraphics[width=0.8\linewidth]{MG6}
	\caption{หน้าประวติและผลการประกวด}
\end{figure}

\indent หน้านี้เปรียบเสมือนคลังเก็บข้อมูลสำหรับกิจกรรมการประกวดทั้งหมดที่ผู้จัดการเคยจัดและ เสร็จสิ้นไปแล้ว (เช่น ประกาศผลแล้ว หรือยกเลิกไป) ดูรายการประกวดที่จบไปแล้ว โดยปกติแล้ว หน้านี้จะแสดงรายการประกวดในอดีตทั้งหมดที่ผู้จัดการคนนี้เคยสร้างไว้ ทำให้สามารถย้อนกลับมาดูได้ว่าเคยจัดกิจกรรมอะไรไปบ้าง ดูผลสรุปการแข่งขัน คือส่วนที่สำคัญที่สุดครับ ในแต่ละรายการประกวดที่จบไปแล้ว จะมีปุ่ม "ดูผลสรุป" (ที่เป็นไอคอนรูปดวงตา) อยู่ข้างๆ เมื่อกดปุ่มนี้ จะมีหน้าต่าง Pop-up แสดงรายละเอียดผลการแข่งขันขึ้นมา ซึ่งจะบอกว่า ใครคือผู้ชนะ 3 อันดับแรก, ได้คะแนนรวมเท่าไหร่, และส่งปลาชื่ออะไรเข้าประกวด

\vspace{\baselineskip}

\begin{figure}[h]
	\centering
	\includegraphics[width=0.8\linewidth]{MG8}
	\caption{หน้าโปรไฟล์ผู้จัดการประกวด}
\end{figure}

\indent ในหน้านี้ ผู้จัดการสามารถทำจัดการข้อมูลส่วนตัว จัดการจะเห็นข้อมูลของตัวเอง เช่น รูปโปรไฟล์, ชื่อ, และอีเมล สามารถกดปุ่ม "แก้ไขข้อมูลส่วนตัว" เพื่อเข้าไป เปลี่ยนชื่อ-นามสกุล หรือ Username ของตัวเองได้ สามารถ คลิกที่รูปโปรไฟล์ เพื่ออัปโหลดรูปประจำตัวใหม่ได้ และยังสามารถภาพรวมและจัดการงานด่วน  ศูนย์ควบคุม จะมีสรุปตัวเลขสำคัญๆ ให้ดูอย่างรวดเร็ว เช่น ตอนนี้มี "ประกวดที่ดูแลอยู่" กี่รายการ, มี "ผู้สมัครรออนุมัติ" กี่คน, และมี "ประกวดที่เสร็จสิ้น" ไปแล้วกี่รายการ รายการที่ต้องจัดการ ส่วนนี้สำคัญมากจะแสดงรายการประกวดที่กำลังดำเนินการอยู่ (Active) ผู้จัดการสามารถดูสถานะคร่าวๆ และกดปุ่ม "จัดการ" ที่ท้ายรายการได้เลย เมื่อกดปุ่ม "จัดการ" ระบบจะพาผู้จัดการเข้าไปที่ "ห้องแข่งขัน" (Live Contest Room) ของการประกวดนั้นๆ ทันที เพื่อไปอนุมัติผู้สมัครหรือจัดการขั้นตอนต่อไป

\endgroup}{}
	
	% บทที่ 4 (ใช้หน้าใน Chapter3_28.tex เป็นหน้าบท)
	\thispagestyle{plain}
	\IfFileExists{Chapter3_28.tex}{%==================== chapter4.tex ====================

\clearpage
\thispagestyle{empty}

\begingroup
\fontsize{16pt}{19.2pt}\selectfont
\justifying
\XeTeXlinebreakskip=0pt plus 1pt minus 0.5pt
\setlength{\parindent}{1.5cm}
\setlength{\parskip}{0pt}

% ---------- หัวบท + เขียนสารบัญบทก่อนหัวข้อย่อย ----------
\phantomsection
\addcontentsline{toc}{chapter}{บทที่ 4 ผลการดำเนินงาน}
\begin{center}
	{\bfseries\fontsize{18pt}{21.6pt}\selectfont บทที่ 4}
\end{center}

\vspace{\baselineskip}

% ---------- ชื่อบท ----------
\begin{center}
	{\bfseries\fontsize{18pt}{21.6pt}\selectfont ผลการดำเนินงาน}
\end{center}

\vspace{\baselineskip}

% ---------- หัวข้อใหญ่ (ชิดซ้าย, หนา 16pt) ----------
\section*{ผลการทดสอบแบบจำลองปัญญาประดิษฐ์}
\addcontentsline{toc}{section}{ผลการทดสอบแบบจำลองปัญญาประดิษฐ์}

% ---------- เนื้อหา (จัดกระจายแบบไทย, ย่อหน้าแรก 1.5 ซม.) ----------
\indent ในการพัฒนาเว็บแอปพลิเคชัน “ศูนย์รวมการจัดประกวดปลากัดไทย” ผู้วิจัยได้สร้างระบบ
จัดเก็บข้อมูลการแข่งขัน ข่าวสาร และเชื่อมต่อกับแบบจำลองสำหรับการจำแนกสายพันธุ์ปลากัด ซึ่ง
โมเดลหลักที่เลือกใช้คือ ResNet50 (pre-trained บน ImageNet) พร้อมการปรับ Fine-tune
ให้เหมาะกับข้อมูลที่รวบรวมมา โดยก่อนการฝึกโมเดลมีการเตรียมข้อมูลภาพ ดังนี้

\begin{sloppypar}
	\begin{enumerate} %[start=4]  % เริ่มที่ 2 (ต่อจาก 1)
		\item \textbf{การตรวจสอบและกรองไฟล์ ภาพ:} ลือกเฉพาะไฟล์ ที่ รองรับ ได้แก่ .jpg, .jpeg, .png, .bmp,
		.webp
		\item \textbf{การทำความสะอาด (Cleaning):} เปิดภาพอย่างปลอดภัยด้วยไลบรารี PIL, แปลงเป็น RGB,ปรับขนาดด้านยาวไม่เกินที่กำหนด และบันทึกใหม่เป็น JPEG
		\item \textbf{การกำจัดภาพซ้ำ (Deduplication):} ใช้วิธี Average Hash เพื่อตรวจจับและตัดภาพที่ซ้ำหรือเกือบซ้ำ
		\item \textbf{การแบ่ง ชุด ข้อมูล (Splitting):} แบ่ง ข้อมูล เป็น Training set และ Validation set ในอัตรา 85:15 โดยสุ่มแยกต่อคลาส
	\end{enumerate}
\end{sloppypar}

\indent หลังการเตรียมข้อมูลได้ชุดข้อมูลรวมจำนวน 356 ภาพแบ่งเป็น Train 294 ภาพและ Validation 50 ภาพ

\begin{table}[h]
	\caption{จำนวนภาพปลากัดหลังการเตรียมข้อมูล}
	{\tablefont
		\setlength{\tabcolsep}{6pt}%
		\begin{tabularx}{\linewidth}{@{}
				>{\raggedright\arraybackslash}X
				>{\centering\arraybackslash}p{2.2cm}
				>{\centering\arraybackslash}p{2.6cm}
				>{\centering\arraybackslash}p{2.2cm}
				@{}}
			\Xhline{1.5pt}
			\bfseries คลาส & \bfseries Train & \bfseries Validation & \bfseries รวม \\
			\Xhline{0.5pt}
			ปลากัดพื้นบ้านภาคอีสานหางลาย & 101 & 17 & 118 \\
			\Xhline{0.5pt}
			ปลากัดพื้นบ้านภาคใต้ & 114 & 20 & 134 \\
			\Xhline{0.5pt}
			ปลากัดพื้นบ้านมหาชัย & 79 & 13 & 92 \\
			\Xhline{0.5pt}
			รวม & 294 & 50 & 344 \\
			\Xhline{1.5pt}
	\end{tabularx}}
\end{table}

\endgroup}{}
	\IfFileExists{Chapter3_29.tex}{%==================== chapter3_29.tex ====================

\clearpage
\thispagestyle{plain}

\begingroup
\fontsize{16pt}{19.2pt}\selectfont
\justifying
\XeTeXlinebreakskip=0pt plus 1pt minus 0.5pt
\setlength{\parindent}{1.5cm}
\setlength{\parskip}{0pt}

\section*{ผลการทดสอบโมเดล ResNet50}
\addcontentsline{toc}{section}{ผลการทดสอบโมเดล ResNet50}

\indent โมเดลที่ใช้ทดลองคือ ResNet50 โดยเปลี่ยนชั้นสุดท้ายให้รองรับ 3 คลาสและทำการ
Fine-tune ด้วย Optimizer แบบ AdamW พร้อม Cosine learning rate schedule และ Early
Stopping เพื่อป้องกัน Overfitting

\begin{sloppypar}
	\begin{enumerate}
		\item ค่า \textbf{Accuracy} บนชุด Validation สูงสุด = 1.0000
		\item ค่า \textbf{Macro-F1} บนชุด Validation สูงสุด = 1.0000
	\end{enumerate}
\end{sloppypar}

\indent การหยุดการฝึกเกิดขึ้นที่ Epoch 20 จากทั้งหมด 30 Epoch ตามเกณฑ์ Early Stopping กราฟการเรียนรู้ แสดงดังรูป ซึ่งเห็นได้ว่าค่า Loss ลดลงต่อเนื่องและ Accuracy
F1-score ของ Validation set เพิ่มขึ้นจนถึง 100\% นอกจากนี้ยังได้สร้าง Confusion Matrix ดังรูป ซึ่งผลลัพธ์แสดงว่าโมเดลสามารถจำแนกได้ถูกต้องครบทุกตัวอย่างในชุด Validation

\vspace{\baselineskip}

\section*{สรุปผลการทดสอบแบบจำลอง}
\addcontentsline{toc}{section}{สรุปผลการทดสอบแบบจำลอง}
\indent โมเดลสามารถจำแนกปลากัดทั้งสามกลุ่ม (ปลากัดพื้นบ้านภาคอีสานหางลาย, ปลากัดพื้นบ้านภาคใต้, ปลากัดพื้นบ้านมหาชัย) ได้อย่างถูกต้องบน
Validation set อย่างไรก็ตาม ขนาดข้อมูลยังค่อนข้างเล็กโดยเฉพาะคลาส ปลากัดพื้นบ้านมหาชัย ที่มีจำนวน
น้อย ซึ่งอาจทำให้โมเดลมี Bias และค่า Accuracy ที่ได้สูงอาจสะท้อน Overfitting ต่อโดเมนข้อมูล
ที่ใช้ ดังนั้นควรมีการทดสอบเพิ่มเติมกับชุดข้อมูลใหม่ที่ไม่ได้อยู่ในกระบวนการฝึก เพื่อประเมินความ
สามารถทั่วไป (generalization) ของโมเดล



\begin{figure}[h]
	\centering
	\includegraphics[width=0.42\linewidth]{GF2}
	\hfill
	\includegraphics[width=0.42\linewidth]{GF1}
	\caption{กราฟ Loss ของ Train และ Validation และ Confusion Matrix ของผลการทดสอบ}
\end{figure}

\endgroup


}{}
	\IfFileExists{Chapter3_29/1.tex}{%==================== chapter3_29/1.tex ====================

\clearpage
\thispagestyle{plain}

\begingroup
\fontsize{16pt}{19.2pt}\selectfont
\justifying
\XeTeXlinebreakskip=0pt plus 1pt minus 0.5pt
\setlength{\parindent}{1.5cm}
\setlength{\parskip}{0pt}



\section*{ขั้นตอนการทำงานการทดลอง}
\addcontentsline{toc}{section}{ขั้นตอนการทำงานการทดลอง}
\indent ระบบประกอบด้วย 4 ขั้นหลัก: (1) Data Preparation สแกนและทำความสะอาดรูป แปลงเป็น RGB/Resize แบ่ง Train/Validation และคัดรูปซ้ำด้วย Average Hash ก่อนบันทึกเป็นโครงสร้าง ImageFolder; (2) Data Loading ใช้ SafeImageFolder+DataLoader พร้อม Augmentation (RandomResizedCrop, Flip, ColorJitter, Affine) และ Normalization; (3) Training ใช้ ResNet50 (pretrained ImageNet) เป็นตัวเข้ารหัสคุณลักษณะ ร่วมกับหัวจำแนก (GlobalAvgPool+Linear) เทรนด้วย AdamW, Label Smoothing, Warmup+Cosine LR, Gradient Accumulation/Clipping, AMP (bf16/fp16), และรองรับ torch.compile; (4) Evaluation \& Artifacts ประเมินด้วย Accuracy และ F1-macro พร้อมรายงาน per-class/Confusion Matrix และบันทึก best.pt, last.pt, metrics.json, class\_to\_idx.json.


\begin{figure}[h]
	\centering
	\includegraphics[width=0.3\linewidth]{Image Classification Pipeline.drawio}
	\caption{Image Classification Pipeline}
\end{figure}

\newpage

\begin{sloppypar}
	\begin{enumerate} %[start=4]  % เริ่มที่ 2 (ต่อจาก 1)
		\item \textbf{รับข้อมูลเข้า:} รูปภาพถูกจัดมาจากหลายโฟลเดอร์ แทนแต่ละ “คลาส” ของป้ายกำกับ
		\item \textbf{Data Preparation:}
			\begin{enumerate}
				\item เปิดรูปอย่างปลอดภัย แปลงเป็น RGB/ย่อขนาด
				\item คัดรูปซ้ำแบบหยาบ (average hash)
				\item แบ่งเป็นชุดฝึก (train) และตรวจสอบ (val)
				\item บันทึกเป็นโครงสร้าง ImageFolder พร้อมสถิติการเตรียมข้อมูล
			\end{enumerate}
		\item \textbf{Data Loading:}
			\begin{enumerate}
			\item ใช้ SafeImageFolder + DataLoader โหลดรูปเป็นแบตช์
			\item ใส่ augmentation สำหรับ train และ normaliz ให้เข้ากับ ImageNet
			\end{enumerate}
		\item \textbf{โมเดล (Encoder + Head):}
			\begin{enumerate}
			\item ResNet-50 (pretrained) ทำหน้าที่ “เข้ารหัส” สกัดฟีเจอร์จากภาพ
			\item อัปเดตพารามิเตอร์ด้วย AdamW, ใช้ Warmup→Cosine LR
			\item รองรับ AMP (bf16/fp16), gradient accumulation/clip, และ (ถ้ามี) torch.compile
			\end{enumerate}
		\item \textbf{การฝึก (Training Loop):}
			\begin{enumerate}
			\item คำนวณ Cross-Entropy (มี label smoothing)
			\item ส่วนหัว (Global Avg Pool + Linear) แปลงฟีเจอร์เป็น logits ตามจำนวนคลาส
			\end{enumerate}
		\item \textbf{การประเมิน (Validation):}
			\begin{enumerate}
			\item วัด Accuracy, F1-macro, รายงานรายคลาส และ Confusion Matrix
			\item ใช้ F1-macro เลือกโมเดลที่ดีที่สุด และตรวจ Early Stopping
			\end{enumerate}
		\item \textbf{อาร์ติแฟกต์/ผลลัพธ์:}
					\begin{enumerate}
			\item เก็บ best.pt, last.pt, metrics.json, class\_to\_idx.json
			\item พร้อมนำไปทดสอบ/เสิร์ฟใช้งานต่อในขั้นตอนถัดไป
		\end{enumerate}
	\end{enumerate}
\end{sloppypar}

\clearpage}{}
	\IfFileExists{Chapter3_29/2.tex}{%==================== chapter3_29/1.tex ====================

\clearpage
\thispagestyle{plain}

\begingroup
\fontsize{16pt}{19.2pt}\selectfont
\justifying
\XeTeXlinebreakskip=0pt plus 1pt minus 0.5pt
\setlength{\parindent}{1.5cm}
\setlength{\parskip}{0pt}


\begin{figure}[h]
	\centering
	\includegraphics[width=0.2\linewidth]{ResNet-50.drawio}
	\caption{ขั้นตอนการทำงาน ResNet-50}
\end{figure}

\begin{sloppypar}
	\begin{enumerate} %[start=4]  % เริ่มที่ 2 (ต่อจาก 1)
		\item อินพุตภาพ 3×224×224 -> ผ่าน Conv 7×7, stride 2 แล้ว MaxPool 3×3, stride 2 เพื่อลดขนาดแผนที่คุณลักษณะ (feature map) อย่างรวดเร็ว
		\item เข้า สเตจของ Residual Bottleneck 4 ช่วงต่อเนื่อง (มีทางลัด residual ทุกบล็อก):
		\begin{enumerate}
			\item Stage1: ขนาดคงที่ 56×56, ช่องออก 256 (บล็อก ×3)
			\item Stage2: ลดสเกลเป็น 28×28 ที่บล็อกแรก (stride 2), ช่องออก 512 (บล็อก ×4)
			\item Stage3: ลดสเกลเป็น 14×14 ที่บล็อกแรก, ช่องออก 1024 (บล็อก ×6)
			\item Stage4: ลดสเกลเป็น 7×7 ที่บล็อกแรก, ช่องออก 2048 (บล็อก ×3)
			แต่ละบล็อกเป็นลำดับ 1×1 (ลด/ขยายช่อง) -> 3×3 (สกัดลวดลายเชิงพื้นที่) -> 1×1 (ขยายช่อง) แล้วบวกกับเส้นลัด (shortcut) เพื่อช่วยให้เทรนลึกได้
		\end{enumerate}
		\item ได้แผนที่คุณลักษณะสุดท้ายขนาด 2048×7×7 -> ทำ Global Average Pooling ให้เป็นเวกเตอร์ 2048-D
		\item ส่งเข้า ชั้น Linear (FC) แปลงเป็น logits = จำนวนคลาส
	\end{enumerate}
\end{sloppypar}}{}	
	% บทที่ 5 (ใช้หน้าใน Chapter3_30.tex เป็นหน้าบท)
	\thispagestyle{plain}
	\IfFileExists{Chapter3_30.tex}{%==================== chapter3_28.tex ====================

\clearpage
\thispagestyle{empty}

\begingroup
\fontsize{16pt}{19.2pt}\selectfont
\justifying
\XeTeXlinebreakskip=0pt plus 1pt minus 0.5pt
\setlength{\parindent}{1.5cm}
\setlength{\parskip}{0pt}

% ---------- หัวบท + เขียนสารบัญบทก่อนหัวข้อย่อย ----------
\phantomsection
\addcontentsline{toc}{chapter}{บทที่ 5 บทสรุป}
\begin{center}
	{\bfseries\fontsize{18pt}{21.6pt}\selectfont บทที่ 5}
\end{center}

\vspace{\baselineskip}

% ---------- ชื่อบท ----------
\begin{center}
	{\bfseries\fontsize{18pt}{21.6pt}\selectfont บทสรุป}
\end{center}

\vspace{\baselineskip}

\indent จากการศึกษาค้นคว้า เรื่องเว็บ แอปพลิเคชัน ศูนย์ รวมการจัด ประกวดปลากัด ไทย และมี
การนำ Ai มาใช้ในการช่วยตรวจสอบประเภทปลากัด ว่าเป็นปลากัดประเภทไหน โดยมีวัตถุประสงค์
เพื่อ พัฒนาระบบเว็บ แอปพลิเคชัน ศูนย์ รวมการจัด ประกวดปลากัด ไทย โดยเราได้ ศึกษาและพัฒนา
ระบบเป็น Web Application โดยใช้ React, Visual Studio code, Supabase
การสร้างโมเดลจาก ResNet-50 เป็นการนำมาใช้ ในการสร้างต้นแบบบน Web App
Appication เนื่องจากมีการสนับสนุนร่วมกับ React,Node.js,Express โดย Web App Appication
สามารถทำการแยกประเภทปลากัด ได้ 3 ประเภทในตอนนี้สามารถทำการจำแนกได้โดยการอัปโหลดรูปภาพ 3 รูปภาพแต่เราเอารูปภาพรูปแรกเพื่อให้โมเดลดูว่าคือปลากัดประเภทไหน

\vspace{\baselineskip}

% ---------- หัวข้อใหญ่ (ชิดซ้าย, หนา 16pt) ----------
\section*{ปัญหาและอุปสรรค}
\addcontentsline{toc}{section}{ปัญหาและอุปสรรค}

\begin{sloppypar}
	\begin{enumerate}
		\item แหล่งข้อมูลในการทำ Dataset ที่เป็นรูปภาพปลากัดของแต่ละประเภท หาได้ค่อนข้างยากเพราะ
		ต้องใช้ภาพจำนวนมากต่อปลา 1 ตัว จึงต้องทำการขออณุญาตใช้ภาพจากกลุ่ม Facebook ชุมชน
		คนเลี้ยงปลากัด ที่อนุเคราะห์ให้ภาพปลากัดมาทำ Dataset
	\end{enumerate}
\end{sloppypar}

% ---------- หัวข้อใหญ่ (ชิดซ้าย, หนา 16pt) ----------
\section*{ข้อเสนอแนะ}
\addcontentsline{toc}{section}{ข้อเสนอแนะ}

\begin{sloppypar}
	\begin{enumerate}
		\item Web Application ทำให้สามารถรับชำระเงินค่าสมัครการเข้าร่วมการประกวด
		\item Web Application เปรียบเทียบปลากัดที่ชนะการประกวด กับปลากัดของตนเอง
		\item เพิ่มจำนวนคลาสให้ครบถ้วนตาม ตามประเภทปลากัดพื้นบ้านของไทย
		\item เพิ่มระบบการแจ้งเตือนภายนอก เพื่อให้ผู้เชี่ยวชาญหรือผู้จัดการประกวด ทราบได้ทันท่วงทีว่ามี
		กิจกรรมอะไรเข้ามา อย่างเช่น ผู้เชี่ยวมีปลากัดรอประเมิน ผู้จัดการประกวด มีกดอนุมัติเข้าร่วม
		การประกวด
	\end{enumerate}
\end{sloppypar}}{}
	
	% ===== ส่วนท้าย =====
	% บรรณานุกรม
	%\chapter*{บรรณานุกรม}
	%\addcontentsline{toc}{chapter}{บรรณานุกรม}
		\clearpage
	\phantomsection
	\addcontentsline{toc}{chapter}{บรรณานุกรม} % ใส่ในสารบัญครั้งเดียว
	\thispagestyle{empty}
	
	\vspace*{\fill}
	\begin{center}
		{\bfseries\fontsize{16pt}{19.2pt}\selectfont บรรณานุกรม}
	\end{center}
	\vspace*{\fill}
	
	\clearpage
	\thispagestyle{plain}
	%\nocite{*}
	\printbibliography[heading=thai-bib]
	
	% ภาคผนวก
	% ===== ภาคผนวก (หน้าเปิดแบบกึ่งกลางทั้งหน้า) =====
	\clearpage
	\phantomsection
	\addcontentsline{toc}{chapter}{ภาคผนวก} % ใส่ในสารบัญครั้งเดียว
	\thispagestyle{empty}
	
	\vspace*{\fill}
	\begin{center}
		{\bfseries\fontsize{16pt}{19.2pt}\selectfont ภาคผนวก}
	\end{center}
	\vspace*{\fill}
	
	\clearpage
	% ตั้งหัว/เลขหน้าตามปกติสำหรับเนื้อหาภาคผนวก
	\pagestyle{plain}
	
	\IfFileExists{Appendix.tex}{%==================== Appendix.tex ====================

\clearpage
\thispagestyle{plain}

\begingroup
% เนื้อหาภาคผนวก: 16pt baseline ~19.2pt ตามสเปกเล่ม
\fontsize{16pt}{19.2pt}\selectfont
\justifying
\XeTeXlinebreakskip=0pt plus 1pt minus 0.5pt
\setlength{\parindent}{1.5cm}
\setlength{\parskip}{0pt}

% ---------- หัวข้อใหญ่ (ชิดซ้าย, หนา 16pt) ----------
\noindent{\bfseries\fontsize{16pt}{19.2pt}\selectfont ภาคผนวก ก}\par
\noindent{\bfseries\fontsize{16pt}{19.2pt}\selectfont รายละเอียดการทดสอบระบบ}\par

% ---------- เนื้อหา (จัดกระจายแบบไทย, ย่อหน้าแรก 1.5 ซม.) ----------
\indent ภาคผนวกนี้แสดงรายละเอียดการทดสอบระบบเว็บแอปพลิเคชันศูนย์รวมการจัดประกวดปลากัดไทย
รวมถึงผลการทดสอบในแต่ละด้านและข้อมูลเพิ่มเติมที่เกี่ยวข้อง

\noindent{\bfseries\fontsize{16pt}{19.2pt}\selectfont ก.1 การทดสอบการทำงานของระบบ}\par

\indent การทดสอบการทำงานของระบบครอบคลุมการทดสอบฟังก์ชันต่าง ๆ ของระบบ
รวมถึงการสมัครแข่งขัน การอัปโหลดรูปภาพปลากัด การประเมินด้วย AI
และการบันทึกผลการประกวด

\noindent{\bfseries\fontsize{16pt}{19.2pt}\selectfont ก.2 การทดสอบประสิทธิภาพ}\par

\indent การทดสอบประสิทธิภาพของระบบใช้เครื่องมือ Load Testing
เพื่อจำลองการใช้งานจริงและวัดประสิทธิภาพของระบบในสภาวะต่าง ๆ

\noindent{\bfseries\fontsize{16pt}{19.2pt}\selectfont ภาคผนวก ข}\par
\noindent{\bfseries\fontsize{16pt}{19.2pt}\selectfont รหัสต้นฉบับโปรแกรม}\par

\indent ภาคผนวกนี้แสดงรหัสต้นฉบับของโปรแกรมที่สำคัญบางส่วน
เพื่อให้ผู้อ่านสามารถเข้าใจการทำงานของระบบได้ดียิ่งขึ้น

\par\endgroup
\clearpage

%================== จบ Appendix.tex ====================}{}

	%\IfFileExists{Appendix1.tex}{

\noindent{\bfseries\fontsize{16pt}{19.2pt}\selectfont ผลการทดสอบโมเดล}\par


\indent โมเดลสามารถจำแนกปลากัดทั้งสามกลุ่ม (ปลากัดพื้นบ้านภาคอีสานหางลาย, ปลากัดพื้นบ้านภาคใต้, ปลากัดพื้นบ้านมหาชัย) ได้อย่างถูกต้องบน
Validation set อย่างไรก็ตาม ขนาดข้อมูลยังค่อนข้างเล็กโดยเฉพาะคลาส ปลากัดพื้นบ้านมหาชัย ที่มีจำนวน
น้อย ซึ่งอาจทำให้โมเดลมี Bias และค่า Accuracy ที่ได้สูงอาจสะท้อน Overfitting ต่อโดเมนข้อมูล
ที่ใช้ ดังนั้นควรมีการทดสอบเพิ่มเติมกับชุดข้อมูลใหม่ที่ไม่ได้อยู่ในกระบวนการฝึก เพื่อประเมินความ
สามารถทั่วไป (generalization) ของโมเดล

\vspace{\baselineskip}

\begin{figure}[h]
	\centering
	\includegraphics[width=0.47\linewidth]{GF2}
	\hfill
	\includegraphics[width=0.47\linewidth]{GF1}
	\caption{กราฟ Loss ของ Train และ Validation และ Confusion Matrix ของผลการทดสอบ}
\end{figure}

\par\endgroup
\clearpage

%================== จบ Appendix.tex ====================}{}
	\IfFileExists{Appendix1_1.tex}{

\noindent{\bfseries\fontsize{16pt}{19.2pt}\selectfont เกณฑ์การตัดสินการประกวดปลากัดป่าในประเทศไทย}\par

% --- ตาราง: ลักษณะเฉพาะของปลากัดป่าภาคใต้และเกณฑ์การให้คะแนนในการประกวด (บังคับ 13pt) ---
\begingroup
\renewcommand{\arraystretch}{1.15}
\setlength{\arrayrulewidth}{0.5pt}

% ถ้ามี hook \AtBeginEnvironment{tabularx}{\tablefont} อยู่ ให้กำหนด \tablefont เฉพาะที่นี่เป็น 13pt
\makeatletter
\@ifundefined{tablefont}{\newcommand{\tablefont}{}}{}%
\makeatother
\renewcommand{\tablefont}{\fontsize{13pt}{15.6pt}\selectfont} % << ฟอนต์คงที่ 13pt

\begin{table}[h]
	\captionsetup{justification=raggedright, singlelinecheck=false,
		labelfont=bf, textfont=bf}
	\caption{ลักษณะเฉพาะของปลากัดป่าภาคใต้และเกณฑ์การให้คะแนนในการประกวด}
	\centering
	
	{\tablefont % ย้ำอีกชั้นให้แน่ใจว่าเนื้อหาในตารางใช้ 13pt
		\begin{tabularx}{\textwidth}{@{}>{\raggedright\arraybackslash}p{2.8cm}
				>{\raggedright\arraybackslash}X
				>{\centering\arraybackslash}p{1.6cm}@{}}
			\Xhline{1.5pt}
			\bfseries ลักษณะ & \bfseries ลักษณะเด่นตามมาตรฐาน & \bfseries คะแนน \\
			\hline
			ส่วนหัวและลำตัว &
			สันหัวเป็นเขม่าดำ ที่แก้มมีขีดสีเขียวถึงฟ้าสองขีด บางตัวอาจมีสีเขียวเคลือบเต็มแก้ม;
			ลำตัวทรงกระบอก สัดส่วนสมดุล & 10 \\
			\hline
			ลำตัวและเกล็ด &
			เกล็ดขนาดเล็กถึงปานกลาง เรียงแนบลำตัวสม่ำเสมอ ปลายเกล็ดมีประกายเขียว–ฟ้าตามธรรมชาติ & 10 \\
			\hline
			ครีบหลัง (Dorsal Fin) &
			โคนกว้าง แผ่ได้เต็ม แนวครีบเรียงเป็นระเบียบ ปลายครีบแหลมเล็กน้อย ขอบครีบเรียบไม่บิดงอ & 15 \\
			\hline
			ครีบก้น (Anal Fin) &
			ครีบยาวต่อเนื่อง ขอบครีบสม่ำเสมอ ไม่ฉีกขาดหรือบิดงอ รูปทรงกลมกลืนกับลำตัว & 15 \\
			\hline
			ครีบเอว/ครีบท้อง (Pelvic Fin) &
			เป็นคู่สมมาตร เรียวยาว ปลายครีบมีแต้มสีขาว/ฟ้าได้ กางได้ดี & 5 \\
			\hline
			ครีบหาง (Caudal Fin) &
			ทรงกลมหรือมน แผ่ได้เต็ม ขอบหางเรียบสมมาตร ไม่บิดงอ & 10 \\
			\hline
			การทรงตัวและการว่ายน้ำ &
			ทรงตัวดี ว่ายเป็นจังหวะต่อเนื่อง ไม่เอียงหรืองอผิดปกติ & 10 \\
			\hline
			การพองสู้ &
			กางครีบเต็ม แผ่นปิดเหงือก (เหงือก) แผ่ได้ดี ตอบสนองต่อสิ่งเร้าชัดเจน & 10 \\
			\hline
			ภาพรวม &
			ความสมบูรณ์ของร่างกาย ความกลมกลืนของสัดส่วนและสีสัน ความแข็งแรงโดยรวม & 15 \\
			\Xhline{0.5pt}
			\bfseries คะแนนรวมทั้งสิ้น & & \bfseries 100 \\
			\Xhline{1.5pt}
		\end{tabularx}
	}% end \tablefont
	%\caption*{ที่มา: อรุณี รอดลอย, 2018, 128 69}
\end{table}
\endgroup}{}
	\IfFileExists{Appendix1_2.tex}{

% --- ตาราง: ลักษณะเฉพาะของปลากัดป่าภาคใต้และเกณฑ์การให้คะแนนในการประกวด (บังคับ 13pt) ---
\begingroup
\renewcommand{\arraystretch}{1.15}
\setlength{\arrayrulewidth}{0.5pt}

% ถ้ามี hook \AtBeginEnvironment{tabularx}{\tablefont} อยู่ ให้กำหนด \tablefont เฉพาะที่นี่เป็น 13pt
\makeatletter
\@ifundefined{tablefont}{\newcommand{\tablefont}{}}{}%
\makeatother
\renewcommand{\tablefont}{\fontsize{13pt}{15.6pt}\selectfont} % << ฟอนต์คงที่ 13pt

\begin{table}[h]
	\captionsetup{justification=raggedright, singlelinecheck=false,
		labelfont=bf, textfont=bf}
	\caption{ลักษณะเฉพาะของปลากัดป่าภาคอีสานและเกณฑ์การให้คะแนนในการประกวด}
	\centering
	
	{\tablefont % ย้ำอีกชั้นให้แน่ใจว่าเนื้อหาในตารางใช้ 13pt
		\begin{tabularx}{\textwidth}{@{}>{\raggedright\arraybackslash}p{2.8cm}
				>{\raggedright\arraybackslash}X
				>{\centering\arraybackslash}p{1.6cm}@{}}
			\Xhline{1.5pt}
			\bfseries ลักษณะ & \bfseries ลักษณะเด่นตามมาตรฐาน & \bfseries คะแนน \\
			\hline
			ส่วนหัวและลำตัว &
			สันหัวเป็นเขม่าดำ สีของเกล็ดมีสีเขียววาวขึ้นบริเวณแผ่นปิดเหงือก
			(แก้ม) ลำตัวทรงกระบอกเรียวยาวกว่าชนิดอื่น (ในกลุ่มก่อหวอด) & 10 \\
			\hline
			แก้มและเกล็ด &
			เกล็ดเรียงแน่นเป็นระเบียบตลอดทั้งลำตัว สีของเกล็ดมีโทนสีเขียว
			ถือว่าเป็นลักษณะเด่น & 15 \\
			\hline
			ครีบหลังหรือกระโดง
			(Dorsal Fin) &
			มีก้านครีบเดี่ยว ก้านครีบมีโทนสีน้ำตาลถึงดำ เยื่อครีบมีพื้นสีเขียว
			หรือฟ้า มีลายสีดำ & 10 \\
			\hline
			ครีบก้นหรือชายน้ำ
			(Anal Fin) &
			ก้านครีบเดี่ยว พื้นเนื้อและก้านครีบมีโทนสีแดง มีแถบสีเขียวหรือ
			ฟ้าแซมระหว่างก้านครีบจากโคนถึงปลายของแต่ละเส้น & 10 \\
			\hline
			ครีบท้องหรือตะเกียบ
			(Pelvic fin) &
			เป็นครีบคู่ต้องยาวเรียว มีโทนสีแดง ก้านครีบเส้นแรกมีสีดำ หรือสี
			น้ำตาลแดง หรือสีขาวสะท้อนสีฟ้า ปลายตะเกียบต้องมีสีขาว และ
			ไม่แตก & 10 \\
			\hline
			ครีบหาง
			(Caudal Fin) &
			ก้านครีบแตกสองเท่านั้น พื้นเนื้อและก้านครีบเป็นสีแดง มีแถบสี
			เขียวหรือฟ้าแซมระหว่างก้านครีบลากจากโคนก้านไปจนสุดปลาย
			ก้านหางของแต่ละเส้น และในระหว่างปลายก้านครีบหางที่แตก
			สองจะมีสีเขียวถึงฟ้าแซมทุกช่อง & 15 \\
			\hline
			การพองสู้และการว่ายน้ำ &
			การว่ายน้ำสง่างาม ปราดเปรียว ปลาควรพองสู้ & 10 \\
			\hline
			ภาพรวม &
			ความสมบูรณ์ของร่างกาย ความกลมกลืนของสัดส่วนและสีสัน ความแข็งแรงโดยรวม & 20 \\
			\Xhline{0.5pt}
			\bfseries คะแนนรวมทั้งสิ้น & & \bfseries 100 \\
			\Xhline{1.5pt}
		\end{tabularx}
	}% end \tablefont
	%\caption*{ที่มา: อรุณี รอดลอย, 2018, 128 69}
\end{table}
\endgroup}{}
	\IfFileExists{Appendix1_3.tex}{% --- ตาราง: ลักษณะเฉพาะของปลากัดป่าภาคใต้และเกณฑ์การให้คะแนนในการประกวด (บังคับ 13pt) ---
\begingroup
\renewcommand{\arraystretch}{1.15}
\setlength{\arrayrulewidth}{0.5pt}

% ถ้ามี hook \AtBeginEnvironment{tabularx}{\tablefont} อยู่ ให้กำหนด \tablefont เฉพาะที่นี่เป็น 13pt
\makeatletter
\@ifundefined{tablefont}{\newcommand{\tablefont}{}}{}%
\makeatother
\renewcommand{\tablefont}{\fontsize{13pt}{15.6pt}\selectfont} % << ฟอนต์คงที่ 13pt

\begin{table}[h]
	\captionsetup{justification=raggedright, singlelinecheck=false,
		labelfont=bf, textfont=bf}
	\caption{ลักษณะเฉพาะของปลากัดป่าภาคกลางและภาคเหนือและเกณฑ์การให้คะแนนในการประกวด}
	\centering
	
	{\tablefont % ย้ำอีกชั้นให้แน่ใจว่าเนื้อหาในตารางใช้ 13pt
		\begin{tabularx}{\textwidth}{@{}>{\raggedright\arraybackslash}p{2.8cm}
				>{\raggedright\arraybackslash}X
				>{\centering\arraybackslash}p{1.6cm}@{}}
			\Xhline{1.5pt}
			\bfseries ลักษณะ & \bfseries ลักษณะเด่นตามมาตรฐาน & \bfseries คะแนน \\
			\hline
			ส่วนหัวและลำตัว &
			ควรจะมีสีเทาถึงสีดำ (ลักษณะเด่น)
			ลำตัวทรงกระบอก หรือแบนข้างเล็กน้อย
			ลำตัวทรงกระบอก สัดส่วนสมดุล & 10 \\
			\hline
			แก้มและเกล็ด &
			แก้มควรจะมีขีดสีแดง หรือสีส้ม หรือสีเงิน 1-2 ขีด (ลักษณะ
			เด่น) เกล็ด สีของเกล็ดส่วนบนลำตัวมีสีเขียวหรือฟ้าหรือเกล็ด
			ไม่มีสี หรือมีจุดสีเขียวหรือสีฟ้าขึ้นประปรายบริเวณหลัง
			จำนวนน้อยถือว่าเป็นลักษณะเด่น & 15 \\
			\hline
			ครีบหลัง (Dorsal Fin)ครีบหลังหรือกระโดง
			(Dorsal Fin) &
			มีก้านครีบเดี่ยว พื้นเนื้อ (เยื่อครีบ) มีพื้นสีเขียวหรือฟ้า และมี
			ลาย อาจพบปลายครีบมีสีแดง & 10 \\
			\hline
			ครีบก้นหรือชายน้ำ
			(Anal Fin) &
			มีเส้นก้านครีบเดียว (ไม่แตกสอง) พื้นเนื้อเป็นโทนสีแดง มีสี
			เขียว หรือฟ้าขึ้นแซมปลายขอบของครีบ ส่วนปลายชายน้ำ
			(ชายธง) ควรมีสีแดงตลอดเส้น & 15 \\
			\hline
			ครีบท้องหรือตะเกียบ
			(Pelvic fin) &
			ก้านครีบเดี่ยว (ไม่แตกสอง) ก้านหน้าตะเกียบสีดำ ตะเกียบมี
			สีแดง ปลายก้านสีขาว & 5 \\
			\hline
			ครีบหาง (Caudal Fin) &
			ก้านครีบแตกสองเท่านั้น พื้นและก้านมีสีแดง ขอบหางมีสีดำ
			ระหว่างก้านครีบมีเส้นสีเขียวหรือฟ้าแซม ลากจากโคนหางไม่
			ถึงสุดปลายหางถือเป็นลักษณะเด่น & 15 \\
			\hline
			การพองสู้และการว่ายน้ำ &
			การว่ายน้ำสง่างาม ปราดเปรียว ปลาควรพองสู้ & 10 \\
			\hline
			ภาพรวม &
			ความสง่างาม ความสมบูรณ์ ความมีเสน่ห์ & 20 \\
			\Xhline{0.5pt}
			\bfseries คะแนนรวมทั้งสิ้น & & \bfseries 100 \\
			\Xhline{1.5pt}
		\end{tabularx}
	}% end \tablefont
	%\caption*{ที่มา: อรุณี รอดลอย, 2018, 128 69}
\end{table}
\endgroup}{}
	\IfFileExists{Appendix1_4.tex}{% --- ตาราง: ลักษณะเฉพาะของปลากัดป่าภาคใต้และเกณฑ์การให้คะแนนในการประกวด (บังคับ 13pt) ---
\begingroup
\renewcommand{\arraystretch}{1.15}
\setlength{\arrayrulewidth}{0.5pt}

% ถ้ามี hook \AtBeginEnvironment{tabularx}{\tablefont} อยู่ ให้กำหนด \tablefont เฉพาะที่นี่เป็น 13pt
\makeatletter
\@ifundefined{tablefont}{\newcommand{\tablefont}{}}{}%
\makeatother
\renewcommand{\tablefont}{\fontsize{13pt}{15.6pt}\selectfont} % << ฟอนต์คงที่ 13pt

\begin{table}[h]
	\captionsetup{justification=raggedright, singlelinecheck=false,
		labelfont=bf, textfont=bf}
	\caption{ลักษณะเฉพาะของปลากัดป่ามหาชัยและเกณฑ์การให้คะแนนในการประกวด}
	\centering
	
	{\tablefont % ย้ำอีกชั้นให้แน่ใจว่าเนื้อหาในตารางใช้ 13pt
		\begin{tabularx}{\textwidth}{@{}>{\raggedright\arraybackslash}p{2.8cm}
				>{\raggedright\arraybackslash}X
				>{\centering\arraybackslash}p{1.6cm}@{}}
			\Xhline{1.5pt}
			\bfseries ลักษณะ & \bfseries ลักษณะเด่นตามมาตรฐาน & \bfseries คะแนน \\
			\hline
			ส่วนหัวและลำตัว &
			สันหัวเป็นเขม่าดำ แก้มมีขีดสีเขียวหรือฟ้าแวววาว 2 ขีด บางตัว
			พบมีสีแก้มเคลือบเขียวไปถึงคาง ลำตัวทรงกระบอกยาว และ
			แบนข้างเล็กน้อย & 10 \\
			\hline
			แก้มและเกล็ด &
			พื้นผิวที่ลำตัวมีสีน้ำตาลแดงถึงดำ สีของเกล็ดมีสีเขียวถึงฟ้าแวว
			วาวเรียงเป็นระเบียบเป็นแนวคล้ายแถวเมล็ดข้าวโพดบนฝัก
			เวลาที่กระพุ้งแก้มเปิดจะเห็นแผ่นปิดเหงือกชั้นใน
			(branchiostegal membrane) มีสีน้ำตาลแดงถึงดำ ซึ่งในปลา
			กัดชนิดอื่นจะมีแผ่นปิดเหงือกชั้นในสีแดงสดถึงแดงเข้มเลือดหมู & 15 \\
			\hline
			ครีบหลัง (Dorsal Fin)ครีบหลังหรือกระโดง
			(Dorsal Fin) &
			ก้านครีบแตก 1 ถึง 2 (ถ้าแตก 1 ถือว่าเป็นลักษณะเด่น) ก้าน
			ครีบสีน้ำตาลถึงดำ เยื่อครีบมีพื้นสีเขียวหรือฟ้า ลายสีดำ & 10 \\
			\hline
			ครีบก้นหรือชายน้ำ
			(Anal Fin) &
			ก้านครีบเดี่ยว พื้นเนื้อและก้านครีบมีโทนสีน้ำตาลแดงถึงดำ มี
			แถบสีเขียวหรือฟ้าแซมจากโคนถึงปลายระหว่างก้านครีบของแต่
			ละเส้น & 10 \\
			\hline
			ครีบท้องหรือตะเกียบ
			(Pelvic fin) &
			เป็นครีบคู่ พื้นตะเกียบเป็นโทนสีน้ำตาลแดงถึงดำ ก้านครีบเส้น
			แรกมีสีเขียวถึงฟ้า ปลายตะเกียบสีขาวและไม่แตก & 10 \\
			\hline
			ครีบหาง (Caudal Fin) &
			รูปทรงของครีบเป็นทรงพัด หรือทรงใบโพธิ์ ก้านครีบมีทั้งแตก 2
			ถึง 4 (ถ้าแตก 2 ถือเป็นลักษณะเด่น) พื้นเนื้อและก้านครีบโทนสี
			น้ำตาลแดงถึงดำ มีแถบสีเขียวหรือฟ้าแซมระหว่างก้านครีบลาก
			จากโคนก้านไปจนสุดปลายก้านหางของแต่ละเส้น และใน
			ระหว่างปลายก้านครีบหางที่แตกสองจะมีสีเขียวถึงฟ้าแซมทุก
			ช่อง & 15 \\
			\hline
			การพองสู้และการว่ายน้ำ &
			การว่ายน้ำสง่างาม ปราดเปรียว ปลาควรพองสู้ & 10 \\
			\hline
			ภาพรวม &
			ความสง่างาม ความสมบูรณ์ ความมีเสน่ห์ & 20 \\
			\Xhline{0.5pt}
			\bfseries คะแนนรวมทั้งสิ้น & & \bfseries 100 \\
			\Xhline{1.5pt}
		\end{tabularx}
	}% end \tablefont
	%\caption*{ที่มา: อรุณี รอดลอย, 2018, 128 69}
\end{table}
\endgroup}{}
	\IfFileExists{Appendix1_5.tex}{% --- ตาราง: ลักษณะเฉพาะของปลากัดป่าภาคใต้และเกณฑ์การให้คะแนนในการประกวด (บังคับ 13pt) ---
\begingroup
\renewcommand{\arraystretch}{1.15}
\setlength{\arrayrulewidth}{0.5pt}

% ถ้ามี hook \AtBeginEnvironment{tabularx}{\tablefont} อยู่ ให้กำหนด \tablefont เฉพาะที่นี่เป็น 13pt
\makeatletter
\@ifundefined{tablefont}{\newcommand{\tablefont}{}}{}%
\makeatother
\renewcommand{\tablefont}{\fontsize{13pt}{15.6pt}\selectfont} % << ฟอนต์คงที่ 13pt

\begin{table}[h]
	\captionsetup{justification=raggedright, singlelinecheck=false,
		labelfont=bf, textfont=bf}
	\caption{ลักษณะเฉพาะของปลากัดป่าภาคตะวันออกและเกณฑ์การให้คะแนนในการประกวด}
	\centering
	
	{\tablefont % ย้ำอีกชั้นให้แน่ใจว่าเนื้อหาในตารางใช้ 13pt
		\begin{tabularx}{\textwidth}{@{}>{\raggedright\arraybackslash}p{2.8cm}
				>{\raggedright\arraybackslash}X
				>{\centering\arraybackslash}p{1.6cm}@{}}
			\Xhline{1.5pt}
			\bfseries ลักษณะ & \bfseries ลักษณะเด่นตามมาตรฐาน & \bfseries คะแนน \\
			\hline
			ส่วนหัวและลำตัว &
			ส่วนหัวเป็นเขม่าดำ ที่แก้มมีจุดหรือขีดสีแดง 1 - 2 ขีด หรือไม่
			มีก็ได้ ลำตัวทรงกระบอก & 10 \\
			\hline
			แก้มและเกล็ด &
			เกล็ดมีสีเขียวถึงฟ้าเข้ม ขึ้นเรียงเต็มตัวหรือขึ้นประปรายก็ได้
			แล้วแต่ลักษณะเฉพาะแหล่งนั้น ๆ แต่ถ้าสีของเกล็ดขึ้นเรียงกัน
			เป็นแถวชัดเจนถือเป็นลักษณะเด่น & 15 \\
			\hline
			ครีบหลัง (Dorsal Fin)ครีบหลังหรือกระโดง
			(Dorsal Fin) &
			มีก้านครีบเดี่ยว ก้านครีบมีโทนสีน้ำตาลถึงดำ เยื่อครีบมีพื้นสี
			เขียวหรือฟ้า ลายสีดำ อาจพบปลายครีบมีสีแดง & 10 \\
			\hline
			ครีบก้นหรือชายน้ำ
			(Anal Fin) &
			พื้นเนื้อและก้านครีบเป็นสีแดง ระหว่างทุกก้านมีสีเขียวหรือฟ้า
			ขึ้นแซม (เส้นสีเขียวหรือฟ้านี้จะขึ้นระหว่างปลายก้านจนถึงโคน
			ก้านหรือไม่ถึงก็ได้ แต่ถ้าถึงโคนถือว่าเป็นลักษณะเด่น) จุดเด่น
			ที่สำคัญคือ ปลายชายน้ำของครีบก้น (ชายธง) ต้องมีสีแดง
			คล้ายรูปหยดน้ำ หรือรูปปลายหอก & 15 \\
			\hline
			ครีบท้องหรือตะเกียบ
			(Pelvic fin) &
			ก้านครีบเดี่ยวเป็นครีบคู่ ก้านครีบเส้นแรกมีสีดำ พื้นตะเกียบมี
			สีแดง ปลายตะเกียบสีขาวและไม่แตก & 5 \\
			\hline
			ครีบหาง (Caudal Fin) &
			รูปทรงของครีบเป็นทรงพัด สีพื้นหางและก้านครีบเป็นสีแดง
			เข้ม ปลายก้านครีบแตกสองเท่านั้น ระหว่างก้านครีบมีเส้นสี
			เขียวหรือฟ้าแซม ลากจากโคนหางไม่ถึงสุดปลายหางถือเป็น
			ลักษณะเด่น ปลายหางมีสีแดง รูปทรงคล้ายพระจันทร์เสี้ยว
			หรือมักเรียกกันว่า “วงพระจันทร์” ถ้าวงพระจันทร์มีขนาด 1
			ใน 3 ส่วน ของครีบหางถือว่าเป็นลักษณะเด่น & 15 \\
			\hline
			การพองสู้และการว่ายน้ำ &
			การว่ายน้ำสง่างาม ปราดเปรียว ปลาควรพองสู้ & 10 \\
			\hline
			ภาพรวม &
			ความสง่างาม ความสมบูรณ์ ความมีเสน่ห์ & 20 \\
			\Xhline{0.5pt}
			\bfseries คะแนนรวมทั้งสิ้น & & \bfseries 100 \\
			\Xhline{1.5pt}
		\end{tabularx}
	}% end \tablefont
	%\caption*{ที่มา: อรุณี รอดลอย, 2018, 128 69}
\end{table}
\endgroup}{}
	\IfFileExists{Appendix1_6.tex}{% --- ตาราง: ลักษณะเฉพาะของปลากัดป่าภาคใต้และเกณฑ์การให้คะแนนในการประกวด (บังคับ 13pt) ---
\begingroup
\renewcommand{\arraystretch}{1.15}
\setlength{\arrayrulewidth}{0.5pt}

% ถ้ามี hook \AtBeginEnvironment{tabularx}{\tablefont} อยู่ ให้กำหนด \tablefont เฉพาะที่นี่เป็น 13pt
\makeatletter
\@ifundefined{tablefont}{\newcommand{\tablefont}{}}{}%
\makeatother
\renewcommand{\tablefont}{\fontsize{13pt}{15.6pt}\selectfont} % << ฟอนต์คงที่ 13pt

\begin{table}[h]
	\captionsetup{justification=raggedright, singlelinecheck=false,
		labelfont=bf, textfont=bf}
	\caption{ลักษณะเฉพาะของปลากัดป่าสายพัฒนาและเกณฑ์การให้คะแนนในการประกวด}
	\centering
	
	{\tablefont % ย้ำอีกชั้นให้แน่ใจว่าเนื้อหาในตารางใช้ 13pt
		\begin{tabularx}{\textwidth}{@{}>{\raggedright\arraybackslash}p{2.8cm}
				>{\raggedright\arraybackslash}X
				>{\centering\arraybackslash}p{1.6cm}@{}}
			\Xhline{1.5pt}
			\bfseries ลักษณะ & \bfseries ลักษณะเด่นตามมาตรฐาน & \bfseries คะแนน \\
			\hline
			ส่วนหัวและลำตัว &
				ลำตัวทรงกระบอก (รูปทรงเป็นลักษณะปลาป่ามากที่สุด) & 20 \\
			\hline
			แก้มและเกล็ด &
			สี และลวดลายต้องชัดเจน ทั้ง 2 ข้าง & 20 \\
			\hline
			ครีบหลัง (Dorsal Fin)ครีบหลังหรือกระโดง
			(Dorsal Fin) &
			ก้านครีบเดี่ยว & 5 \\
			\hline
			ครีบก้นหรือชายน้ำ
			(Anal Fin) &
			ก้านครีบเดี่ยว ครีบเหยียดตรง เสมอกันทั้ง 2 ข้าง & 5 \\
			\hline
			ครีบท้องหรือตะเกียบ
			(Pelvic fin) &
			รูปทรงของครีบเป็นทรงใบโพธิ์ หรือพัด ก้านครีบแตกสอง ถือว่า
			เป็นลักษณะเด่น & 10 \\
			\hline
			ครีบหาง (Caudal Fin) &
			รูปทรงของครีบเป็นทรงใบโพธิ์ หรือพัด ก้านครีบแตกสอง ถือว่า
			เป็นลักษณะเด่น & 10 \\
			\hline
			การพองสู้และการว่ายน้ำ &
			การว่ายน้ำสง่างาม ปราดเปรียว ปลาควรพองสู้ & 10 \\
			\hline
			ภาพรวม &
			ความสง่างาม ความสมบูรณ์ ความมีเสน่ห์ & 20 \\
			\Xhline{0.5pt}
			\bfseries คะแนนรวมทั้งสิ้น & & \bfseries 100 \\
			\Xhline{1.5pt}
		\end{tabularx}
	}% end \tablefont
	%\caption*{ที่มา: อรุณี รอดลอย, 2018, 128 69}
\end{table}
\endgroup}{}
	\IfFileExists{Appendix1_7.tex}{%==================== Appendix.tex ====================

\clearpage
\thispagestyle{empty}

\begingroup
% เนื้อหาภาคผนวก: 16pt baseline ~19.2pt ตามสเปกเล่ม
\fontsize{16pt}{19.2pt}\selectfont
\justifying
\XeTeXlinebreakskip=0pt plus 1pt minus 0.5pt
\setlength{\parindent}{1.5cm}
\setlength{\parskip}{0pt}

% ---------- หัวข้อใหญ่ (ชิดซ้าย, หนา 16pt) ----------
\noindent{\bfseries\fontsize{16pt}{19.2pt}\selectfont ภาคผนวก ข}\par

\noindent{\bfseries\fontsize{16pt}{19.2pt}\selectfont โค้ดส่วนระบบหลังบ้าน}\par

\indent Api ส่วนระบบหลังบ้านของเว็บแอปพลิเคชันศูนย์รวมการจัดประกวดปลากัดไทย

\vspace{\baselineskip}

\begin{figure}[h]
	\centering
	\includegraphics[width=0.6\linewidth]{Screenshot from 2025-09-21 22-54-12}
	\caption{ฟังก์ชันสำหรับสมัครสมาชิก (Sign Up)}
\end{figure}

\indent ฟังก์ชัน signUp(userData) รับข้อมูลจากฟอร์ม (อีเมล รหัสผ่าน ชื่อ–นามสกุล และชื่อผู้ใช้) แล้วเรียก supabase.auth.signUp เพื่อสมัครสมาชิก พร้อมแนบข้อมูลโปรไฟล์ไปใน options.data โดยบังคับ role: 'user' เพื่อความปลอดภัยและให้ทริกเกอร์ในฐานข้อมูลสร้างโปรไฟล์อัตโนมัติ จากนั้นตรวจสอบผลตอบกลับ: ถ้ามีข้อผิดพลาดจะจับเคสอีเมลซ้ำแปลงเป็นข้อความไทยให้อ่านง่าย และกรณีอื่นก็ส่งต่อข้อความจาก Supabase; ยังมีการกันเหนียวหากโครงสร้าง data ผิดปกติหรือไม่มี user จะโยนข้อผิดพลาดทันที สุดท้ายคืน data (แม้ v2 อาจไม่ให้ session ตอนสมัคร) ให้เลเยอร์ที่เรียกใช้นำไปจัดการต่อ โดยมีการล็อกเพื่อดีบักแต่ควรระวังข้อมูลอ่อนไหวในสภาพแวดล้อมจริง.

\newpage

\begin{figure}[h]
	\centering
	\includegraphics[width=0.6\linewidth]{Screenshot from 2025-09-21 23-30-59}
	\caption{ฟังก์ชันสำหรับเข้าสู่ระบบ (Sign In)}
\end{figure}

\indent ฟังก์ชัน signIn(email, password) ทำงานโดยส่งอีเมลและรหัสผ่านไปที่ supabase.auth.signInWithPassword เพื่อขอเข้าสู่ระบบ ถ้าเกิดข้อผิดพลาดจะตรวจจับข้อความทั่วไปแล้วแปลงเป็นข้อความไทยอ่านง่าย เช่น “อีเมลหรือรหัสผ่านไม่ถูกต้อง” หรือ “กรุณายืนยันอีเมลก่อนเข้าสู่ระบบ” จากนั้นเมื่อเข้าสู่ระบบสำเร็จจะได้ data ซึ่งมี user และ session นำ user.id ไปค้นตาราง profiles ด้วย select('*').eq('id', user.id).single() เพื่อดึงโปรไฟล์ที่ตรงกัน หากหาไม่เจอหรือมีปัญหาในการอ่านข้อมูลจะโยนข้อผิดพลาดทันที สุดท้ายคืนค่าเป็นอ็อบเจ็กต์ที่ประกอบด้วย token (ใช้ session.access\_token) และ profile เพื่อให้เลเยอร์ Controller นำไปใช้งานต่อ.

\newpage

\begin{figure}[h]
	\centering
	\includegraphics[width=0.6\linewidth]{Screenshot from 2025-09-21 23-39-30}
	\caption{ฟังก์ชันสำหรับออกจากระบบ (Sign Out)}
\end{figure}

\indent ฟังก์ชัน signOut() เรียก supabase.auth.signOut() เพื่อทำลายเซสชันของผู้ใช้ที่ล็อกอินอยู่ หากมีข้อผิดพลาดระหว่างออกจากระบบก็จะโยน Error พร้อมข้อความอธิบายเพื่อให้ส่วนที่เรียกใช้งานจัดการต่อ เมื่อสำเร็จจะส่งออบเจ็กต์ { message: 'ออกจากระบบสำเร็จ' } กลับไปเป็นการยืนยันว่าได้ออกจากระบบเรียบร้อยแล้ว.

\vspace{\baselineskip}

\begin{figure}[h]
	\centering
	\includegraphics[width=0.6\linewidth]{Screenshot from 2025-09-21 23-43-28}
	\caption{ฟังก์ชันหาผู้เชี่ยวชาญที่เหมาะกับ “ประเภทปลากัด” ที่รับเข้ามา}
\end{figure}

\indent ฟังก์ชัน findMatchingExperts(fishType) มีหน้าที่หาผู้เชี่ยวชาญที่เหมาะกับ “ประเภทปลา” ที่รับเข้ามา โดยเริ่มจากคืน [] ทันทีถ้าไม่ได้ส่ง fishType มา จากนั้นใช้ supabaseAdmin ดึงโปรไฟล์ผู้ใช้ที่มี role = 'expert' พร้อมฟิลด์ specialities; ถ้ามี error หรือไม่พบผู้เชี่ยวชาญก็คืน [] แล้วจบ เมื่อได้รายการมา จะปรับ fishType เป็นตัวพิมพ์เล็กและตัดช่องว่างเพื่อให้เปรียบเทียบแบบไม่สนตัวพิมพ์ จากนั้นกรองรายชื่อผู้เชี่ยวชาญโดยเช็คว่า specialities (ต้องเป็นอาร์เรย์) มีรายการใด “เข้าคู่” กับชนิดปลาหรือไม่ ด้วยเงื่อนไขหลากหลาย: ตรงเท่ากันทุกตัวอักษร, เป็นส่วนหนึ่งของกันและกัน (ทั้ง “A อยู่ใน B” และ “B อยู่ใน A”) หรือ “คล้ายกัน” ตามตัวช่วย this.isSimilarType() ถ้ากรองแล้วไม่เหลือใคร จะ fallback เป็นคืนผู้เชี่ยวชาญทั้งหมดเพื่อให้ระบบสุ่ม/เลือกต่อไป แต่ถ้ามี ก็คืนเฉพาะกลุ่มที่แมตช์ พร้อม log จำนวนที่พบ ไหลทั้งหมดถูกครอบด้วย try/catch เพื่อจับข้อผิดพลาดและคืน [] หากเกิดปัญหา.

\vspace{\baselineskip}

\begin{figure}[h]
	\centering
	\includegraphics[width=0.6\linewidth]{Screenshot from 2025-09-21 23-51-49}
	\caption{ฟังก์ชันตรวจสอบภาระงานของผู้เชี่ยวชาญ}
\end{figure}

\indent ฟังก์ชัน getExpertWorkload(expertId) ใช้สำหรับคำนวณภาระงานของผู้เชี่ยวชาญแต่ละคน โดยเรียก supabaseAdmin.from('assignments').select('*', { count: 'exact', head: true }) เพื่อ “นับจำนวนแถว” อย่างเดียว (ไม่ดึงข้อมูลจริง) ของงานที่มี evaluator\_id ตรงกับ expertId และมีสถานะ pending ถ้ามี error ระหว่างดึงข้อมูลจะพิมพ์ log แล้วคืนค่า 999 เพื่อบอกให้ระบบมองว่าคนนี้งานล้น จึงไม่ควรมอบหมายงานเพิ่ม กรณีปกติจะคืนจำนวนงานค้าง (count) หรือ 0 หากไม่มีงานค้าง ทั้งหมดถูกครอบด้วย try/catch เพื่อกันข้อผิดพลาดและใช้ 999 เป็นค่า fallback เมื่อเกิดปัญหา.

\newpage

\begin{figure}[h]
	\centering
	\includegraphics[width=0.6\linewidth]{Screenshot from 2025-09-21 23-57-27}
	\caption{ฟังก์ชันเลือกผู้เชี่ยวชาญที่เหมาะสมที่สุด}
\end{figure}

\indent ฟังก์ชัน selectBestExpert(experts) ใช้เลือกผู้เชี่ยวชาญที่ “ภาระงานน้อยที่สุด” จากลิสต์ที่ส่งเข้ามา โดยถ้าไม่มีข้อมูลจะคืน null ทันที จากนั้นภายใน try มันจะเรียก getExpertWorkload แบบขนานด้วย Promise.all เพื่อเติมฟิลด์ workload ให้ผู้เชี่ยวชาญแต่ละคน แล้วนำผลมาจัดเรียงตาม workload จากน้อยไปมาก เลือกคนแรกเป็นผู้ชนะ (selectedExpert) พร้อม log ชื่อกับภาระงานไว้เพื่อดีบัก ก่อนจะส่งคนนั้นกลับไป หากเกิดข้อผิดพลาดระหว่างคำนวณหรือเรียงลำดับ จะจับใน catch และ fallback ด้วยการสุ่มเลือกใครสักคนจากลิสต์เพื่อให้ระบบยังเดินต่อได้.

\vspace{\baselineskip}

\begin{figure}[h]
	\centering
	\includegraphics[width=0.6\linewidth]{Screenshot from 2025-09-22 00-08-35}
	\caption{ฟังก์ชันสร้าง assignment อัตโนมัติสำหรับ submission ที่ approved}
\end{figure}

\indent ฟังก์ชัน createAutoAssignment(submissionId, fishType) มีหน้าที่สร้างงานมอบหมายผู้เชี่ยวชาญให้กับคำส่ง (submission) แบบอัตโนมัติ โดยเริ่มตรวจสอบว่ามี submissionId หรือไม่ ถ้าไม่มีจะ log และคืน null แล้วจบ จากนั้นภายใน try จะ log รายละเอียด แล้วเรียก findMatchingExperts(fishType) เพื่อดึงรายชื่อผู้เชี่ยวชาญที่เหมาะกับชนิดปลา ถ้าไม่พบใครก็คืน null ต่อมาเรียก selectBestExpert(matchingExperts) เพื่อเลือกคนที่เหมาะที่สุด (เช่น ภาระงานน้อยสุด) ถ้าเลือกไม่ได้ก็คืน null สุดท้ายตรวจสอบในตาราง assignments ว่างานสำหรับ submission\_id นั้นกับผู้เชี่ยวชาญคนที่เลือก (evaluator\_id) ถูกสร้างไว้แล้วหรือไม่ ถ้ามีอยู่แล้วจะคืนงานเดิมทันที (กันการสร้างซ้ำ) ก่อนจะไปขั้นถัดไปของการสร้างงานใหม่ (ซึ่งอยู่นอกสไนเป็ตนี้).

\vspace{\baselineskip}

\begin{figure}[h]
	\centering
	\includegraphics[width=0.6\linewidth]{Screenshot from 2025-09-22 00-17-40}
	\caption{ฟังก์ชันสร้าง assignment อัตโนมัติสำหรับ submission ที่ approved ต่อส่วนที่ 4 และ 5}
\end{figure}

\indent ส่วนต่อของ createAutoAssignment นี้ทำหน้าที่ “สร้างงานมอบหมายจริง” และแจ้งเตือนผู้เชี่ยวชาญที่ถูกเลือก โดยเริ่มจาก insert แถวใหม่ลงตาราง assignments ด้วย submission\_id, evaluator\_id, สถานะเริ่มต้น pending และเวลา assigned\_at เป็น ISO string แล้วเรียก .select().single() เพื่อให้ได้เรคอร์ดที่สร้างกลับมาทันที หาก insert ผิดพลาดจะ log แล้วคืน null จากนั้นพยายามส่งการแจ้งเตือนพื้นฐานให้ผู้เชี่ยวชาญผ่าน NotificationService.createNotification โดยแปลง fishType เป็นชื่อแสดงผลด้วย getFishTypeDisplayName และชี้ลิงก์ไปคิวงาน (/expert/queue) หากแจ้งเตือนล้มเหลวจะเพียงเตือน (warn) แต่ไม่ให้กระทบการมอบหมายงาน สุดท้าย log หมายเลข assignment ที่สร้างสำเร็จและคืนออบเจ็กต์ newAssignment ทั้งบล็อกห่อด้วย try/catch เพื่อคืน null เมื่อเกิดข้อผิดพลาดใด ๆ ระหว่างกระบวนการ.

\clearpage}{}
	\IfFileExists{Appendix1_8.tex}{%==================== Appendix.tex ====================

\clearpage
\thispagestyle{plain}

\begingroup
% เนื้อหาภาคผนวก: 16pt baseline ~19.2pt ตามสเปกเล่ม
\fontsize{16pt}{19.2pt}\selectfont
\justifying
\XeTeXlinebreakskip=0pt plus 1pt minus 0.5pt
\setlength{\parindent}{1.5cm}
\setlength{\parskip}{0pt}

\begin{figure}[h]
	\centering
	\includegraphics[width=0.6\linewidth]{Screenshot from 2025-09-22 00-24-03}
	\caption{ฟังก์ชันมอบหมายงานให้ผู้เชี่ยวชาญสำหรับ submissions ที่ approved แล้วแต่ยังไม่มี assignment}
\end{figure}

\indent ฟังก์ชัน processUnassignedSubmissions() มีหน้าที่สแกน “คำส่งที่ยังไม่ได้มอบหมายผู้เชี่ยวชาญ” แล้วสร้างงานมอบหมายให้อัตโนมัติ โดยเริ่มดึง submissions ที่สถานะเป็น pending หรือ approved และต้องเป็นงานประเมินคุณภาพเท่านั้น (contest\_id เป็น null) หากดึงผิดพลาดหรือไม่พบข้อมูลจะคืน 0 จากนั้นวนเช็กทีละรายการว่าในตาราง assignments มีงานของ submission\_id นั้นอยู่แล้วหรือไม่ด้วยการนับแบบ head: true, count: 'exact' ถ้าไม่มีจึงเก็บไว้ในลิสต์ประมวลผล แล้วไล่เรียก createAutoAssignment(submission.id, submission.fish\_type) เพื่อสร้าง assignment ทีละรายการ พร้อมนับจำนวนที่สร้างสำเร็จและพัก 100ms ระหว่างรอบเพื่อลดภาระระบบ สุดท้าย log ผลรวมและคืนค่าเป็นจำนวน assignments ที่สร้างได้ ทั้งกระบวนการห่อด้วย try/catch หากเกิดปัญหาจะคืน 0.

\clearpage}{}
	\IfFileExists{Appendix1_9.tex}{%==================== Appendix.tex ====================

\clearpage
\thispagestyle{empty}

\begingroup
% เนื้อหาภาคผนวก: 16pt baseline ~19.2pt ตามสเปกเล่ม
\fontsize{16pt}{19.2pt}\selectfont
\justifying
\XeTeXlinebreakskip=0pt plus 1pt minus 0.5pt
\setlength{\parindent}{1.5cm}
\setlength{\parskip}{0pt}

% ---------- หัวข้อใหญ่ (ชิดซ้าย, หนา 16pt) ----------
\noindent{\bfseries\fontsize{16pt}{19.2pt}\selectfont ภาคผนวก ค}\par

\noindent{\bfseries\fontsize{16pt}{19.2pt}\selectfont คู่มือการใช้งานโปรแกรม}\par

\indent วิธีการใช้งานในส่วนของผู้เลี้ยงปลากัด เพื่อเป็นแนวทางในการใช้งานในส่วนที่ซับซ้อน และอาจทำให้สับสนในการใช้งานได้ จึงจัดทำคู่มือการใช้งานขึ้นมาเพื่ออำนวยความสะดวก

\vspace{\baselineskip}

\begin{figure}[h]
	\centering
	\includegraphics[width=0.8\linewidth]{LG1}
	\caption{หน้าเข้าสู่ระบบ}
\end{figure}

\noindent{\bfseries\fontsize{16pt}{19.2pt}\selectfont วิธีใช้งานหน้าเข้าสู่ระบบ}\par

\begin{sloppypar}
	\begin{enumerate}
		\item เตรียม อีเมลหรือชื่อผู้ใช้ และ รหัสผ่าน ที่ลงทะเบียนไว้
		\item หากยังไม่มีบัญชี ให้กด “สมัครสมาชิก” ด้านล่างฟอร์ม
	\end{enumerate}
\end{sloppypar}

\noindent{\bfseries\fontsize{16pt}{19.2pt}\selectfont ขั้นตอนใช้งาน}\par

\begin{sloppypar}
	\begin{enumerate}
		\item เปิดหน้า เข้าสู่ระบบ (Login)
		\item ที่ช่อง อีเมลหรือชื่อผู้ใช้ พิมพ์อีเมลหรือชื่อผู้ใช้ของคุณ
		\item ที่ช่อง รหัสผ่าน พิมพ์รหัสผ่านของคุณ
		\item กดปุ่ม “เข้าสู่ระบบ” หรือกดปุ่ม Enter
	\end{enumerate}
\end{sloppypar}

\newpage

\vspace{\baselineskip}

\begin{figure}[h]
	\centering
	\includegraphics[width=0.8\linewidth]{RG1}
	\caption{หน้าสมัครสมาชิก}
\end{figure}

\noindent{\bfseries\fontsize{16pt}{19.2pt}\selectfont วิธีใช้งานหน้าหน้าสมัครสมาชิก}\par

\begin{sloppypar}
	\begin{enumerate}
		\item ใช้สำหรับสร้างบัญชีใหม่บนแพลตฟอร์ม
		\item กรอกข้อมูลตามช่องต่อไปนี้ให้ครบถ้วน
		\begin{enumerate}
			\item ชื่อ และ นามสกุล
			\item ชื่อผู้ใช้ (อย่างน้อย 3 ตัวอักษร)
			\item อีเมล (รูปแบบอีเมลที่ถูกต้อง)
			\item รหัสผ่าน (อย่างน้อย 6 ตัวอักษร)
			\item ยืนยันรหัสผ่าน (ต้องตรงกับรหัสผ่าน)
		\end{enumerate}
		\item ไอคอนรูปตาในช่องรหัสผ่านใช้เพื่อ แสดง/ซ่อน รหัสผ่านได้
		\item เมื่อกดส่งฟอร์ม ระบบจะป้องกันการกดซ้ำและแสดงสถานะ “กำลังดำเนินการ...”
		\item หากสมัครสำเร็จ จะมีแจ้งเตือนและพาไปหน้า เข้าสู่ระบบ 
		\item หากมีบัญชีอยู่แล้ว กดลิงก์ “เข้าสู่ระบบที่นี่” ด้านล่างฟอร์ม
	\end{enumerate}
\end{sloppypar}

\noindent{\bfseries\fontsize{16pt}{19.2pt}\selectfont ขั้นตอนใช้งาน}\par


\begin{sloppypar}
	\begin{enumerate}
		\item เปิดหน้า สมัครสมาชิก
		\item กรอก ชื่อ และ นามสกุล
		\item กรอก ชื่อผู้ใช้ (≥ 3 ตัวอักษร)
		\item กรอก อีเมล ให้ถูกต้องตามรูปแบบ
		\item กรอก รหัสผ่าน (≥ 6 ตัวอักษร)
		\begin{enumerate}
			\item หากต้องการตรวจสอบตัวสะกด กดไอคอนรูปตาเพื่อแสดงรหัสผ่าน
		\end{enumerate}
		\item กรอก ยืนยันรหัสผ่าน ให้ ตรงกับรหัสผ่าน
		\item กดปุ่ม “สมัครสมาชิก”
		\begin{enumerate}
			\item ถ้าข้อมูลไม่ครบ/ไม่ถูกต้อง ระบบจะแสดงข้อความสีแดงใต้ช่องนั้น ๆ ให้แก้ไขแล้วกดส่งใหม่
		\end{enumerate}
		\item เมื่อสมัครสำเร็จ รอแจ้งเตือน แล้วระบบจะพาไปหน้า เข้าสู่ระบบ โดยอัตโนมัติ
	\end{enumerate}
\end{sloppypar}

\begin{figure}[h]
	\centering
	\includegraphics[width=0.8\linewidth]{EV1}
	\caption{หน้าส่งปลากัดเพื่อประเมินคุณภาพ}
\end{figure}

\noindent{\bfseries\fontsize{16pt}{19.2pt}\selectfont วิธีใช้งานหน้าส่งปลากัดเพื่อประเมินคุณภาพ}\par

\begin{sloppypar}
	\begin{enumerate}
		\item ต้องอัปโหลด รูปปลากัดให้ครบ 3 รูป (JPG/PNG/WEBP) — ระบบ AI จะตรวจสอบว่าทั้ง 3 รูปเป็น ประเภทเดียวกัน
		\begin{enumerate}
			\item ถ้าพบรูปที่วิเคราะห์ไม่ได้/คนละประเภท ระบบจะ ลบเฉพาะรูปนั้น และแจ้งเตือนให้อัปโหลดใหม่
		\end{enumerate}
		\item วิดีโอเป็น ตัวเลือกเสริม (MP4/MOV/AVI) เพื่อแสดงการเคลื่อนไหว (แนะนำ ≤ 50MB)
		\item กรอกข้อมูลปลา: ชื่อปลากัด, อายุ (เดือน – ไม่บังคับ), ประเภทปลากัด
		\begin{enumerate}
			\item AI จะเสนอประเภทให้อัตโนมัติเมื่อรูปครบ 3 รูป; หากเลือกต่างจากที่ AI แนะนำ ระบบจะถามยืนยันอีกครั้ง
		\end{enumerate}
		\item ปุ่ม “ส่ง” จะกดได้เมื่อ อัปโหลดรูปครบ 3 รูป และ AI ตรวจสอบเสร็จ แล้วเท่านั้น
		\item ก่อนส่งจริง จะมี หน้าต่างยืนยัน แสดงสรุปข้อมูล (เช่น ประเภท/ชื่อปลา) ให้ตรวจทาน
	\end{enumerate}
\end{sloppypar}

\noindent{\bfseries\fontsize{16pt}{19.2pt}\selectfont ขั้นตอนใช้งาน}\par


\begin{sloppypar}
	\begin{enumerate}
		\item ที่ส่วน รูปภาพปลากัด ให้ ลาก-วาง หรือ กดคลิกเพื่อเลือกไฟล์ จน ครบ 3 รูป
		\begin{enumerate}
			\item ตรวจสอบตัวอย่างรูปที่อัปโหลดด้านล่าง และลบรูปใด ๆ ด้วยปุ่ม x หากต้องการเปลี่ยน
		\end{enumerate}
		\item รอระบบ AI ตรวจสอบ ความตรงกันของ 3 รูป (จะมีสถานะกำลังตรวจสอบ และแสดงเปอร์เซ็นต์ความมั่นใจเมื่อเสร็จ)
		\item (ไม่บังคับ) อัปโหลด วิดีโอปลากัด เพื่อเสริมการประเมิน แล้วกดเล่นเช็คตัวอย่างได้
		\item ที่ส่วน ข้อมูลปลากัด
		\item ตรวจสอบแถบแจ้งเตือนด้านล่างปุ่มส่ง
		\begin{enumerate}
			\item กรอก ชื่อปลากัด
			\item กรอก อายุ (เดือน) หากทราบ
			\item เลือก ประเภทปลากัด (ระบบมักเติมตามที่ AI แนะนำให้แล้ว)
			\item หากเลือกต่างจาก AI ระบบจะแสดง หน้าต่างยืนยันการเปลี่ยนประเภท
		\end{enumerate}
		\item กดปุ่ม “ส่งประเมินปลากัด” หรือ “ส่งเข้าร่วมประกวด” (ตามโหมด)
		\begin{enumerate}
			\item ถ้ายังไม่ครบ 3 รูป  ระบบจะแจ้งให้ อัปโหลดเพิ่ม
			\item ถ้า AI ยังไม่เสร็จ  ระบบจะแจ้งให้ รอการตรวจสอบ
		\end{enumerate}
		\item ในหน้าต่างยืนยัน ให้ตรวจทานข้อมูล แล้วกด “ยืนยันการส่ง”
		\begin{enumerate}
			\item ระบบจะแสดงสถานะ กำลังส่ง… และเมื่อสำเร็จจะแจ้งเตือน พร้อมพาไปหน้า ประวัติการส่ง
		\end{enumerate}
	\end{enumerate}
\end{sloppypar}

\newpage

\begin{figure}[h]
	\centering
	\includegraphics[width=0.8\linewidth]{HP5}
	\caption{หน้ารวมการประกวดปลากัด}
\end{figure}

\noindent{\bfseries\fontsize{16pt}{19.2pt}\selectfont วิธีใช้งานหน้าหน้ารวมการประกวดปลากัด}\par

\begin{sloppypar}
	\begin{enumerate}
		\item หน้านี้แสดง รายการการประกวดทั้งหมด เป็นการ์ด พร้อมโปสเตอร์ (ถ้าไม่มีจะแสดงภาพแทน), ชื่อ, คำอธิบายย่อ, และช่วงวันที่จัดงาน
		\item ที่มุมซ้ายบนรูป จะมี ป้ายสถานะ อัปเดตอัตโนมัติ
		\begin{enumerate}
			\item กำลังจะเริ่ม = ยังไม่ถึงวันเริ่ม
			\item เปิดรับสมัคร = อยู่ในช่วงเริ่ม–สิ้นสุด
			\item สิ้นสุดแล้ว = เลยวันสิ้นสุด
			\item กำลังดำเนินการ = ไม่เข้ากรณีข้างต้น
		\end{enumerate}
		\item คลิกที่การ์ดเพื่อไปหน้า รายละเอียดการประกวด
		\item หากกดสมัครจากปุ่ม/ลิงก์ที่เกี่ยวข้องและคุณ เข้าสู่ระบบแล้ว จะเปิด หน้าต่างสมัคร (Modal) ถ้ายัง ไม่ได้ล็อกอิน ระบบจะพาไปหน้า เข้าสู่ระบบ ก่อน
	\end{enumerate}
\end{sloppypar}

\noindent{\bfseries\fontsize{16pt}{19.2pt}\selectfont ขั้นตอนใช้งาน}\par

\begin{sloppypar}
	\begin{enumerate}
		\item เปิดหน้า การประกวดปลากัด
		\item เลื่อนดูรายการการประกวด และสังเกต ป้ายสถานะ บนรูปโปสเตอร์
		\item คลิกการ์ดที่สนใจเพื่อดู รายละเอียดเพิ่มเติม
		\item หากต้องการสมัคร
		\begin{enumerate}
			\item ถ้า ล็อกอินแล้ว: ระบบเปิด หน้าต่างสมัคร ให้กรอกข้อมูลและอัปโหลดสื่อ
			\item ถ้า ยังไม่ล็อกอิน: ระบบจะพาไปหน้า เข้าสู่ระบบ แล้วกลับมาทำรายการต่อ
		\end{enumerate}
		\item กลับมาที่หน้านี้เมื่อไหร่ก็ได้ เพื่อดูรายการใหม่ ๆ หรือสถานะที่อัปเดตแล้ว
	\end{enumerate}
\end{sloppypar}

\newpage

\begin{figure}[h]
	\centering
	\includegraphics[width=0.8\linewidth]{HP6}
	\caption{หน้าต่างสมัครเข้าร่วมประกวด}
\end{figure}


\noindent{\bfseries\fontsize{16pt}{19.2pt}\selectfont วิธีใช้งานหน้าต่างสมัครเข้าร่วมประกวด}\par

\begin{sloppypar}
	\begin{enumerate}
		\item ใช้สำหรับ ส่งปลากัดเข้าร่วมการประกวด ภายในหน้าต่าง (Modal) เดียว
		\item กรอกข้อมูลหลัก: ชื่อปลากัด, อายุ (เดือน–ไม่บังคับ), ประเภทปลากัด
		\begin{enumerate}
			\item ถ้าเวทีนี้กำหนดประเภทเดียว ปุ่มเลือกประเภทจะ ล็อกอัตโนมัติ
		\end{enumerate}
		\item อัปโหลดสื่อ
		\begin{enumerate}
			\item รูปภาพ ได้สูงสุด 3 รูป (JPG/PNG/WebP) พร้อมแสดงตัวอย่างและปุ่มลบ
			\item วิดีโอ (ไม่บังคับ) รองรับ MP4/WebM/QuickTime พร้อมตัวอย่างและปุ่มลบ
		\end{enumerate}
		\item ระบบมี AI ตรวจเช็กแบบสด จากรูปและประเภทที่เลือก
		\begin{enumerate}
			\item จะแจ้งว่า ตรงเงื่อนไข หรือ ไม่ตรงเงื่อนไขแต่ยังส่งได้ หรือ ไม่มั่นใจ
		\end{enumerate}
		\item ต้องติ๊ก ยืนยันความถูกต้องของข้อมูล ก่อนกด “ยืนยันการสมัคร”
		\item กดปุ่ม X มุมขวาบนเพื่อปิดหน้าต่างโดยไม่ส่งข้อมูล
	\end{enumerate}
\end{sloppypar}

\noindent{\bfseries\fontsize{16pt}{19.2pt}\selectfont ขั้นตอนใช้งาน}\par


\begin{sloppypar}
	\begin{enumerate}
		\item เปิดหน้าต่าง สมัครเข้าร่วมประกวด
		\item กรอก ชื่อปลากัด และ (ถ้ามี) อายุ (เดือน)
		\item เลือก ประเภทปลากัด (ถ้าถูกล็อกแสดงว่าเวทีกำหนดประเภทเดียว)
		\item อัปโหลด รูปภาพ โดยลาก–วาง หรือคลิกเลือก (แนะนำอัปอย่างน้อย 1 รูป และได้สูงสุด 3 รูป)
		\begin{enumerate}
			\item ตรวจดูตัวอย่างรูป และกดปุ่ม ลบ (x) หากต้องการเปลี่ยน
		\end{enumerate}
		\item (ไม่บังคับ) อัปโหลด วิดีโอ เพื่ือแสดงการเคลื่อนไหว
		\item รอสถานะ AI กำลังตรวจสอบ… แล้วดูผลลัพธ์ว่า ตรงเงื่อนไข / ไม่ตรง / ไม่มั่นใจ
		\item ติ๊กช่อง ยืนยันการส่งข้อมูลเข้าร่วมประกวด
		\item กดปุ่ม ยืนยันการสมัคร
		\begin{enumerate}
			\item ขณะส่ง ระบบจะแสดง กำลังส่งข้อมูล… และแจ้งผลสำเร็จ/ข้อผิดพลาดด้วยการแจ้งเตือน (toast)
		\end{enumerate}
	\end{enumerate}
\end{sloppypar}


\begin{figure}[h]
	\centering
	\includegraphics[width=0.8\linewidth]{HT2}
	\caption{หน้าประวัติการเข้าร่วมการแข่งขัน}
\end{figure}

\noindent{\bfseries\fontsize{16pt}{19.2pt}\selectfont วิธีใช้งานหน้าประวัติการเข้าร่วมการแข่งขัน}\par

\begin{sloppypar}
	\begin{enumerate}
		\item หน้านี้จะแสดง ประวัติที่คุณส่งปลากัดเข้าร่วมการแข่งขัน ในรูปแบบตาราง
		\item รายการแต่ละแถวมีข้อมูล: ชื่อการประกวด, ชื่อปลากัด, วันที่ส่ง, สถานะการแข่งขัน, สถานะการสมัคร, คะแนน, ปุ่มดูรายละเอียด
		\item ป้ายสถานะ (Badge) จะแสดงด้วยสีอ่านง่าย
		\begin{enumerate}
			\item สถานะการแข่งขัน: กำลังดำเนินการ, ปิดรับสมัคร, ตัดสิน, ประกาศผล, ยกเลิก ฯลฯ
			\item สถานะการสมัคร: รอตรวจสอบ, ผ่านการคัดเลือก, ประกาศผล, ถูกปฏิเสธ
		\end{enumerate}
		\item กด ดูรายละเอียด เพื่อเปิดหน้าต่าง (Modal) แสดงข้อมูลเชิงลึก เช่น ภาพ/วิดีโอ คะแนน เหตุผลการปฏิเสธ ฯลฯ
		\item ถ้าไม่มีรายการ ระบบจะแจ้งว่า ยังไม่มีประวัติ
		\item ขณะโหลดข้อมูลจะขึ้นข้อความ กำลังโหลด… และถ้าเกิดปัญหาจะแสดง ข้อความผิดพลาด
	\end{enumerate}
\end{sloppypar}

\noindent{\bfseries\fontsize{16pt}{19.2pt}\selectfont ขั้นตอนใช้งาน}\par

\begin{sloppypar}
	\begin{enumerate}
		\item เปิดหน้า ประวัติการเข้าร่วมการแข่งขัน
		\item รอให้ระบบ โหลดข้อมูล (หากช้าเล็กน้อยเป็นปกติ)
		\item ดูภาพรวมจากตาราง
		\begin{enumerate}
			\item เช็ค ชื่อการประกวด / ชื่อปลากัด / วันที่ส่ง
			\item ดู สถานะการแข่งขัน และ สถานะการสมัคร จากป้ายสี
			\item ตรวจสอบ คะแนน (ถ้ามี)
		\end{enumerate}
		\item ต้องการดูรายละเอียดเพิ่มเติม ให้กดปุ่ม ดูรายละเอียด ที่แถวรายการนั้น
		\item ภายในหน้าต่างรายละเอียด ตรวจสอบข้อมูลรูป/วิดีโอ คะแนน และเหตุผลต่าง ๆ จากนั้นกดปิดเพื่อกลับสู่ตาราง
		\item หากตารางว่าง ให้กลับไปหน้า การประกวด เพื่อสมัครรายการใหม่ เมื่อมีผลส่งแล้วจะมาแสดงที่หน้านี้อัตโนมัติ
	\end{enumerate}
\end{sloppypar}

\begin{figure}[h]
	\centering
	\includegraphics[width=0.8\linewidth]{HT1}
	\caption{หน้าประวัติการประเมินคุณภาพ}
\end{figure}

\noindent{\bfseries\fontsize{16pt}{19.2pt}\selectfont วิธีใช้งานหน้าประวัติการประเมินคุณภาพ}\par

\begin{sloppypar}
	\begin{enumerate}
		\item หน้านี้แสดง ประวัติการส่งประเมินคุณภาพปลากัด ของคุณในรูปแบบตาราง
		\item ระหว่างโหลดข้อมูลจะแสดง สัญลักษณ์หมุน (Loader) และหากเกิดปัญหาจะแสดง กล่องข้อความผิดพลาด
		\item เมื่อไม่มีข้อมูล จะแสดงข้อความ “ยังไม่มีประวัติการประเมินคุณภาพ”
		\item คอลัมน์ในตาราง
		\begin{enumerate}
			\item ชื่อปลากัด
			\item วันที่ส่ง (รูปแบบวันที่ไทย)
			\item สถานะ (แสดงเป็นป้ายสี เช่น รอการตรวจสอบ/กำลังประเมิน/ประเมินเสร็จสิ้น/ถูกปฏิเสธ ฯลฯ)
			\item คะแนน (ถ้ามีจะแสดงเป็นตัวหนาสีเขียว มิฉะนั้นเป็น “-”)
			\item รายละเอียด (ปุ่มเปิดดูข้อมูลเชิงลึก)
		\end{enumerate}
		\item ปุ่ม ดูรายละเอียด จะเปิดหน้าต่าง (Modal) แสดงข้อมูลฉบับเต็มของการประเมิน
	\end{enumerate}
\end{sloppypar}

\noindent{\bfseries\fontsize{16pt}{19.2pt}\selectfont ขั้นตอนใช้งาน}\par

\begin{sloppypar}
	\begin{enumerate}
		\item เข้าหน้า ประวัติการประเมินคุณภาพ
		\item รอให้ระบบ โหลดข้อมูล (หากมีปัญหาจะเห็นข้อความแจ้งข้อผิดพลาด)
		\item ตรวจดูตาราง
		\begin{enumerate}
			\item เช็ค ชื่อปลากัด และ วันที่ส่ง
			\item ดู สถานะ จากป้ายสี และ คะแนน หากมี
		\end{enumerate}
		\item กด ดูรายละเอียด ในแถวที่ต้องการ เพื่อเปิด หน้าต่างรายละเอียด
		\item อ่านข้อมูลเชิงลึกใน Modal แล้ว ปิดหน้าต่าง เพื่อกลับสู่ตาราง
		\item หากยังไม่มีรายการ ให้กลับไปหน้า ส่งประเมินคุณภาพ ก่อน แล้วข้อมูลจะมาแสดงในหน้านี้โดยอัตโนมัติ
	\end{enumerate}
\end{sloppypar}

\clearpage}{}
	\IfFileExists{Appendix1_10.tex}{%==================== Appendix.tex ====================

\clearpage
\thispagestyle{plain}

\begingroup
% เนื้อหาภาคผนวก: 16pt baseline ~19.2pt ตามสเปกเล่ม
\fontsize{16pt}{19.2pt}\selectfont
\justifying
\XeTeXlinebreakskip=0pt plus 1pt minus 0.5pt
\setlength{\parindent}{1.5cm}
\setlength{\parskip}{0pt}

\indent วิธีการใช้งานในส่วนของผู้จัดการประกวด เพื่อเป็นแนวทางในการใช้งานในส่วนที่ซับซ้อน และอาจทำให้สับสนในการใช้งานได้ จึงจัดทำคู่มือการใช้งานขึ้นมาเพื่ออำนวยความสะดวก

\vspace{\baselineskip}

\begin{figure}[h]
	\centering
	\includegraphics[width=0.8\linewidth]{MG1}
	\caption{หน้าแดชบอร์ดผู้จัดการ}
\end{figure}

\noindent{\bfseries\fontsize{16pt}{19.2pt}\selectfont วิธีใช้งานหน้าแดชบอร์ดผู้จัดการ}\par

\begin{sloppypar}
	\begin{enumerate}
		\item ดูภาพรวม สถิติการประกวด ผ่านการ์ดตัวเลข (ทั้งหมด/ร่าง/ดำเนินการ/ปิดรับสมัคร/ตัดสิน/ประกาศผล) และกราฟแท่ง “ภาพรวมสถานะการประกวด”
		\item พื้นที่ การแจ้งเตือน แสดงข้อความเหตุการณ์ล่าสุด สามารถ
		\begin{enumerate}
			\item กด ทำเป็นอ่านทั้งหมด เพื่อล้างรายการ
			\item คลิกลิงก์ ดูรายละเอียด (ถ้ามี) เพื่อไปยังหน้าที่เกี่ยวข้อง
		\end{enumerate}
		\item หากกำลังโหลดจะแสดง วงล้อหมุน; กรณีผิดพลาดจะแสดง ข้อความข้อผิดพลาด
		\item แดชบอร์ดจะแสดงชื่อผู้ใช้มุมบน (“ยินดีต้อนรับ, ผู้จัดการ/ชื่อจริง”)
	\end{enumerate}
\end{sloppypar}

\noindent{\bfseries\fontsize{16pt}{19.2pt}\selectfont ขั้นตอนใช้งาน}\par

\begin{sloppypar}
	\begin{enumerate}
		\item เข้าสู่หน้า แดชบอร์ดผู้จัดการ
		\item ดูการ์ดสรุปด้านบนเพื่อทราบจำนวนการประกวดในแต่ละสถานะอย่างรวดเร็ว
		\item เลื่อนลงดู กราฟแท่ง เพื่อเปรียบเทียบสัดส่วนสถานะต่าง ๆ ชัดเจน
		\item ที่กล่อง การแจ้งเตือน
		\begin{enumerate}
			\item อ่านข้อความล่าสุด และคลิก ดูรายละเอียด หากต้องการไปยังรายการนั้น
			\item ต้องการล้างทั้งหมด กด ทำเป็นอ่านทั้งหมด
			\item ต้องการลบทีละรายการ กดปุ่ม X ข้างข้อความ
		\end{enumerate}
		\item หากเห็นข้อความ กำลังโหลด… รอให้ข้อมูลแสดงผลครบ; ถ้าขึ้น ข้อผิดพลาด ให้รีเฟรชหน้าหรือกลับมาใหม่ภายหลัง
	\end{enumerate}
\end{sloppypar}

\vspace{\baselineskip}

\begin{figure}[h]
	\centering
	\includegraphics[width=0.8\linewidth]{MG2}
	\caption{หน้าจัดการกิจกรรม — สร้างการประกวด/ข่าว}
\end{figure}

\noindent{\bfseries\fontsize{16pt}{19.2pt}\selectfont วิธีใช้งานหน้าจัดการกิจกรรม — สร้างการประกวด/ข่าว}\par

\begin{sloppypar}
	\begin{enumerate}
		\item ใช้สำหรับ สร้างกิจกรรมใหม่ ได้ 2 ประเภท: การประกวด หรือ ข่าวสาร
		\item กรอกข้อมูลหลัก: ประเภท, ชื่อโครงการ/หัวข้อข่าว, คำอธิบายย่อ, รายละเอียดเต็ม (Rich Text)
		\item ไฟล์โปสเตอร์: อัปโหลดภาพ (PNG/JPG/WEBP) ขนาดไม่เกิน 5MB มีตัวอย่างแสดงและปุ่มลบ
		\item เมื่อเลือกประเภทเป็น การประกวด จะมีฟิลด์เพิ่มเติม
		\begin{enumerate}
			\item สถานะเริ่มต้น: แบบร่าง (Draft) หรือ เริ่มดำเนินการ (Published)
			\item วันที่เริ่มรับสมัคร / วันที่สิ้นสุด (ปฏิทินภาษาไทย, กันเลือกวันที่ย้อนหลัง/สิ้นสุดก่อนเริ่ม)
			\item ประเภทปลากัดที่เปิดรับ (เลือกได้ 1 ประเภท)
			\item คณะกรรมการ เลือกได้สูงสุด 3 คน (ค้นหาชื่อได้, รายการถูกกรองให้สัมพันธ์กับประเภทที่เลือก)
		\end{enumerate}
		\item ปุ่ม “บันทึกและสร้างกิจกรรม” จะส่งข้อมูลขึ้นระบบและแสดงสถานะกำลังทำงาน
	\end{enumerate}
\end{sloppypar}

\noindent{\bfseries\fontsize{16pt}{19.2pt}\selectfont ขั้นตอนใช้งาน}\par

\begin{sloppypar}
	\begin{enumerate}
		\item ไปที่หน้า จัดการกิจกรรม (สร้าง)
		\item เลือก ประเภท: การประกวด หรือ ข่าวสารทั่วไป/ประชาสัมพันธ์
		\item กรอก ชื่อโครงการ/หัวข้อข่าว, คำอธิบายย่อ, และพิมพ์ รายละเอียดเต็ม ในช่อง Rich Text
		\item อัปโหลด โปสเตอร์กิจกรรม (≤ 5MB)
		\begin{enumerate}
			\item ตรวจสอบตัวอย่างภาพได้ และกดปุ่ม ถังขยะ เพื่อลบ/อัปโหลดใหม่
		\end{enumerate}
		\item หากเลือกเป็น การประกวด
		\begin{enumerate}
			\item เลือก สถานะเริ่มต้น
			\item เลือก วันที่เริ่มรับสมัคร (ไม่น้อยกว่าวันปัจจุบัน) และ วันที่สิ้นสุด (ต้องไม่ก่อนวันเริ่ม)
			\item เลือก ประเภทปลากัดที่เปิดรับ (1 รายการ)
			\item ค้นหาและติ๊กเลือก คณะกรรมการ ได้สูงสุด 3 คน
		\end{enumerate}
		\item ตรวจทานความถูกต้อง แล้วกด “บันทึกและสร้างกิจกรรม”
		\begin{enumerate}
			\item ขณะส่ง ระบบจะแสดงไอคอนหมุน กำลังสร้างกิจกรรม...
			\item หากสำเร็จจะแจ้ง สร้างกิจกรรมสำเร็จ!; หากมีข้อผิดพลาดจะมีข้อความเตือนให้แก้ไข
		\end{enumerate}
	\end{enumerate}
\end{sloppypar}

\vspace{\baselineskip}

\begin{figure}[h]
	\centering
	\includegraphics[width=0.8\linewidth]{MG9}
	\caption{หน้าจัดการกิจกรรม — สร้างการประกวด/ข่าว}
\end{figure}

\noindent{\bfseries\fontsize{16pt}{19.2pt}\selectfont วิธีใช้งานหน้ารายการกิจกรรมทั้งหมด (แก้ไข/ลบ)}\par

\begin{sloppypar}
	\begin{enumerate}
		\item หน้านี้ใช้สำหรับดูรายการกิจกรรมทั้งหมดของผู้จัดการ (การประกวด/ข่าว) พร้อมโปสเตอร์ ชื่อ ช่วงวันเวลา และสถานะ (เฉพาะการประกวด)
		\item มีช่องค้นหาตามชื่อกิจกรรม และตัวกรองตามประเภท (การประกวด/ข่าว) และสถานะการประกวด (ร่าง/กำลังดำเนินการ/ปิดรับสมัคร/ตัดสิน/ประกาศผล/ยกเลิก)
		\item การ์ดกิจกรรมแต่ละใบมีปุ่ม ดูรายละเอียด (ไอคอนตา), แก้ไข (ดินสอ) และ ลบ (ถังขยะ)
		\item กด ดูรายละเอียด เพื่อเปิด Modal แสดงข้อมูลย่อ โปสเตอร์ วันเริ่ม–สิ้นสุด และรายชื่อกรรมการ (ถ้ามี)
		\item กด แก้ไข เพื่อเปิดหน้าต่างแก้ไขข้อมูลกิจกรรม (Edit Modal)
		\item กด ลบ เพื่อแสดงหน้าต่างยืนยันการลบ และยืนยันก่อนลบจริง
		\item ระหว่างดึงข้อมูลจะแสดงสถานะกำลังโหลด; หากเกิดข้อผิดพลาดจะแสดงข้อความแจ้งเตือน
	\end{enumerate}
\end{sloppypar}

\noindent{\bfseries\fontsize{16pt}{19.2pt}\selectfont ขั้นตอนใช้งาน}\par

\begin{sloppypar}
	\begin{enumerate}
		\item เปิดหน้า รายการกิจกรรมทั้งหมด (แก้ไข/ลบ)
		\item ใช้ช่องค้นหาเพื่อค้นหาตามชื่อ และ/หรือเลือกประเภทกิจกรรมที่ต้องการแสดง
		\item หากเลือกประเภทเป็น การประกวด ให้เลือกสถานะเพิ่มเติมเพื่อกรองผลลัพธ์
		\item ตรวจดูรายการที่แสดง (โปสเตอร์ ชื่อ วันเริ่ม–สิ้นสุด ป้ายสถานะสำหรับการประกวด)
		\item ดำเนินการกับรายการ:
		\begin{enumerate}
			\item กด ดูรายละเอียด เพื่อเปิด Modal และตรวจสอบข้อมูลย่อ/โปสเตอร์/กรรมการ
			\item กด แก้ไข เพื่อเปิด Edit Modal ปรับแก้ข้อมูล จากนั้นบันทึก
			\item กด ลบ เพื่อเปิดหน้าต่างยืนยัน ยืนยันการลบ เพื่อดำเนินการลบถาวร
		\end{enumerate}
		\item หากมีการแก้ไข/ลบสำเร็จ ระบบจะแจ้งผลและรีเฟรชรายการให้อัตโนมัติ
		\item หากไม่พบรายการที่ตรงเงื่อนไข จะปรากฏข้อความ ไม่พบรายการที่ตรงกับเงื่อนไข ให้ปรับการค้นหา/ตัวกรองใหม่
	\end{enumerate}
\end{sloppypar}

\vspace{\baselineskip}

\begin{figure}[h]
	\centering
	\includegraphics[width=0.8\linewidth]{MG10}
	\caption{หน้ามอบหมายกรรมการ}
\end{figure}

\noindent{\bfseries\fontsize{16pt}{19.2pt}\selectfont วิธีใช้งานหน้ามอบหมายกรรมการ}\par

\begin{sloppypar}
	\begin{enumerate}
		\item หน้านี้ใช้สำหรับมอบหมายกรรมการให้การประกวด โดยจำกัดจำนวนกรรมการสูงสุดต่อการประกวดไว้ที่ 3 คน
		\item ส่วนเลือกการประกวดจะแสดงการ์ดกิจกรรมที่สถานะเหมาะกับการมอบหมาย (ร่าง, กำลังดำเนินการ, ปิดรับสมัคร) พร้อมโปสเตอร์ ชื่อ และช่วงวันที่
		\item เมื่อเลือกการประกวดแล้ว จะแสดงรายชื่อกรรมการปัจจุบัน พร้อมสถานะตอบรับ เช่น รอดำเนินการ, ยอมรับแล้ว, ปฏิเสธแล้ว และมีปุ่มลบออกจากคณะกรรมการ
		\item ส่วนเลือกผู้เชี่ยวชาญเพิ่มเติมจะแสดงเฉพาะผู้เชี่ยวชาญที่ตรงกับประเภท/สาขาที่การประกวดเปิดรับ และสามารถค้นหาชื่อได้
		\item ผู้เชี่ยวชาญที่ถูกเลือกจะแสดงสถานะเลือกแล้ว และหากตรงความเชี่ยวชาญจะมีป้ายกำกับบอก
		\item ปุ่มล้างการเลือกทั้งหมดใช้เพื่อล้างรายชื่อผู้เชี่ยวชาญที่เลือกไว้ก่อนบันทึก
		\item ปุ่มบันทึกการมอบหมายใช้เพื่อยืนยันการเพิ่มกรรมการ ระบบจะตรวจสอบโควตาและแสดงสถานะกำลังมอบหมาย
		\item ระหว่างโหลดข้อมูลจะแสดงสถานะกำลังโหลด และหากเกิดข้อผิดพลาดจะแจ้งข้อความให้ทราบ
	\end{enumerate}
\end{sloppypar}

\noindent{\bfseries\fontsize{16pt}{19.2pt}\selectfont ขั้นตอนใช้งาน}\par

\begin{sloppypar}
	\begin{enumerate}
		\item เปิดหน้า มอบหมายกรรมการ
		\item เลือกการประกวดจากการ์ดรายการที่แสดง หากยังไม่พบรายการที่มอบหมายได้ ให้กลับไปสร้างหรือปรับสถานะการประกวดให้เหมาะสม
		\item ตรวจดูกรรมการปัจจุบันของการประกวด และลบรายชื่อที่ต้องการออกได้จากปุ่มลบ
		\item ในส่วนเลือกผู้เชี่ยวชาญเพิ่มเติม ให้พิมพ์ค้นหาชื่อหรือตรวจดูจากรายการที่ระบบคัดกรองให้ แล้วติ๊กเลือกผู้เชี่ยวชาญตามต้องการ
		\item สังเกตตัวนับโควตากรรมการ ว่าจำนวนรวมของที่มีอยู่แล้วและที่เลือกใหม่ไม่เกิน 3 คน
		\item หากต้องการเริ่มเลือกใหม่ทั้งหมด ให้กด ล้างการเลือกทั้งหมด
		\item เมื่อตรวจสอบถูกต้อง กด บันทึกการมอบหมาย ระบบจะเพิ่มกรรมการตามที่เลือกและรีเฟรชข้อมูล
		\item หากต้องการลบกรรมการภายหลัง ให้เลือกการประกวดเดิม กดลบที่รายชื่อนั้น ระบบจะลบและส่งการแจ้งเตือนให้ผู้เชี่ยวชาญ
	\end{enumerate}
\end{sloppypar}

\newpage

\vspace{\baselineskip}

\begin{figure}[h]
	\centering
	\includegraphics[width=0.8\linewidth]{MG11}
	\caption{หน้าห้องจัดการแข่งขัน}
\end{figure}

\noindent{\bfseries\fontsize{16pt}{19.2pt}\selectfont วิธีใช้งานหน้าห้องจัดการแข่งขัน}\par

\begin{sloppypar}
	\begin{enumerate}
		\item ใช้สำหรับบริหารการประกวดแบบเรียลไทม์: เลือกเวทีที่กำลังดำเนินการ ดู/คัดกรองรายชื่อผู้สมัคร ตรวจสถานะ AI ความคืบหน้าการให้คะแนน อนุมัติ/ปฏิเสธแบบเดี่ยวหรือหลายรายการ และเปลี่ยนสถานะการประกวดจนถึงประกาศผล
		\item การ์ดการแข่งขันแต่ละรายการแสดง โปสเตอร์ ชื่อ ช่วงวัน สถานะปัจจุบัน จำนวนกรรมการที่ตอบรับ และจำนวนผู้สมัคร คลิกการ์ดเพื่อเข้าสู่แดชบอร์ดของเวทีนั้น
		\item แถบสรุปบนแดชบอร์ดแสดงผู้สมัครทั้งหมด (นับคน) จำนวนปลากัด รวมถึงจำนวนตามสถานะ: รออนุมัติ อนุมัติแล้ว ประเมินแล้ว ปฏิเสธ และจำนวนกรรมการ
		\item ปุ่มจัดการสถานะการประกวด (มุมขวาบนของแดชบอร์ด) จะเปลี่ยนไปตามสถานะ: ปิดรับสมัคร  เริ่มการตัดสิน  คำนวณและประกาศผล
		\item ระหว่างสถานะ “ตัดสิน” จะแสดงความคืบหน้าการให้คะแนน: จำนวนกรรมการที่ตอบรับ จำนวนงานที่ครบทุกกรรมการ และเปอร์เซ็นต์ความครอบคลุมเฉลี่ย พร้อมปุ่มรีเฟรช
		\item ตารางรายชื่อผู้สมัครสามารถคัดกรองตามสถานะ (ทั้งหมด/รออนุมัติ/อนุมัติแล้ว/ประเมินแล้ว/ปฏิเสธ) และกรองย่อยด้วยสถานะ AI (ตรง/ไม่ตรง/ยังไม่มั่นใจ)
		\item สถานะ AI ต่อแถวแสดงเป็นป้ายสี: ตรง, ไม่ตรง, ยังไม่มั่นใจ และหากไม่ตรง สามารถเปิดดูรายละเอียดประเภทที่อนุญาตของเวทีได้
		\item แถวผู้สมัครมีปุ่มจัดการอย่างรวดเร็ว: ดูรายละเอียด อนุมัติ ปฏิเสธ (หรือยกเลิกการอนุมัติ), เปิดดูคะแนนกรรมการ รวมถึงกล่องติ๊กเพื่อทำรายการแบบกลุ่ม
		\item ในหน้า “ดูรายละเอียดผู้สมัคร” มีปุ่มให้ AI วิเคราะห์ภาพแรกของผลงานเพื่อช่วยพิจารณาความตรงประเภท พร้อมคำแนะนำให้ อนุมัติ/ตรวจเพิ่ม
		\item เมื่ออยู่ในแท็บ “ประเมินแล้ว” สามารถอนุมัติผลคะแนนแบบหลายรายการได้ในครั้งเดียว
	\end{enumerate}
\end{sloppypar}

\noindent{\bfseries\fontsize{16pt}{19.2pt}\selectfont ขั้นตอนใช้งาน}\par

\begin{sloppypar}
	\begin{enumerate}
		\item เข้าเมนู ห้องจัดการแข่งขัน แล้วเลือกการประกวดที่ต้องการบริหารโดยคลิกการ์ดของเวทีนั้น
		\item ตรวจสอบสถานะปัจจุบันของการประกวดในแถบด้านบน หากพร้อม
		\begin{enumerate}
			\item จาก กำลังดำเนินการ เลือก ปิดรับสมัคร
			\item จาก ปิดรับสมัคร เลือก เริ่มการตัดสิน เพื่อให้กรรมการเริ่มให้คะแนน
			\item จาก ตัดสิน เมื่อคะแนนครบ เลือก คำนวณและประกาศผล
		\end{enumerate}
		\item ใช้การ์ดสถิติเลือกมุมมอง เช่น รออนุมัติ/อนุมัติแล้ว/ประเมินแล้ว/ปฏิเสธ หรือดูรายชื่อกรรมการ
		\item ในมุมมองรายชื่อผู้สมัคร
		\begin{enumerate}
			\item เลือกกรองสถานะหลัก (เช่น รออนุมัติ) และถ้าต้องการให้แคบผลลัพธ์ยิ่งขึ้น ให้เลือกกรองสถานะ AI (ทั้งหมด/ตรง/ไม่ตรง/ยังไม่มั่นใจ)
			\item ติ๊กเลือกหลายรายการเพื่อทำงานแบบกลุ่ม แล้วกด อนุมัติที่เลือก / ปฏิเสธที่เลือก หรือในมุมมอง “ประเมินแล้ว” ให้กด อนุมัติผลคะแนนที่เลือก
			\item จัดการรายเดี่ยวด้วยปุ่มบนแถว: ดูรายละเอียด อนุมัติ ปฏิเสธ (หรือยกเลิกการอนุมัติ) และ ดูคะแนน
		\end{enumerate}
		\item เมื่อต้องการใช้ AI ช่วยตรวจประเภท
		\begin{enumerate}
			\item กด ดูรายละเอียด บนแถวผู้สมัคร
			\item กด วิเคราะห์ด้วย AI ระบบจะแสดงผลการทำนาย ความมั่นใจ และคำแนะนำ (อนุมัติ/ตรวจเพิ่ม)
		\end{enumerate}
		\item ระหว่างสถานะ “ตัดสิน” ตรวจความคืบหน้าการให้คะแนนจากแถบสรุป และกด รีเฟรช เพื่ออัปเดตล่าสุด
		\item เมื่อคะแนนครบถ้วน ให้กด คำนวณและประกาศผล ระบบจะยืนยันก่อนดำเนินการและปิดขั้นตอนของเวทีนั้น
	\end{enumerate}
\end{sloppypar}

\newpage

\begin{figure}[h]
	\centering
	\includegraphics[width=0.8\linewidth]{MG12}
	\caption{หน้าประวัติและผลการประกวด}
\end{figure}

\noindent{\bfseries\fontsize{16pt}{19.2pt}\selectfont วิธีใช้งานหน้าประวัติและผลการประกวด}\par

\begin{sloppypar}
	\begin{enumerate}
		\item ใช้สำหรับดูประวัติการแข่งขันที่ผ่านมา พร้อมสรุปผลผู้ชนะและคะแนนเรียงลำดับจากมากไปน้อย
		\item มีช่องค้นหาเพื่อกรองรายการตามชื่อการประกวดในอดีตแบบเรียลไทม์
		\item การ์ดแต่ละรายการแสดงชื่อการประกวด วันที่สิ้นสุด สถานะล่าสุด และไฮไลต์ผู้ชนะ (เมื่อสถานะเป็น “ประกาศผล”)
		\item ปุ่ม ดูผลสรุป จะเปิดหน้าต่างสรุปผล แสดงอันดับที่ 1–3 พร้อมชื่อปลา เจ้าของ และคะแนนรวม
		\item ระหว่างโหลดข้อมูลจะแสดงสถานะกำลังโหลด และหากเกิดข้อผิดพลาดจะแสดงข้อความแจ้งเตือน
		\item หากไม่พบรายการตามคำค้นหาหรือยังไม่มีประวัติ จะแสดงข้อความบอกสถานะให้ทราบ
	\end{enumerate}
\end{sloppypar}

\noindent{\bfseries\fontsize{16pt}{19.2pt}\selectfont ขั้นตอนใช้งาน}\par

\begin{sloppypar}
	\begin{enumerate}
		\item เข้าเมนู ประวัติและผลการประกวด
		\item พิมพ์คำค้นในช่อง ค้นหาชื่อการประกวดในอดีต... เพื่อกรองรายการ (ไม่พิมพ์ก็เห็นทั้งหมด)
		\item ดูสรุปในแต่ละการ์ด: วันที่สิ้นสุด สถานะ และผู้ชนะ (ถ้ามี)
		\item คลิกปุ่ม ดูผลสรุป ของรายการที่สนใจ
		\item ในหน้าต่างสรุปผล ตรวจดูอันดับ 1–3 พร้อมคะแนนรวมและข้อมูลเจ้าของปลา
		\item ปิดหน้าต่างสรุปเมื่อเสร็จสิ้น หรือเลือกดูรายการอื่น ๆ ต่อได้ตามต้องการ
	\end{enumerate}
\end{sloppypar}

\clearpage}{}
	\IfFileExists{Appendix1_11.tex}{%==================== Appendix.tex ====================

\clearpage
\thispagestyle{plain}

\begingroup
% เนื้อหาภาคผนวก: 16pt baseline ~19.2pt ตามสเปกเล่ม
\fontsize{16pt}{19.2pt}\selectfont
\justifying
\XeTeXlinebreakskip=0pt plus 1pt minus 0.5pt
\setlength{\parindent}{1.5cm}
\setlength{\parskip}{0pt}

\indent วิธีการใช้งานในส่วนของผู้เชี่ยวชาญ เพื่อเป็นแนวทางในการใช้งานในส่วนที่ซับซ้อน และอาจทำให้สับสนในการใช้งานได้ จึงจัดทำคู่มือการใช้งานขึ้นมาเพื่ออำนวยความสะดวก

\vspace{\baselineskip}

\begin{figure}[h]
	\centering
	\includegraphics[width=0.8\linewidth]{EP1}
	\caption{หน้าแดชบอร์ดผู้เชี่ยวชาญ}
\end{figure}

\noindent{\bfseries\fontsize{16pt}{19.2pt}\selectfont วิธีใช้งานหน้าแดชบอร์ดผู้เชี่ยวชาญ}\par

\begin{sloppypar}
	\begin{enumerate}
		\item ใช้ดูภาพรวมงานของผู้เชี่ยวชาญแบบรวดเร็ว ได้แก่ จำนวน “รอประเมินคุณภาพ”, “รอตอบรับการแข่งขัน”, และ “งานที่เสร็จสิ้น”
		\item มีการ์ดสรุปสถิติ 3 ใบ (คลิกได้)
		\begin{enumerate}
			\item รอประเมินคุณภาพ — ลิงก์ไปที่ /expert/evaluations
			\item รอตอบรับการแข่งขัน — ลิงก์ไปที่ /expert/judging
			\item งานที่เสร็จสิ้น — ลิงก์ไปที่ /expert/history
		\end{enumerate}
		\item ส่วน “การแจ้งเตือนและงานด่วน” จะแสดงกล่องแจ้งเตือนจำนวนงานที่ต้องทำทันที (ถ้ามี) เช่น จำนวนงานรอประเมิน และจำนวนคำเชิญเป็นกรรมการที่ยังไม่ตอบรับ
		\item ขณะโหลดจะแสดงสัญลักษณ์กำลังประมวลผล และหากไม่มีงานค้าง ระบบจะแจ้งว่า “ไม่มีงานที่ต้องดำเนินการในขณะนี้”
		\item แสดงชื่อผู้ใช้งานด้านบนเพื่อความชัดเจนว่าเป็นแดชบอร์ดของผู้เชี่ยวชาญ
	\end{enumerate}
\end{sloppypar}

\noindent{\bfseries\fontsize{16pt}{19.2pt}\selectfont ขั้นตอนใช้งาน}\par

\begin{sloppypar}
	\begin{enumerate}
		\item เข้าสู่ระบบด้วยบทบาทผู้เชี่ยวชาญ แล้วเปิด “แดชบอร์ดผู้เชี่ยวชาญ”
		\item ดูตัวเลขสรุปในแต่ละการ์ดเพื่อประเมินภาพรวมงานปัจจุบัน
		\item หากต้องประเมินงาน ให้คลิกการ์ด “รอประเมินคุณภาพ” เพื่อไปยังหน้ารายการที่ต้องประเมิน
		\item หากมีคำเชิญเป็นกรรมการ ให้คลิกการ์ด “รอตอบรับการแข่งขัน” เพื่อเข้าไปตอบรับ/ปฏิเสธ
		\item ต้องการดูผลงานที่ดำเนินการเสร็จแล้ว ให้คลิกการ์ด “งานที่เสร็จสิ้น” เพื่อดูประวัติ
		\item ตรวจสอบส่วน “การแจ้งเตือนและงานด่วน” หากมีกล่องแจ้งเตือน แนะนำให้เริ่มดำเนินการตามที่ระบุ
		\item หากตัวเลขยังไม่อัปเดตหลังดำเนินการ สามารถรีเฟรชหน้าเพื่อดึงข้อมูลล่าสุดได้
	\end{enumerate}
\end{sloppypar}

\begin{figure}[h]
	\centering
	\includegraphics[width=0.8\linewidth]{EP2}
	\caption{หน้าคิวงานประเมินคุณภาพ (ผู้เชี่ยวชาญ)}
\end{figure}

\noindent{\bfseries\fontsize{16pt}{19.2pt}\selectfont วิธีใช้งานหน้าคิวงานประเมินคุณภาพ (ผู้เชี่ยวชาญ)}\par

\begin{sloppypar}
	\begin{enumerate}
		\item หน้าแสดง “คิวงานประเมินคุณภาพ” แบ่ง 2 แท็บ:
		\emph{รอการตอบรับ} (งานที่ผู้เชี่ยวชาญต้องกดรับหรือปฏิเสธ) และ
		\emph{ที่ต้องให้คะแนน} (งานที่ตอบรับแล้วและพร้อมเปิดฟอร์มให้คะแนน)
		\item แต่ละรายการจะแสดงภาพปลากัด ชื่อปลา ประเภท ผู้ส่ง และสถานะปัจจุบัน
		\item ปุ่มสำหรับงานในแท็บ \emph{รอการตอบรับ}:
		“ตอบรับ” เพื่อรับงาน, “ปฏิเสธ” เพื่อระบุเหตุผลและคืนงาน
		\item ปุ่มสำหรับงานในแท็บ \emph{ที่ต้องให้คะแนน}:
		“ให้คะแนน” เพื่อเปิดฟอร์มกรอกคะแนนและส่งผล
		\item มีสถานะระหว่างทำงาน: แสดงสัญลักษณ์กำลังโหลดเมื่อดึงข้อมูล,
		และแสดงหน้าว่างพร้อมข้อความเมื่อไม่มีรายการในคิว
		\item เมื่อดำเนินการสำเร็จ (ตอบรับ/ปฏิเสธ/ส่งคะแนน) ระบบจะแจ้งเตือนและรีเฟรชคิวอัตโนมัติ
	\end{enumerate}
\end{sloppypar}

\noindent{\bfseries\fontsize{16pt}{19.2pt}\selectfont ขั้นตอนใช้งาน}\par

\begin{sloppypar}
	\begin{enumerate}
		\item เปิดเมนู “คิวงานประเมินคุณภาพ”
		\item เลือกแท็บที่ต้องการ
		\begin{enumerate}
			\item แท็บ “รอการตอบรับ”: ตรวจรายละเอียด แล้วกด “ตอบรับ” หรือ “ปฏิเสธ”
			\item แท็บ “ที่ต้องให้คะแนน”: กด “ให้คะแนน” เพื่อเปิดฟอร์ม
		\end{enumerate}
		\item หากปฏิเสธ ระบบจะเปิดหน้าต่างให้กรอกเหตุผล กดยืนยันเพื่อส่งกลับคิว
		\item หากให้คะแนน ให้กรอกคะแนนในฟอร์มให้ครบ แล้วกด “ส่งคะแนน”
		\item ตรวจสอบข้อความแจ้งเตือนความสำเร็จ จากนั้นรายการจะย้ายสถานะ/หายจากคิวตามเงื่อนไข
	\end{enumerate}
\end{sloppypar}


\begin{figure}[h]
	\centering
	\includegraphics[width=0.8\linewidth]{EP3}
	\caption{หน้าห้องตัดสิน — รายการปลาที่ต้องให้คะแนน}
\end{figure}

\noindent{\bfseries\fontsize{16pt}{19.2pt}\selectfont วิธีใช้งานหน้าห้องตัดสิน — รายการปลาที่ต้องให้คะแนน}\par

\begin{sloppypar}
	\begin{enumerate}
		\item หน้าแสดงรายชื่อปลาที่ต้องให้คะแนนของการแข่งขันแต่ละรายการ หากผู้จัดการยังไม่เปิดสถานะ “ตัดสิน” ระบบจะแจ้งว่า “ยังไม่เปิดการตัดสิน”
		\item แสดงการ์ดปลา: รูปแรก ชื่อปลา ชื่อผู้ส่ง ปุ่ม “ให้คะแนน” และเช็กบ็อกซ์ “เปรียบเทียบ”
		\item แถบควบคุมด้านบนบอกจำนวน “เลือกไว้: X/10” พร้อมปุ่ม “เปรียบเทียบ” และ “ล้างการเลือก”
		\item โหมดเปรียบเทียบ: แสดงปลาที่เลือกแบบกริด เลือกเลื่อนรูปได้ (ปุ่มลูกศร/ภาพย่อ) และกรอก “คะแนนรวม (0–100)” แบบ Quick Score ต่อแต่ละตัว
		\item ปุ่ม “บันทึกคะแนนที่กรอก” จะบันทึกเฉพาะคะแนนรวม (โหมด Quick) ของรายการที่กรอกตัวเลขไว้
		\item ปุ่ม “ให้คะแนน” บนการ์ด จะเปิดฟอร์มให้คะแนนละเอียด (ScoringFormModal) เพื่อกรอกคะแนนตามเกณฑ์และส่งผล
		\item หลังบันทึกคะแนนหรือส่งคะแนนสำเร็จ ระบบจะแจ้งเตือน รีเฟรชรายการ และล้างการเลือกโดยอัตโนมัติ
		\item ปุ่ม “กลับไปหน้ารายการ” ใช้กลับไปหน้าเชิญตัดสิน/รายการแข่งขันของผู้เชี่ยวชาญ
	\end{enumerate}
\end{sloppypar}

\noindent{\bfseries\fontsize{16pt}{19.2pt}\selectfont ขั้นตอนใช้งาน}\par

\begin{sloppypar}
	\begin{enumerate}
		\item เข้าเมนูผู้เชี่ยวชาญ ไปยัง “ห้องตัดสิน — รายการปลาที่ต้องให้คะแนน”
		\item หากขึ้นข้อความ “ยังไม่เปิดการตัดสิน” ให้รอผู้จัดการเปลี่ยนสถานะ แล้วกดรีเฟรชเพื่อโหลดใหม่
		\item ให้คะแนนแบบละเอียด (แนะนำ)
		\begin{enumerate}
			\item เลือกปลาที่ต้องการ แล้วกด “ให้คะแนน”
			\item กรอกคะแนนตามหัวข้อในฟอร์ม และกด “ส่งคะแนน”
		\end{enumerate}
		\item ให้คะแนนแบบรวดเร็ว (Quick Score) หลายตัวพร้อมกัน
		\begin{enumerate}
			\item ติ๊ก “เปรียบเทียบ” ที่การ์ดปลา (เลือกได้สูงสุด 10 ตัว) แล้วกด “เปรียบเทียบ”
			\item ในหน้าต่างเปรียบเทียบ เลื่อนดูรูป (ลูกศร/ภาพย่อ) และกรอก “คะแนนรวม (0–100)” ต่อแต่ละตัว
			\item กด “บันทึกคะแนนที่กรอก” เพื่อบันทึกเฉพาะคะแนนรวมของรายการที่มีตัวเลข
		\end{enumerate}
		\item ต้องการล้างการเลือก กด “ล้างการเลือก” ที่แถบควบคุมหรือในหน้าต่างเปรียบเทียบ
		\item เสร็จงานแล้ว กด “กลับไปหน้ารายการ” เพื่อย้อนกลับไปยังหน้ารวมการตัดสิน
	\end{enumerate}
\end{sloppypar}


\begin{figure}[h]
	\centering
	\includegraphics[width=0.8\linewidth]{EP4}
	\caption{หน้า “ประวัติการทำงานผู้เชี่ยวชาญ”}
\end{figure}

\noindent{\bfseries\fontsize{16pt}{19.2pt}\selectfont วิธีใช้งานหน้า “ประวัติการทำงานผู้เชี่ยวชาญ”}\par
\begin{sloppypar}
	\begin{enumerate}
		\item หน้าสรุปผลงานที่เคยทำ แบ่ง 2 หมวด: \textit{การตัดสินการแข่งขัน} และ \textit{การประเมินคุณภาพ} (สลับด้วยปุ่มแท็บด้านบน)
		\item ระบบแสดงตารางตามหมวดที่เลือก: ชื่อปลา/ชื่อการแข่งขัน, เจ้าของ/ชื่อปลา, ประเภท, คะแนนรวม, วันที่เสร็จสิ้น
		\item คะแนนจะแสดงทศนิยม 2 ตำแหน่ง; หากเป็นหมวดประเมินคุณภาพและรายการยังไม่ “ประเมินแล้ว” จะแสดง “–”
		\item วันที่แสดงในรูปแบบไทยอัตโนมัติ; หากข้อมูลไม่สมบูรณ์จะแสดง “–”
		\item มีสถานะโหลดและหน้าว่าง (Empty State) เมื่อไม่มีข้อมูลในหมวดนั้น ๆ
		\item ระบบจำหมวดล่าสุดที่เลือกไว้ (บันทึกในเบราว์เซอร์) กลับมาเปิดอีกครั้งจะอยู่ที่หมวดเดิม
	\end{enumerate}
\end{sloppypar}

\noindent{\bfseries\fontsize{16pt}{19.2pt}\selectfont ขั้นตอนใช้งาน}\par
\begin{sloppypar}
	\begin{enumerate}
		\item เปิดเมนู “ประวัติการทำงาน” ของผู้เชี่ยวชาญ
		\item เลือกหมวดที่ต้องการดู:
		\begin{enumerate}
			\item “การตัดสินการแข่งขัน” — ดูประวัติให้คะแนนในเวทีประกวด
			\item “การประเมินคุณภาพ” — ดูประวัติการตรวจคุณภาพรายปลา
		\end{enumerate}
		\item ตรวจรายการในตาราง:
		\begin{enumerate}
			\item ตรวจชื่อปลา/ชื่อการแข่งขัน และเจ้าของ/ชื่อปลา (ตามหมวด)
			\item ดูประเภทและคะแนนรวม (ทศนิยม 2 ตำแหน่ง)
			\item ตรวจวันที่เสร็จสิ้นของงานนั้น ๆ
		\end{enumerate}
		\item หากไม่พบข้อมูล ระบบจะแสดงข้อความ “ไม่พบประวัติในหมวดนี้” ให้สลับหมวดหรือกลับมาภายหลัง
		\item ออกจากหน้าได้ทันที — ระบบจะจดจำหมวดที่เลือกไว้สำหรับการใช้งานครั้งถัดไป
	\end{enumerate}
\end{sloppypar}}{}
	
	
	
	% ประวัติผู้วิจัย
	\thispagestyle{plain}
	\IfFileExists{Biography.tex}{\clearpage
\thispagestyle{empty}


% ทั้งหน้า 13pt (baseline ~15.6pt)
\fontsize{16pt}{24pt}\selectfont

% จัดเต็มบรรทัดแบบไทย + ย่อหน้า 1.5 ซม.
\justifying
\XeTeXlinebreakskip=0pt plus 1pt minus 0.5pt
\setlength{\parindent}{1.5cm}
\setlength{\parskip}{0pt}

\phantomsection
\addcontentsline{toc}{chapter}{ประวัติผู้วิจัย}
\begin{center}
	{\bfseries\fontsize{16pt}{19.2pt}\selectfont ประวัติผู้วิจัย}
\end{center}
\vspace{\baselineskip}



% ---------- จัดบรรทัด "ป้ายกำกับ : ค่า" ให้ตรงคอลัมน์ ----------

\settowidth{\FieldLabelWidth}{\bfseries ประเภทสารนิพนธ์ :}

% ป้ายกำกับบรรทัดแรก "ตัวหนา 13pt" + ค่า "13pt ธรรมดา" อยู่บรรทัดเดียวกัน
% ถ้าค่ายาวจะตัดบรรทัดภายในคอลัมน์ขวาและชิดซ้ายใต้คอลัมน์ค่าเดิมเสมอ
\newcommand{\Field}[2]{%
	\noindent
	\makebox[\FieldLabelWidth][l]{\bfseries #1 :}%
	\hspace{\FieldSep}%
	\parbox[t]{\dimexpr\linewidth-\FieldLabelWidth-\FieldSep\relax}{#2}%
	\par
}

% ---------- บล็อกข้อมูลหัวเรื่อง ----------
\Field{ชื่อ-สกุล}{เอกสิทธิ์ อัศวดารา}
\Field{วัน เดือน ปี เกิด}{17 กรกฎาคม 2546}
\Field{สถานที่เกิด}{จังหวัดพิษณุโลก}
\Field{วุฒิการศึกษา}{-}
\Field{ที่อยู่ปัจจุบัน}{66/4 หมู่ที่ 4 ตำบลชาติตระการ อพเภาชาติตระการ จังหวัดพิษณุโลก 65170}

% เว้น 1 บรรทัด
\vspace{\baselineskip}}{}
	
\end{document}
