% --- ตาราง: ลักษณะเฉพาะของปลากัดป่าภาคใต้และเกณฑ์การให้คะแนนในการประกวด (บังคับ 13pt) ---
\begingroup
\renewcommand{\arraystretch}{1.15}
\setlength{\arrayrulewidth}{0.5pt}

% ถ้ามี hook \AtBeginEnvironment{tabularx}{\tablefont} อยู่ ให้กำหนด \tablefont เฉพาะที่นี่เป็น 13pt
\makeatletter
\@ifundefined{tablefont}{\newcommand{\tablefont}{}}{}%
\makeatother
\renewcommand{\tablefont}{\fontsize{13pt}{15.6pt}\selectfont} % << ฟอนต์คงที่ 13pt

\begin{table}[h]
	\captionsetup{justification=raggedright, singlelinecheck=false,
		labelfont=bf, textfont=bf}
	\caption{ลักษณะเฉพาะของปลากัดป่ามหาชัยและเกณฑ์การให้คะแนนในการประกวด}
	\centering
	
	{\tablefont % ย้ำอีกชั้นให้แน่ใจว่าเนื้อหาในตารางใช้ 13pt
		\begin{tabularx}{\textwidth}{@{}>{\raggedright\arraybackslash}p{2.8cm}
				>{\raggedright\arraybackslash}X
				>{\centering\arraybackslash}p{1.6cm}@{}}
			\Xhline{1.5pt}
			\bfseries ลักษณะ & \bfseries ลักษณะเด่นตามมาตรฐาน & \bfseries คะแนน \\
			\hline
			ส่วนหัวและลำตัว &
			สันหัวเป็นเขม่าดำ แก้มมีขีดสีเขียวหรือฟ้าแวววาว 2 ขีด บางตัว
			พบมีสีแก้มเคลือบเขียวไปถึงคาง ลำตัวทรงกระบอกยาว และ
			แบนข้างเล็กน้อย & 10 \\
			\hline
			แก้มและเกล็ด &
			พื้นผิวที่ลำตัวมีสีน้ำตาลแดงถึงดำ สีของเกล็ดมีสีเขียวถึงฟ้าแวว
			วาวเรียงเป็นระเบียบเป็นแนวคล้ายแถวเมล็ดข้าวโพดบนฝัก
			เวลาที่กระพุ้งแก้มเปิดจะเห็นแผ่นปิดเหงือกชั้นใน
			(branchiostegal membrane) มีสีน้ำตาลแดงถึงดำ ซึ่งในปลา
			กัดชนิดอื่นจะมีแผ่นปิดเหงือกชั้นในสีแดงสดถึงแดงเข้มเลือดหมู & 15 \\
			\hline
			ครีบหลัง (Dorsal Fin)ครีบหลังหรือกระโดง
			(Dorsal Fin) &
			ก้านครีบแตก 1 ถึง 2 (ถ้าแตก 1 ถือว่าเป็นลักษณะเด่น) ก้าน
			ครีบสีน้ำตาลถึงดำ เยื่อครีบมีพื้นสีเขียวหรือฟ้า ลายสีดำ & 10 \\
			\hline
			ครีบก้นหรือชายน้ำ
			(Anal Fin) &
			ก้านครีบเดี่ยว พื้นเนื้อและก้านครีบมีโทนสีน้ำตาลแดงถึงดำ มี
			แถบสีเขียวหรือฟ้าแซมจากโคนถึงปลายระหว่างก้านครีบของแต่
			ละเส้น & 10 \\
			\hline
			ครีบท้องหรือตะเกียบ
			(Pelvic fin) &
			เป็นครีบคู่ พื้นตะเกียบเป็นโทนสีน้ำตาลแดงถึงดำ ก้านครีบเส้น
			แรกมีสีเขียวถึงฟ้า ปลายตะเกียบสีขาวและไม่แตก & 10 \\
			\hline
			ครีบหาง (Caudal Fin) &
			รูปทรงของครีบเป็นทรงพัด หรือทรงใบโพธิ์ ก้านครีบมีทั้งแตก 2
			ถึง 4 (ถ้าแตก 2 ถือเป็นลักษณะเด่น) พื้นเนื้อและก้านครีบโทนสี
			น้ำตาลแดงถึงดำ มีแถบสีเขียวหรือฟ้าแซมระหว่างก้านครีบลาก
			จากโคนก้านไปจนสุดปลายก้านหางของแต่ละเส้น และใน
			ระหว่างปลายก้านครีบหางที่แตกสองจะมีสีเขียวถึงฟ้าแซมทุก
			ช่อง & 15 \\
			\hline
			การพองสู้และการว่ายน้ำ &
			การว่ายน้ำสง่างาม ปราดเปรียว ปลาควรพองสู้ & 10 \\
			\hline
			ภาพรวม &
			ความสง่างาม ความสมบูรณ์ ความมีเสน่ห์ & 20 \\
			\Xhline{0.5pt}
			\bfseries คะแนนรวมทั้งสิ้น & & \bfseries 100 \\
			\Xhline{1.5pt}
		\end{tabularx}
	}% end \tablefont
	%\caption*{ที่มา: อรุณี รอดลอย, 2018, 128 69}
\end{table}
\endgroup