%==================== chapter1_1.tex ====================

\clearpage
\thispagestyle{plain}

\begingroup
\fontsize{16pt}{19.2pt}\selectfont
\justifying
\XeTeXlinebreakskip=0pt plus 1pt minus 0.5pt
\setlength{\parindent}{1.5cm}
\setlength{\parskip}{0pt}

% ---------- วัตถุประสงค์ของงานวิจัย ----------
\section*{วัตถุประสงค์ของงานวิจัย}
\addcontentsline{toc}{section}{วัตถุประสงค์ของงานวิจัย}
\indent 1.\,พัฒนาแพลตฟอร์มที่เป็นศูนย์รวมการจัดประกวดปลากัดไทยบนเว็บแอปพลิเคชัน

\vspace{\baselineskip} % เว้น 1 บรรทัด

% ---------- ขอบเขตของงานวิจัย ----------
\section*{ขอบเขตของงานวิจัย}
\addcontentsline{toc}{section}{ขอบเขตของงานวิจัย}

\indent งานวิจัยนี้มุ่งเน้นการออกแบบและพัฒนาเว็บแอปพลิเคชันศูนย์รวมการจัดประกวดปลากัดไทย
โดยใช้หลักการ Responsive Design รองรับการใช้งานทุกแพลตฟอร์ม โดยมีขอบเขตดังนี้

% ===== รูปแบบ enumerate ให้ตรงสเปกเดิม (พร้อมตัวแปรครบ) =====
% ต้องมี \usepackage{enumitem}

% กำหนดค่าแต่ละตัว
\setlength{\LoneLabelSep}{0.5em}                                 % ระยะเลขกับข้อความ (ระดับ 1)
\settowidth{\LoneLabelWidth}{9.}                                 % กว้างพอ "9."
\setlength{\LoneContentCol}{\dimexpr 1.5cm + \LoneLabelWidth + \LoneLabelSep\relax}

\setlength{\LtwoLabelSep}{0.5em}                                 % ระยะเลขกับข้อความ (ระดับ 2)
\settowidth{\LtwoLabelWidth}{9.9.}                               % กว้างพอ "9.9."
\setlength{\ExtraAlign}{-2.8em}                                  % ปรับชิดเพิ่ม/ลดตามต้องการ

\setlength{\LthreeLabelSep}{0.5em}                               % ระยะเลขกับข้อความ (ระดับ 3)
\settowidth{\LthreeLabelWidth}{9.9.9.}                           % กว้างพอ "9.9.9."
\setlength{\NestedStep}{2em}                                     % ให้ระดับ 3 ลึกกว่า level 2 อีก 2em (ปรับได้)

% ระดับ 1
\setlist[enumerate,1]{%
	label=\arabic*., align=left,
	leftmargin=1.5cm, labelindent=0pt,
	labelwidth=\LoneLabelWidth, labelsep=\LoneLabelSep,
	itemsep=0pt, topsep=0.5\baselineskip
}

% ระดับ 2 (ชิฟต์ด้วย \ExtraAlign ให้ตรงสเปกเดิม)
\setlist[enumerate,2]{%
	label*=\arabic*., align=left,
	leftmargin=*,
	labelwidth=\LtwoLabelWidth, labelsep=\LtwoLabelSep,
	labelindent=\dimexpr \LoneContentCol + \ExtraAlign - \LtwoLabelWidth - \LtwoLabelSep\relax,
	itemsep=0pt, topsep=0pt
}

% ระดับ 3 (จุดเริ่มข้อความ = \LoneContentCol + \ExtraAlign + \NestedStep)
\setlist[enumerate,3]{%
	label*=\arabic*., align=left,
	leftmargin=*,
	labelwidth=\LthreeLabelWidth, labelsep=\LthreeLabelSep,
	labelindent=\dimexpr \LoneContentCol + \ExtraAlign + \NestedStep
	- \LthreeLabelWidth - \LthreeLabelSep\relax,
	itemsep=0pt, topsep=0pt
}

\begin{enumerate}
	\item ผู้ใช้งานที่ไม่มีบัญชีผู้ใช้งาน
	\begin{enumerate}
		\item สามารถลงทะเบียนสมัครสมาชิกได้
		\item สามารถเข้าชมข่าวสารเกี่ยวกับการแข่งขันหรือข้อมูลอื่น ๆ ได้
	\end{enumerate}
	
	\item ผู้เลี้ยงปลากัด
	\begin{enumerate}
		\item ลงทะเบียนและจัดการบัญชีผู้ใช้งานของตนเอง
		\item อัปโหลดข้อมูลปลากัด เช่น รูปภาพและรายละเอียดต่าง ๆ
		\item ขอรับบริการประเมินคุณภาพปลากัดโดยผู้เชี่ยวชาญ
		\item ติดตามผลการประเมินและประวัติการประเมิน
		\item รับการแจ้งเตือนกิจกรรมและข้อมูลใหม่จากระบบ
		\item เข้าร่วมการประกวดปลากัด
	\end{enumerate}
	
	\item ผู้จัดการประกวดปลากัด
	\begin{enumerate}
		\item เพิ่มรายละเอียดการจัดการประกวด
		\item ลงทะเบียนและจัดการบัญชีผู้ใช้งาน
		\item กำหนดการจัดประกวดปลากัดไทย
		\item กำหนดกรรมการตัดสินในการประกวดแต่ละครั้ง
		\item ประกาศผลการแข่งขัน
		\item ดูประวัติการจัดการประกวด
		\item ดูผลคะแนนการแข่งขันทั้งหมดของผู้เลี้ยงปลากัด
	\end{enumerate}
	
	\item ผู้เชี่ยวชาญด้านปลากัด
	\begin{enumerate}
		\item ลงทะเบียนและจัดการโปรไฟล์ผู้เชี่ยวชาญ
		\item ตรวจสอบข้อมูลปลากัดที่ถูกอัปโหลดโดยผู้เลี้ยง
		\item ให้บริการประเมินคุณภาพปลากัดและบันทึกผลการประเมิน
		\item เข้าร่วมการประเมินแบบกลุ่มและกิจกรรมออนไลน์
	\end{enumerate}
\newpage
	\item ผู้บริหารจัดการข้อมูล
	\begin{enumerate}
		\item ดูแลและจัดการระบบทั้งหมด
		\item จัดการข้อมูลผู้ใช้งาน เช่น การปรับปรุงและลบข้อมูล
		\item ติดตามผลการประเมินและกิจกรรมต่าง ๆ ภายในระบบ
		\item จัดการการแจ้งเตือนและข้อมูลสำคัญของระบบ
	\end{enumerate}
\end{enumerate}

% --- เว้น 1 บรรทัด ---
\vspace{\baselineskip}

% ---------- แนวคิดและหลักการ ----------
\noindent{\bfseries\fontsize{16pt}{19.2pt}\selectfont แนวคิดและหลักการ}\par
\indent แนวคิดและหลักการของการพัฒนาเว็บแอปพลิเคชันศูนย์รวมการจัดประกวดปลากัดไทย
คือการสร้างแพลตฟอร์มที่สามารถเชื่อมโยงผู้เลี้ยงปลากัด ผู้เชี่ยวชาญ และผู้จัดการประกวดเข้าด้วยกันอย่างมีประสิทธิภาพ
โดยมีเป้าหมายเพื่อเพิ่มความสะดวกในการเข้าถึงบริการต่าง ๆ เช่น การประเมินคุณภาพปลากัด การให้คำปรึกษาด้านการเพาะพันธุ์
และการติดตามผลการประกวด ระบบจะรองรับการใช้งานทั้งบนอุปกรณ์พกพาและเว็บเบราว์เซอร์ โดยใช้หลักการออกแบบแบบ
Responsive Design เพื่อให้ผู้ใช้งานสามารถเข้าถึงข้อมูลและบริการได้ทุกที่ทุกเวลา

% --- เว้น 1 บรรทัด ---
\vspace{\baselineskip}

% ---------- เครื่องมือที่ใช้ในการพัฒนาโปรแกรม ----------
\noindent{\bfseries\fontsize{16pt}{19.2pt}\selectfont เครื่องมือที่ใช้ในการพัฒนาโปรแกรม}\par

\begin{enumerate}
	\item คอมพิวเตอร์ส่วนบุคคล (Personal Computer)
	\begin{enumerate}
		\item Device name: MacBook Air (M1, 2020)
		\item Processor: Apple M1 chip, 8-core CPU (4 performance cores,\\4 efficiency cores)
		\item Installed RAM: 8 GB unified memory
		\item System type: 64-bit operating system, ARM-based processor
	\end{enumerate}
\end{enumerate}

\par\endgroup
\clearpage
\clearpage
%================== จบ chapter1_1.tex ====================
