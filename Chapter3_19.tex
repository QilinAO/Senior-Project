%==================== chapter3_19.tex ====================

\clearpage
\thispagestyle{plain}

\begingroup
\fontsize{16pt}{19.2pt}\selectfont
\justifying
\XeTeXlinebreakskip=0pt plus 1pt minus 0.5pt
\setlength{\parindent}{1.5cm}
\setlength{\parskip}{0pt}

\vspace{\baselineskip}


% ============================ Class: ContestJudge ============================
\begin{table}[h]
	\caption{Class Description : ContestJudge}
	{\tablefont\setlength{\tabcolsep}{6pt}%
		\begin{tabularx}{\linewidth}{@{} >{\raggedright\arraybackslash}p{3.6cm} X @{}}
			\Xhline{1.5pt}
			\textbf{Class Name :} & ContestJudge \\
			\Xhline{0.5pt}
			\textbf{Description :} & การเป็นกรรมการของผู้เชี่ยวชาญในกิจกรรมการประกวด รวมถึงสถานะคำเชิญ \\
			\Xhline{0.5pt}
			\textbf{Attribute :} &
			\begin{tabular}{@{}l@{}}
				id: long — รหัสแถว (PK) \\
				judge\_id: UUID — ผู้เชี่ยวชาญ (Profile.id) \\
				contest\_id: UUID — การประกวด \\
				status: string — สถานะคำเชิญ/บทบาท (invited, accepted, declined ฯลฯ) \\
				assigned\_at: datetime — เวลามอบหมาย
			\end{tabular} \\
			\Xhline{0.5pt}
			\textbf{Method :} &
			\begin{tabular}{@{}l@{}}
				getJudgingContests(): ดึงรายการประกวดที่เกี่ยวข้อง \\
				respondToJudgeInvitation(contestId, response, reason): ตอบรับ/ปฏิเสธ \\
				assignJudgeToContest(contestId, judgeId): มอบหมายกรรมการ \\
				removeJudgeFromContest(contestId, judgeId): ถอดกรรมการ \\
				notifyJudgeRemoval(contestId, judgeId, options): แจ้งเตือนการถอด
			\end{tabular} \\
			\Xhline{1.5pt}
	\end{tabularx}}
\end{table}

%\newpage

% ============================ Class: Assignment ============================
\begin{table}[h]
	\caption{Class Description : Assignment}
	{\tablefont\setlength{\tabcolsep}{6pt}%
		\begin{tabularx}{\linewidth}{@{} >{\raggedright\arraybackslash}p{3.6cm} X @{}}
			\Xhline{1.5pt}
			\textbf{Class Name :} & Assignment \\
			\Xhline{0.5pt}
			\textbf{Description :} & งานที่มอบหมายให้ผู้เชี่ยวชาญประเมินผลงาน เก็บคะแนนย่อยและผลรวม \\
			\Xhline{0.5pt}
			\textbf{Attribute :} &
			\begin{tabular}{@{}l@{}}
				id: long — รหัสงาน (PK) \\
				submission\_id: UUID — อ้างอิงผลงาน \\
				evaluator\_id: UUID — ผู้ประเมิน (Profile.id) \\
				status: string — สถานะงาน (queued, accepted, rejected, evaluated ฯลฯ) \\
				scores: JSON — รายละเอียดคะแนน \\
				total\_score: decimal — คะแนนรวม \\
				assigned\_at: datetime — เวลาได้รับมอบหมาย \\
				evaluated\_at: datetime — เวลาประเมินเสร็จ
			\end{tabular} \\
			\Xhline{0.5pt}
			\textbf{Method :} &
			\begin{tabular}{@{}l@{}}
				getEvaluationQueue(): ดึงคิวงานประเมินของผู้เชี่ยวชาญ \\
				respondToEvaluation(assignmentId, status, reason): ตอบรับ/ปฏิเสธงาน \\
				submitQualityScores(assignmentId, scoresData): ส่งคะแนนโหมดคุณภาพ \\
				getScoringSchema(bettaType, options): ดึงเกณฑ์การให้คะแนนแบบไดนามิก
			\end{tabular} \\
			\Xhline{1.5pt}
	\end{tabularx}}
\end{table}

\newpage

% ============================ Class: Notification ============================
\begin{table}[h]
	\caption{Class Description : Notification}
	{\tablefont\setlength{\tabcolsep}{6pt}%
		\begin{tabularx}{\linewidth}{@{} >{\raggedright\arraybackslash}p{3.6cm} X @{}}
			\Xhline{1.5pt}
			\textbf{Class Name :} & Notification \\
			\Xhline{0.5pt}
			\textbf{Description :} & การแจ้งเตือนของระบบที่ส่งถึงผู้ใช้แต่ละคน ใช้แจ้งผลการประกวด การมอบหมาย หรือการเปลี่ยนสถานะ \\
			\Xhline{0.5pt}
			\textbf{Attribute :} &
			\begin{tabular}{@{}l@{}}
				id: long — รหัสแจ้งเตือน (PK) \\
				user\_id: UUID — ผู้รับ \\
				message: string — ข้อความ \\
				link\_to: string — ลิงก์ไปยังหน้าที่เกี่ยวข้อง \\
				is\_read: boolean — อ่านแล้วหรือยัง \\
				type: string — ประเภทแจ้งเตือน \\
				created\_at: datetime — เวลาแจ้งเตือน
			\end{tabular} \\
			\Xhline{0.5pt}
			\textbf{Method :} &
			\begin{tabular}{@{}l@{}}
				getMyNotifications(opts): ดึงรายการแจ้งเตือนของผู้ใช้ \\
				markNotificationRead(id): ทำรายการเดียวเป็นอ่านแล้ว \\
				markAllNotificationsRead(): ทำทั้งหมดเป็นอ่านแล้ว \\
				countUnread(): นับจำนวนที่ยังไม่อ่าน
			\end{tabular} \\
			\Xhline{1.5pt}
	\end{tabularx}}
\end{table}

\clearpage