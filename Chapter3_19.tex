%==================== chapter3_19.tex ====================

\clearpage
\thispagestyle{plain}

\begingroup
\fontsize{16pt}{19.2pt}\selectfont
\justifying
\XeTeXlinebreakskip=0pt plus 1pt minus 0.5pt
\setlength{\parindent}{1.5cm}
\setlength{\parskip}{0pt}

\vspace{\baselineskip}


% ============================== Class: ManagerService ===========================
\begin{table}[h]
	\caption{Class Description : ManagerService}
	{\tablefont\setlength{\tabcolsep}{6pt}%
		\begin{tabularx}{\linewidth}{@{} >{\raggedright\arraybackslash}p{3.6cm} D @{}}
			\Xhline{1.5pt}
			\textbf{Class Name :} & ManagerService \\
			\Xhline{0.5pt}
			\textbf{Description :} & Service ที่รวบรวม Business Logic หลักสำหรับ Manager เช่น การสร้าง/อัปเดตการประกวด และการจัดการกรรมการ ติดต่อ SupabaseAdmin และส่งการแจ้งเตือนผ่าน NotificationService \\
			\Xhline{0.5pt}
			\textbf{Attribute :} & - \\
			\Xhline{0.5pt}
			\textbf{Method :} &
			\begin{tabular}{@{}l@{}}
				createContestOrNews(managerId, data): สร้างการประกวด/ข่าวสารลงในฐานข้อมูล \\
				assignJudgeToContest(contestId, judgeId): บันทึกการมอบหมายกรรมการลงในฐาน\\ข้อมูล
			\end{tabular} \\
			\Xhline{1.5pt}
	\end{tabularx}}
\end{table}

% ============================ Class: ExpertController ===========================
\begin{table}[h]
	\caption{Class Description : ExpertController}
	{\tablefont\setlength{\tabcolsep}{6pt}%
		\begin{tabularx}{\linewidth}{@{} >{\raggedright\arraybackslash}p{3.6cm} D @{}}
			\Xhline{1.5pt}
			\textbf{Class Name :} & ExpertController \\
			\Xhline{0.5pt}
			\textbf{Description :} & Controller จุดรับ Endpoint ของ Expert จัดการคำขอเกี่ยวกับการประเมินผลและการให้คะแนน \\
			\Xhline{0.5pt}
			\textbf{Attribute :} & - \\
			\Xhline{0.5pt}
			\textbf{Method :} &
			\begin{tabular}{@{}l@{}}
				getEvaluationQueue(req, res): ดึงคิวงานประเมินคุณภาพ \\
				submitQualityScores(req, res): ส่งผลคะแนนการประเมิน
			\end{tabular} \\
			\Xhline{1.5pt}
	\end{tabularx}}
\end{table}

% ============================== Class: ExpertService ============================
\begin{table}[h]
	\caption{Class Description : ExpertService}
	{\tablefont\setlength{\tabcolsep}{6pt}%
		\begin{tabularx}{\linewidth}{@{} >{\raggedright\arraybackslash}p{3.6cm} D @{}}
			\Xhline{1.5pt}
			\textbf{Class Name :} & ExpertService \\
			\Xhline{0.5pt}
			\textbf{Description :} & Service จัดการ Logic สำหรับผู้เชี่ยวชาญ เช่น การดึงคิวงาน การตอบรับงาน และการบันทึกคะแนน \\
			\Xhline{0.5pt}
			\textbf{Attribute :} & - \\
			\Xhline{0.5pt}
			\textbf{Method :} &
			\begin{tabular}{@{}l@{}}
				getEvaluationQueue(expertId): ดึงคิวงานประเมินคุณภาพที่ยังไม่เสร็จสิ้นของ\\ผู้เชี่ยวชาญ 
				submitQualityScores(assignmentId, scoresData): \\ บันทึกผลคะแนนลงฐานข้อมูลและอัปเดตสถานะงาน
			\end{tabular} \\
			\Xhline{1.5pt}
	\end{tabularx}}
\end{table}