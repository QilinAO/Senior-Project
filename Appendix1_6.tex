% --- ตาราง: ลักษณะเฉพาะของปลากัดป่าภาคใต้และเกณฑ์การให้คะแนนในการประกวด (บังคับ 13pt) ---
\begingroup
\renewcommand{\arraystretch}{1.15}
\setlength{\arrayrulewidth}{0.5pt}

% ถ้ามี hook \AtBeginEnvironment{tabularx}{\tablefont} อยู่ ให้กำหนด \tablefont เฉพาะที่นี่เป็น 13pt
\makeatletter
\@ifundefined{tablefont}{\newcommand{\tablefont}{}}{}%
\makeatother
\renewcommand{\tablefont}{\fontsize{13pt}{15.6pt}\selectfont} % << ฟอนต์คงที่ 13pt

\begin{table}[h]
	\captionsetup{justification=raggedright, singlelinecheck=false,
		labelfont=bf, textfont=bf}
	\caption{ลักษณะเฉพาะของปลากัดป่าสายพัฒนาและเกณฑ์การให้คะแนนในการประกวด}
	\centering
	
	{\tablefont % ย้ำอีกชั้นให้แน่ใจว่าเนื้อหาในตารางใช้ 13pt
		\begin{tabularx}{\textwidth}{@{}>{\raggedright\arraybackslash}p{2.8cm}
				>{\raggedright\arraybackslash}X
				>{\centering\arraybackslash}p{1.6cm}@{}}
			\Xhline{1.5pt}
			\bfseries ลักษณะ & \bfseries ลักษณะเด่นตามมาตรฐาน & \bfseries คะแนน \\
			\hline
			ส่วนหัวและลำตัว &
				ลำตัวทรงกระบอก (รูปทรงเป็นลักษณะปลาป่ามากที่สุด) & 20 \\
			\hline
			แก้มและเกล็ด &
			สี และลวดลายต้องชัดเจน ทั้ง 2 ข้าง & 20 \\
			\hline
			ครีบหลัง (Dorsal Fin)ครีบหลังหรือกระโดง
			(Dorsal Fin) &
			ก้านครีบเดี่ยว & 5 \\
			\hline
			ครีบก้นหรือชายน้ำ
			(Anal Fin) &
			ก้านครีบเดี่ยว ครีบเหยียดตรง เสมอกันทั้ง 2 ข้าง & 5 \\
			\hline
			ครีบท้องหรือตะเกียบ
			(Pelvic fin) &
			รูปทรงของครีบเป็นทรงใบโพธิ์ หรือพัด ก้านครีบแตกสอง ถือว่า
			เป็นลักษณะเด่น & 10 \\
			\hline
			ครีบหาง (Caudal Fin) &
			รูปทรงของครีบเป็นทรงใบโพธิ์ หรือพัด ก้านครีบแตกสอง ถือว่า
			เป็นลักษณะเด่น & 10 \\
			\hline
			การพองสู้และการว่ายน้ำ &
			การว่ายน้ำสง่างาม ปราดเปรียว ปลาควรพองสู้ & 10 \\
			\hline
			ภาพรวม &
			ความสง่างาม ความสมบูรณ์ ความมีเสน่ห์ & 20 \\
			\Xhline{0.5pt}
			\bfseries คะแนนรวมทั้งสิ้น & & \bfseries 100 \\
			\Xhline{1.5pt}
		\end{tabularx}
	}% end \tablefont
	%\caption*{ที่มา: อรุณี รอดลอย, 2018, 128 69}
\end{table}
\endgroup